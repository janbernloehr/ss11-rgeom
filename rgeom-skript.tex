% =============================================================================
% Titel:	Riemannsche Geometrie
% Erstellt:	SS 2011, Universität Stuttgart
% Dozent:	Prof Dr. Uwe Semmelmann
% =============================================================================
\documentclass[%
	paper=a5,%
	fleqn,%
	DIV=18,%
	BCOR=0mm,
	fontsize=11pt,
	titlepage=false,%
	bibliography=totoc,
	DIV=18,%
	twoside=true,
	pdftitle=Riemannsche Geometrie,
	pdfauthor=Uwe Semmelmann,
	numbers=noendperiod]%
	{scrbook}
	
\usepackage{scrpage2}
\usepackage[lc,color,lgerman]{jancommon}
\usepackage{janmath,custom}
 
\makeindex

\begin{document}

% titlepage
\begin{titlepage}
\bfseries\color{darkgray}
\vspace*{2mm}
\noindent
Prof. Dr. Uwe Semmelmann, Universität Stuttgart

\begin{center}
\vspace*{10mm}
\noindent
{\huge\color{darkblue} Riemannsche Geometrie}

\vspace*{4mm}

Stuttgart, Sommersemester 2011
\end{center}

\vspace*{\fill}

\begin{flushright}
\small
Version: \today\
\vspace*{5mm}

Für Hinweise auf Druckfehler und Kommentare jeder Art bin ich 
dankbar.\footnote{\color{darkgray}
Prof. Dr. Uwe Semmelmann, Universität Stuttgart,
\href{mailto:uwe.semmelmann@mathematik.uni-stuttgart.de}{uwe.semmelmann@mathematik.uni-stuttgart.de};\\
Jan-Cornelius Molnar,
\href{mailto:jan.molnar@studentpartners.de}{jan.molnar@studentpartners.de}}
Viel Spaß!
\end{flushright}
\end{titlepage}


% Inhaltsverzeichnis
\tableofcontents


\chapter{Hauptfaserbündel und assoziierte Faserbündel}

\section{Lokal triviale Faserung}

\begin{defn}
\index{Faser}\index{Faserung}\index{Faserung!lokal-triviale}
\index{Faser!-typ}\index{lokale Trivialität}\index{Basis}
\index{Totalraum}\index{Projektion}
Sei $\pi\colon E\to M$ eine glatte Abbildung zwischen Mannigfaltigkeiten $E$ und $M$
und $F$ eine weitere Mannigfaltigkeit. Man nennt das Tupel $(E,\pi,M;F)$
\emph{lokal-triviale Faserung vom Fasertyp $F$}, falls es zu jedem Punkt $x\in
M$ eine Umgebung $U\subset M$ von $x$ und einen Diffeomorphismus
\begin{align*}
\phi_U : \pi^{-1}(U)\to U\times F\tag{*}
\end{align*}
gibt mit $\pr_1\circ\phi_U = \pi$.\fish
\end{defn}


Eigenschaft (*) nennt man \emph{lokale Trivialität} und die Abbildungen
$\phi_U$ \emph{lokale Trivialisierungen}. $E$ heißt \emph{Totalraum}, $M$
\emph{Basis} und $\pi$ \emph{Projektion}.
Für jedes $x\in M$ ist die \emph{Faser} $E_x \defl \pi^{-1}(x) \subset E$
eine Untermannigfaltigkeit von $E$ und unter dem Diffeomorphismus
\begin{align*}
\phi_{U,x} = \pr_2\circ\phi_U\big|_{E_x} \colon E_x\to F
\end{align*}
diffeomorph zu $F$. Insbesondere sind alle Fasern $E_x$ für $x\in M$ zueinander
diffeomorph.

\begin{figure}
\centering
\begin{pspicture}(0,-1.81)(3.84,1.67)
\color{darkgray}

\psframe[fillstyle=solid,fillcolor=lightgray](2.0,1.67)(0.44,-0.35)
\psframe(3.04,1.67)(0.0,-0.35)

\psline(0.04,-1.31)(0.46,-1.31)
\psline(1.96,-1.31)(3.06,-1.31)
\psline[linecolor=darkblue](0.46,-1.31)(1.96,-1.31)
\psline{->}(3.46,0.55)(3.46,-1.05)

\rput(3.47,0.875){$E$}
\rput(3.48,-1.325){$M$}
\rput(3.72,-0.185){$\pi$}
\rput(1.2,1.395){$\pi^{-1}(U)$}
\rput(0.99,-1.565){$x$}

\psdots[linecolor=darkblue,dotsize=0.12](1.0,-1.31)
\psarc(0.56,-1.31){0.12}{90.0}{270.0}
\psarc(1.86,-1.31){0.12}{270.0}{90.0}
\end{pspicture}
% ============================================
\qquad\qquad\qquad\qquad
\begin{pspicture}(0,-1.81)(3.32,1.77)
\color{darkgray}

\psframe[linecolor=darkblue,fillstyle=solid,fillcolor=darkblue,opacity=0.25]%
	(0.46,1.77)(1.98,-0.25)
\psframe[linecolor=purple,fillstyle=solid,fillcolor=purple,opacity=0.25]%
	(2.7,1.77)(1.16,-0.25)
% 
% \psline[linecolor=purple](2.7,1.75)(2.7,-0.23)
% \psline[linecolor=purple](1.16,1.75)(1.16,-0.23)
% \psline[linecolor=darkblue](0.46,1.75)(0.46,-0.23)
% \psline[linecolor=darkblue](1.98,1.75)(1.98,-0.23)

\psframe(3.28,1.77)(0.0,-0.25)

\rput(0.98,-1.585){$U_i$}
\rput(2.25,-1.585){$U_j$}

\psline(0.04,-1.21)(0.44,-1.21)
\psline[linecolor=darkblue](0.44,-1.21)(1.16,-1.21)
\psline[linecolor=purple](1.99,-1.21)(2.74,-1.21)
\psline[linecolor=darkyellow](1.16,-1.21)(1.98,-1.21)
\psline(2.74,-1.21)(3.3,-1.21)

\psarc[linecolor=darkblue](0.56,-1.21){0.12}{90.0}{270.0}
\psarc[linecolor=darkblue](1.86,-1.21){0.12}{270.0}{90.0}
\psarc[linecolor=purple](1.28,-1.21){0.12}{90.0}{270.0}
\psarc[linecolor=purple](2.62,-1.21){0.12}{270.0}{90.0}

\end{pspicture} 
\caption{Zur lokalen Faserung und zur Übergangsabbildung.}
\end{figure}

\begin{ex}
Die triviale Faserung $E= M\times F$, wobei $\pi$ dann der Projektion auf den
ersten Faktor entspricht.\boxc
\end{ex}

Analog zu Kartenwechseln auf Mannigfaltigkeiten, kann man nun
Abbildungen definieren, die zwischen den lokalen Trivialisierungen wechseln.

\begin{defn}
\index{Übergangsfunktion}
Sei $\setd{U_i}$ eine offene Überdeckung von $M$ und $\phi_i\defl \phi_{U_i}$
die zugehörigen lokalen Trivialisierungen von $E$. Dann heißt die Abbildung
\begin{align*}
\phi_i\circ\phi_k^{-1} \colon (U_i\cap U_k)\times F\to (U_i\cap U_k)\times F
\end{align*}
\emph{Übergangsfunktion} zwischen den Bündelkarten $(U_i,\phi_i)$ und
$(U_k,\phi_k)$.\fish
\end{defn}


Durch diese Definition werden Abbildungen
\begin{align*}
\phi_{ik}\colon U_i\cap U_k\to \Diff(F),\qquad x\mapsto
\phi_{i,x}\circ\phi_{k,x}^{-1}
\end{align*}
induziert. Später werden wir sehen, dass das Bündel bereits vollständig durch
die $\phi_{ik}$ beschrieben wird. Das ermöglicht die Konstruktion von Bündeln
durch Vorgabe der $\phi_{ik}$ und eine Klassifikation der Bündel über $M$.

Als unmittelbare Folgerung aus der Definition erhalten wir folgende Aussage.

\begin{lem}
\index{Kozyklen-Bedingung}
Die Abbildungen $\phi_{ik}$ erfüllen die \emph{Kozyklen-Bedingung},
\begin{propenum}
\item $\phi_{ii} = \id$,
\item $\phi_{ik}(x)\circ\phi_{kj}(x) = \phi_{ij}(x)$,\qquad
auf $ \;U_i\cap U_j\cap U_k$.\fish
\end{propenum}
\end{lem}


\begin{defn}
\index{Faserung!isomorph}
Zwei lokal triviale Faserungen $(E,\pi,M;F)$ und
$(\tilde{E},\tilde{\pi},M;\tilde{F})$ heißen \emph{isomorph}, falls ein
Diffeomorphismus
\begin{align*}
\psi\colon E\to \tilde{E}
\end{align*}
existiert, mit $\tilde{\pi}\circ\psi = \pi$.\fish
\end{defn}

Ein solcher Diffeomorphismus überführt Fasern in Fasern. Insbesondere sind $F$
und $\tilde{F}$ diffeomorph.

\begin{figure}
\centering
\begin{tikzpicture}[description/.style={fill=white,inner sep=2pt}]
\matrix (m) [matrix of math nodes, row sep=3em,
column sep=2.5em, text height=1.5ex, text depth=0.25ex]
{ E & & \tilde{E} \\
& M & \\ };
\path[->,font=\scriptsize]
(m-1-1) edge node[auto] {$ \psi $} (m-1-3)
edge node[auto,swap] {$ \pi $} (m-2-2)
(m-1-3) edge node[auto] {$ \tilde{\pi} $} (m-2-2);
\end{tikzpicture}
\caption{Kommutatives Diagramm für isomorphe Faserungen.}
\end{figure}


\begin{rem}
\index{Faserung!trivial}
Eine Faserung heißt \emph{trivial}, falls sie isomorph ist zu $(M\times
F,\pr_1,M;F)$.
%Ein wichtiges Problem in der Differentialgeometrie ist es, Bedingungen für die
%Trivialität von Faserungen zu finden.\map
\end{rem}

\begin{ex}
\begin{exenum}
\item \textit{Überlagerungen} $p\colon \tilde{M}\to M$ sind lokal triviale
Faserungen mit diskreter Faser.
\item \textit{Tangentialbündel} $(TM,\pi,M;\R^n)$ und
\textit{Formenbündel} $(\Lambda^k TM,\pi,M;\Lambda^k\R^n)$ sind lokal triviale
Faserungen, euklidische Vektorräume bilden ihre Faser.
\item \index{Rahmenbündel}\textit{Rahmenbündel}. Gegeben eine Mannigfaltigkeit
$M$, definiert man
\begin{align*}
E = P_{\GL_n} \defl
\setdef{p}{p\text{ ist Basis von }T_xM,\text{ für ein }x\in M}.
\end{align*}
Dann ist $E$ eine Mannigfaltigkeit und durch $(E,\pi,M;F)$ eine lokal triviale
Faserung gegeben, wobei $\pi$ die Fußpunktabbildung ist, d.h.
\begin{align*}
\pi(p) = x,\qquad \text{falls } p = \setd{e_1,\ldots,e_n} \text{ Basis von
}T_xM.
\end{align*}
Die Faser $E_x$ ist gerade die Menge aller Basen von $T_xM$ und folglich ist $F
= \GL_n$.
\item \index{Pullback-Bündel}\textit{Pullback-Bündel}. Sei $f\colon N\to M$ eine
glatte Abbildung und $(E,\pi,M;F)$ eine lokal triviale Faserung. Man definiert nun
\begin{align*}
f^*E \defl \setdef{(y,e)\in N\times E}{f(y) = \pi(e)} \subset N\times E, 
\end{align*}
mit Projektionsabbildung $\bar{\pi}(y,e) = y$.
Für den Spezialfall einer Untermannigfaltigkeit $i\colon N\opento M$ ist gerade $i^*E
= E\big|_N$.\boxc
\end{exenum}
\end{ex}


\begin{figure}[H]
\centering
\begin{tikzpicture}[description/.style={fill=white,inner sep=2pt}]
\matrix (m) [matrix of math nodes, row sep=3em,
column sep=2.5em, text height=1.5ex, text depth=0.25ex]
{ f^*E & & E \\
N &  & M \\ };
\path[->,font=\scriptsize]
(m-1-1) edge node[auto] {$ \pr_2 $} (m-1-3)
	    edge node[auto,swap] {$ \bar{\pi} $} (m-2-1)
(m-2-1) edge node[auto] {$ f $} (m-2-3)
(m-1-3) edge node[auto] {$ \pi $} (m-2-3);
\end{tikzpicture}
\caption{Kommutatives Diagramm für das Pullback-Bündel.}
\end{figure}


\begin{defn}
\index{Schnitt!glatter}
Ein \emph{glatter Schnitt} einer lokal trivialen Faserung $(E,\pi,M;F)$ ist eine
glatte Abbildung
\begin{align*}
s: M\to E    \qquad \text{mit }   \;  \pi\circ s = \id_M.
\end{align*}
D.h. für alle $x\in M$ ist $s(x)\in E_x = \pi^{-1}(x)$. Der \emph{Raum aller
Schnitte} wird mit $\Gamma(E)$ bezeichnet.
\fish
\end{defn}

Die Existenz glatter Schnitte, d.h. $\Gamma(E)\neq \varnothing$, ist nicht per
se gesichert, sondern an topologische und geometrische Bedingungen an die
Faserung geknüpft. Für hinreichend kleine Umgebungen $U\subset M$ existieren
jedoch immer lokale Schnitte.

\begin{defn}
\index{Schnitt!lokaler}
Sei $U\subset M$ eine offene Umgebung in $M$. Ein Schnitt in $E_U \defl
E\big|_U= \pi^{-1}(U)$ heißt \emph{lokaler Schnitt} von
$E$ über $U$. Der Raum der lokalen Schnitte über $U$ wird mit $\Gamma(U,E)$
bezeichnet.\fish
\end{defn}

\begin{rem}
Ist der Fasertyp ein euklidischer Raum, d.h. $F\cong \R^n$, so existiert ein
allgemeiner Fortsetzungssatz: Sei $A\subset M$ abgeschlossen und $s\colon A\to E$
glatt --- d.h. $s\colon A^\circ \to E$ ist glatt und für jedes $x\in\partial A$
existiert eine Umgebung $U\subset M$ von $x$, so dass $s\colon U\to E$ ebenfalls
glatt ist ---, dann existiert eine glatte globale Fortsetzung von $s$ zu
\begin{align*}
\tilde{s}\colon M\to E,\qquad \tilde{s}\big|_A = s.
\end{align*}
Da es immer möglich ist, einen Schnitt zu finden, der in einem Punkt glatt ist, lässt
sich auch $\Gamma(E)\neq\varnothing$ immer garantieren.\map
\end{rem}

\section{Hauptfaserbündel}

Von besonderem Interesse sind Faserungen, auf denen eine Lie-Gruppe wirkt,
die mit der Faserung verträglich ist.

\begin{defn}
\index{Hauptfaserbündel (HFB)}
Sei $G$ eine Lie-Gruppe und $\pi\colon P\to M$ eine glatte Abbildung. Das Tupel
$(P,\pi,M;G)$ heißt \emph{$G$-Hauptfaserbündel ($G$-HFB) auch
$G$-Prinzipalbündel} über $M$, falls gelten:
\begin{defnenum}
  \item Die Gruppe $G$ wirkt von rechts auf $P$ und die Wirkung ist \emph{fasertreu},
  d.h.
  \begin{align*}
  P_x\cdot g \subset P_x,\qquad \text{für alle }x\in M\text{ und }g\in G,
  \end{align*}
  und die Gruppenwirkung ist \emph{einfach transitiv} auf den Fasern, d.h. für
  $x\in M$ und zu $p,q\in P_x$ gegeben existiert genau ein $g\in G$, so dass
  $p\cdot g = q$.
  \item Es gibt einen Bündelatlas $\setd{(U_i,\phi_i)}$ aus $G$-äquivarianten
  Bündelkarten, d.h.
  \begin{align*}
  \phi_i : P_{U_i}\to U\times G,\qquad
  \phi_i(p) = (\pi(p),S_{U_i}(p)),
  \end{align*}
  ist ein äquivarianter  Diffeomorphismus, wobei $S_{U_i}\colon P_{U_i}\to G$ glatt, und
  \begin{align*}
  \phi_i(p\cdot g) = \phi_i(p)\cdot g \defr (\pi(p),S_{U_i}(p)g),\qquad \forall g\in
  G,\quad p\in P_{U_i},
  \end{align*}
  d.h. es gilt $S_{U_i}(p\cdot g) = S_{U_i}(p) \cdot g$ für alle $g \in G, p \in P_{U_i}$. \fish
\end{defnenum}
\end{defn}

In der Eichfeldtheorie nennt man die Wahl der Bündelkarten $\{U_i, \phi_{U_i}\}$ auch Wahl
einer \emph{Eichung}.

\begin{rem}[Notation.]
Um ein $G$-Hauptfaserbündel $(P,\pi,M;G)$ zu bezeichnen, schreiben wir auch oft
nur $\pi\colon P\to M$ bzw. lassen die Projektion ganz weg und bezeichnen das
$G$-Hauptfaserbündel mit $P\to M$.\map
\end{rem}

\begin{rem}[Bemerkungen.]
\begin{remenum}
\item
Die Rechtswirkung von $G$ lässt sich auch als eine Darstellung
\begin{align*}
r: G\to \Diff(P),\qquad g\mapsto r_g,
\end{align*}
beschreiben, d.h.  $r_e = \id$ und $r_g\circ r_h = r_{h\cdot g}$. Um die Notation
kurz zu halten, setzen wir zur Abkürzung $r_g(p)\defl p\cdot g$.
\item
$M$ ist der $G$-Quotientenraum von $P$,
\begin{align*}
M = P/G = P/_\sim,\qquad p_1\sim p_2 \iff \exists g : p_2 = p_1\cdot g.
\end{align*}
Man hätte das Hauptfaserbündel auch über diese Äquivalenzrelation als
Quotientenraum definieren können. Insbesondere wirkt $G$ nach dieser Definition
frei auf $P$. Die Faser $\pi^{-1}(x)$ ist dann die $G$-Bahn durch einen Punkt $p\in E$ mit
$\pi(p) = x$ und die Abbildung
\begin{align*}
G \to \pi^{-1}(x),\qquad g\mapsto p\cdot g
\end{align*}
ist ein Diffeomorphismus, d.h. alle Fasern $P_x$ sind diffeomorph zu $G$.\map 
\end{remenum}
\end{rem}





Zwei lokal triviale Faserungen $\pi\colon E\to M$ und $\tilde{\pi}\colon \tilde{E}\to M$
heißen isomorph, falls es einen Diffeomorphismus $\psi\colon E\to \tilde{E}$ gibt,
der Fasern in Fasern überführt. Bei Hauptfaserbündeln ist es sinnvoll zusätzlich
noch zu fordern, dass dieser Isomorphismus mit der $G$-Wirkung verträglich ist.

\begin{defn}
\index{Hauptfaserbündel!isomorph}
Zwei $G$-Hauptfaserbündel über der Basis $M$, $(P,\pi,M;G)$ und
$(\tilde{P},\tilde{\pi},M;G)$ heißen \emph{isomorph}, falls es einen
$G$-äquivarianten Isomorphismus $\psi\colon P\to \tilde{P}$ gibt, der die Fasern
erhält, d.h.  $\psi(p \cdot g) = \psi(p) \cdot g $ und  $\tilde{\pi} \circ \psi=\pi$. 
Folgendes Diagramm kommutiert:

\centering
\begin{tikzpicture}[description/.style={fill=white,inner sep=2pt}]
\matrix (m) [matrix of math nodes, row sep=3em,
column sep=2.5em, text height=1.5ex, text depth=0.25ex]
{ P & & \tilde{P} \\
& M & \\ };
\path[->,font=\scriptsize]
(m-1-1) edge node[auto] {$ \psi $} (m-1-3)
edge node[auto,swap] {$ \pi $} (m-2-2)
(m-1-3) edge node[auto] {$ \tilde{\pi} $} (m-2-2);
\end{tikzpicture}
\fish
\end{defn}

\begin{rem}
\index{Hauptfaserbündel!trivial}
Ein $G$-Hauptfaserbündel heißt \emph{trivial}, falls es isomorph ist zum
trivialen Bündel
$
(M\times G,\pr_1,M;G).
$, d.h. falls eine globale Trivialisierung existiert.

%Es ist in der Regel nicht offensichtlich, ob ein gegebenes Bündel trivial ist.
%Entweder findet man direkt eine globale Trivialisierung $\psi$ oder man findet
%topologische Obstruktionen, die die Existenz einer solchen Trivialisierung
%ausschließen. Im Falle von Hauptfaserbündeln besteht außerdem eine bijektive
%Beziehung zwischen globalen Trivialisierungen und globalen Schnitten. Es
%ist oftmals einfacher, die Existenz eines globalen Schnitts
%nachzuweisen, als direkt den Diffeomorphismus $\psi$ anzugeben.\map
\end{rem}






\begin{ex}
\begin{exenum}
\item Für jede Lie-Gruppe $G$ ist das triviale $G$-Hauptfaserbündel definiert
durch $P=M\times G\to M$ mit der $G$-Wirkung $(x,g)\cdot h\defl (x,g\cdot h)$.
\item \index{Rahmenbündel}
Das \textit{Rahmenbündel} ist definiert durch
\begin{align*}
P_{\GL_n} \defl \setdef{ p= (e_1,\ldots,e_n)}{(e_1,\ldots,e_n)\text{ Basis von
}T_xM\text{ für ein }x\in M},
\end{align*} 
mit der $\GL_n$-Wirkung $(e_1,\ldots,e_n)\cdot g = (e_1',\ldots,e_n')$,
\begin{align*}
e_k' = \sum_{1\le l\le n} g_{kl}e_l.
\end{align*}
Die Fasern $P_x = \pi^{-1}(x)$ sind dann gerade alle Vektorraum-Basen von
$T_xM$, d.h. $P_x \cong \GL_n$, für $n= \dim M$.

Die lokale Trivialität wird beschrieben durch die Abbildungen
\begin{align*}
S_{U_i}\colon P_{U_i} \to \GL_n,
\end{align*}
für $U_i\subset M$ offen, mit
\begin{align*}
S_{U_i}(x,e_1,\ldots,e_n) = (\dx_k(e_l))_{kl},
\end{align*}
wobei $(\dx_k(e_l))_{kl}$ die Transformationsmatrix beschreibt, die die
Basis $(e_1,\ldots,e_n)$ in die Standardbasis überführt.
\item \textit{Homogene Räume}. Sei $H\subset G$ eine abgeschlossene Untergruppe,
dann ist $G/H$ ein homogener Raum und durch
\begin{align*}
\pi\colon G\to G/H,\qquad g\mapsto [g]
\end{align*}
ist ein $H$-Hauptfaserbündel $\pi\colon G\to G/H$ gegeben. Die Gruppenwirkung
entspricht gerade der Rechtsmultiplikation mit Elementen aus $H\subset G$.
\item Die $n$-dimensionale Sphäre lässt sich als homogener Raum auffassen, 
\begin{align*}
S^n =\O(n+1)/\O(n).
\end{align*}
Somit ist $O(n+1)\to S^n$ ein $\O(n)$-Hauptfaserbündel.
\item Jede Mannigfaltigkeit $M$ besitzt eine universelle Überlagerung
$\tilde{M}\to M$, d.h. die Fundamentalgruppe von $\tilde{M}$ ist trivial
$\pi_1(\tilde{M}) = 1$. Durch die universelle Überlagerung ist  ein
$\pi_1(M)$-Hauptfaserbündel $\tilde{M}\to M$ gegeben.
\item \textit{Hopf-Faserung}. 
Man betrachtet die $2n+1$-dimensionale Sphäre eingebettet in $\C^{n+1}$ und
identifiziert jeden Punkt auf der Sphäre mit einer Geraden in der komplexen
Ebene,
\begin{align*}
\pi\colon S^{2n+1}\subset\C^{n+1}\to \CP^n,\qquad v\mapsto [v].
\end{align*}
Man erhält ein $\U(1)$-Hauptfaserbündel $\pi\colon S^{2n+1}\to \CP^n$; alle Fasern sind
diffeomorph zu $S^1\cong \U(1)$. Somit ist $S^{2n+1}\to \CP^n$ ein
$\U(1)$-Hauptfaserbündel und die Gruppenwirkung ist gerade die
Standard-Skalarmultiplikation in $\C^{n+1}$,
\begin{align*}
r_z(v) = v\cdot z,\qquad z\in S^1 \subset \C,\quad v\in S^{2n+1}.
\end{align*}
Die Standardüberdeckung von $\CP^n$ lässt sich in homogenen Koordinaten
ausdrücken als
\begin{align*}
U_i \defl \setdef{[z_0:\ldots:z_i:\ldots :z_n]}{z_i\neq 0},\qquad 0\le i\le n.
\end{align*}
Man definiert nun Abbildungen
\begin{align*}
S_{U_i} \colon \pi^{-1}(U_i) \to  \U(1),\qquad
(z_0,\ldots,z_n)\mapsto \frac{z_i}{\abs{z_i}},\qquad 0\le i\le n.
\end{align*}
Diese spielen gerade die Rolle der $S_{U_i}=\pr_2\circ \phi_{U_i}$ in der Definition des
Hauptfaserbündels.\boxc
\end{exenum}
\end{ex}

\begin{defn}
Sei $P$ ein $G$-Hauptfaserbündel über $M$ und $\setd{U_i}$ eine offene
Überdeckung von $M$. Man definiert nun Abbildungen
\begin{align*}
g_{ij} \colon U_i\cap U_j\to G,\qquad g_{ij}(x)\defl S_{U_i}(p)S_{U_j}(p)^{-1},
\end{align*}
für $x\in M$ und $\pi(p)=x$.\fish
\end{defn}

Die Abbildungen $g_{ij}$ sind wohldefiniert, denn falls $\pi(p) = x = \pi(q)$,
so existiert genau ein $g\in G$, so dass $q = p\cdot g$ und folglich ist
\begin{align*}
S_{U_i}(q)S_{U_j}(q)^{-1}
&=
S_{U_i}(p\cdot g)S_{U_j}(p\cdot g)^{-1}
= S_{U_i}(p)gg^{-1}S_{U_j}(p)^{-1}\\
&= S_{U_i}(p)S_{U_j}(p)^{-1}.
\end{align*}
Nach dem folgenden Lemma entsprechen die Übergangsfunktionen des Haupfaserbündels $P \to M$ gerade den
Linkstranslationen mit $g_{ij}$:

\begin{lem}
\label{lem:g-phi-relation}
Seien $U_i,U_j$ Umgebungen einer offenen Überdeckung von $M$. Dann
gilt,
\begin{propenum}
\item $g_{ij}(x) = \phi_{ij}(x)(e)$, für alle $x\in U_i\cap U_j$.
\item $\phi_{ij}(x)(g) = g_{ij}(x)\cdot g$, für alle $x\in M$ und $g\in
G$.\fish
\end{propenum}
\end{lem}
\begin{proof}
Aufgrund der Äquivarianz gilt $\phi_{U_i}(p\cdot g) =
\phi_{U_i}(p)\cdot g$ für die lokalen Trivialisierungen. Dies gilt auch für die
$\phi_{U_i,x} \defl \pr_2\circ\phi_{U_i}\big|_{P_x}$, denn die $G$-Wirkung ist
fasertreu. Dadurch ist die $G$-Wirkung auf $P$ durch die lokalen Trivialisierungen
festgelegt,
\begin{align*}
p\cdot g \defl \phi_{U_i,x}^{-1}(\phi_{U_i,x}(p)\cdot g),\qquad p\in P,\quad
g\in G.
 \tag{*}
\end{align*}
Insbesondere ist $\phi_{U_i,x}^{-1}(h\cdot g) =\phi_{U_i,x}^{-1}(h)\cdot g$ für
$g,h\in G$.

a): Wir haben bereits gesehen, dass die Definition der $g_{ij}(x) \defl
S_{U_i}(p)\cdot S_{U_j}(p)^{-1}$ unabhängig vom gewählten $p\in P_x$ ist.
Wähle daher $p$ so, dass $S_{U_j}(p) = e$. Dann ist
\begin{align*}
p = S_{U_j}^{-1}(e) = \phi_{U_j,x}^{-1}(e).
\end{align*}
Eine kurze Rechnung ergibt,
\begin{align*}
g_{ij}(x) &= S_{U_i}(p)S_{U_j}(p)^{-1}
= S_{U_i}(p) = \phi_{U_i,x}(p)\\
&= \phi_{U_i,x}(\phi_{U_j,x}^{-1}(e))
= \phi_{ij}(x)(e).
\end{align*}
b): Analog erhält man die zweite Behauptung:
\begin{align*}
g_{ij}(x)\cdot g &= \phi_{U_i,x}(\phi_{U_j,x}^{-1}(e))\cdot g\\
&= \phi_{U_i,x}(\phi_{U_j,x}^{-1}(e)\cdot g)\\
&= \phi_{U_i,x}(\phi_{U_j,x}^{-1}(g)).\qed
\end{align*}
\end{proof}

Die Abbildungen $g_{ij}$ haben ähnliche Eigenschaften wie die
Übergangsfunktionen $\phi_{ij}$. In der Literatur findet man daher für die
$g_{ij}$ oftmals ebenfalls die Bezeichnung Übergangsfunktionen. Sie erfüllen
auch die Kozyklenbedingung:

\begin{lem}
Die Abbildungen $g_{ij}\colon U_i\cap U_j\to G$ für eine offene Überdeckung
$\setd{U_i}$ von $M$ sind $G$-Kozyklen, d.h.
\begin{propenum}
\item $g_{ii}(x) = e$ für alle $x\in U$.
\item $g_{ij}(x)\cdot g_{jk}(x) = g_{ik}(x)$ für $x\in U_i\cap U_j\cap U_k$.
\end{propenum}
Insbesondere gilt $g_{ij}(x)^{-1} = g_{ji}(x)$.\fish
\end{lem}

\begin{prop}
Sei $G$ eine Lie-Gruppe und $\pi\colon P\to M$ glatt. Das Tupel $(P,\pi,M;G)$ ist ein
$G$-Hauptfaserbündel genau dann, wenn es einen Bündelatlas
$\setd{U_i,\phi_{i}}$ und $G$-Kozyklen $g_{ij}$ gibt, so dass die Kozyklen
$\phi_{ij}$ durch Linkstranslation mit $g_{ij}$ gegeben sind,
\begin{align*}
\phi_{ij}(x) = L_{g_{ij}(x)},\qquad x\in U_i\cap U_j.
\end{align*}
Die $G$-Wirkung auf $P$ ist dann gegeben durch
\begin{align*}
p\cdot g = \phi_{U_i,x}^{-1}(\phi_{U_i,x}(p)\cdot g),\qquad \pi(p) = x.\fish
\end{align*}
\end{prop}

Die Kozykeleigenschaft bildet quasi die >>Verklebungsvorschrift<< nach der man
aus den $g_{ij}$ ein Bündel erhält.


\begin{proof}
$\Rightarrow$: Folgt direkt aus Lemma \ref{lem:g-phi-relation}.

$\Leftarrow$: Man erhält die $\phi_{U_i,x}$ aus den $g_{ij}$ und definiert die
$G$-Wirkung durch obige Formel. Dann ist diese fasertreu und einfach transitiv,
denn die $\phi_{U_i,x}$ sind Diffeomorphismen. Offensichtlich sind auch die $\phi_{U_i}$ nach dieser Definition $G$-äquivariant.
Somit ist $(P,\pi,M;G)$ ein Hauptfaserbündel.\qed
\end{proof}

\begin{lem}
Es besteht eine bijektive Beziehung zwischen lokalen Schnitten und lokalen
Trivialisierungen. Insbesondere ist ein Hauptfaserbündel genau dann trivial, wenn
es einen globalen Schnitt besitzt. \fish
\end{lem}
\begin{proof}
%Die Projektion $\pi\colon P\to M$  erfüllt

$\Rightarrow$: Gegeben sei ein lokaler Schnitt $s_U$. Für $x\in U$ ist
$s_U(x)\in P_x$ und folglich existiert für jedes $p\in P_x$ ein eindeutiges
$g\in G$, so dass
\begin{align*}
p = s_U(x)\cdot g.
\end{align*}
Man definiert nun als lokale Trivialisierung
\begin{align*}
\phi_U(p)\defl \phi_U(s_U(x))\cdot g = (x,g).
\end{align*}
Die Glattheit von $\phi_U$ folgt aus der Glattheit von $\sigma_U$.

$\Leftarrow$: Gegeben sei eine lokale Trivialisierung $\phi_U\colon \pi^{-1}(U)\to
U\times G$. Zu $x\in U$ definiert man nun
\begin{align*}
s_U\defl \phi_U^{-1}(x,e),
\end{align*}
dann ist $s_U$ glatt und $s_U(x)\in P_x$.\qed
\end{proof}

\begin{rem}
Sei $(\setd{U_i},s_i)$ eine offene Überdeckung von $M$, dann gilt
\begin{align*}
s_{U_i}(x) = s_{U_j}(x) \cdot g_{ji}(x),\qquad x \in U_i\cap U_j,
\end{align*}
denn es gilt ja $p = s_i(x)S_{U_i}(p) = s_j(x)S_{U_j}(p)$.\map
\end{rem}






\begin{ex}
Sei $P=P_{\GL_n}$ das Rahmenbündel über einer $n$-dimensionalen
Mannigfaltigkeit $M$. Eine Karte $(U,x)$  von $M$, definiert einen zugehörigen lokalen
Schnitt
\begin{align*}
\sigma_U(x) = \left(\frac{\partial}{\partial
x_1},\ldots,\frac{\partial}{\partial x_n} \right).
\end{align*}
Das Rahmenbündel $P_{\GL_n}$ ist also genau dann trivial, wenn $M$ parallelisierbar ist, d.h.
wenn $n$ global definierte  punktweise linear unabhängige Vektorfelder existieren. In diesem Fall sind auch
das Tangentialbündel und das Formenbündel trivial. Die Trivialität des
Rahmenbündels ist also eine sehr starke Forderung.

Von den Sphären sind lediglich die zwei Lie-Gruppen $S^1$ und $S^3$, sowie die
$S^7$ parallelisierbar. Außerdem ist jede Lie-Gruppe parallelisierbar.\boxc
\end{ex}

\section{Assoziierte Faserbündel}

Hauptfaserbündel sind wichtige Objekte der Differentialgeometrie, aus denen man viele interessante
Vektorbündel als assoziierte Faserbündel wiedergewinnen kann. Eigenschaften wie
der Zusammenhang übertragen sich --- einmal für das Hauptfaserbündel definiert
--- dann auf diese Vektorbündel. Exemplarisch behandeln wir in diesem Abschnitt
das Tangentialbündel.

Sei $\pi\colon P\to M$ ein $G$-Hauptfaserbündel, dann ist $P = \dot{\bigcup}_{x\in
M} P_x$ und alle Fasern $P_x$ sind diffeomorph zu $G$.
Wirke $G$ außerdem von links auf einer Mannigfaltigkeit $F$, d.h. es existiert
eine Abbildung
\begin{align*}
l : G\to \Diff(F),\quad g\mapsto l_g,\qquad
\text{mit}\quad l_e=\id \quad\text{und}\quad l_{g\cdot h}(v) = l_g\circ l_h(v).
\end{align*}
% mit $l_e = \id$. 
Schreiben wir $l_g(v) = g\cdot v$, dann wirkt $G$ von rechts auf $P\times F$
durch,
\begin{align*}
(p,v)\cdot g \defl (p\cdot g, g^{-1}\cdot v),\qquad p\in P,v\in F, g\in G
\end{align*}
Man definiert nun $E=(P\times F)/G \defr P\times_G F$, wobei die Punkte in $E$
Äquivalenzklassen $[p,v]$ unter folgender Äquivalenzrelation sind
\begin{align*}
(p,v)\sim (p\cdot g,g^{-1}\cdot v)\qquad \text{für alle }g\in G.
\end{align*}
Man identifiziert also alle Punkte in $P\times F$, die in einer $G$-Bahn liegen.
Die Projektion von $E$ auf $M$ ist dann gegeben durch $\hat{\pi}\colon E\to M$,
$[p,v] \mapsto \pi(p)$.

\begin{prop}
\index{Faserbündel!assoziiertes}
Das Tupel $(E,\hat{\pi},M;F)$ ist eine lokal triviale Faserung über $M$ mit
Faser $F$. Man sagt $E$ ist das zum $G$-Hauptfaserbündel $P$ \emph{assoziierte
Faserbündel}.\fish
\end{prop}

\begin{proof}
Sei $x\in M$ und $(U,\phi_U)$ eine Bündelkarte um $P_x$ von $P$,
\begin{align*}
\phi_U : P_U \to U\times G,\qquad p \mapsto (\pi(p),s_U(p)). 
\end{align*}
Man definiert nun als lokale Trivialisierung von $E$,
\begin{align*}
\psi_U : \hat{\pi}^{-1}(U) = E_U \to U\times F,
\qquad [p,v] \mapsto (\pi(p),\,s_U(p)\cdot v).
\end{align*}
Dann ist $\psi_U$ wohldefiniert, da $G$ fasertreu und  $s_U$ $G$-äquivariant ist. Außerdem ist $\psi_U$ glatt.\qed
\end{proof}

\begin{rem}
Die Kozyklen von $E = P \times_G F$ sind gegeben durch die Linkswirkung
$l_{g_{ij}(x)}$ der Kozyklen von $P$.\map
\end{rem}

\begin{ex}
\begin{exenum}
\item 
Das Tangentialbündel ist assoziiert zum Rahmenbündel. Genauer gilt
für eine $n$-dimensionale Mannigfaltigkeit $M$
\begin{align*}
TM \cong
P_{\GL_n}\times_{\GL_n} \R^n,
\end{align*}
mit der Gruppenwirkung
\begin{align*}
\GL_n \times\R^n\to \R^n,\qquad (g,v)\mapsto g\cdot v.
\end{align*}
Sei dazu $m\in M$ und $p=(e_1,\ldots,e_n)$ eine Basis von $T_mM$, d.h. $p\in
P_m$. Dann ist der Diffeomorphismus gegeben durch
\begin{align*}
P_{\GL_n}\times_{\GL_n}\R^n \ni [p,v] \mapsto \sum_{i=1}^n v_i e_i \in
T_mM,\qquad
v = (v_1, \ldots, v_n)\in\R^n.
\end{align*}
\item 
Analog ist auch das Formenbündel zum Rahmenbündel assoziiert.
Die Gruppenwirkung von $\GL_n$ auf $\Lambda^k \R^n$ ist dabei gegeben durch
\begin{align*}
\rho_k : \GL_n\to \Aut(\Lambda^k \R^n),\qquad
g\cdot (v_1\wedge \ldots\wedge v_k)
\defl (g\cdot v_1\wedge \ldots \wedge g\cdot v_k).
\end{align*}
Die Abbildung $\rho_k$ ist eine Darstellung von $\GL_n$ auf $\Lambda^k \R^n$ und man hat,
\begin{align*}
\Lambda^k TM \cong P_{\GL_n} \times_{\rho_k} \Lambda^k \R^n.
\end{align*}
Die Äquivalenzklassen sind bezüglich der Darstellung zu verstehen als
\begin{align*}
[p,v] \sim [p\cdot g,\rho(g)^{-1}v],\qquad p\in P_x,\; g\in G.
\end{align*}
\item
Viele interessante Vektorbündel erhält man als assoziierte Faserbündel, indem
man das $G$-Hauptfaserbündel $P\to M$ festhält und nur die Darstellung
variiert.~\boxc
\end{exenum}
\end{ex}

\begin{rem}
Sei $P$ ein $G$-Hauptfaserbündel und $E$ zu $P$ assoziiert mit Fasertyp $F$. Sei
$x\in M$, dann gilt $E_x\cong F$. Für jedes $p\in P_x$ 
ist nun der sogenannte Faserdiffeomorphismus gegeben durch,
\begin{align*}
[p] : F\to P_x\times_G F = E_x,\qquad
v\mapsto [p,v].\map
\end{align*}
Offensichtlich gilt $[p \cdot g] = [p] \cdot l_g$ für alle $p\in P, g\in G$.
\end{rem}

\begin{lem}
Sei $E = P\times_G F$ ein zum $G$-Hauptfaserbündel $P$ assoziiertes
Faserbündel mit Fasertyp $F$, dann besteht eine bijektive Beziehung zwischen
Schnitten in $E$ und $G$-äquivarianten Funktionen,
\begin{align*}
\Gamma(E) \cong \Cs^\infty(P,F)^G \defl
\setdef{f : P\to F}{f(p\cdot g) = g^{-1}f(p)\text{ für }p\in P,g\in G}.\fish
\end{align*}
\end{lem}
\begin{proof}
$\supset$: Sei $\bar{s}\in \Cs^\infty(P,F)^G$, dann definiert man für $x\in M$,
\begin{align*}
s(x) \defl [p,\bar{s}(p)],\qquad \text{für ein }p\in P_x.
\end{align*}
Der Schnitt $s\colon M\to E$ ist wohl definiert, d.h. unabhängig von der Wahl von $p \in E_x$,
denn für $q\in P_x$ beliebig existiert ein
$g\in G$, so dass $q = p\cdot g$ und es gilt
\begin{align*}
[q,\bar{s}(q)] = [p\cdot g,\bar{s}(p\cdot g)]
= [p\cdot g,g^{-1}\bar{s}(p)] = [p,\bar{s}(p)].
\end{align*}
Weiterhin ist $s$ glatt, denn $\bar{s}$ ist glatt, und somit ist $s$ ein Schnitt
in $E$.

$\subset$: Sei $s\in \Gamma(E)$, dann ist $s(x) \in E_x$ und man definiert für
$p\in P_x$,
\begin{align*}
\bar{s}(p) = [p]^{-1}(s\circ \pi(p)). 
\end{align*}
Dann ist die Funktion $\bar{s}\colon P\to F$ $G$-äquivariant, denn für $p\in P$ und
$g\in G$ ist
\begin{align*}
\bar{s}(p\cdot g) = [p\cdot g]^{-1}(s\circ \pi(p\cdot g))
= g^{-1}[p]^{-1}(s\circ \pi(p)) = g^{-1}\bar{s}(p).
\end{align*} 
Somit ist $\bar{s}\in \Cs^\infty(P,F)^G$.\qed
\end{proof}

\begin{ex}
Die Schnitte im Tangentialbündel sind gerade die Vektorfelder auf
$M$. Nach dem vorigen Lemma gilt $\chi(M)\cong
\Cs^\infty(P_{\GL_n},\R^n)^{\GL_n}$. Um dies zu veranschaulichen, wähle Karten
$(U,x)$ und $(V,y)$ um $m \in M$, dann sind durch
\begin{align*}
p = \left(\frac{\partial}{\partial x_1},\ldots,\frac{\partial}{\partial x_n}
\right)\qquad\text{und}\qquad
\tilde{p} = 
\left(\frac{\partial}{\partial y_1},\ldots,\frac{\partial}{\partial y_n}
\right),
\end{align*}
zwei Basen von $T_mM$ gegeben. Zwischen den Koordinatenvektorfeldern besteht
die Beziehung
\begin{align*}
\frac{\partial}{\partial y_i} = \sum_{j=1}^n a_{ji}\frac{\partial }{\partial
x_j},\qquad
a_{ji} = \frac{\partial x_j}{\partial y_i}
= \frac{\partial (x_j\circ y^{-1})}{\partial y_i}\bigg|_0,
\end{align*}
mit der sich die Basen ineinander überführen lassen,
\begin{align*}
\tilde{p} = p\cdot g,\qquad g = (a_{ji})_{1\le j,i\le n}.
\end{align*}
Sei nun $X$ ein Vektorfeld auf $M$, dann besitzt dieses bezüglich den Karten $x$
bzw. $y$ die lokale Darstellung,
\begin{align*}
X\big|_{U\cap V} = \sum_{i=1}^n v_i \frac{\partial}{\partial y_i}
= \sum_{i,j=1}^n a_{ji}v_i \frac{\partial}{\partial x_j}.\tag{*}
\end{align*}
Man kann $X$ somit wie folgt eine Abbildung $f$ zuordnen
\begin{align*}
X\mapsto f,\qquad f(\tilde{p}) = (v_1,\ldots,v_n).
\end{align*}
Nach Gleichung (*) ist $f$ außerdem $G$-äquivariant, denn $f(p) = g\cdot
f(\tilde{p})$. Somit ist
$f\in\Cs^\infty(P_{\GL_n}, \R^n)^{\GL_n}$. Die Funktion
$f$ beschreibt das Vektorfeld $X$ eindeutig, indem es vorgibt, wie ein
Basiswechsel, die Koeffizienten von $X$ verändert.\boxc
\end{ex}

\begin{prop}
\label{prop:Kozyklen-bestimmen-AFB-HFB}
Sei $M$ eine Mannigfaltigkeit und $G$ eine Lie-Gruppe, die von links auf einer
Mannigfaltigkeit $F$ wirkt. Weiter sei $\setd{U_i}$ eine offene Überdeckung
von $M$ und
\begin{align*}
g_{ij}\colon U_i\cap U_j\to G
\end{align*}
eine Familie von $G$-Kozyklen. Dann gibt es bis auf Isomorphie genau eine lokal
triviale Faserung $(E,\hat{\pi},M;F)$ deren Kozyklen durch Linkstranslation der
$g_{ij}(x)$ gegeben sind.\fish
\end{prop}
\begin{proof}
\textit{Konstruktion von $E$}. Man definiert zunächst
\begin{align*}
\hat{E} = \bigcup_{i} U_i\times F,
\end{align*}
und für $(x_i,v_i)\in U_i\times F$, $(x_k,v_k)\in U_k\times F$ eine
Äquivalenzrelation
\begin{align*}
(x_i,v_i)\sim (x_k,v_k) 
\iff x_i = x_k = x \text{ und }
v_k = g_{ki}v_i.
\end{align*}
Setzt man $E = \hat{E}/_\sim$ und $\hat{\pi}([x,v]) = x$, so ist $\pi\colon E\to M$
eine lokal triviale Faserung mit lokalen Trivialisierungen
\begin{align*}
\psi_i : E_{U_i} = \pi^{-1}(U_i) \to U_i\times F,\qquad [x_i,v]\mapsto (x_i,v).
\end{align*}
Sei $x\in M$, dann sind durch Konstruktion die \textit{Kartenwechsel} gegeben
durch,
\begin{align*}
\psi_{ix}\circ \psi_{kx}^{-1}(v) = g_{ik}(x)v.
\end{align*}
Analog erhält man das $G$-Hauptfaserbündel $P$, indem man in obiger Konstruktion
$F$ durch $G$ ersetzt.
Dann ist $E$ zu $P$ assoziiert, d.h. die Abbildung
\begin{align*}
P\times_G F\to E,\qquad [[x,g],v]\mapsto [x,g\cdot v]
\end{align*}
ist ein Isomorphismus.

Es verbleibt, die Eindeutigkeit von $E$ zu zeigen.\qed
\end{proof}

\section{Vektorbündel}

\begin{defn}
\index{Vektorbündel}
Ein \emph{Vektorbündel vom Rang $r$} ist eine lokal triviale Faserung
$(E,\pi,M;V)$ mit folgenden Eigenschaften:
\begin{defnenum}
\item Der Fasertyp $V$ ist ein $r$-dimensionaler $\K$-Vektorraum.
\item Für den Bündelatlas $\setd{(U_i,\phi_i)}$ und $x\in M$ sind die
Abbildungen
\begin{align*}
\phi_{i,x} = \phi_i\big|_{E_x}  : E_x \to V
\end{align*}
Vektorraumisomorphismen.\fish
\end{defnenum}
\end{defn}

Ein Vektorbündel ist also eine Faserung deren Faser ein Vektorraum ist und
deren lokale Trivialisierungen eingeschränkt auf die Fasern
Vektorraumisomorphismen ergeben.

\begin{rem}[Bezeichnung.]
\index{Geradenbündel}
Für $r = \mathrm{rk}(E) = 1$ nennt man $E$ ein \emph{Geradenbündel} (auch
\emph{Linienbündel}).\map
\end{rem}


Ähnlich wie bei den Hauptfaserbündeln, ist es sinnvoll für die Isomorphie zweier
Vektorbündel zusätzlich zu fordern, dass auch die lineare Struktur übertragen
wird.

\begin{defn}
\index{Vektorbündel!isomorph}\index{Vektorbündel!-Homomorphismus}
Zwei Vektorbündel $(E,\pi,M;V)$ und $(\tilde{E},\tilde{\pi},M;\tilde V)$ heißen
\emph{isomorph}, wenn es einen fasertreuen Diffeomorphismus
\begin{align*}
\psi\colon E\to \tilde{E}
\end{align*}
gibt, so dass $\psi_x \defl \psi\big|_{E_x} \colon E_x\to \tilde{E}_x$ für jedes
$x\in M$ ein Vektorraumisomorphismus ist.

Allgemeiner heißt eine differenzierbare, fasertreue Abbildung $\psi\colon E\to \tilde{E}$
\emph{Homomorphismus der Vektorbündel}, wenn alle $\psi_x$
lineare Abbildungen sind.\fish
\end{defn}


\begin{lem}
Sei $\pi\colon P\to M$ ein $G$-Hauptfaserbündel, und seien zwei Vektorbündel
\begin{align*}
E_1 = P\times_{\rho_1} V_1,\qquad E_2 = P\times_{\rho_2} V_2, 
\end{align*}
definiert mit den Darstellungen
\begin{align*}
\rho_1 : G\to \GL(V_1),\qquad \rho_2 : G\to \GL(V_2).
\end{align*}
Jede $G$-äquivariante Abbildung $f\colon V_1\to V_2$ induziert
einen Homomorphismus von Vektorbündeln
\begin{align*}
\hat{f}\colon E_1\to E_2,\qquad [p,v] \mapsto [p,f(v)].\fish
\end{align*}
\end{lem}
\begin{proof}
Die Abbildung $\hat{f}$ ist wohldefiniert, denn seien $g\in G$, $p\in P$ und
$v\in V_1$, so ist $[p,v] = [p\cdot g,\rho_1(g^{-1})v]$ und es gilt
\begin{align*}
\hat{f}([p\cdot g,\rho_1(g^{-1})v]) &= 
[p\cdot g,f(\rho_1(g^{-1})v)] = [p\cdot g,\rho_2(g^{-1})f(v)] \\
&= [p,f(v)] = \hat{f}([p,v]).
\end{align*}
Weiterhin ist $\hat{f}$ per definitionem fasertreu und
$\hat{f}$ eingeschränkt auf eine Faser linear, da $f$ linear.\qed
\end{proof}


\begin{rem}
Der Raum der Schnitte 
$\Gamma(E)$ wird ein $\Cs^\infty(M)$-Modul, indem man für einen Schnitt $s$ in
$E$ und eine glatte Funktion $f\in\Cs^\infty(M)$ definiert,
\begin{align*}
(f\cdot s)(x) = f(x)\cdot s(x),\qquad x\in M.\map
\end{align*}
\end{rem}

\begin{defn}
\index{Vektorbündel!trivial}
Ein Vektorbündel heißt \emph{trivial}, falls es isomorph ist zum trivialen
Bündel $E=M\times \K^n\to M$ ist.\fish
\end{defn}

\begin{ex}
\begin{exenum}
\item
Das Tangentialbündel ist ein Vektorbündel
\begin{align*}
TM \cong P_{\GL_n}\times_{\GL_n}\R^n, 
\end{align*}
denn der Fasertyp ist der euklidische Vektorraum $\R^n$.
\item Das Formenbündel ist ein Vektorbündel vom Rang $\binom{n}{k}$
\begin{align*}
\Lambda^k T_M \cong P_{\GL_n}\times_{\rho}\Lambda^k\R^n,
\end{align*}
denn $\Lambda^k\R^n$ ist isomorph zu $\R^{\binom{n}{k}}$.
\item Betrachte den $n$-dimensionalen komplex projektiven Raum 
\begin{align*} 
\CP^n = \setdef{L\subset \C^{n+1}}{L\text{ Unterraum der Dimension
}1}.
\end{align*}
Das \emph{kanonische Linienbündel} (auch \emph{tautologisches Bündel}) ist
gegeben durch
\begin{align*}
\gamma_n \defl \setdef{(L,\xi)\in \CP^n\times\C^{n+1}}{\xi\in L}
\overset{\pi}{\longrightarrow} \CP^n,
\end{align*}
mit der Projektion $\pi(L,\xi) = L$ und Fasern
\begin{align*}
\pi^{-1}(L) = \setdef{\xi\in\C^{n+1}}{\xi\in L} = L.
\end{align*}
\begin{lem*}
$\gamma_n$ ist ein komplexes Vektorbündel vom Rang 1.\fish
\end{lem*}

Betrachte das $\U(1)$-Hauptfaserbündel (Hopf-Bündel)
\begin{align*}
S^{2n+1}\subset \C^{n+1} \to \CP^n,\qquad z\mapsto [z],
\end{align*}
wobei $z \sim z\cdot \lambda$ für $\lambda\in \U(1)$. Jede irreduzible
Darstellung von $\U(1)$ ist beschrieben durch einen Index $k\in\Z$,
\begin{align*}
\rho_k : \U(1)\to \GL(\C),\qquad \rho_k(z)\cdot v = z^kv,\qquad
\text{für }z\in\U(1),\; v\in \C.
\end{align*}
Alle Bündel die man als assoziierte Bündel aus dem Hopf-Bündel erhält haben die
Gestalt
\begin{align*}
S^{2n+1}\times_{\rho_k}\C = \gamma_n^k =
\begin{cases}
\gamma_n^{\otimes k}, & k > 0,\\
\bar{\gamma_n}^{\otimes k}, & k < 0,\\
\text{trivial}, & k=0,
\end{cases}
\end{align*}
wobei $\gamma_n^{\otimes k} = \gamma_n \underbrace{\otimes \ldots
\otimes}_{\text{k-mal}} \gamma_n$.

Eine Überdeckung von $S^{2n+1}$ lässt sich in homogenen Koordinaten beschreiben
als
\begin{align*}
U_j = \setdef{[z_0:\ldots:z_n]}{z_j\neq 0},\qquad 0\le j \le n.
\end{align*}
Zu dieser Überdeckung finden sich lokale Schnitte des Hopf-Bündels der
Form $\sigma_j : U_j \to S^{2n+1}$ mit
\begin{align*}
\sigma_j([w_0:\ldots:w_n]) =
\Biggl(\frac{w_0}{w_j},\ldots,
\frac{w_{j-1}}{w_j},1,\frac{w_{j+1}}{w_j},\ldots,\frac{w_n}{w_j}
\Biggr)\cdot\Biggl(1 + \sum_{k\neq j}
\abs{\frac{w_k}{w_j}}^2\Biggr)^{-1/2}\!\!\!\!\!\!\!\!\!\!\!.
\end{align*}
Damit lassen sich die lokalen Trivialisierungen des tautologischen Bündels
angeben durch
\begin{align*}
\phi_i\colon \gamma_n\big|_{U_i} \to U_i\times \C,\qquad (L,\xi) \mapsto (L,z),\qquad
\xi = \sigma_i(L)\cdot z.
\end{align*}
Die Abbildungen $\phi_i$ sind bijektiv, glatt und linear Isomorphismen in den
Fasern. Die Übergangsfunktionen
\begin{align*}
\phi_j\circ\phi_i^{-1}\colon (U_i\cap U_j)\times\C\to (U_i\cap U_j)\times\C,\qquad
(L,z) \mapsto (L,g_{ij}(L)\cdot z)
\end{align*}
sind ebenfalls glatt und somit ist $\gamma_n$ assoziiert zu dem
Hauptfaserbündel, das durch die $g_{ij}$ erzeugt wird.\boxc
\end{exenum}

\end{ex}

Sei nun $P$ ein $G$-Hauptfaserbündel, $V$ ein Vektorraum und $\rho$
eine $G$-Darstellung, d.h.
$
\rho\colon G\to \Aut(V)=\GL(V)
$
ist ein Gruppen-Homomorphismus. Dann wirkt $G$ auf $V$ durch
$
g\cdot v \defl \rho(g)v.
$

\begin{lem}
$E_\rho\defl P\times_\rho V$ ist ein Vektorbündel.\fish
\end{lem}

\begin{proof}
Die lokalen Trivialisierungen von $P$ sind gegeben durch
\begin{align*}
\phi_U : P_U\to U\times G,\qquad p\mapsto (\pi(p),s_U(p)).
\end{align*}
Man erhält nun die lokalen Trivialisierungen von $E$ durch
\begin{align*}
\psi_U : E_U \to U\times V,\qquad
[p,v]\mapsto (\pi(p),\rho(s_U(p))v).
\end{align*}
Die Vektorraumstruktur auf $E_x$ ist dann gegeben durch
\begin{align*}
\lambda[p,v]+\mu[p,w] = [p,\lambda v + \mu w],\qquad
v,w\in V,\quad \lambda,\mu\in\K.
\end{align*}
Dadurch werden die Faserdiffeomorphismen zu Vekotrraumisomorphismen,
\begin{align*}
[p]: V\to E_x,\qquad v\mapsto [p,v].\qed
\end{align*}
\end{proof}

\begin{lem}
\label{lem:VB-assoziert-GLn}
Jedes Vektorbündel ist assoziiert zu einem $G$-Hauptfaserbündel mit linearer
Strukturgruppe $G$, d.h. $G$ ist eine Untergruppe der $\GL_n$.\fish
\end{lem}
\begin{proof}
Sei $\pi\colon E\to M$ ein Vektorbündel mit Fasertyp $V$ und $G=\GL(V)$. Sei
$\setd{(U_i,\psi_i)}$ ein Bündelatlas von $E$, dann sind für jedes $x\in M$ die
Abbildungen $\psi_{i,x}\circ\psi_{k,x}^{-1}$ lineare Isomorphismen von $V$.
Folglich bilden die Kozyklen nach $G$ ab,
\begin{align*}
g_{ij}\colon U_i\cap U_k \to G.
\end{align*}
Nach Satz \ref{prop:Kozyklen-bestimmen-AFB-HFB} existiert ein zum $G$-Hauptfaserbündel
assoziiertes Faserbündel mit $G$-Kozyklen $g_{ij}$. Da dieses bis auf Isomorphie
eindeutig ist, folgt die Behauptung.~\qed
\end{proof}

\section{Lineare Algebra von Vektorbündeln}

Da die Fasern der Vektorbündel selbst Vektorräume sind, lassen sich einige
Konstruktionen der Linearer Algebra direkt auf Vektorbündel
verallgemeinern. Dazu wendet man die Konstruktion zunächst auf den Fasern
an und betrachtet das Ergebnis dann als Faser der Konstruktion für das
Vektorbündel.

\begin{defn}
\index{Vektorbündel!Whitney-Summe}
\index{Vektorbündel!Tensorprodukt}
\index{Vektorbündel!duales}
\index{Vektorbündel!Homomorphismenbündel}
Seien $(E,\pi,M;V)$ und $(\tilde{E},\tilde{\pi},M;\tilde{V})$ zwei Vektorbündel.
\begin{defnenum}
\item Man definiert die \emph{Whitney-Summe} als
\begin{align*}
E\oplus \tilde{E} = \bigcup_{x\in M} E_x\oplus \tilde{E}_x,
\end{align*}
mit Projektion $\hat{\pi} \colon E\oplus \tilde{E}\to M$, $e_x\oplus
\tilde{e}_x\mapsto x$. Die lokalen Trivialisierungen sind gegeben durch
\begin{align*}
\hat{\pi}^{-1}(U)\to U\times (V\oplus \tilde{V}),\qquad
e_x\oplus \tilde{e}_x \mapsto (x,\phi_{U,x}(e_x)\oplus
\tilde{\phi}_{U,x}(\tilde{e}_x)).
\end{align*}
\item Das \emph{Tensorprodukt} ist definiert als,
\begin{align*}
E\otimes \tilde{E} = \bigcup_{x\in M} E_x\otimes \tilde{E}_x,
\end{align*}
mit Projektion und lokalen Trivialisierungen wie oben.
\item Das zu $E$ \emph{duale Bündel $E^*$} ist definiert durch,
\begin{align*}
E^* = \bigcup_{x\in M} E_x^*.
\end{align*}
\item Das \emph{Homomorphismenbündel} ist definiert durch,
\begin{align*}
\Hom(E,\tilde{E}) = \bigcup_{x\in M} \Hom(E_x,\tilde{E}_x)
\end{align*}
und es gilt $\Hom(E,\tilde{E}) =  E^* \otimes \tilde{E}$.\fish
\end{defnenum}
\end{defn}

Für einen komplexen Vektorraum $V$ definiert man den \textit{konjugierten
Vektorraum} $\bar{V}$, indem man die abelschen Gruppenoperationen von $V$
übernimmt und als skalare Multiplikation definiert
\begin{align*}
\C\times \bar{V}\to \bar{V},\qquad (\lambda,v)\mapsto \bar{\lambda}v,
\end{align*}
wobei diese Multiplikation wieder in $V$ aufzufassen ist.

\begin{defn}
Das zu einem Vektorbündel $E$ \emph{konjugierte Bündel} ist definiert durch
\begin{align*}
\bar{E}\defl\bigcup_{x\in M} \bar{E_x}. 
\end{align*}
\end{defn}


\begin{ex}
Sei $P$ ein $G$-Hauptfaserbündel und sind zwei Vektorbündel gegeben als
\begin{align*}
E = P\times_\rho V,\qquad \tilde{E} = P \times_{\tilde{\rho}} \tilde V,
\end{align*}
mit Darstellungen
\begin{align*}
\rho : G\to \GL(V),\qquad \tilde{\rho}\colon G\to \GL(\tilde V).
\end{align*}
Dann ist die Whitney-Summe gegeben durch $E\oplus \tilde{E} =
P\times_{\rho\oplus \tilde{\rho}} (V\oplus \tilde V)$, mit der Darstellung
\begin{align*}
\rho\oplus\tilde{\rho} \colon G\to \GL(V\oplus \tilde V),\qquad 
g\mapsto 
\begin{pmatrix}
\rho(g)\\
& \tilde{\rho}(g)
\end{pmatrix}.\boxc
\end{align*}
\end{ex}

%%% Vorlesung vom 11. Mai 2011
%TODO: Nachtragen, dass man im reellen Fall \bar{E}=E setzt.
%TODO: Nachtragen, Notation für HFB \pi\colon P\to M, P \to M. 

\section{Bündelmetriken}

Lemma \ref{lem:VB-assoziert-GLn} des vorletzten Abschnittes besagt, dass jedes
Vektorbündel zu einem $G$-Hauptfaserbündel mit linearer Strukturgruppe
assoziiert ist. Für reelle oder komplexe Vektorbündel kann man diese Aussage
noch verschärfen, denn dann ist das Vektorbündel sogar zu einem $\O(n)$ bzw.
$\U(n)$-Hauptfaserbündel assoziiert.

\begin{defn}
\index{Bündelmetrik}
Sei $E\to M$ ein reelles oder komplexes Vektorbündel. Eine
\emph{Bündelmetrik} auf $E$ ist ein Schnitt $h\in\Gamma(E^*\otimes \bar{E}^*)$,
so dass $h_x$ für jedes $x\in M$ eine nicht-ausgeartete und symmetrische (für
$E$ reell) bzw. hermitische (für $E$ komplex) Bilinearform ist.\fish
\end{defn}

\begin{prop}
Auf jedem reellen oder komplexen Vektorbündel existiert eine positiv definite
Bündelmetrik.\fish
\end{prop}
\begin{proof}
Sei $\setd{U_i}$ eine offene Überdeckung der Mannigfaltigkeit $M$ und
$E\to M$ ein reelles oder komplexes Vektorbündel vom Fasertyp $V$ mit
Trivialisierungen
\begin{align*}
\phi_i : \pi^{-1}(U_i) \to U_i\times V.
\end{align*}
Sei weiter $e_1,\ldots,e_n$ eine Basis in $V$, dann definiert man lokale
Schnitte
\begin{align*}
s_{i,a} = U_i\to E,\qquad s_{i,a}(x) = \phi_{i}^{-1}(x,e_a),
\end{align*}
für $a=1,\ldots,n$. Für jedes $x\in U_i$ bilden die Schnitte
$s_{i,1}(x),\ldots,s_{i,n}(x)$ eine Basis der Faser $E_x$. Durch Festlegung
einer Orthonormalbasis wird nun lokal auf $U_i$ eine Bündelmetrik induziert,
\begin{align*}
h_i : U_i\to E_{U_i}^*\otimes \bar{E}_{U_i}^*,\qquad
h_i(x)(s_{i,a}(x),s_{i,b}(x)) \defl \delta_{a,b},\qquad 1\le a,b\le n.
\end{align*}
Man wählt nun eine der Überdeckung $\setd{U_i}$ untergeordnete Zerlegung der
Eins $\setd{f_i}$, die lokal endlich ist, und setzt
\begin{align*}
h \defl \sum_{i} f_i\ h_i.
\end{align*}
Dann ist $h$ auf ganz $M$ definiert und für jedes $x\in M$ sind nur endlich
viele $f_i(x)$ von Null verschieden. Somit ist $h$ ein Schnitt in
$\Gamma(E^*\otimes \bar{E}^*)$ und die Symmetrie bzw. Hermitizität
der $h_i$ übertragen sich auf $h$. Außerdem ist $h$ nicht ausgeartet, da die
$h_i$ positiv definit sind.\qed
\end{proof}

\begin{ex}
Die Riemannsche Metrik ist eine Bündelmetrik auf dem Tangentialbündel. Während
nach dem vorangegangenen Satz stets eine positiv definite Bündelmetrik
gegeben ist, ist die Existenz von Bündelmetriken beliebiger Signatur an
topologische Bedingungen an die Mannigfaltigkeit $M$ geknüpft.\boxc
\end{ex}

\begin{prop}
Sei $E\to M$ ein Vektorbündel vom Rang $n$.
\begin{propenum}
\item Ist $E$ reell, so ist $E$ assoziiert zu einem
$O(n)$-Hauptfaserbündel.
\item Ist $E$ komplex, so ist $E$ assoziiert zu einem
$U(n)$-Hauptfaserbündel.\fish
\end{propenum}
\end{prop}
\begin{proof}
Sei $E\to M$ ein Vektorbündel vom Rang $n$. Zu $x\in M$ definiere $P_x$ als die
Menge der Orthonormalbasen in $E_x$. Dann ist durch
\begin{align*}
P = \bigcup_{x\in M} P_x\to M
\end{align*}
eine lokal triviale Faserung gegeben.

a): Ist das Vektorbündel reell, so sind alle Fasern $E_x$ isomorph zu einem
reellen Vektorraum $V$. Zu je zwei Orthonormalbasen über $E_x$ existiert genau
eine orthogonale Abbildung, die die Orthonormalbasen ineinander überführt. Somit
wirkt $\O(n)$ einfach transitiv auf den Fasern $P_x$, also ist $P\to M$ ein
$\O(n)$-Hauptfaserbündel. Der Diffeomorphismus, der $E$ zu $P$ assoziiert, ist
gegeben durch
\begin{align*}
P\times_{\O_n} \R^n\to E,\qquad
[(s_1,\ldots,s_n),(v_1,\ldots,v_n)]\mapsto \sum_{i=1}^n v_i s_i,
\end{align*}
für eine Basis $(s_1,\ldots,s_n)$ in $E_x$ und $(v_1,\ldots,v_n)\in\R^n$.

b): Ist das Vektorbündel komplex, so sind alle Fasern $E_x$ isomorph zu einem
unitären Vektorraum $V$. Somit wirkt $U(n)$ einfach transitiv auf den Fasern
$P_x$ und $E$ ist zu einem $\U(n)$-Hauptfaserbündel $P\to M$ assoziiert.\qed
\end{proof}

\begin{ex}
\begin{exenum}
\item \textit{Konstruktion von Bündelmetriken in assoziierten Vektorbündeln}.
Sei $\pi\colon P\to M$ ein $G$-Hauptfaserbündel und
\begin{align*}
\rho\colon G\to \GL(V)
\end{align*}
eine Darstellung auf $V$. Weiter sei $\lin{\cdot,\cdot}_V$ ein $G$-invariantes
Skalarprodukt auf $V$, d.h.
\begin{align*}
\lin{\rho(g)v,\rho(g)w}_V = \lin{v,w}_V,\qquad \text{für alle }g\in G,\; v,w\in
V.
\end{align*}
Für einen reellen Vektorraum $V$ impliziert dies $\im
\rho\subset\O(n)$. Ist $V$ dagegen komplex, so folgt $\im\rho\subset\U(n)$.

Nun existiert auf $E=P\times_\rho V$ eine Bündelmetrik so, dass für $e_a =
[p,v_a]$ für ein beliebiges $p\in P_x$ und $v_a\in V$, $a=1,2$ gilt
\begin{align*}
h(x)(e_1,e_2) \defl \lin{v_1,v_2}_V.
\end{align*}
Die Bündelmetrik hängt nicht von der Wahl von $p$ ab, denn für ein anderes
$q\in P_x$ existiert ein $g\in G$, so dass $q = p\cdot g$ und damit $[p,v_a] =
[q,\rho(g)^{-1} v_a]$. Mit der $G$-Invarianz von $\lin{\cdot,\cdot}_V$, folgt
nun
\begin{align*}
\lin{\rho(g)^{-1}v_1,\rho(g)^{-1}v_2}_V = \lin{v_1,v_2}_V = h(x)(e_1,e_2).
\end{align*}
Somit ist $h$ wohldefiniert.
\item Das vorangegangene Beispiel ist typisch: Ein unter der Darstellung
invariantes Objekt auf dem Vektorraum induziert ein Objekt auf dem Bündel.\boxc  
\end{exenum}
\end{ex}

\section{Reduktion der Strukturgruppe}

Ein Vektorbündel vom Rang $n$ ist in natürlicher Weise assoziiert zu einem
$\GL_n$-Hauptfaserbündel. Ist es außerdem reell bzw. komplex, so ist es
zusätzlich assoziiert zu einem $\O_n$- bzw. $U_n$- Hauptfaserbündel, wobei
$\O_n$ und $\U_n$ Untergruppen der $\GL_n$ sind. Dieser Zusammenhang soll nun
formalisiert werden.

\begin{defn}
\index{$\lambda$-Reduktion}
Sei $\pi_P\colon P\to M$ ein $G$-Hauptfaserbündel und $\lambda\colon H\to G$ ein
Gruppenhomomorphismus von Lie-Gruppen. Eine \emph{$\lambda$-Reduktion von $P$}
ist ein $H$-Hauptfaserbündel $\pi_Q\colon Q\to M$ und eine Abbildung $f\colon Q\to P$ mit
folgenden Eigenschaften:
\begin{defnenum}
  \item $f$ ist fasertreu, d.h. $\pi_P\circ f = \pi_Q$.
  \item $f(q\cdot h) = f(q)\cdot \lambda(h)$ für alle $q\in Q$ und $h\in
  H$.
\end{defnenum}

\centering
\begin{tikzpicture}[description/.style={fill=white,inner sep=2pt}]
\matrix (m) [matrix of math nodes, row sep=3em,
column sep=2.5em, text height=1.5ex, text depth=0.25ex]
{ H & & G \\
Q & & P\\
& M & \\ };
\path[->,font=\scriptsize]
(m-1-1) edge node[auto] {$\lambda$} (m-1-3)
(m-1-1) edge node[auto] { } (m-2-1)
(m-1-3) edge node[auto] { } (m-2-3)
(m-2-1) edge node[auto] {$f$} (m-2-3)
(m-2-1) edge node[auto,swap] {$ \pi_Q $} (m-3-2)
(m-2-3) edge node[auto] {$ \pi_P $} (m-3-2);
\end{tikzpicture}\fish
\end{defn}

\begin{rem}[Bemerkungen.]
\begin{remenum}
\item
Ist $H\subset G$ eine Lie-Untergruppe und $\lambda = i: H\opento G$ die
Inklusion, so nennt man die $\lambda$-Reduktion auch einfach Reduktion von
$P$ auf ein $H$-Hauptfaserbündel $Q$.
Die Strukturgruppe von $P$ wird also reduziert auf die Untergruppe $H\subset
G$. In diesem Fall ist $Q$ sogar ein Unterbündel von $P$.
\item \textit{Warnung.} Ob für eine gegebene Lie-Untergruppe $H\subset G$ eine $\lambda$-Reduktion existiert, hängt jedoch von der Mannigfaltigkeit $M$ ab.\map
\end{remenum}
\end{rem}

%Die Reduktion von Hauptfaserbündeln lässt sich rein topologisch betrachten.

\begin{ex}
Sei $P_{\GL_n}\to M$ das Rahmenbündel von $M$.
\begin{exenum}
\item \textit{Riemannsche Mannigfaltigkeiten}. Eine Riemannsche Metrik $g$ auf
$M$ definiert eine Reduktion von $P_{\GL_n}$ auf ein
$P_{\O_n}$-Hauptfaserbündel, wobei $i\colon \O_n\opento \GL_n$ eine Lie-Untergruppe
ist.

{
\centering
\begin{tikzpicture}[description/.style={fill=white,inner sep=2pt}]
\matrix (m) [matrix of math nodes, row sep=3em,
column sep=2.5em, text height=1.5ex, text depth=0.25ex]
{ P_{\O_n} & & P_{\GL_n}\\
& M  & \\};
\path[->,font=\scriptsize]
(m-1-1) edge node[auto] {} (m-1-3)
(m-1-1) edge node[auto,swap] { } (m-2-2)
(m-1-3) edge node[auto] { } (m-2-2);
\end{tikzpicture}

}

Sei umgekehrt $f\colon Q = P_{\O_n} \to P_{\GL_n}$ eine $\O_n$-Reduktion von
$P_{\GL_n}$. Dann definiert $(Q,f)$ wie folgt eine Riemannsche Metrik auf
$M$: Seien $x\in M$ und $q\in Q_x$, dann ist $f(q) = (v_1,\ldots,v_n)$ eine
Basis von $T_xM$. Man definiert die Metrik $g$ nun punktweise durch die
Festsetzung
\begin{align*}
g(x)(v_i,v_j) = \delta_{ij}.
\end{align*}
Damit ist $g$ wohldefiniert, denn für eine andere Wahl von $\tilde{q}\in P_x$,
ist $\tilde{q} = q\cdot g$ und
\begin{align*}
f(\tilde{q}) = f(q\cdot g) = f(q)\cdot g = (\tilde{v}_1,\ldots,\tilde{v}_n),
\end{align*}
ist eine zu $f(q)$ orthogonal konjugierte Basis von $T_xM$ und erzeugt somit
dieselbe Metrik $g$.

Zusammenfassend existiert genau dann eine Riemannsche Metrik auf $M$, wenn
$P_{\GL}$ eine $\O_n$-Reduktion besitzt.
\item \textit{Orientierbare Mannigfaltigkeiten}. Eine Mannigfaltigkeit $M$ ist
genau dann orientierbar, wenn sich $P_{\GL_n}$ auf $P_{\GL_n^+}$ reduzieren
lässt, wobei
\begin{align*}
\GL_n^+\defl \setdef{A\in\GL_n}{\det A > 0}.
\end{align*}
Das Hauptfaserbündel $P_{\GL_n^+}$ ist gerade die Menge aller Basen in
$P_{\GL_n}$ mit positiver Orientierung.

Ist $(M,g)$ außerdem eine Riemannsche Mannigfaltigkeit, so ist $M$ genau dann
orientierbar, wenn eine Reduktion von $P_{\GL_n}$ auf $P_{\SO_n}\opento
P_{\O_n}\opento P_{\GL_n}$ existiert.
\item \textit{Riemannsche Produkte}. Sei $M^n = N_1^a \times N_2^b$ mit $a+b =
n$, dann ist das Tangentialbündel von $M$ die Whitney-Summe der Tangentialbündel
von $N_1$ und $N_2$,
\begin{align*}
TM = TN_1\oplus TN_2.
\end{align*}
Somit reduziert sich $P_{\GL_n}$ auf $P_G$ mit
\begin{align*}
G = 
\begin{pmatrix}
\GL_a & 0 \\
0 & \GL_b
\end{pmatrix}.
\end{align*}
\item \textit{Fastkomplexe Mannigfaltigkeiten}. Eine Mannigfaltigkeit $M$ ist
eine komplexe Mannigfaltigkeit, wenn ihre Karten  auf offene
Teilmengen des $\C^n$ abbilden und die Kartenwechsel biholomorphe Abbildungen sind. 
Eine größere Klasse bilden die fastkomplexen
Mannigfaltigkeiten, für die ein Endomorphismus $J\in \End(TM)$ existiert
mit $J^2 = -\Id$. Durch $J_x$ wird jeder Tangentialraum $T_xM$ zu einem komplexen
Vektorraum.
Diese Mannigfaltigkeiten haben notwendigerweise gerade
Dimension und solch ein $J$ existiert genau dann, wenn das Tangentialbündel $TM$
ein komplexes Vektorbündel ist.

Eine Mannigfaltigkeit $M^{2n}$ ist genau dann fastkomplex, wenn sich
$P_{\GL_{2n}}$ auf $P_G$ reduziert mit
\begin{align*}
G = 
\setdef{
\begin{pmatrix}
A & -B\\
B & A
\end{pmatrix}
}{A,B\in \GL_n} \cong \GL_n(\C) .
\end{align*}

Ist $M^{2n}$ außerdem Riemannsch, dann ist $M$ genau dann fastkomplex, wenn sich
$P_{\GL_{2n}}$ auf $P_{\U_n}$ reduziert, wobei die unitäre Gruppe  $U_n$ als $U_n = O_{2n}\cap G$
zu verstehen ist.
\item \textit{Spin Strukturen}.
Die nichttriviale zweifache Überlagerung der $\SO_n$
\begin{align*}
\lambda\colon \Spin_n \overset{2x}{\longrightarrow} \SO(n)
\end{align*}
ist eine Lie-Gruppe, die sogenannte \emph{Spin-Gruppe} $\Spin_n$. Sie ist
zusammenhängend für $n\ge 2$ und einfach zusammenhängend für $n\ge 3$, d.h.
$\pi_1(\Spin_n) = 1$. Eine \emph{Spin-Struktur} auf einer orientierten
Riemannschen Mannigfaltigkeit $(M^n,g)$ ist eine $\lambda$-Reduktion von
$P_{\SO_n}$ auf
\begin{align*}
P_{\Spin_n}\to P_{\SO_n}.
\end{align*}

Nicht jede orientierte Riemannsche Mannigfaltigkeit trägt eine solche
Struktur. Ihre Existenz ist an topologische Bedingungen geknüpft, die man
als Orientierbarkeit 2. Ordnung verstehen kann.

Es gibt Darstellungen von $\Spin_n$ die keine Darstellungen von $\SO_n$ sind.
Mit ihnen lassen sich neue Bündel, sogenannte \emph{Spinor Bündel} definieren.
In der Quantenelektrodynamik interpretiert man Elementarteilchen wie das
Elektron als Schnitte in diesen Spinor Bündeln.
\item \textit{Triviales Hauptfaserbündel}.
Wie jede Gruppe besitzt die $\GL_n$ die triviale Untergruppe
$\setd{e}\subset G$. Sei $f\colon Q\to P$ eine $\setd{e}$-Reduktion von $P$. Dann
ist jede Faser $Q_x$ für $x\in M$ diffeomorph zu $\setd{e}$ und folglich
\begin{align*}
Q = \bigcup_{x\in M} Q_x = M\times \setd{e}\cong M. 
\end{align*} 
Die lokalen Trivialisierungen von $Q$ haben die Form
\begin{align*}
\phi_U : \pi^{-1}(U) \to U\cong U\times\setd{e}
\end{align*}
und somit existiert ein globaler Schnitt von $Q$
\begin{align*}
\sigma : M\to Q,\qquad x\mapsto \hat{x} = \setd{x}\times\setd{e}.
\end{align*}
Durch $s = f\circ \sigma\colon M\to P$ ist somit ein globaler Schnitt von $P$ gegeben
und folglich ist $P$ trivial.

Existiert umgekehrt ein globaler Schnitt $s\colon M\to P$, dann definiert die
Abbildung
\begin{align*}
f: Q=M\times\setd{e}\to P,\qquad (x,e)\mapsto s(x)
\end{align*}
eine $\setd{e}$-Reduktion von $P$, denn $\pi_P\circ f(x,e) = \pi_P\circ s(x) = x
= \pi_Q(x,e)$.

Somit ist $P$ genau dann trivial, wenn $P$ eine $\setd{e}$-Reduktion
besitzt.\boxc
\end{exenum}
\end{ex}

\begin{rem}
Das Rahmenbündel lässt sich auch auffassen als
\begin{align*}
P_{\GL_n} = \setdef{f: \R^n\to T_x M}{f \text{ ist ein linearer Isomorphismus}},
\end{align*}
denn eine Abbildung $f\colon \R^n\to T_xM$ ist genau dann ein Isomorphismus, wenn sie
Basen von $\R^n$ in Basen von $T_xM$ überführt.
Analog dazu betrachtet man auch
\begin{align*}
&P_{\O_n} = \setdef{f: \R^n\to T_x M}{f\text{ ist Isometrie}},\\
&P_{\GL_n^+} = \setdef{f: \R^n\to T_x M}{f\text{ ist orientierbarer
Isomorphismus}}.\map
\end{align*} 
\end{rem}

\begin{prop}
Sei $\lambda\colon H\to G$ ein Gruppenhomomorphismus von Lie-Gruppen, $\rho\colon G\to
\GL(V)$ eine Darstellung von $G$ über einem Vektorraum $V$ und $P\to M$ ein
$G$-Hauptfaserbündel mit einer $\lambda$-Reduktion $(Q,f)$. Dann gilt
\begin{align*}
P\times_\rho V \cong Q\times_{\rho\circ\lambda} V.
\end{align*}

\centering
\begin{tikzpicture}[description/.style={fill=white,inner sep=2pt}]
\matrix (m) [matrix of math nodes, row sep=3em,
column sep=2.5em, text height=1.5ex, text depth=0.25ex]
{ H & & G & & \GL(V) \\
Q & & P\\
& M & \\ };
\path[->,font=\scriptsize]
(m-1-1) edge node[auto] {$\lambda$} (m-1-3)
(m-1-3) edge node[auto] {$\rho$} (m-1-5)
(m-1-1) edge node[auto] { } (m-2-1)
(m-1-3) edge node[auto] { } (m-2-3)
(m-2-1) edge node[auto] {$f$} (m-2-3)
(m-2-1) edge node[auto,swap] {} (m-3-2)
(m-2-3) edge node[auto] {} (m-3-2);
\end{tikzpicture}\fish
\end{prop}

\begin{ex}
Sei $(M,g)$ eine Riemannsche Mannigfaltigkeit. Dann besitzt $M$ eine
Reduktion $P_{\O_n}\opento P_{\GL_n}$ und es gilt
\begin{align*}
TM = P_{\GL_n}\times_\rho \R^n \cong P_{\O_n}\times_{\rho\circ\lambda}\R^n,
\end{align*}
mit $\rho = \id$ und $\lambda = i: \O_n\opento \GL_n$.\boxc
\end{ex}

\begin{proof}[Beweis des Satzes.]
Wir geben direkt die Abbildung
\begin{align*}
\psi\colon Q\times_{\rho\circ\lambda} V \to P\times_\rho V,\qquad
[q,v]\mapsto [f(q),v],
\end{align*}
an, und weisen im Folgenden nach, dass es sich dabei um einen 
Vektorbündel-Isomorphismus handelt. Sei dazu $x\in M$ fest.

\textit{$\psi$ ist wohldefiniert}. Sei $\tilde{q}\in Q_x$ eine andere Wahl von
$q$, so existiert ein $h\in H$ mit $\tilde{q} = q\cdot h$ und
$[q,v] = [q\cdot h,\rho(\lambda(h))^{-1}v]$. Folglich ist
\begin{align*}
[f(q\cdot h),\rho(\lambda(h))^{-1}v] = 
[f(q)\cdot\lambda(h),\rho(\lambda(h))^{-1}v] = [f(q),v] = \psi([q,v]).
\end{align*}

\textit{$\psi$ ist linear auf Fasern}. Dies ist offensichtlich der Fall, denn
\begin{align*}
\psi(\lambda[q,v] + \mu[q,w]) = [f(q),\lambda v + \mu w]
= \lambda\psi([q,v]) + \mu\psi([q,w]),
\end{align*}
für $q\in Q_x$, $\lambda,\mu\in \K$ und $v,w\in V$.

\textit{$\psi$ ist injektiv}. Da $\psi$ fasertreu ist, genügt es die
Injektivität auf einer Faser zu zeigen. Sei also $\psi([q,v]) =
\psi([\tilde{q},\tilde{v}])$ für $q,\tilde{q}\in Q_x$ und $v,\tilde{v}\in V$.
Dann ist $\tilde{q} = q\cdot h$ und nach Annahme
\begin{align*}
\psi([q,v]) = [f(q),v] = [f(\tilde{q}),\rho(\lambda(h))^{-1}v]
\overset{!}{=}[f(\tilde{q}),\tilde{v}].
\end{align*}
Somit ist $\tilde{v} = \rho(\lambda(h))^{-1}v$ und folglich
\begin{align*}
[q,v] = [q\cdot h,\rho(\lambda(h))^{-1}v] = [\tilde{q},\tilde{v}].
\end{align*}

\textit{$\psi$ ist surjektiv}. Sei $[p,v]\in P\times_\rho V$ für ein $p\in P_x$.
Wähle $q\in Q_x$ beliebig, dann existiert ein $g\in G$, so dass $f(q) = p\cdot
g$. Folglich ist
\begin{align*}
\psi([q,\rho(g^{-1})v]) = [f(q),\rho(g^{-1})v]
= [p\cdot g,\rho(g^{-1})(v)] = [p,v].
\end{align*}
Es verbleibt zu zeigen, dass $\psi$ tatsächlich eine glatte Abbildung ist,
was in lokalen Trivialisierungen nachzuweisen ist.
\qed
\end{proof}

% Vorlesung vom 25.05.2011

\chapter{Zusammenhänge in Hauptfaserbündeln}

Im vergangenen Semester haben wir für Funktionen $f\in\Cs^\infty(M)$ auf einer
Mannigfaltigkeit $M$ den vom $\R^n$ wohlbekannten Begriff der Ableitung zum
Differential $\df$ verallgemeinert. Anschließend haben wir
Differentialformen $\omega\in\Omega^k(M)$ betrachtet und auch für diese ein
Differential
\begin{align*}
\ddd : \Omega^k(M)\to \Omega^{k+1}(M)
\end{align*}
gefunden, das für $k=0$ mit dem Differential von Funktionen übereinstimmt. Nun
haben wir erkannt, dass $k$-Formen wiederum spezielle Schnitte in einem
Vektorbündel sind. Das Ziel dieses Abschnittes ist es, das Differential auch auf
solche Objekte zu verallgemeinern. Dabei spielt der Zusammenhang auf dem
Hauptfaserbündel eine zentrale Rolle, denn dieser induziert ein
Differential auf allen assoziierten Faserbündeln. 

\section{Definitionen und Beispiele}

Sei wieder $(P,\pi,M;G)$ ein $G$-Hauptfaserbündel über einer Mannigfaltigkeit
$M$.

\begin{defn}
\label{defn:Distribution}
\index{Distribution}
Eine \emph{differenzierbare Distribution $\Delta$ vom Rang $r$} auf einer
Mannigfaltigkeit $N$ ist ein Unterbündel von $TN$ vom Rang $r$.\fish
\end{defn} 

\begin{rem}
Äquivalent dazu ist die Definition, dass $\Delta$ eine differenzierbare
Zuordnung
\begin{align*}
\Delta : N\to TN,\qquad u\mapsto \Delta_u
\end{align*}
ist, wobei $\Delta_u$ einen $r$-dimensionalen Unterraum von $T_uN$ beschreibt.
Die Differenzierbarkeit der Distribution ist hier so zu verstehen, dass $\Delta$
lokal durch differenzierbare Vektorfelder aufgespannt wird.\map
\end{rem}

Die Projektion $\pi\colon P\to M$ des Hauptfaserbündels ist eine Submersion, d.h. das
Differential $\dpi$ ist surjektiv. Aus dem Satz vom regulären Wert folgt nun,
dass jede Faser $P_x = \pi^{-1}(x)$ für $x\in M$ eine differenzierbare
Untermannigfaltigkeit von $P$ ist. Insbesondere ist dann $T_uP_x\subset T_uP$
ein Unterraum für alle $u\in P$.

\begin{defn}
\label{defn:Vertikaler-Tangentialraum}
\index{Tangentialraum!vertikaler}
Sei $x\in M$ und $u\in P_x$, dann heißt der Unterraum
\begin{align*}
T^v_uP \defl T_uP_x\subset T_uP
\end{align*}
\emph{vertikaler Tangentialraum} an $P$ im Punkt $u$.\fish 
\end{defn}

\begin{figure}[H]
\centering
\begin{pspicture}(0,-1.49)(4.76,1.51)
\psbezier(0.4,-1.47)(1.38,-0.99)(2.76,-1.05)(3.58,-1.29)
\psbezier(0.0,1.07)(0.72,0.55)(0.78,-0.55)(0.9,-1.09)
\psbezier(0.78,1.19)(1.32,0.75)(1.46,-0.41)(1.56,-0.97)
\psbezier[linecolor=purple](1.68,1.19)(2.34,0.51)(2.2,-0.49)(2.12,-0.97)
\psbezier(3.14,1.19)(3.32,0.19)(2.88,-0.61)(2.66,-0.95)
\psbezier(3.88,1.11)(3.94,-0.19)(3.52,-0.65)(3.2,-1.05)

\rput(4.28,1.3){\color{darkgray}$P$}

\rput(4.28,-1.285){\color{darkgray}$M$}


\rput(4.59,0.095){\color{darkgray}$\pi$}

\rput(2.09,-1.285){\color{darkgray}$x$}


\rput(1.89,1.335){\color{purple}$P_x$}

\rput(2.7,0.4){\color{darkblue}$T^v_uP_x$}

\rput(1.99,-0.05){\color{darkgray}$u$}
\psline{->}(4.28,1.07)(4.28,-1.11) 
\psline[linecolor=darkblue](2.2,1.23)(2.2,-0.99)
\psdots[dotsize=0.12](2.1,-1.09)
\psdots[dotsize=0.12](2.2,-0.07)
\end{pspicture} 
\caption{Zum vertikalen Tangentialraum.}
\end{figure}

Die Strukturgruppe $G$ des Hauptfaserbündels ist eine Lie-Gruppe. Ihr
Tangentialraum an das Einselement $e\in G$ lässt sich mit den linksinvarianten
Vektorfeldern auf $G$ identifizieren,
\begin{align*}
T_e G \cong \g\defl \setdef{X\in \Gamma(TG)}{l_g^* X = X}.
\end{align*}
Wir suchen nun eine ähnliche Identifikation für den vertikalen Tangentialraum.

\begin{defn}
\index{Vektorfeld!fundamentales}
Wirke eine Lie-Gruppe $G$ auf einer Mannigfaltigkeit $N$, und bezeichne $\g$ die
Lie-Algebra von $G$. Dann ist das \emph{fundamentale
Vektorfeld} zu $X\in\g$ definiert durch
\begin{align*}
\tilde{X}(u) \defl \left. \frac{\ddd}{\dt} u\cdot \exp(tX)\right|_{t=0},\qquad
u\in N,
\end{align*}
wobei $\exp\colon \g\to G$ die Exponentialabbildung bezeichnet.\fish
\end{defn}

\begin{prop}
Für jedes $u\in P$ gelten:
\begin{propenum}
\item $T^v_uP = \ker\dpi_u$.
\item Die Abbildung $\g\to T^v_u$, $X\mapsto \tilde{X}(u)$
ist ein linearer Isomorphismus und insbesondere ist
\begin{align*}
T^v_uP = \setdef{\tilde{X}(u)}{X\in \g}.\fish
\end{align*}
\end{propenum}
\end{prop}
\begin{proof}
a): Sei $x\in M$ und $u\in P_x$. Zu $X\in T^v_uP$ betrachte eine Kurve
$\gamma$ in $P_x$ durch $u$ mit $\dot{\gamma}(0) = X$, dann gilt
$\pi\circ\gamma \equiv x$ und folglich
\begin{align*}
0 = \frac{\ddd}{\dt}\pi\circ\gamma(t) \bigg|_{t=0}
= \dpi_u(\dot{\gamma}(0)) = \dpi_u(X).
\end{align*}
Somit ist $T^v_uP \subset \ker \dpi_u$. Sei $U\subset M$ eine Umgebung von
$x$, dann gilt lokal $P_U \cong U\times G$, sowie insbesondere $P_x\cong G$.
Folglich ist 
\begin{align*}
\dim T_uP = \dim T_x M + \dim T_e G = \dim M + \dim T^v_uP.
\end{align*}
Andererseits ist $\pi\colon P\to M$ eine Submersion, also gilt 
\begin{align*}
\dim \ker \dpi_u + \dim M = \dim T_uP.
\end{align*}
Somit ist $\dim \ker \dpi_u = \dim T^v_uP$ und daher $\ker\dpi_u = T^v_uP$.

b): Sei $x\in M$ und $u\in P_x$. Für $X\in\g$ ist $u\cdot \exp(tX) \in P_x$ für
alle $t\in\R$ und somit ist $\tilde{X}(u) \in T^v_uP$. Weiterhin ist die
Abbildung
\begin{align*}
f_u : \g\to T^v_uP,\qquad X\mapsto \tilde{X}(u) 
\end{align*}
linear und da $\dim G = \dim T^v_uP$ genügt es zu zeigen, dass $f_u$
injektiv ist. Sei also $f_u(X) = \tilde{X}(u) = 0$, dann ist für $t\in\R$
\begin{align*}
\frac{\ddd}{\dt} u\cdot \exp(tX)
=
\frac{\ddd}{\ds} u\cdot \exp((t+s)X)\bigg|_{s=0}
=
\ddd R_{\exp(tX)}(\tilde{X}(u)) = 0.
\end{align*}
Folglich ist $u\cdot \exp(tX) \equiv u$ für alle $t\in\R$, jedoch ist die
Exponentialabbildung ein lokaler Diffeomorphismus um $0$ und folglich $X=0$.
Somit ist $f_u$ injektiv.\qed
\end{proof}

\begin{cor}
\index{Tangentialbündel!vertikales}
Die Zuordnung
\begin{align*}
T^v P : P\to TP,\qquad  u \mapsto T^v_uP
\end{align*}
definiert eine rechtsinvariante Distribution auf $P$. Man nennt $T^vP$ das
\emph{vertikale Tangentailbündel von $P$}.\fish
\end{cor}
\begin{proof}
Sei $u\in P$, dann lässt sich nach dem vorigen Satz
der vertikale Tangentialraum schreiben als $T^v_uP = \ker\dpi_u$.
Weiterhin lässt sich $\ker\dpi_u$ lokal mit Hilfe von Koordinaten als
Aufspann von glatten Vektorfeldern darstellen. Weiter ist die Dimension von $\ker\dpi_u$
lokal konstant  und folglich ist $T^v_uP$ eine
Distribution.

Für die Rechtsinvarianz ist zu zeigen, dass $\ddd R_g T^v_u P = T^v_{ug}P$
für jedes $g\in G$. Sei also $X\in T^v_uP$ und betrachte eine
Kurve $\gamma$ durch $u$, die ganz in $P_x$ verläuft mit $\dot{\gamma}(0) = X$. 
Dann ist für $g\in G$,
\begin{align*}
\ddd R_g(X) = \left.\frac{\ddd}{\dt} \tilde{\gamma}(t)\right|_{t=0},\qquad
\tilde{\gamma}(t) = R_g\circ\gamma(t),
\end{align*}
und $\tilde{\gamma}$ ist eine Kurve durch $u\cdot g$, die aufgrund der
Fasertreue der $G$-Wirkung ebenfalls ganz in $P_x$ verläuft, mit
$\dot{\tilde{\gamma}}(0) = \ddd R_g(X)$. Somit ist $\ddd R_g(X) \in
T^v_{ug}P$ und da $\ddd R_g^{-1} = \ddd R_{g^{-1}}$ folgt die
Rechtsinvarianz von $T^vP$.\qed
\end{proof}

\begin{rem}
Für jedes $X\in\g$ ist das fundamentale Vektorfeld $\tilde{X}\colon P\to T^vP$ global
definiert und glatt. Falls $X\neq 0$, dann ist $\tilde X$ auch nullstellenfrei. Insbesondere ist das vertikale
Tangentialbündel $T^vP$ somit parallelisierbar.\map
\end{rem}


\begin{defn}
\index{Tangentialraum!horizontaler}
Sei $u\in P$, dann nennt man einen zu $T^v_uP\subset T_uP$ komplementären
Unterraum \emph{horizontal}.\fish
\end{defn}

\begin{figure}[H]
\centering
\begin{pspicture}(0,-1.69)(3.46,1.69)
\psbezier(0.74,-1.29)(0.84,-1.65)(2.14,-1.67)(2.3,-1.29)(2.46,-0.91)(0.64,-0.93)(0.74,-1.29)
\psdots[dotsize=0.12](1.24,-1.27)

\psbezier(0.68,1.67)(1.42,1.11)(1.44,-0.49)(1.26,-1.15)
\psline[linecolor=darkblue](1.12,1.65)(1.52,-1.05)
\pspolygon[linestyle=none,fillstyle=solid,fillcolor=white,opacity=0.5](0.18,0.45)(0.44,-0.17)(2.3,0.15)
\pspolygon[linecolor=purple](0.18,0.45)(0.44,-0.17)(2.3,0.15)(1.96,0.73)
\psline[linecolor=darkblue](1.12,1.65)(1.32,0.29)
\psdots[dotsize=0.12](1.32,0.29)

\rput(1.49,-1.265){\color{darkgray}$x$}

\rput(1.72,1.475){\color{darkblue}$T^v_uP_x$}

\rput(0.69,1.295){\color{darkgray}$P_x$}

\rput(0.48,-0.345){\color{purple}$T^h_uP_x$}

\rput(3.04,1.515){\color{darkgray}$P$}

\rput(3.0,-1.225){\color{darkgray}$M$}

\rput(3.29,0.395){\color{darkgray}$\pi$}
\rput(1.12,0.29){\color{darkgray}$u$}
\psline{->}(3.0,1.21)(3.0,-0.97)
\end{pspicture}
\caption{Zum horizontalen und vertikalen Tangentialraum.}
\end{figure}

\begin{defn}
\label{defn:Zusammenhang}
\index{Zusammenhang}
\index{Tangentialbündel!horizontales}
Ein \emph{Zusammenhang} auf dem Hauptfaserbündel $(P,\pi,M;G)$ ist eine
rechtsinvariante Distribution bestehend aus horizontalen Tangentialräumen
\begin{align*}
T^hP: P\to TP,\qquad u\mapsto T^h_uP,
\end{align*}
d.h. $T^h_u$ ist komplementär zu $T^v_uP$ und $\ddd R_g(T^h_uP) =
T^h_{ug}P$ für alle $g\in G$ und $u\in P$. $T^hP$ nennt man \emph{horizontales
Tangentialbündel}.\fish
\end{defn}

\begin{rem}[Bemerkungen.]
\begin{remenum}
\item Während $T^vP\subset TP$ kanonisch durch $\pi$ induziert wird, involviert die
Definition von $T^hP\subset TP$ die Wahl eines Vektorraumkomplements, welches nicht
eindeutig gegeben ist. Insbesondere ist dadurch die Wahl eines Zusammenhangs
auf $P$ nicht kanonisch.
\item Sei $u\in P_x$ für ein $x\in M$, dann ist das Differential der Projektion
$\dpi_u : T^h_uP\to T_xM$ injektiv und somit ein Isomorphismus.\map
\end{remenum}
\end{rem}

\section{Zusammenhangsformen}

Sei $N$ eine Mannigfaltigkeit und $\Omega^k(N)$ bezeichne die Menge der
$k$-Formen auf $N$ mit Werten in $\R$. Allgemeiner betrachten wir nun für einen
reellen Vektorraum $V$ die Menge $\Omega^k(N,V)$ der $k$-Formen auf $N$ mit
Werten in $V$.

Auf einer Lie-Gruppe $G$, lässt sich mit der Links- und Rechtsmultiplikation
die Konjugation mit einem Gruppenelement $g\in G$  bilden,
\begin{align*}
\alpha_g = L_g\circ R_{g^{-1}},\qquad h\mapsto g\cdot h\cdot g^{-1}.
\end{align*}
Dann ist $\alpha_g(e) = e$ für jedes $g\in G$ und folglich ist das
Differential $(\ddd \alpha_g)_e$ ein Isomorphismus. Das Differential heißt
\emph{adjungierte Darstellung} von $G$ auf $\g$ und man schreibt
\begin{align*}
\Ad(g) \defl (\ddd\alpha_g)_e.
\end{align*}

\begin{defn}
\label{defn:Zusammenhangsform}
\index{Zusammenhangsform}
Eine \emph{Zusammenhangsform} auf einem Hauptfaserbündel $(P,\pi,M;G)$ ist eine
1-Form $A\in\Omega^1(P,\g)$ mit Werten in $\g$ und folgenden Eigenschaften:
\begin{defnenum}
\item $R_g^*A = \Ad(g^{-1})A$ für alle $g\in G$,
\item $A(\tilde{X}) = X$ für jedes $X\in \g$.
\end{defnenum}
\end{defn}

Unser Ziel ist es nun eine bijektive Beziehung zwischen Zusammenhangsformen und
Zusammenhängen zu beschreiben. Dafür benötigen wir noch folgendes

\begin{lem}
\label{lem:Rechts-Translation-Fundamentales-Vektorfeld}
Sei $g\in G$ und $X\in \g$, dann gilt
\begin{align*}
\ddd R_g(\tilde{X}) = \widetilde{\Ad(g^{-1})X}.\fish
\end{align*}
\end{lem}
\begin{proof}
Sei $u\in P_x$ für $x\in M$, dann rechnet man direkt nach
\begin{align*}
\ddd R_g(\tilde{X})(u) &= 
(\ddd R_{g})_{ug^{-1}}(\tilde{X}(ug^{-1})) = 
(\ddd R_{g})_{ug^{-1}}\left(\frac{\ddd}{\dt}
ug^{-1}\cdot\exp(tX)\bigg|_{t=0} \right)\\
&= \frac{\ddd}{\dt}
ug^{-1}\cdot\exp(tX)\cdot g\bigg|_{t=0}
= \frac{\ddd}{\dt} u\cdot \alpha_{g^{-1}}\circ\exp(tX)\bigg|_{t=0}\\
&= \frac{\ddd}{\dt} u\cdot \exp(t\Ad(g^{-1})X) \bigg|_{t=0}
= \widetilde{\Ad(g^{-1})X}(u).\qed
\end{align*}
\end{proof}

\begin{prop}
\label{prop:Zusammenhang-Zusammenhangsform-1:1}
Zusammenhänge und Zusammenhangsformen auf einem $G$-Hauptfaserbündel
$\pi\colon P\to M$ stehen in bijektiver Beziehung zueinander:
\begin{propenum}
\item Sei $T^hP\colon P\to TP$, $u\mapsto T^h_uP$ ein Zusammenhang auf $P$, dann
wird durch
\begin{align*}
A_u(\tilde{X}(u)+Y_h) \defl X,
\end{align*}
für $u\in P$, $X\in \g$ und $Y_h \in T^h_uP$ eine Zusammenhangsform auf $P$
definiert.
\item Sei $A\in \Omega^1(P,\g)$ eine Zusammenhangsform auf $P$, dann wird durch
\begin{align*}
T^hP: P\to TP,\qquad u\mapsto T^h_uP \defl \ker A_u
\end{align*}
ein Zusammenhang auf $P$ definiert.\fish
\end{propenum}
\end{prop}
\begin{proof}
a): Wir überprüfen die Bedingungen für Zusammenhangsformen nach Definition
\ref{defn:Zusammenhangsform}. Für jedes $X\in \g$ und $u\in P$ ist nach
Definition
\begin{align*}
A(\tilde{X}(u)) = A(\tilde{X}(u)+Y_h) = X,
\end{align*}
indem man $Y_h = 0$ setzt. Sei nun $Y_h$ horizontal in $u$, dann ist nach
Voraussetzung für jedes $g\in G$, $\ddd R_g(Y_h)$ horizontal in $u\cdot g$.
Somit ist nach dem vorigen Lemma,
\begin{align*}
R_g^* A(\tilde{X}(u)+Y_h) &= 
A(\ddd R_g (\tilde{X}(u)) + \ddd R_g (Y_h))
= A(\widetilde{\Ad(g^{-1})X}(u\cdot g) + \ddd R_g (Y_h))\\
&= \Ad(g^{-1})X
= \Ad(g^{-1})A(\tilde{X}(u) + Y_h).
\end{align*}
Somit ist $R_g^* A = \Ad(g^{-1})A$.

b): Es ist zu zeigen, dass $\ker A$ eine horizontale, rechtsinvariante,
differenzierbare Distribution auf $P$ ist.

\textit{$\ker A$ ist horizontal}. Sei $u\in P$ und $Y\in \ker A_u\cap T^v_uP$.
Dann ist insbesondere $Y\in T^v_uP$ und lässt sich folglich schreiben als
$Y=\tilde{Y}(u)$. Somit gilt
\begin{align*}
0 = A_u(Y) = A(\tilde{Y})(u) = Y.  
\end{align*}
Daher sind $A_u$ und $T^v_uP$ tatsächlich
disjunkt. Weiterhin ist $A_u\colon T^v_uP \to \g$ nach der Formel $A(\tilde{X}) =
X$ surjektiv und folglich
\begin{align*}
\dim \ker A_u = \dim T_uP - \dim \g = \dim T_uP - \dim T^v_uP.
\end{align*}
Also ist $T_uP = \ker A_u\oplus T^v_uP$ und $\ker_u$ ist horizontal.

\textit{$\ker A$ ist rechtsinvariant}. Es ist zu zeigen, dass $\ddd R_g(\ker
A_u) = \ker A_{u\cdot g}$ für jedes $g\in
G$. Sei also $Y\in \ker A_u$, so verifiziert man sofort, dass
\begin{align*}
A_{ug}(\ddd R_g(Y)) = (R_g^* A)_u(Y)
= \Ad(g^{-1})A_u(Y) = 0,
\end{align*}
also ist $\ker A_u \subset \ker A_{u\cdot g}$. Weiterhin ist $\ker A_u$
horizontal, also folgt mit der Dimensionsformel, dass $\dim \ker A_u = \dim
\ker A_{u\cdot g}$ und folglich gilt $\ker A_u = \ker A_{u\cdot g} = \ddd
R_{g} \ker A_u$.

\textit{$\ker A$ ist differenzierbar}. Sei $(W,x_1,\ldots,x_n)$ eine Karte um
$u\in P$, $(e_1,\ldots,e_r)$ eine Basis von $\g$ und $Y\in T_uP$. In lokalen
Koordinaten schreibt sich nun
\begin{align*}
Y = \sum_{i=1}^n \xi_i \frac{\partial}{\partial x_i}(u),\qquad \xi_i\in \R,
\end{align*}
sowie analog
\begin{align*}
A\left(\frac{\partial}{\partial x_i} \right) = 
\sum_{j=1}^r A_{ij} e_j,\qquad 1\le i\le n,
\end{align*}
mit glatten Funktionen $A_{ij}$. Dann ist $Y\in \ker A_u$ genau dann, wenn
\begin{align*}
A_u(Y) = \sum_{i=1}^n\sum_{j=1}^r \xi_i A_{ji}(u)e_j = 0. 
\end{align*}
Dies ist ein lineares Gleichungssystem für $\xi_i$ und da die $A_{ji}$ glatt
sind, hängt die Lösung ebenfalls glatt von $u$ ab. Also wird $\ker A$ lokal
durch glatte Vektorfelder aufgespannt.\qed
\end{proof}

\begin{ex}
\index{Zusammenhang!kanonischer flacher}
\textit{Der kanonisch flache Zusammenhang}.
Wir betrachten das triviale $G$-Hauptfaserbündel $(P=M\times G,\pr_1,M;G)$ über
einer Mannigfaltigkeit $M$. An einem Punkt $p=(x,g)\in P$ ist der
vertikale Tangentialraum gegeben durch
\begin{align*}
T^v_{(x,g)}P = T_{(x,g)}(\setd{x}\times G) \cong T_gG,
\end{align*}
%und somit trivial. 
Die $G$-Wirkung auf $P$ ist einfach durch die
Multiplikation in der zweiten Komponente gegeben,
\begin{align*}
(x,g)\cdot h = (x,g\cdot h),\qquad (x,g)\in P,\qquad h\in G.
\end{align*}
Somit berechnet sich das fundamentale Vektorfeld für ein $Y\in \g$ zu
\begin{align*}
\tilde{Y}(x,g) &= \frac{\ddd}{\dt} (x,g)\cdot \exp(tY)\bigg|_{t=0}
= \frac{\ddd}{\dt} (x,g\cdot \exp(tY))\bigg|_{t=0}\\
&= (0,\ddd L_g(Y)) \in T_{x}M\oplus T_gG \cong T_{(x,g)}P.
\end{align*}
Wählt man als horizontalen Tangentialraum den zu $T^v_{(x,g)}P =
T_{(x,g)}(\setd{x}\times G)$ komplementären Tangentialraum
\begin{align*}
T^h_{(x,g)}P \defl T_{(x,g)}(M\times \setd{g}) \cong T_xM,
\end{align*}
erhält man einen Zusammenhang auf $P$, den sogenannten \emph{kanonischen
flachen Zusammenhang}.\boxc
\end{ex}

\begin{figure}[H]
\centering
\begin{pspicture}(0,-1.99)(5.3,1.99)
\psframe(4.08,1.51)(0.0,-0.91)
\psline(0.08,-1.63)(4.18,-1.63)
\psline(0.82,1.51)(0.82,-0.87)
\psline(1.64,1.49)(1.64,-0.89)
\psline(3.22,1.49)(3.22,-0.89)
\psline[linecolor=darkblue](2.42,1.49)(2.42,-0.89)
\psdots[dotsize=0.12](2.42,-1.63)

\rput(2.51,1.735){\color{darkgray}$P_x$}
\rput(2.53,-1.845){\color{darkgray}$x$}
\rput(4.86,1.815){\color{darkgray}$P$}
\rput(4.8,-1.565){\color{darkgray}$M$}
\rput(5.13,0.395){\color{darkgray}$\pi$}

%\rput(2.2,0.4){\color{darkgray}$u$}
\rput(2.8,0.3){\color{darkblue}$T^v_uP$}
\rput(2.03,0.8){\color{purple}$T^h_uP$}

\psline{->}(4.8,1.61)(4.8,-1.29)
\psline(0.54,0.59)(1.08,0.59)
\psline(1.36,0.59)(1.9,0.59)
\psline[linecolor=purple](2.14,0.59)(2.68,0.59)
\psline(2.94,0.59)(3.48,0.59)
\end{pspicture} 
\caption{Zum kanonischen Zusammenhang.}
\end{figure}


Nach Satz \ref{prop:Zusammenhang-Zusammenhangsform-1:1} induziert der
kanonische flachen Zusammenhang eine Zusammenhangsform. Diese wollen wir nun
genauer untersuchen.

\begin{defn}
\index{Maurer-Cartan-Form}
Sei $\mu_G \in \Omega^1(G,\g)$ die durch
\begin{align*}
(\mu_G)_g(X) \defl \ddd L_{g^{-1}}(X),\qquad g\in G,\quad X\in T_gG, 
\end{align*}
definierte 1-Form auf $G$ mit Werten in $\g$. Sie heißt \emph{kanonische 1-Form}
oder auch \emph{Maurer-Cartan-Form} von $G$.\fish
\end{defn} 

\begin{lem}
Die Zusammenhangsform $A\in\Omega^1(P,\g)$ des kanonischen flachen Zusammenhangs
ist gegeben durch
\begin{align*}
A_{(x,g)} \colon T_xM\oplus T_gG \to \g,\qquad (X,Y) \mapsto \mu_G(Y).\fish
\end{align*}
\end{lem}
\begin{proof}
Nach Satz \ref{prop:Zusammenhang-Zusammenhangsform-1:1} existiert eine
Zusammenhangsform auf $P=M\times G$ und ist gegeben durch
\begin{align*}
A_{(x,g)}(\tilde{Z}(x,g)+X_h) = Z,
\end{align*}
für alle $Z\in \g$ und $X_h\in T^h_{(x,g)}P \cong T_xM$.
Für das fundamentale Vektorfeld berechnet sich zu gegebenem $X\in T_gG\cong
T^v_{(x,g)}P$ eine Lösung der Gleichung
\begin{align*}
\tilde{Z}(x,g) = \ddd L_{g}(Z) \overset{!}{=} X
\end{align*}
zu $Z = \ddd L_{g^{-1}}(X) \in \g$. Somit ist
\begin{align*}
A_{(x,g)}(X+Y_h) = 
A_{(x,g)}(\tilde{Z}(x,g) + Y_h) = Z = \mu_G(X).\qed
\end{align*}
\end{proof}

\section{Existenz von Zusammenhängen}

Für das triviale $G$-Hauptfaserbündel konnten wir konkret einen Zusammenhang
angeben und seine Zusammenhangsform mit Hilfe der Maurer-Cartan-Form von $G$
beschreiben. Ob ein allgemeines Hauptfaserbündel überhaupt einen Zusammenhang
besitzt, haben wir jedoch noch nicht geklärt. Jedes Hauptfaserbündel ist allerdings
lokal trivial. Dies wollen wir nun ausnutzen, um lokal auf dem
Hauptfaserbündel einen Zusammenhang zu definieren und diesen dann mit der
Zerlegung der Eins zu einem Zusammenhang auf dem ganzen Hauptfaserbündel zu
verkleben.

\begin{prop}
Auf jedem Hauptfaserbündel existiert ein Zusammenhang.\fish
\end{prop}
\begin{proof}
Sei $(P,\pi,M;G)$ ein beliebiges $G$-Hauptfaserbündel und $\setd{U_i}$ eine
offene Überdeckung von $M$, so dass $P$ über $\setd{U_i}$ trivial ist, d.h.
\begin{align*}
P_{U_i} = \pi^{-1}(U_i) \cong U_i \times G.
\end{align*}
Weiterhin sei $\setd{f_i}$ eine der Überdeckung $\setd{U_i}$ untergeordnete
Zerlegung der Eins. Da $P_{U_i}$ trivial ist, existiert ein kanonisch
flacher Zusammenhang auf $P_{U_i}$ mit Zusammenhangsform $A_i\in
\Omega^1(P_{U_i},\g)$. Wir definieren nun eine 1-Form $A\in\Omega^1(P,\g)$
durch
\begin{align*}
A\defl \sum_{i} (f_i\circ \pi) A_i.
\end{align*}
Da die $f_i$ lokal endlich sind, ist $A$ wohldefiniert. Sei nun $X\in\g$
und $p\in P$, dann gilt
\begin{align*}
A(\tilde{X}(p))
= \sum_i f_i(\pi(p)) A_i(\tilde{X}(p))
= \biggl(\sum_i f_i(\pi(p))\biggr)\cdot X
= X,
\end{align*}
denn $\sum_i f_i(\pi(p)) = 1$. Sei weiter $Y\in T_gP$ und $g\in G$, dann
ist
\begin{align*}
(R_g^* A)_p (Y) &= A_{pg}(\ddd R_g(Y))
= \sum_i f_i(\pi(p\cdot g)) A_{i,p\cdot g}(\ddd R_g(Y))\\
&= \sum_i f_i(\pi(p\cdot g)) \Ad(g^{-1})A_{i,p}(Y)\\
&= \Ad(g^{-1})\sum_i f_i(\pi(p)) A_{i,p}(Y)\\
&= \Ad(g^{-1})A_p(Y),
\end{align*}
denn $\pi(p\cdot g) = \pi(p)$. Somit ist $A$ eine Zusammenhangsform auf $P$.\qed
\end{proof}

\begin{ex}
\newcommand{\dqomega}{\ddd \bar{\omega}}
\textit{Ein Zusammenhang auf dem Hopfbündel}.
Wir betrachten die Hopf-Faserung
\begin{align*}
(S^3,\pi,\CP^1;S^1),\qquad \CP^1 \cong S^2,
\end{align*}
mit Strukturgruppe $G = S^1\cong \U(1) = \setdef{\e^{\ii t}}{t\in\R}$ und
Lie-Algebra $\g = \ii\R$. Die 3-Sphäre schreiben wir als
\begin{align*}
S^3 = \setdef{(w_1,w_2)}{\abs{w_1}^2 + \abs{w_2}^2 = 1} \subset\C^2,
\end{align*}
so dass ihr Tangentialraum im Punkt $p=(w_1,w_2)\in S^3$ beschrieben ist durch
\begin{align*}
T_{(w_1,w_2)}S^3\subset T_{(w_1,w_2)}\C^2 \cong \C^2.
\end{align*}
Seien nun $\dom_j$ und $\dqomega_j$ die durch
\begin{align*}
\dom_j(X_1,X_2) = X_j,\qquad \dqomega_j(X_1,X_2) = \bar{X_j},
\end{align*}
definierten 1-Formen auf $S^3$ und $A\colon TS^3\to \ii\R$ die 1-Form
\begin{align*}
A_{(w_1,w_2)} \defl \frac{1}{2}\left(\bar{w}_1 \dom_1 - w_1\dqomega_1
+ \bar{w}_2 \dom_2 - w_2\dqomega_2\right).
\end{align*}
Dann ist $\bar{A_{(w_1,w_2)}} = - A_{(w_1,w_2)}$ und folglich nimmt $A$ Werte in
$\g = \ii\R$ an.
Wir zeigen nun, dass $A$ eine Zusammenhangsform auf dem
Hopfbündel definiert.

Zunächst bemerken wir, dass die Strukturgruppe $\U(1)$
abelsch und folglich die Konjugation trivial ist, denn für $z,w\in \U(1)$
ist $\alpha_z(w) = zwz^{-1} = w$, also $\alpha_z = \id$ und $\Ad(z) = \Id$.
Sei nun $X\in T_{(w_1,w_2)}S^3$ und $\gamma_1$ und $\gamma_2$ Kurven in $\C$
durch $w_1$ bzw. $w_2$ mit $\abs{\gamma_1}^2 + \abs{\gamma_2}^2
= 1$ und
\begin{align*}
X = \frac{\ddd}{\dt} (\gamma_1(t),\gamma_2(t))\bigg|_{t=0}
= (\dot{\gamma}_1(0),\dot{\gamma}_2(0)) = (X_1,X_2).
\end{align*} 
Für ein $z\in \U(1)$ gilt dann
\begin{align*}
\ddd R_z(X) = \frac{\ddd}{\dt} (\gamma_1(t)z,\gamma_2(t)z)\bigg|_{t=0}
= (\dot{\gamma}_1(0)z,\dot{\gamma}_2(0)z) = (X_1z,X_2z).
\end{align*}
Somit erhalten wir schließlich
\begin{align*}
(R_z^*A)_{(w_1,w_2)}(X) &= A_{(w_1z,w_2z)}(\ddd R_{z}(X))
= A_{(w_1z,w_2z)}(X_1z,X_2z)\\
&= \frac{1}{2}
\left(\bar{w_1z}X_1z - w_1z\bar{X_1z} + \bar{w_2z}X_2z - w_2z\bar{X_2z}
\right)\\
&= \frac{1}{2}
\left(\bar{w_1}X_1 - w_1\bar{X_1} + \bar{w_2}X_2 - w_2\bar{X_2}
\right)\\
&= A_{(w_1,w_2)}(X_1,X_2) = \Ad(z^{-1})A_{(w_1,w_2)}(X), 
\end{align*}
denn $1 = \abs{z}^2 = z\bar{z}$.

Sei weiter $\ii x\in \g = i\R$, dann rechnet man direkt nach, dass
\begin{align*}
A_{(w_1,w_2)}(\widetilde{\ii x}(w_1,w_2)) &= 
A_{(w_1,w_2)}(w_1\ii x,w_2\ii x) \\ &= 
\frac{1}{2}
\left(\bar{w_1}w_1\ii x - w_1\bar{w_1\ii x}
+ \bar{w_2}w_2\ii x - w_2\bar{w_2\ii x}
\right) \\ 
&=
\frac{1}{2}
\ii \left((\abs{w_1}^2 + \abs{w_2}^2)x + (\abs{w_1}^2 + \abs{w_2}^2)x
\right)\\
&= \ii x.\boxc
\end{align*}
\end{ex}

% Vorlesung vom 30.04.2011

\section{Lokale Zusammenhangsformen}

Es ist ein wichtiges Hilfsmittel der Differentialgeometrie globale Objekte in
lokale Objekte zu zerlegen, die eine gewisse >>Verklebungsvorschrift<< erfüllen,
und umgekehrt aus diesen lokalen Objekten mit Hilfe der Verklebungsvorschrift
die globalen Objekte wiederzugewinnen. So haben wir beispielsweise bereits
gezeigt, dass ein $G$-Hauptfaserbündel eindeutig durch einen Bündelatlas
$(U_i,\phi_i)$ und den $G$-Kozyklen $g_{ij}$ als Verklebungsvorschrift bestimmt
ist. Wir suchen nun eine analoge Zerlegung und Verklebungsvorschrift für den
Zusammenhang auf einem $G$-Hauptfaserbündel.

Sei wieder $(P,\pi,M;G)$ ein $G$-Hauptfaserbündel. Zunächst beweisen wir ein
nützliches Lemma um die Ableitung eines Produktes einer Kurve im
Hauptfaserbündel und einer Kurve in der Lie-Gruppe $G$ berechnen zu können.

\begin{lem}[Produktformel]
\label{prop:Produktformel}
\index{Produktformel für Kurven}
Gegeben sei eine Mannigfaltigkeit $P$ auf der eine
Lie-Gruppe $G$ von rechts wirkt. Sei $p(t)$ eine Kurve in $P$ mit $p(0) = p$ und
$g(t)$  eine Kurve in $G$ mit $g(0) = g$ und $z(t) = p(t)\cdot g(t)$ ihr
Produkt. Dann gilt
\begin{align*}
\dot{z}(0) = \dR_g(\dot{p}(0)) + \widetilde{\mu_G(\dot{g}(0))}(p\cdot g).\fish
\end{align*}
\end{lem}

\begin{proof}
Wir bezeichnen die glatte $G$-Wirkung auf $P$ mit
\begin{align*}
\phi\colon P\times G\to P,\qquad (p,g)\mapsto p\cdot g,
\end{align*}
und für $p\in P$ und $g\in G$ die Multiplikationen mit
\begin{align*}
R_g : P\to P,\qquad p\mapsto  p\cdot g,\qquad\qquad \ph_p : G\to P,\qquad
g\mapsto p\cdot g.
\end{align*}
Die Ableitung der Kurve $z$ berechnet sich nun zu
\begin{align*}
\dot{z}(0) &= \frac{\ddd}{\dt}\bigg|_{t=0} \phi(p(t),g(t))
= \dphi_{(p,g)}(\dot{p}(0),\dot{g}(0))\\
&= \dphi_{(p,g)}(\dot{p}(0),0) + \dphi_{(p,g)}(0,\dot{g}(0))\\
&= \frac{\ddd}{\dt}\bigg|_{t=0} \phi(p(t),g)
+ \frac{\ddd}{\dt}\bigg|_{t=0} \phi(p,g(t))\\
&= \frac{\ddd}{\dt}\bigg|_{t=0} R_g(p(t))
+ \frac{\ddd}{\dt}\bigg|_{t=0} \ph_p(g(t))\\
&= \dR_g(\dot{p}(0)) + \dph_p(\dot{g}(0)). 
\end{align*}
Beachtet man, dass $\exp(\dL_{g^{-1}} \cdot ) = L_{g^{-1}}\circ \exp(\cdot)$,
folgt schließlich
\begin{align*}
\widetilde{\mu_G(\dot g(0))}(p\cdot g) &= 
\frac{\ddd}{\ds}\bigg|_{s=0} (p\cdot g) \exp(s \mu_G(\dot{g}(0)))\\
&=\frac{\ddd}{\ds}\bigg|_{s=0} \ph_p(\exp(s \dot{g}(0)))\\
&= \dph_p(\dot{g}(0)).\qed
\end{align*}
\end{proof}

\begin{defn}
\index{Zusammenhangsform!lokale}
Sei $A\in\Omega^1(P,\g)$ eine Zusammenhangsform auf $P$, $U\subset M$ eine
offene Umgebung und $s\colon U\to P$ ein lokaler Schnitt von $P$. Die lokale 1-Form
\begin{align*}
A^s \defl s^* A =  A\circ \ds \in \Omega^1(U,\g)
\end{align*}
heißt die \emph{durch $s$ definierte lokale Zusammenhangsform auf $U$}.\fish
\end{defn}

Wir wollen nun untersuchen, wie sich die lokale Zusammenhangsform $A^s$
verändert, wenn man den lokalen Schnitt $s$ durch einen anderen lokalen Schnitt 
ersetzt. Seien also $U_i,U_j\subset M$ offene Umgebung mit $U_i\cap U_j\neq
\varnothing$ und $s_i : U_i\to P$ und $s_j : U_j\to P$ lokale Schnitte. Dann
existiert eine Abbildung
\begin{align*}
g_{ij}  : U_i\cap U_j\to G,
\end{align*}
so dass sich die lokalen Schnitte durch Translation mit der Abbildung
$g_{ij}$ ineinander überführen lassen
\begin{align*}
s_j(x) = s_i(x)\cdot g_{ij}(x),\qquad x\in U_i\cap U_j.
\end{align*}
Wir definieren nun die \emph{zurückgezogene kanonische 1-Form} als
\begin{align*}
\mu_{ij} = g_{ij}^* \mu_G \in \Omega^1(U_i\cap U_j,\g),
\end{align*}
d.h. für einen Tangentialvektor $X\in T_x(U_i\cap U_j)$ an einen Punkt $x\in
U_i\cap U_j$ ist
\begin{align*}
\mu_{ij}(X) = \mu_G(\dg_{ij}(X)).
\end{align*}
Nach dieser Vorbereitung können wir nun formulieren, wie sich $A^s$ beim
Übergang zu einem anderen lokalen Schnitt verhält. Mehr noch, denn wir gewinnen
eine Verklebungsvorschrift für die lokalen 1-Formen $A^s$ mit deren Hilfe man
die Zusammenhangsform auf dem Hauptfaserbündel aus den lokalen
Zusammenhangsformen wiedergewinnen kann.

\begin{figure}[h]
\centering
\begin{pspicture}(0,-1.75)(6,1.73)
\psline(0.4,-1.19)(3.8,-1.19)
\psbezier(0.64,1.39)(1.3,0.89)(1.34,-0.51)(0.92,-0.97)
\psbezier(1.72,1.41)(2.38,0.91)(2.42,-0.49)(2.0,-0.95)
\psbezier(2.84,1.43)(3.5,0.93)(3.54,-0.47)(3.12,-0.93)
\psarc(0.98,-1.19){0.14}{90.0}{270.0}
\psarc(3.18,-1.19){0.14}{270.0}{90.0}

\rput(1.99,-1.525){\color{darkgray}$U_i\cap U_j$}
\psbezier[linecolor=darkblue](0.0,-0.25)(0.84,-0.81)(1.5,-0.13)(2.04,0.07)(2.58,0.27)(2.84,-0.79)(3.36,-0.45)
\psbezier[linecolor=purple](0.78,1.19)(1.56,1.47)(1.74,0.99)(2.34,0.77)(2.94,0.55)(3.68,1.71)(4.6,1.25)
\rput(3.74,-0.345){\color{darkblue}$s_i$}
\rput(4.87,1.275){\color{purple}$s_j$}

\rput(5.4,1.275){\color{darkgray}$P$}
\rput(5.4,-1.19){\color{darkgray}$M$}
\psline{->}(5.4,1)(5.4,-0.9)
\rput(5.58,0){\color{darkgray}$\pi$}
\end{pspicture} 
\caption{Zwei Schnitte in einem Hauptfaserbündel.}
\end{figure}

\begin{prop}
\begin{propenum}
\item
Sei $A\in \Omega^1(P,\g)$ eine Zusammenhangsform und $(s_i,U_i)$ und $(s_j,U_j)$
lokale Schnitte mit $U_i\cap U_j\neq \varnothing$. Dann gilt
\begin{align*}
A^{s_j} = \Ad(g_{ij}^{-1})A^{s_i} + \mu_{ij}.
\end{align*}
\item Ist umgekehrt eine Überdeckung von $M$ durch lokale Schnitte $(s_i,U_i)$
gegeben sowie eine Familie von 1-Formen $A_i\in \Omega^1(U_i,\g)$, so dass auf
$U_i\cap U_j\neq \varnothing$ gilt
\begin{align*}
A_j = \Ad(g_{ij}^{-1})\circ A_i + \mu_{ij},\tag{*}
\end{align*}
dann existiert eine Zusammenhangsform $A$ auf $P$ mit $A^{s_i} = A_i$ für alle
lokalen Schnitte $s_i$.\fish
\end{propenum}
\end{prop}

\begin{rem}[Spezialfälle.]
\begin{remenum}
\item Sei die Strukturgruppe $G\subset \GL(n,\K)$ eine Matrizengruppe, dann ist
die Gruppenwirkung auf dem Hauptfaserbündel linear und für alle $g\in G$ und
$X\in\g \subset \gl(n,\K)$ gilt
\begin{align*}
\dL_g(X) = g\cdot X,\qquad \Ad(g)X = gXg^{-1}.
\end{align*}
Die (*)-Bedingung übersetzt sich somit zu
\begin{align*}
A_j = g_{ij}^{-1}A_i g_{ij} + g_{ij}^{-1}\dg_{ij}.
\end{align*}
\item Ist $P$ trivial, dann existiert ein globaler Schnitt $s\colon M\to P$ und die
Zusammenhangsform auf ganz $P$ ist bereits  durch $A^s\in
\Omega^1(M,\g)$ gegeben.\map
\end{remenum}
\end{rem}

\begin{proof}[Beweis des Satzes.]
a): Sei $x\in U_i$, dann ist für $X\in T_xM$ die lokale Zusammenhangsform auf
$U_i$ definiert durch
\begin{align*}
A^{s_i}(X) \defl (s_i^* A)(X) = A(\ds_i(X)). 
\end{align*}
Weiterhin lässt sich $s_i$ auf $U_i\cap U_j\neq \varnothing$ mit Hilfe der
Kozyklen $g_{ij}$ in $s_j$ überführen
\begin{align*}
s_j = s_i \cdot g_{ij}. 
\end{align*}
Um $\ds_j$ über $\ds_i$ auszudrücken, betrachte eine Kurve $\gamma(t)$ 
in $M$ mit $\gamma(0) = x$ und $\dot{\gamma}(0) = X$, dann ist nach dem
Produktformel für Kurven
\begin{align*}
\ds_j(X) &=\frac{\ddd}{\dt}\bigg|_{t=0} s_j\circ\gamma(t)
= \frac{\ddd}{\dt}\bigg|_{t=0} s_i\circ\gamma(t) \cdot
g_{ij}\circ\gamma(t)\\
&=\dR_{g_{ij}(x)} \ds_i(X) + \widetilde{\mu_G(\dg_{ij}(X))}(s_i(x)\cdot
g_{ij}(x))\\
&= \dR_{g_{ij}(x)} \ds_i(X) + \widetilde{\mu_{ij}(X)}(s_i(x)).
\end{align*}
Somit erhalten wir schließlich aus den Rechenregeln für Zusammenhangsformen,
\begin{align*}
A^{s_j}(X) &= A(\ds_j(X))
= A(\dR_{g_{ij}(x)} \ds_i(x) + \widetilde{\mu_{ij}(X)}(s_i(x)))\\
&= R_{g_{ij}(x)}^*A(\ds_i(X)) + A(\widetilde{\mu_{ij}(X)}(s_i(x)))\\
&= \Ad(g_{ij}^{-1}(x)) A^{s_i}(X) + \mu_{ij}(X).
\end{align*}

b): Sei also eine Überdeckung von $P$ durch Schnitte $(s_i,U_i)$ gegeben sowie
eine Familie von 1-Formen $A_i\in\Omega^1(U_i,\g)$, so dass (*) erfüllt ist. Wir
konstruieren zunächst mit den $A_i$ Zusammenhangsformen auf $P_{U_i} =
\pi^{-1}(U_i)$.

Sei dazu $x\in U_i$ und $u= s_i(x)$. Dann ist $P_{U_i}\cong U_i\times G$ über
den Diffeomorphismus
\begin{align*}
U_i\times G\to P_{U_i},\qquad (x,g) \mapsto s_i(x)\cdot g. 
\end{align*}
Somit ist $T_x U_i$ komplementär zu $T_gG \cong T^v_uP_{U_i}$, d.h.
\begin{align*}
T_uP_{U_i} = T^v_uP_{U_i} \oplus \ds_i(T_xU_i).
\end{align*}

\begin{figure}[H]
\centering
\begin{pspicture}(-0.2,-1.55)(5.26,1.55)
\psline(0.68,-1.39)(4.08,-1.39)
\psbezier(0.92,1.19)(1.58,0.69)(1.62,-0.71)(1.2,-1.17)
\psbezier(2.0,1.21)(2.66,0.71)(2.7,-0.69)(2.28,-1.15)
\psbezier(3.12,1.23)(3.78,0.73)(3.82,-0.67)(3.4,-1.13)
\psbezier[linecolor=darkyellow](0.78,0.23)(1.56,0.49)(1.36,0.15)(2.02,-0.11)(2.68,-0.37)(3.42,0.75)(4.34,0.29)

\rput(4.44,0.575){\color{darkyellow}$s_i$}
\rput(4.94,1.055){\color{darkgray}$P_{U_i}$}
\rput(4.96,-1.385){\color{darkgray}$U_i$}
\psline{->}(4.94,0.79)(4.96,-1.07)
\psline[linecolor=darkblue](0.76,-0.71)(4.26,0.49)
\psline[linecolor=purple](2.5,1.45)(2.6,-1.29)

\rput(2.03,1.375){\color{purple}$T^v_uP$}
\rput(0.69,-0.245){\color{darkblue}$\ds_i(T_xU_i)$}
\end{pspicture} 
\caption{Tangentialvektoren an den Schnitt sind horizontal.}
\end{figure}

Für festes $u = s_i(x)\in P_{U_i}$ definieren wir nun eine 1-Form $A_u :
T_uP_{U_i} \to \g$ durch
\begin{align*}
A_u(\tilde{Y}(u) + \ds_i(X)) = Y + A_i(X),\qquad Y\in \g,\quad
X\in T_xU_i,
\end{align*}
und setzten diese auf die Faser $P_x$ fort durch die Vorschrift
\begin{align*}
A_{ug} = \Ad(g^{-1})A_u\circ \dR_{g^{-1}},\qquad g\in G. 
\end{align*}
Dann ist $A$ eine Zusammenhangsform auf $P_x$, denn 
nach Lemma \ref{lem:Rechts-Translation-Fundamentales-Vektorfeld} gilt
\begin{align*}
A_{ug}(\tilde{Y}(ug)) &= \Ad(g^{-1})A_u(\dR_{g^{-1}}(\tilde{Y}(ug)))\\
&= \Ad(g^{-1})(A_u(\widetilde{\Ad(g)Y}(u)))\\
&= \Ad(g^{-1})\Ad(g)Y = Y.
\end{align*}
Weiterhin ist für $a\in G$ und $Z\in T_{ug}P$,
\begin{align*}
(R_a^* A)_{ug}(Z) &= 
A_{uga}(\dR_a(Z))
= \Ad((ga)^{-1})A_u(\dR_{(ga)^{-1}}\dR_a(Z))\\
&= \Ad(a^{-1})\Ad(g^{-1})A_u(\dR_{g^{-1}}(Z))
= \Ad(a^{-1})A_{ug}(Z).
\end{align*}
Für jedes $x\in U_i$ erhalten wir somit eine Zusammenhangsform auf der Faser
$P_x$ und somit eine lokale Zusammenhangsform auf $P_{U_i}$.

Damit durch diese
Konstruktion eine Zusammenhangsform auf ganz $P$ gegeben ist, müssen zwei
Zusammenhangsformen $A$ und $\hat{A}$, die auf $P_{U_i}$ bzw. $P_{U_j}$ definiert sind,
auf dem Durchschnitt $P_{U_i}\cap P_{U_j}$ übereinstimmen.
Auf vertikalen Vektoren gilt bereits
\begin{align*}
A_u(\tilde{X}(u)) = X = \hat{A}_u(\tilde{X}(u)),\qquad u\in P_{U_i}\cap P_{U_j}.
\end{align*}
Weiterhin ist die Summe $T_uP_{U_j} = \ds_j(T_xU_j)\oplus T^v_uP$ direkt, es
genügt daher zu zeigen, dass $A$ und $\hat{A}$ auch auf dem Bild von $\ds_j$
übereinstimmen. Sei also $X\in T_xM$ für ein $x\in U_i\cap U_j\neq \varnothing$. Wir stellen wiederum $\ds_j$ über $\ds_i$ dar und erhalten,
\begin{align*}
A_{s_j(x)}(\ds_j(X)) &= 
{A}_{s_i(x)g_{ij}(x)}(\dR_{g_{ij}(x)}\ds_i(X) + \widetilde{\mu_{ij}(X)}(s_i(x))) \\
&=
(R_{g_{ij}(x)}^*{A})_{s_i(x)}(\ds_i(X)) + \mu_{ij}(X)\\
&=
\Ad(g_{ij}^{-1}(x)){A}_{s_i(x)}(\ds_i(X)) + \mu_{ij}(X)\\
&=
\Ad(g_{ij}^{-1}(x))A_i(X) + \mu_{ij}(X)\\
&=A_j(X) = \hat{A}_{s_j(x)}(\ds_j(X)).
\end{align*}
Die lokal auf $P_{U_i}$ definierten Zusammenhangsformen können also zu einer
global auf $P$ definierten Zusammenhangsform verklebt werden.\qed
\end{proof}


\section{Zusammenhänge und kovariante Ableitungen}

Sei $M$ eine $n$-dimensionale Mannigfaltigkeit. Dann ist das Rahmenbündel
$\GL(M)$ über $M$ ein $\GL_n$-Hauptfaserbündel und trägt somit einen
Zusammenhang. Ist $M$ beispielsweise eine Riemannsche Mannigfaltigkeit, dann
trägt das Tangentialbündel $TM$ eine kovariante Ableitung $\nabla$, den
Levi-Civita-Zusammenhang. Nun ist das Tangentialbündel außerdem ein zu $\GL(M)$
assoziiertes Vektorbündel und es stellt sich die Frage, in welcher Beziehung
Zusammenhänge auf $\GL(M)$ und kovariante Ableitungen auf $TM$ stehen. 


\begin{prop}
Es gibt eine bijektive Beziehung zwischen kovarianten Ableitungen auf $TM$ und
Zusammenhängen auf $\GL(M)$.
\begin{align*}
\nabla\text{ kovariante Ableitung auf }TM \leftrightsquigarrow A \text{
Zusammenhangsform auf }\GL(M).\fish
\end{align*}
\end{prop}

\begin{proof}

$\leftsquigarrow$: Sei $A\colon T\GL(M) \to \gl(n)$ eine Zusammenhangsform auf
$\GL(M)$. Die Standard-Einheitsmatrizen $E_{ij}$, die eine 1 in der $i$-ten
Zeile und $j$-ten Spalte haben und sonst nur aus der Null bestehen, bilden eine
Basis von $\gl(n)$. Somit lässt sich $A$ schreiben als
\begin{align*}
A = \sum_{i,j=1}^n \omega_{ij} E_{ij},
\end{align*}
mit 1-Formen $\omega_{ij}\colon T\GL(M)\to \R$. Sei nun 
\begin{align*}
s=(s_1,\ldots,s_n): U\to \GL(M)
\end{align*}
ein lokaler Schnitt in $\GL(M)$, dann bilden die Vektorfelder $s_1,\ldots,s_n$
in jedem Punkt eine Basis von $TM$. Wir definieren nun lokal eine kovariante
Ableitung auf $TU$ durch
\begin{align*}
\nabla_X s_k \defl \sum_{i=1}^n \omega_{ik}^s(X)s_i
\defl \sum_{i=1}^n \omega_{ik}(\ds(X))s_i,\qquad 1\le k\le n,
\end{align*}
setzen diese linear auf ganz $TU$ fort und verlangen für $f\in\Cs^\infty(M)$,
dass
\begin{align*}
\nabla_X (f s_k) = \df(X) s_k + f\nabla_X s_k.
\end{align*}
Dann ist $\nabla$ tatsächlich eine kovariante Ableitung auf $TU$, die eventuell
noch von der Wahl der $s_1,\ldots,s_n$ abhängt. Man zeigt nun, dass für einen
anderen Schnitt $s'$ gilt
\begin{align*}
(\nabla_X Y)_s = (\nabla_X Y)_{s'},\qquad X,Y\in\chi(U).
\end{align*}
Dazu schreibt man $Y = \sum_{k=1}^n y_k s_k = \sum_{l=1}^n y_l' s_l'$ und
überprüft
\begin{align*}
(\nabla_X Y)_s &= 
\sum_{k=1}^n \nabla_X (y_k s_k)
= \sum_{k=1}^n \left(\dy_k(X) s_k + y_k \nabla_X s_k\right)\\
&= \sum_{k=1}^n \dy_k(X) s_k   + \sum_{i,k=1}^n y_k \omega_{ik}^s s_i.
\end{align*}
Andererseits ist
\begin{align*}
(\nabla_X Y)_{s'} = 
\sum_{l=1}^n \dy_l'(X) s_l'   + \sum_{j,l=1}^n y_l' \omega_{jl}^{s'} s_j'.
\end{align*}
Verwendet man nun
\begin{align*}
s_k(x) = \sum_{i=1}^n s_j'(x)g_{jk}(x),\qquad y_k =
\sum_{k,l=1}^ny_l'(g^{-1})_{kl},
\end{align*}
sowie dass für lineare Strukturgruppen
\begin{align*}
A^s = g^{-1}A^{s'}g + g^{-1}\dg,
\end{align*}
erhält man nach einer längeren Rechnung $(\nabla_X Y)_s = (\nabla_X Y)_{s'}$.

$\rightsquigarrow$: Sei nun eine kovariante Ableitung auf $TM$ gegeben, dann
erhält man zu einem lokalen Schnitt $s\colon U \to \GL(M)$ analog zu obiger
Definition die 1-Formen $\omega_{ik}$. Diese setzt man nun zu einer lokalen
1-Formen auf $U$ zusammen und verifiziert, dass es sich um eine
lokale Zusammenhangsform handelt. Anschließend verklebt man diese lokalen Formen
zu einer globalen Form auf $T\GL(M)$.\qed
\end{proof}

\begin{rem}[Bemerkungen.]
\begin{remenum}
\item
Auf dem Tangentialbündel jeder Mannigfaltigkeit existiert eine kovariante
Ableitung.
\item
Sei $\nabla$ ein metrischer Zusammenhang auf dem Tangentialbündel, d.h. für
$X,Y,Z\in\chi(M)$ gilt
\begin{align*}
Z(g(X,Y)) = g(\nabla_Z X,Y) + g(X,\nabla_Z Y).
\end{align*}
Dann ist die Metrik parallel und $\nabla$ induziert auf dem Rahmenbündel eine
Zusammenhangsform
\begin{align*}
A = \sum_{i,j=1}^n \omega_{ij}E_{ij},
\end{align*}
mit $\omega_{ij} = - \omega_{ji}$.
\item Obiges Konstruktionsschema funktioniert allgemein. Eine
Zusammenhangsform $A$ auf dem Hauptfaserbündel induziert in analoger Weise eine
kovariante Ableitung auf allen assoziierten Faserbündeln.

Umgekehrt induziert eine kovariante Ableitung auf einem Vektorbündel $E\to M$
eine Zusammenhangsform auf dem Rahmenbündel über $E$.\map
\end{remenum}
\end{rem}

\begin{ex}
\textit{Linksinvariante Zusammenhänge auf reduktiven homogenen Räumen}. Sei $G$
eine Lie-Gruppe und $H\subset G$ eine abgeschlossene Untergruppe. Dann ist $H$
ebenfalls eine Lie-Gruppe und wirkt durch die Multiplikation in
natürlicher Weise von rechts auf $G$. Der Quotient $M=G/H$ ist eine
differenzierbare Mannigfaltigkeit, ein sogenannter \emph{homogener Raum} und
$(G,\pi,M;H)$ ist ein $H$-Hauptfaserbündel.
\begin{figure}[H]
\centering
\begin{tikzpicture}[description/.style={fill=white,inner sep=2pt}]
\matrix (m) [matrix of math nodes, row sep=3em,
column sep=2.5em, text height=1.5ex, text depth=0.25ex]
{ H & & G \\
&  & M \\ };
\path[->,font=\scriptsize]
(m-1-1) edge node[auto] {$ g\cdot h $} (m-1-3)
(m-1-3) edge node[auto] {$ \pi $} (m-2-3);
\end{tikzpicture}
\end{figure}
Man nennt $G/H$ \emph{reduktiv}, falls es eine Vektorraumzerlegung gibt, so dass
\begin{align*}
\g = \h\oplus\m,\qquad \text{und}\qquad \Ad(H)\m\subset \m.
\end{align*}
Aus Dimensionsgründen folgt dann sogar $\Ad(H)\m = \m$.

Sei nun $G/H$ reduktiv, also $T_eG = \g = \h\oplus \m$, dann ist $T_gG =
\dL_g(\g)\oplus \dL_g(\m)$.
Betrachte zu $X\in\h$ das fundamentale Vektorfeld $\tilde{X}$ auf $G$, dann ist
für $g\in G$
\begin{align*}
\tilde{X}(g) = \frac{\ddd}{\dt}\bigg|_{t=0} g\cdot \exp(tX)
= \dL_g(X).
\end{align*}
Also ist der vertikale Tangentialraum an einen Punkt $g\in G$ durch
Linkstranslation von $\h$ gegeben, $T^v_g G = \setdef{\tilde{X}(g)}{X\in \h} =
\dL_g(\h)$. Wir definieren nun
\begin{align*}
T^h \colon G\to TG,\qquad g\mapsto T_g^hG = \dL_g(\m), 
\end{align*}
dann ist $T^h$ offenbar linksinvariant, denn
\begin{align*}
\dL_a(T^h_gG) = \dL_{ag}(\m) = T_{ag}^hG.
\end{align*}
Außerdem ist $\dL_g(\m)$ komplementär zu $T_g^vG = \dL_g(\h)$ und
rechtsinvariant, denn
\begin{align*}
\dR_a(T_g^hG) &= \dR_a\dL_g(\h) = 
\dL_g\dR_a(\h) = 
\dL_g\dL_a \dL_{a^{-1}}\dR_a(\h)\\
&= \dL_{ga}\Ad(a^{-1})(\h)
=   \dL_{ga}\h = T_{ga}^h G.
\end{align*}
Somit ist durch die obige Definition von $T^h$ tatsächlich ein horizontales
Tangentialbündel an $G$ gegeben. Die zugehörige Zusammenhangsform lässt sich als
Projektion der Maurer-Cartan Form schreiben
\begin{align*}
A = \pr_{\h}\circ\mu_G \in\Omega^1(G,\h),
\end{align*} 
denn sei $X\in\h$ und $Y_h\in T_g^h G$, dann gilt nach Definition der
Zusammenhangsform
\begin{align*}
A(\tilde{X}(g) + Y_h) = X.
\end{align*}
Andererseits ist $\dL_{g^{-1}}(Y_h) \in \m$ und damit gilt auch
\begin{align*}
\pr_{\h}\circ \mu_G(\tilde{X}(g) + Y_h) &= 
\pr_{\h}(\mu_G(\dL_g(X)+Y_h))
= \pr_{\h}(\dL_{g^{-1}}\dL_g(X) + \dL_{g^{-1}}(Y))\\
&= \pr_{\h}(X) + \pr_{\h}(\dL_{g^{-1}}(Y))
= X.\boxc
\end{align*}
\end{ex}

\chapter{Das Differential eines Zusammenhangs}

\section{Differentialformen mit Werten in Vektorbündeln}

Sei $E\to M$ ein Vektorbündel über $M$.

\begin{defn}
\index{Differentialform!in Vektorbündeln}
Eine \emph{$k$-Form $\omega$ auf $M$ mit Werten in $E$} ist eine
differenzierbare Zuordnung
\begin{align*}
x\in M\mapsto \omega_x,\qquad \omega_x : T_xM\times\cdots \times T_xM\to E_x,
\end{align*}
mit $\omega_x$ multilinear und alternierend. Differenzierbar heißt, dass für jede hinreichend kleine offene Menge
$U\subset M$ und alle glatten Vektorfelder $X_1,\ldots,X_k$ auf $U$ die
Abbildung
\begin{align*}
x\mapsto \omega_x(X_1(x),\ldots,X_k(x))
\end{align*}
ein Schnitt in $E_{U}$ ist. Den \emph{Raum aller $k$-Formen mit Werten in $E$}
bezeichnet man mit $\Omega^k(M,E)$.\fish
\end{defn}

\begin{rem}[Bemerkungen.]
\begin{remenum}
\item
Alternativ ist eine $k$-Form mit Werten in $E$ eine $\Cs^\infty(M)$
multilineare, alternierende Abbildung
\begin{align*}
\omega\colon \underbrace{\chi(M)\times \cdots \times \chi(M)}_{k\text{-mal}}\to
\Gamma(E)
\end{align*}
\item Eine weitere alternative Definition ist $\Omega^k(M,E)
=\Gamma(\Lambda^kT^*M\otimes E)$.
\item Die $0$-Formen sind gerade die Schnitte in $E$, d.h. $\Omega^0(M,E) =
\Gamma(E)$.
\item Ist das Vektorbündel $E$ trivial und vom Rang $r$, dann ist
$\omega\in\Omega^k(M,E)$ wieder eine vektorwertige $k$-Form $\omega =
(\omega_1,\ldots,\omega_r)$ mit gewöhnlichen reellen $k$-Formen
$\omega_i\in\Omega^k(M,\R)$.\map
\end{remenum}
\end{rem}

Sei nun $\pi\colon P\to M$ ein $G$-Hauptfaserbündel, $V$ ein Vektorraum und $\rho\colon
G\to \GL(V)$ eine $G$-Darstellung. Das Vektorbündel $E$ sei weiterhin als
assoziiertes Faserbündel
\begin{align*}
E = P\times_\rho V
\end{align*}
gegeben. Wir haben bereits gesehen, dass sich Schnitte in $E$, also $0$-Formen
mit Werten in $E$, gerade mit $G$-äquivarianten Funktionen identifizieren lassen
\begin{align*}
\Gamma(E) &\cong \Cs^\infty(P,V)^{(G,\rho)}\\
&= \setdef{f: P\to V}{f(p\cdot g) = \rho(g)^{-1}f(p),\qquad p\in P,\; g\in G}.
\end{align*}
Man nennt diese Funktionen auch \emph{Schnitte vom Typ $\rho$}.
Wir wollen nun eine analoge Identifikation für allgemeine $k$-Formen mit Werten
in $E$ erarbeiten.

\begin{defn}
\index{Differentialform!horizontal}
\index{Differentialform!vom Typ $\rho$}
Sei $\omega\in \Omega^k(P,V)$, dann heißt $\omega$
\begin{defnenum}
\item \emph{horizontal}, falls $\omega(X_1,\ldots,X_k) = 0$ wenn ein $X_i$
vertikal ist (d.h. $\dpi(X_i) = 0$), und
\item \emph{vom Typ $\rho$}, falls für alle $g\in G$
\begin{align*}
R_g^* \omega = \rho(g)^{-1}\omega.
\end{align*}
\end{defnenum}
Den \emph{Raum der horizontalen $k$-Formen auf $P$ mit Werten in $V$ vom Typ
$\rho$} bezeichnet man mit $\Omega^k_\hor(P,V)^{(P,\rho)}$.\fish
\end{defn}

\begin{ex}
Wir betrachten die Menge der Zusammenhänge $\Cs(P)$ auf dem
$G$"=Hauptfaserbündel $\pi\colon P\to M$. Sind nun $A_1$ und $A_2$
Zusammenhangsformen, dann sind sie nach Voraussetzung vom Typ $\Ad$ aber
\textit{nicht} horizontal. Jedoch ist ihre Differenz horizontal und daher
\begin{align*}
A_1-A_2 \in \Omega^1_\hor(P,\g)^{(G,\Ad)}.
\end{align*}
Umgekehrt sei $A$ eine Zusammenhangsform und
$\omega\in\Omega_\hor^1(P,\g)^{(G,\Ad)}$, dann ist $A+\omega$ offenbar wiederum
eine Zusammenhangsform.

Somit bilden die Zusammenhänge $\Cs(P)$ auf $P$ einen affinen Raum über dem
Vektorraum $\Omega_\hor^1(P,\g)^{(G,\Ad)}$.\boxc
\end{ex}

Der folgende Satz verallgemeinert die Beziehung
\begin{align*}
\Omega^0(M,E) = \Gamma(E)\cong \Cs^\infty(P,V)^{(G,\rho)} = \Omega_\hor^0(P,V)^{(G,\rho)}.
\end{align*}

\begin{prop}
\label{prop:Identifikation-E-Formen-auf-M}
Der Vektorraum
$\Omega^k(M,E)$ der $k$-Formen auf $M$ mit Werten in $E$ ist isomorph zum
Vektorraum $\Omega_\hor^k(P,V)^{(G,\rho)}$ der horizontalen $k$-Formen auf
$P$ vom Typ $\rho$.\fish
\end{prop}
\begin{proof}[Beweisskizze.]
Zu Konstruieren ist ein Isomorphismus
\begin{align*}
\Omega^k(M,E) \to \Omega_\hor^k(P,V)^{(G,\rho)},\qquad \omega\mapsto
\bar{\omega}.
\end{align*}
Dazu verwenden wir für $p\in P_x$ zu $x\in M$ den Faserdiffeomorphismus
\begin{align*}
[p]: V\to E_x,\qquad v\mapsto [p,v].
\end{align*}

Sei $\bar{\omega}\in \Omega^k(P,V)^{(G,\rho)}$ gegeben und $X_1,\ldots,X_k\in
T_xM$, dann definiert man
\begin{align*}
\omega_x(X_1,\ldots,X_k) \defl
[p,\bar{\omega}_p(\tilde{X}_1,\ldots,\tilde{X}_k)],
\end{align*}
wobei $\tilde{X}_i\in T_pP$, so dass $\dpi(\tilde{X}_i) = X_i$ und $p\in P_x$
beliebig.

Sei umgekehrt $\omega\in \Omega^k(M,E)$ gegeben und $T_1,\ldots,T_k\in T_pP$,
dann definiert man
\begin{align*}
\bar{\omega}_p(T_1,\ldots,T_k) \defl
[p]^{-1}\omega_{\pi(p)}(\dpi(T_1),\ldots,\dpi(T_k)).
\end{align*}

Es ist nun zu zeigen, dass diese Abbildung wohldefiniert ist und $\omega$ und
$\bar{\omega}$ tatsächlich glatte Abbildungen sind. Für Details siehe \cite[Satz
3.5]{Baum:2009wk}.\qed
\end{proof}

\begin{defn}
Das \emph{adjungierte Vektorbündel} zu einem $G$-Hauptfaserbündel $P$ ist
definiert als
\begin{align*}
\Ad(P) \defl P\times_\Ad \g.\fish
\end{align*}
\end{defn}

\begin{rem}
Nach obiger Identifikation und der vorangegangen Bemerkung ist der Raum der
Zusammenhänge $\Cs(P)$ auf $P$ ein affiner Raum über dem Vektorraum
$\Omega^1(M,\Ad(P))\cong \Omega^1_\hor(P,V)^{(G,\Ad)}$.\map
\end{rem}

\section{Das absolute Differential eines Zusammenhangs}

Sei wieder $\pi\colon P\to M$ ein $G$-Hauptfaserbündel, $V$ ein Vektorraum, $\rho\colon
G\to \GL(V)$ eine $G$-Darstellung und
\begin{align*}
E = P\times_\rho V.
\end{align*}

\begin{defn}
\index{kovariante Ableitung}
Eine \emph{kovariante Ableitung $\nabla$} auf dem Vektorbündel $\Gamma(E)$ ist
eine Abbildung
\begin{align*}
\nabla \colon \Gamma(E)\to \Gamma(T^*M\otimes E),\qquad
e \mapsto \nabla e,
\end{align*}
so dass für $f\in\Cs^\infty(M)$ gilt
\begin{align*}
\nabla (fe) = \df\otimes e + f\nabla e.
\end{align*}
Schreiben wir $\nabla_X e \defl (\nabla e)(X)$ für ein Vektorfeld $X\in\chi(M)$,
dann ist
\begin{align*}
\nabla_X(fe) = \df(X)e + f \nabla_X e.\fish
\end{align*} 
\end{defn}

Mit den bisherigen Definitionen lässt sich $\nabla$ auf $E=P\times_\rho V$ auch
schreiben als
\begin{align*}
\nabla \colon \Omega^0(M,E) \to \Omega^1(M,E).
\end{align*}
Ziel dieses Abschnittes ist es, für einen Zusammenhang $A$ auf $P$ ein
Differential
\begin{align*}
\ddd_A \colon \Omega^k(M,E)\to \Omega^{k+1}(M,E)
\end{align*}
auf den $E$-wertigen Formen zu definieren, so dass $\nabla$ gerade die $0$-te
Komponente von $\ddd_A$ wird. Dazu werden wir das Differential zunächst
auf $\Omega_\hor^k(M,V)^{(G,\rho)}$ definieren und anschließend mit Hilfe von
Satz \ref{prop:Identifikation-E-Formen-auf-M} auf $\Omega^k(M,E)$ übersetzen.

Sei also $A$ ein Zusammenhang auf $P$, dann zerlegt sich das Tangentialbündel an
$P$ in die direkte Summe
\begin{align*}
TP = T^hP\oplus T^vP,\qquad T^vP = \ker(\dpi).
\end{align*}
Die Projektion auf die horizontale Komponente bezeichnen wir mit $\pr_h$
\begin{align*}
\pr_h \colon TP \to T^hP.
\end{align*}

\begin{defn}
\index{Differential!absolutes}
Sei $A$ ein Zusammenhang auf $P$, dann heißt die lineare Abbildung 
\[D_A \colon
\Omega^k(P,V)\to \Omega^{k+1}(P,V),\]
die für Vektorfelder $T_1,\ldots,T_k$ in $P$ gegeben ist durch
\begin{align*}
(D_A\omega)(T_1,\ldots,T_k) \defl \dom(\pr_h(T_1),\ldots,\pr_h(T_k)),
\end{align*}
\emph{absolutes Differential} von $A$ auf $P$.\fish
\end{defn}

Offenbar liegt das Bild von $D_A$ in den horizontalen Formen. Mit Hilfe der
Identifikation $\Omega_\hor^k(P,V)^{(G,\rho)}\cong \Omega^k(M,E)$ wollen wir
$D_A$ nun auf $E$-wertige Formen übersetzen. Dazu ist es jedoch zunächst einmal erforderlich nachzuweisen, dass
$D_A$ auch den Typ $\rho$ erhält.

\begin{prop}
\label{prop:Absolutes-Differential}
\begin{propenum}
\item Das absolute Differential erhält horizontale Formen vom Typ $\rho$, d.h.
\begin{align*}
D_A\big|_{\Omega_\hor^k(P,V)^{(G,\rho)}} \colon \Omega_\hor^k(P,V)^{(G,\rho)} \to
\Omega_\hor^{k+1}(P,V)^{(G,\rho)}.
\end{align*}
\item Sei $\omega\in\Omega_\hor^k(P,V)^{(G,\rho)}$, dann gilt
\begin{align*}
D_A\omega = \dom + \rho_*(A)\wedge \omega,
\end{align*}
wobei für Vektorfelder $T_0,\ldots,T_k$ in $P$ der letztere Term gegeben ist
durch
\begin{align*}
(\rho_*(A)\wedge \omega)(T_0,\ldots,T_k) = 
\sum_{i=0}^k (-1)^i \rho_*(A(T_i))\omega(\ldots,\hat{T}_i,\ldots).\fish
\end{align*}
\end{propenum}
\end{prop}

Bevor wir den Satz beweisen können, benötigen wir noch ein Verfahren um
Vektorfelder auf $M$ zu horizontalen Vektorfeldern auf $P$ zu >>liften<<. 

\begin{defn}
Sei $X$ ein Vektorfeld auf $M$. Der \emph{horizontale Lift} von $X$ ist ein
Vektorfeld $X^*$ auf $P$ mit
\begin{defnenum}
\item $X^*_p\in T_p^hP$\quad für $p\in P$,\quad und
\item $\dpi(X^*) \equiv X$.\fish
\end{defnenum}
\end{defn}

\begin{prop}[Proposition]
\begin{propenum}
\item Zu jedem Vektorfeld $X$ auf $M$ existiert genau ein horizontaler Lift
$X^*$ auf $P$.
\item Sei $T$ ein horizontales, rechtsinvariantes Vektorfeld auf $P$, dann
existiert genau ein Vektorfeld $X$ auf $M$ mit $X^* = T$. 
\item Der horizontale Lift ist rechtsinvariant, d.h.
\begin{align*}
X^*_{p\cdot g} = \dR_g(X_p^*),\qquad \text{für alle }p\in P\text{ und }g\in G.
\end{align*}
\item Der horizontale Lift kommutiert mit vertikalen Vektorfeldern, d.h.
\begin{align*}
[\tilde{Y},X^*] = 0,\qquad \text{für }X\in\chi(M)\text{ und }Y\in\g.\fish
\end{align*}
\end{propenum}
\end{prop}

Wir werden die Proposition später beweisen.

\begin{proof}[Beweis des Satzes.]
a): Das Bild von $D_A$ liegt in den horizontalen Formen, denn sei
$\omega\in\Omega^k(P,V)$ und $T_0,\ldots,T_k\in T_pP$ mit einem $T_i$ vertikal,
dann ist $\pr_h(T_i) = 0$ und folglich
\begin{align*}
(D_A\omega)(T_0,\ldots,T_k) = \dom(\pr_h(T_0),\ldots,\pr_h(T_k)) = 0.
\end{align*}

Es verbleibt zu zeigen, dass $D_A$ auch den Typ erhält. Sei dazu
$\omega\in\Omega^k(P,V)$ vom Typ $\rho$. Dann ist für $g\in G$ zu
zeigen, dass $R_g^*D_A\omega = \rho(g)^{-1}D_A\omega$. Da der horizontale
Tangentialraum eine rechtsinvariante Distribution ist, gilt $\pr_h\circ \dR_g =
\dR_g\circ \pr_h$ und folglich
\begin{align*}
(R_g^* D_A \omega) &
= \dom(\pr_h\circ \dR_g)
= \dom(\dR_g\circ \pr_h)\\
&= (R_g^*\dom)\circ\pr_h
= (\ddd (R_g^*\omega))\circ\pr_h\\
&= \rho(g)^{-1} \dom\circ\pr_h
=\rho(g)^{-1} D_A\omega.
\end{align*}
Somit erhält $D_A$ horizontale Formen vom Typ $\rho$.

b): Seien $T_0,\ldots,T_k\in T_pP$, dann ist zu zeigen, dass
\begin{align*}
(D_A\omega)(T_0,\ldots,T_k) = 
\dom(T_0,\ldots,T_k) + (\rho_*(A)\wedge \omega)(T_0,\ldots,T_k).
\end{align*}
Aufgrund der Multilinearität genügt es $T_i$ zu betrachten die
entweder vertikal oder horizontal sind.

\textit{1. Fall, alle $T_i$ sind horizontal}. Dann ist $\pr_h(T_i) = T_i$ und
das absolute Differential entspricht dem gewöhnlichen,
\begin{align*}
D_A\omega(T_0,\ldots,T_k) = \dom(T_0,\ldots,T_k).
\end{align*}
Andererseits ist $A(T_i) = 0$ für alle $T_i$, und somit ist $(\rho_*(A)\wedge
\omega)(T_0,\ldots,T_k) = 0$. Also gilt die Formel.

\textit{2. Fall, mindestens zwei $T_i$ sind vertikal, die übrigen horizontal}.
Nach a) ist $D_A\omega$ horizontal und folglich
$D_A\omega(T_0,\ldots,T_k) = 0$. Auch $\omega$ ist horizontal, also
\begin{align*}
\omega(\ldots,\hat{T}_i,\ldots) = 0,\qquad 0\le i\le k.
\end{align*}
Somit ist $(\rho_*(A)\wedge\omega)(T_0,\ldots,T_k) = 0$ und es verbleibt zu
zeigen, dass $\dom(T_0,\ldots,T_k) =0$. Dazu verwenden wir
folgende wichtige Formel für das gewöhnliche Differential
\begin{align*}
\dom(T_0,\ldots,T_k) &= 
\sum_{i=0}^k (-1)^i T_i(\omega(\ldots,\hat{T}_i,\ldots))\\
&+
\sum_{i<j} (-1)^{i+j}
\omega([T_i,T_j],\ldots,\hat{T}_i,\ldots,\hat{T}_j,\ldots)).
\end{align*}
In der vorderen Summe sind alle Terme Null, denn $\omega$ ist horizontal.
Weiterhin sind in der hinteren Summe alle Terme Null bis auf eventuell den Term,
für den beide vertikalen Vektoren im Kommutator stehen, d.h.
\begin{align*}
\dom(T_0,\ldots,T_k) = (-1)^{i+j}
\omega([T_i,T_j],\ldots,\hat{T}_i,\ldots,\hat{T}_j,\ldots),\qquad
T_i,T_j\text{ vertikal}.
\end{align*}
Nun existieren $X_i,X_j\in \g$, so dass für die fundamentalen Vektorfelder gilt
$\tilde{X_i}(p) = T_i$ und $\tilde{X_j}(p) = T_j$. Dann ist aber auch
\begin{align*}
[T_i,T_j] = [\tilde{X}_i,\tilde{X}_j]\bigg|_{p} =
\widetilde{[X_i,X_j]}\bigg|_{p} \in T_p^vP,
\end{align*}
und somit ist auch der verbliebene Term Null.

\textit{3. Fall, $T_0$ ist vertikal und die übrigen sind horizontal}. Dann ist
$D_A\omega = 0$ und es existiert ein $X\in \g$, so dass $\widetilde{X}(p) = T_0$
und somit
\begin{align*}
(\rho_*(A)\wedge\omega)(T_0,\ldots,T_k) = 
\rho_*(X)\omega(T_1,\ldots,T_k).
\end{align*}
Zu zeigen ist nun, dass $\dom(T_0,\ldots,T_k) =
-\rho_*(X)\omega(T_1,\ldots,T_k)$. Dazu bezeichne $X_1^*$,\ldots $X_k^*$ die
horizontalen Lifts zu $T_1,\ldots,T_k$, dann ergibt die Formel für das
gewöhnliche Differential und die Horizontalität von $\omega$, dass
\begin{align*}
\dom(T_0,\ldots,T_k) &= 
\tilde{X}(\omega(X_1^*,\ldots,X_k^*))
+ \sum_{i=1}^k
(-1)^{i} 
\omega([\tilde{X},X_i^*],\ldots,\hat{X}_i^*,\ldots)
\bigg|_{p}
\\
&= \tilde{X}(\omega(X_1^*,\ldots,X_k^*))\bigg|_{p},
\end{align*}
denn $[\tilde{X},X_i^*] = 0$. Weiterhin ist nach Definition der Lie-Ableitung,
\begin{align*}
\tilde{X}(\omega(X_1^*,\ldots,X_k^*))\bigg|_{p} &= 
\frac{\ddd}{\dt}\bigg|_0
\omega(X_1^*(p\cdot \exp(tX)),\ldots,X_k^*(p\cdot \exp(tX)))\\
&=
\frac{\ddd}{\dt}\bigg|_0
\omega(\dR_{\exp(tX)}X_1^*,\ldots,\dR_{\exp(tX)}X_k^*)\\
&= \frac{\ddd}{\dt}\bigg|_0
(R_{\exp(tX)}^*\omega)(X_1^*,\ldots,X_k^*)\\
&= \frac{\ddd}{\dt}\bigg|_0
\rho(\exp(-tX))\omega(X_1^*,\ldots,X_k^*)\\
&= \frac{\ddd}{\dt}\bigg|_0
\e^{-t\ddd\rho(X)}\omega(X_1^*,\ldots,X_k^*)\\
&= -\ddd\rho(X)\omega(X_1^*,\ldots,X_k^*)\\
&= -\rho_*(X)\omega(X_1^*,\ldots,X_k^*).
\end{align*}
Somit ist der Beweis vollständig.\qed
\end{proof}

% Vorlesung vom 06.06.2011

Das absolute Differential ist also ein Differentialoperator 1. Ordnung auf
horizontalen Formen vom Typ $\rho$,
\begin{align*}
D_A \colon \Omega_\hor^k(P,V)^{(G,\rho)}\to \Omega_\hor^{k+1}(P,V)^{(G,\rho)}.
\end{align*}
Mit Hilfe des Isomorphismus $\Omega_\hor^k(P,V)^{(G,\rho)}\to \Omega^k(M,E)$,
$\bar{\omega}\mapsto \omega$ können wir nun ein Differential für Formen
mit Werten in $E$ definieren.

\begin{defn}
\index{Differential}
Sei $A$ ein Zusammenhang auf $P$, dann ist das \emph{Differential $\ddd_A$}
ein Endormorphismus von $\Omega^*(M,E)$ definiert durch
\begin{align*}
\ddd_A \colon \Omega^k(M,E)\to\Omega^{k+1}(M,E),\qquad \bar{\ddd_A \omega} \defl
D_A\bar{\omega},
\end{align*}
wobei $\bar{\omega}$ über den Isomorphismus aus Satz
\ref{prop:Identifikation-E-Formen-auf-M} gegeben ist.\fish
\end{defn}

Das Differential $\ddd$ auf den gewöhnlichen Formen $\Omega^*(M)$ hat die
Eigenschaft, dass $\ddd^2 = 0$. Unser Ziel ist es nun, $\ddd_A^2$ zu
berechnen. Wir werden feststellen, dass $\ddd_A^2$ im Allgemeinen nicht
verschwindet.

\begin{rem}[Bemerkungen.]
\begin{remenum}
\item Seien $U\subset M$ offen, $\pi(p) = x\in M$ und $X_0,\ldots,X_k$
Tangentialvektoren in $T_xM$ mit horizontalen Lifts $X_0^*,\ldots,X_k^*$.
Weiterhin sei $\omega\in\Omega^k(M,E)$, dann hat das Differential von $\omega$
im Punkt $x$ die Darstellung
\begin{align*}
(\ddd_A\omega)_x(X_0,\ldots,X_k) = 
[p,D_A\bar{\omega}_p(X_0^*,\ldots,X_k^*)]= 
[p,\ddd \bar{\omega}_p(X_0^*,\ldots,X_k^*)].
\end{align*}

Alternativ kann man das Differential auch ohne horizontale Lifts beschreiben.
Sei dazu $s\colon U\to P$ ein lokaler Schnitt, dann gilt
\begin{align*}
(\ddd_A\omega)_x(X_0,\ldots,X_k) = 
[s(x),(D_A \bar{\omega})_p(\ds(X_0),\ldots,\ds(X_k))].
\end{align*}
\item Für gewöhnliche Differentialformen $\sigma,\omega\in\Omega^*(M)$ kennen
wir das Dachprodukt $\sigma\wedge \omega\in\Omega^*(M)$. Für zwei bündelwertige
Formen $\sigma,\omega\in \Omega^*(M,E)$  lässt sich 
$\sigma\wedge\omega$ jedoch nicht sinnvoll definieren. Allerdings operieren
die gewöhnlichen Differentialformen auf den bündelwertigen Formen  durch
\begin{align*}
\wedge : \Omega^k(M)\times \Omega^l(M,E) \to \Omega^{k+l}(M,E),\qquad
(\sigma,\omega) \mapsto \sigma\wedge \omega,
\end{align*}
wobei das Dachprodukt aus einer gewöhnlichen und einer bündelwertigen Form
definiert ist durch
\begin{align*}
&(\sigma\wedge \omega)_x(X_1,\ldots,X_{k+l}) \\ &\qquad \defl 
\frac{1}{k!l!} \sum_{\pi\in S_{k+l}} \sign(\sigma)
\sigma(X_{\pi(1)},\ldots,X_{\pi(k)})\omega(X_{\pi(k+1)},\ldots,X_{\pi(k+l)}),
\end{align*}
für $x\in M$ und $X_1,\ldots,X_{k+l}\in T_xM$. Für dieses Dachprodukt gelten die
üblichen Rechenregeln.\map
\end{remenum}
\end{rem}

\begin{lem}
Sei $A$ ein Zusammenhang auf $P$ und seien $\sigma\in\Omega^k(M)$ und
$\omega\in \Omega^*(M,E)$. Dann gilt
\begin{align*}
\ddd_A(\sigma\wedge \omega) = \ddd\sigma\wedge \omega + (-1)^k\sigma \wedge
\ddd_A\omega.
\end{align*}
\end{lem}
\begin{proof}
Seien $\sigma\in \Omega^k(M)$ und $\omega\in\Omega^l(M,E)$. Man überzeugt sich
anhand der expliziten Formel für das Dachprodukt und für den Isomorphismus
$\bar{\omega}\mapsto \omega$, dass $\bar{\sigma\wedge\omega} =
(\pi^*\sigma)\wedge\bar{\omega}$ und für diese gewöhnlichen Differentialformen gilt bei komponentenweiser Anwendung des Differentials,
\begin{align*}
\ddd((\pi^*\sigma)\wedge \bar{\omega}) = 
(\pi^* \ddd\sigma) \wedge\bar{\omega}
+(-1)^k (\pi^* \sigma) \wedge \ddd\bar{\omega}. 
\end{align*}
Seien nun $X_1,\ldots,X_{k+l}$ Vektorfelder auf $M$, dann gilt 
\begin{align*}
&\ddd_A (\sigma\wedge \omega)(X_1,\ldots,X_{k+l})\\
&\qquad= [p,\ddd \bar{\sigma\wedge \omega}(X_1^*,\ldots,X_{k+l}^*)]\\
&\qquad= [p,((\pi^*\dsg)\wedge \bar{\omega} +
(-1)^k(\pi^*\sigma)\wedge\ddd\bar{\omega})(X_1^*,\ldots,X_{k+l}^*)]\\
&\qquad=[p,(\bar{\dsg\wedge \omega} +
(-1)^k\bar{\sigma\wedge\ddd\omega})(X_1^*,\ldots,X_{k+l}^*)]\\
&\qquad=(\ddd\sigma\wedge \omega +
(-1)^k\sigma\wedge\dom)(X_1,\ldots,X_{k+l}).\qed
\end{align*}
\end{proof}

\begin{rem}
Für den Spezialfall, dass $\sigma\in\Omega^k(M)$ und $e\in\Gamma(E) =
\Omega^0(M,E)$ ist $\sigma\wedge\omega\in \Gamma(T^*M\times \ldots
\times T^*M \otimes E)$ und man schreibt auch
\begin{align*}
\ddd_A(\sigma\otimes e) = 
\dsg\otimes e + (-1)^k \sigma\wedge \nabla^A e.\map
\end{align*}
\end{rem}

\begin{defn}[Definition und Satz]
\index{kovariante Ableitung!induzierte}
Man nennt die Abbildung
\begin{align*}
\nabla^A \defl \ddd_A \colon \Gamma(E)=\Omega^0(M,E)\to
\Omega^1(M,E) = \Gamma(T^*M \otimes E),\quad e\mapsto \nabla^A e,
\end{align*}
die \emph{durch den Zusammenhang $A$ induzierte kovariante Ableitung} auf $E$.
Für ein Vektorfeld $X\in \chi(M)$ schreibt man auch kurz $\nabla^A_Xe \defl
(\nabla^A e)(X)$.\fish
\end{defn}
\begin{proof}
Zu zeigen ist, dass $\nabla^A$ tatsächlich eine kovariante Ableitung auf $E$
ist. Sei $e\in\Gamma(E)$, dann ist $\nabla^A e\in\Omega^1(M,E)$ ein 1-Form und
folglich $\nabla^A_X e = \ddd_A e$ in natürlicher Weise $\Cs^\infty(M)$ linear
in $X$. Sei nun $f\in\Cs^\infty(M)$, dann gilt
\begin{align*}
\nabla_X^A (fe) = \ddd_A (fe)(X)
= \ddd f(X)e + f\ddd_A e(X)
= X(f)e + f\nabla_X^A e.
\end{align*}
Also definiert $\nabla^A$ tatsächlich eine kovariante Ableitung auf $E$.\qed
\end{proof}

\begin{prop}
Seien $e\in\Gamma(E)$ und $X\in\chi(M)$. Dann existiert zu $x\in M$ eine
Umgebung $U$ von $x$, ein glatter Schnitt $s\colon U\to P$ und eine glatte
Funktion $v\in \Cs^\infty(U,V)$, so dass für jedes $x\in U$,
\begin{align*}
(\nabla_X^A e)_x = [s(x),\dv(X(x))+\rho_*(A^s(X(x)))v(x)]].\fish
\end{align*}
\end{prop}
\begin{proof}
Sei $x\in M$, dann existiert eine Umgebung $U\subset M$ von $x$, so dass
\begin{align*}
e\big|_U = [s,v],
\end{align*}
mit einem glatten Schnitt $s\colon U\to P$ und einer glatten Funktion
$v\in\Cs^\infty(U,V)$. Andererseits existiert auch eine glatte Funktion
$\bar{e}\in \Cs^\infty(P,V)^{(G,\rho)}$, so dass
\begin{align*}
e(x) = [p,\bar{e}(p)],
\end{align*}
für jedes $p\in P_x$. Folglich ist $v(x) = \bar{e}(s(x))$ und man rechnet nach,
dass
\begin{align*}
(\nabla_X^A e)_x &= (\ddd_A e)_x(X_x)
= [s(x),(D_A \bar{e})_x(\ds(X_x))]\\
&= [s(x),\ddd \bar{e}_x(\ds(X_x)) + \rho_*(A_x(\ds(X_x)))\bar{e}(s(x))]\\
&= [s(x),\ddd v(X_x) + \rho_*(A^s(X_x))v(x)].\qed
\end{align*}
\end{proof}

\begin{rem}
Sei $\lin{\cdot,\cdot}_V$ ein $G$-invariantes Skalarprodukt auf $V$, dann ist
\begin{align*}
\lin{e,\tilde{e}}_E\defl  \lin{v,\tilde{v}}_V,
\end{align*}
für Schnitte $e = [p,v]$, und $\tilde{e}=[p,\tilde{v}]\in\Gamma(E)$
eine Bündelmetrik auf $E$. Sei $U\subset M$ offen und $s\colon U\to P$ ein lokaler
Schnitt in $P$, dann schreibt sich
\begin{align*}
\nabla_X^A e = [s(x),\ddd v_x(X_x) + \rho_*(A^s(X_x))v(x)],
\end{align*}
so dass man direkt verifiziert
\begin{align*}
&\lin{\nabla_X^A e,\tilde{e}}_E + \lin{e,\nabla_X^A \tilde{e}}_E \\ 
&\qquad = 
\lin{\ddd v(X) + \rho_*(A^s(X))v,\tilde{v}}_V + \lin{v,\ddd\tilde{v}(X) +
\rho_*(A^s(X))\tilde{v}}_V\\
&\qquad=
\lin{\ddd v(X),\tilde{v}}_V + 
\lin{v,\ddd \tilde{v}(X)}_V +
\lin{\rho_*(A^s(X))v,\tilde{v}}_V + \lin{v,\rho_*(A^s(X))\tilde{v}}_V\\
&\qquad=
X(\lin{v,\tilde{v}}) + \lin{\rho_*(A^s(X))v,\tilde{v}}_V +
\lin{v,\rho_*(A^s(X))\tilde{v}}_V.
\end{align*}
Nun ist $A^s(X_x)\in \g$, wobei allgemein für $a\in\g$ gilt
\begin{align*}
\lin{\ddd\rho(a)v,\tilde{v}}_V &= 
\frac{\ddd}{\dt}\bigg|_{t=0} \lin{\rho(\exp(ta))v,\tilde{v}}_V
=
\frac{\ddd}{\dt}\bigg|_{t=0} \lin{v,\rho(\exp(-ta))\tilde{v}}_V\\
&=
-\lin{v,\ddd\rho(a)\tilde{v}}_V,
\end{align*}
denn $\lin{\cdot,\cdot}_V$ ist $G$-invariant. Somit gilt
\begin{align*}
\lin{\nabla_X^A e,\tilde{e}}_E + \lin{e,\nabla_X^A \tilde{e}}_E = 
X(\lin{v,\tilde{v}}_V) = 
X(\lin{e,\tilde{e}}_E),
\end{align*}
d.h. die Bündelmetrik ist parallel bezüglich $\nabla^A$.\map
\end{rem}

\begin{lem}
Sei $A$ ein Zusammenhang auf $P$, $e\in \Gamma(E)$ ein Schnitt in $E$ und $X$
ein Vektorfeld in $M$ mit horizontalem Lift $X^*$. Sei $x\in X$, so gilt für
jedes $p\in P_x$
\begin{align*}
(\nabla_X^A e)_x = [p](X^*(\bar{e})_p).\fish
\end{align*} 
\end{lem}
\begin{proof}
Man rechnet ohne Umwege nach, dass
\begin{align*}
(\nabla_X^A e)_x = \ddd_A e_x(X_x) = 
[p,D_A \bar{e}(X^*)_p]
=
[p,\ddd \bar{e}(X^*)_p]
= [p,X^*(\bar{e})_p].\qed
\end{align*}
\end{proof}

\section{Die Krümmung eines Zusammenhangs}
\label{sec:Krümmung-eines-Zusammenhangs}

\begin{defn}
\index{Krümmung!eines Zusammenhangs}
Sei $A$ ein Zusammenhang auf $P$, dann ist
\begin{align*}
F^A \defl D_A A \in\Omega^2(P,\g)
\end{align*}
die \emph{Krümmung} von $A$.\fish
\end{defn}

\begin{rem}
Es gilt sogar $F^A \in \Omega_\hor^2(P,\g)^{(G,\Ad)} \cong \Omega^2(M,\Ad(P))$.
Denn das Bild von $D_A$ sind horizontale Formen und $A$ ist vom Typ $\Ad$, wobei
$D_A$ den Typ erhält. Also ist $F^A = D_A A$ horizontal und vom Typ $\Ad$, also
$F^A\in\Omega^2(P,\g)^{(G,\Ad)}$ und $\Ad(P) = P\times_\Ad \g$, also ist
$\Omega_\hor^2(P,\g)^{(G,\Ad)} \cong \Omega^2(M,\Ad(P))$.\map 
\end{rem}

\begin{defn}[Definition und Satz]
\index{Kommutator!$\g$-wertiger Formen}
Sei $N$ eine Mannigfaltigkeit auf der eine Lie-Gruppe $G$ mit Lie-Algebra $\g$
wirkt. Weiter sei $(a_1,\ldots,a_r)$
eine Basis von $\g$ und gegeben seien zwei $\g$-wertige Formen $\omega\in
\Omega^k(N,\g)$ und $\tau\in\Omega^l(N,\g)$ mit
\begin{align*}
\omega = \sum_{i=1}^r \omega^i a_i,\qquad \tau = \sum_{i=1}^r \tau^i a_i,
\end{align*}
wobei $\omega_i\in\Omega^k(N)$ und $\tau^i\in\Omega^l(N)$. Dann ist der
\emph{Kommutator} von $\omega$ und $\tau$ definiert durch
\begin{align*}
[\omega,\tau]\defl \sum_{i,j=1}^r \omega^i \wedge \tau^j [a_i,a_j]
\in\Omega^{k+l}(N,\g),
\end{align*}
d.h. $[\cdot,\cdot]\colon \Omega^k(N,\g)\times\Omega^l(N,\g) \to
\Omega^{k+l}(N,\g)$.\fish
\end{defn}
\begin{proof}
Zu zeigen ist, dass die Definition von $[\cdot,\cdot]$ unabhängig von der Wahl
der Basis ist.\qed
\end{proof}

\begin{lem}
Seien $\omega\in\Omega^k(N,\g)$ und $\tau\in\Omega^l(N,\g)$. Dann gelten:
\begin{propenum}
\item $[\omega,\tau] = (-1)^{kl+1} [\tau,\omega]$.
\item $\ddd[\omega,\tau] = [\dom,\tau] + (-1)^k [\omega,\dtau]$.
\item Für $\omega \in\Omega^1(N,\g)$ und $X,Y\in\chi(N)$ gilt
\begin{align*}
[\omega,\omega](X,Y) = 2[\omega(X),\omega(Y)].\fish
\end{align*}
\end{propenum}
\end{lem}
\begin{proof}
Sei also $(a_1,\ldots,a_r)$ eine Basis von $\g$ und
$\omega\in\Omega^k(N,\g)$ und $\tau\in\Omega^l(N,\g)$  bezüglich dieser Basis
durch gewöhnliche Differentialformen $\omega^i\in\Omega^k(N)$ bzw.
$\tau^i\in\Omega^l(N)$ dargestellt.

a): Es gilt, dass $\omega^i\wedge \tau^j = (-1)^{kl}\tau^j\wedge\omega^i$ sowie
$[a_i,a_j] = -[a_j,a_i]$, also
\begin{align*}
[\omega,\tau] &=  (-1)^{kl+1} \sum_{i,j=1}^r \tau^j \wedge \omega^i [a_j,a_i]
= (-1)^{kl+1}[\tau,\omega].
\end{align*}

b): Weiterhin gilt
$\ddd(\omega^i\wedge \tau^j) = \dom^i \wedge \tau^j + (-1)^k \omega^i
\wedge \ddd\tau^j$ und folglich,
\begin{align*}
\ddd[\omega,\tau] &= \sum_{i,j=1}^r \ddd(\omega^i \wedge \tau^j) [a_i,a_j]
= \sum_{i,j=1}^r \left(\dom^i \wedge \tau^j + (-1)^k \omega^i
\wedge \ddd\tau^j\right) [a_i,a_j]\\
&= [\dom,\tau] + (-1)^k [\omega,\ddd\tau].
\end{align*}

c): Seien $X,Y\in\chi(N)$, dann verifiziert man direkt, dass
\begin{align*}
[\omega,\omega](X,Y) 
&=
\sum_{i,j} \left(\omega^i(X)\omega^j(Y)-\omega^j(X)\omega^i(Y)\right)
[a_i,a_j]\\
&= 
\left[\sum_i \omega^i(X) a_i, \sum_j \omega^j(X) a_j \right]
-
\left[\sum_i \omega^i(Y) a_i, \sum_j \omega^j(X) a_j \right]\\
&= [\omega(X),\omega(Y)]-[\omega(Y),\omega(X)]
= 2[\omega(X),\omega(Y)].\qed
\end{align*}
\end{proof}

\begin{prop}[Strukturgleichung]
\index{Strukturgleichung}
Sei $A$ ein Zusammenhang auf $P$, dann gilt
\begin{align*}
F^A = \ddd A + \frac{1}{2}[A,A].\fish
\end{align*}
\end{prop}

\begin{rem}[Bemerkungen.]
\begin{remenum}
\item Ist die Lie-Algebra $\g$ von $G$ abelsch, dann ist $[A,A] = 0$ und die
Strukturgleichung nimmt eine einfachere Form an,
\begin{align*}
F^A = \ddd A.
\end{align*}
Insbesondere ist $F^A$ dann exakt. In der Physik spielt die $\U(1)$-Theorie eine
fundamentale Rolle. Hier ist $G=\U(1)$ also $P$ ein $\U(1)$-Hauptfaserbündel
und die Lie-Algebra zu $G$ ist $\g = \u(1)$ und damit abelsch.
\item Sei $G$ eine Matritzengruppe, d.h. $G$ ist eine Untergruppe der $\GL_n$.
Dann lässt sich die Zusammenhangsform als Matrix $A=(A_{ij})$ auffassen mit
gewöhnlichen 1-Formen $A_{ij}\in\Omega^1(P)$ als Komponenten. Seien
$X,Y\in\chi(P)$, dann schreibt sich der Kommutator als
\begin{align*}
\frac{1}{2}[A,A](X,Y) &= 
[A(X),A(Y)]_\g = A(X)A(Y)-A(Y)A(X)\\
&= \left( \sum_k A_{ij}(X)A_{kj}(Y) - A_{ik}(Y)A_{ki}(X) \right)_{ij}\\
&= \left( \sum_k A_{ik}\wedge A_{kj}(X,Y) \right)_{ij} \defr A\wedge A(X,Y).
\end{align*}
Dadurch hat die Strukturgleichung die Form
\begin{align*}
F^A = \ddd A + A\wedge A.
\end{align*}
In den allermeisten Anwendungen ist $G$ eine Matrizengruppe und dieser
Zusammenhang gilt.\map
\end{remenum}
\end{rem}

\begin{proof}[Beweis der Strukturgleichung.]
Aufgrund der Multilinearität genügt es wieder die Formel für Tangentialvektoren
zu zeigen, die horizontal bzw. vertikal sind.

a): Seien $X,Y$ horizontal, dann gilt $A(X) = A(Y) = 0$,
\begin{align*}
F^A(X,Y) = \ddd A(\pr_h(X),\pr_h(Y)) = \ddd
A(X,Y)
\end{align*}
und $[A,A](X,Y) = 2[A(X),A(Y)] =  0$.

b): Sei $X$ horizontal und $Y$ vertikal. Dann ist
\begin{align*}
F^A(X,Y) = \ddd A(\pr_h(X),\pr_h(Y)) = 0,
\end{align*}
und ebenso $[A,A](X,Y) = 2[A(X),A(Y)] = 0$.
Sei $V^*$ ein horizontaler Lift von $X$ und $\tilde{Z}_p = Y$ für ein $p\in P$
und $Z\in\g$. Dann ist $A(X) = 0$ und $A(Y) = Z$ und es gilt
\begin{align*}
\ddd A(X,Y) = V^*(A(\tilde{Z})) - \tilde{Z}(A(V^*)) - A([\tilde{Z},V^*])
= V^*(Z) = 0, 
\end{align*}
denn $Z$ ist ein konstanter Vektor, also verschwindet die Lie-Ableitung nach
$V^*$.

c): Seien $X$ und $Y$ vertikal, dann ist $F^A(X,Y) = 0$. Seien $\tilde{V}_p = X$
und $\tilde{W}_p = Y$ für ein $p\in P$ und $X,W\in\g$. Dann ist
\begin{align*}
\frac{1}{2}[A,A](X,Y) = [A(X),A(Y)] = [V,W].
\end{align*}
Andererseits ist
\begin{align*}
\ddd A(X,Y) &= \tilde{V}(A(\tilde{W})) - \tilde{W}(A(\tilde{V})) -
A([\tilde{V},\tilde{W}])\\
&= \tilde{V}(W) - \tilde{W}(V)
- A(\widetilde{[V,W]})
= -[V,W].\qed 
\end{align*}
\end{proof}



\begin{prop}[Bianchi Identität]
\index{Bianchi Identität}
Sei $A$ ein Zusammenhang auf $P$, dann gilt
\begin{align*}
D_A F^A = 0.\fish
\end{align*}
\end{prop}

\begin{proof}
Aus der Strukturgleichung erhalten wir für das Differential der Krümmung,
\begin{align*}
\ddd F^A = \ddd\left(\ddd A + \frac{1}{2}[A,A]\right)
= \frac{1}{2}\left([\ddd A,A] - [A,\ddd A] \right)
= [\ddd A,A].
\end{align*}
Somit gilt für das absolute Differential,
\begin{align*}
D_A F^A = \ddd F^A\circ\pr_h = [\ddd A,A]\circ\pr_h = 0, 
\end{align*}
denn $A$ verschwindet auf horizontalen Vektorfeldern.\qed
\end{proof}

\begin{rem}
Im Beweis der Bianchi Identität haben wir ganz allgemein gezeigt, dass
\begin{align*}
\ddd F^A = [\ddd A,A].
\end{align*}
Für Matrizengruppe geht dies über in
\begin{align*}
\ddd F^A = F^A\wedge A - A\wedge F^A.
\end{align*}
Um das einzusehen, betrachte
\begin{align*}
\ddd F^A = [\ddd A ,A] = [F^A - \frac{1}{2}[A,A],A]
= [F^A,A] - \frac{1}{2}[[A,A],A].
\end{align*}
Wähle nun eine Basis $(e_1,\ldots,e_r)$ von $\g$, dann ist 
\begin{align*}
A = \sum_{i=1}^r A^i e_i,\qquad A^i\in\Omega^1(M).
\end{align*}
und folglich schreibt sich der Kommutator als
\begin{align*}
&[[A,A],A] 
\qquad= \sum_{i,k,j=1}^r A^i\!\wedge\! A^j\! \wedge\! A^k [[e_i,e_j],e_k]\\
&\qquad= \frac{1}{3}\Biggl(\sum_{i,k,j=1}^r A^i\!\wedge\! A^j\! \wedge\! A^k
[[e_i,e_j],e_k]+ A^k\!\wedge\! A^i\! \wedge\! A^j [[e_k,e_i],e_j]\\
&\qquad\qquad\qquad\qquad+
A^j\!\wedge\! A^k\! \wedge\! A^i [[e_j,e_k],e_i]
\Biggr)\\ 
&\qquad= \frac{1}{3}\left(\sum_{i,k,j=1}^r A^i\!\wedge\! A^j\! \wedge\!
A^k\left( [[e_i,e_j],e_k]+
 [[e_k,e_i],e_j]+
 [[e_j,e_k],e_i]\right)
\right) \\ &\qquad= 0,
\end{align*}
aufgrund der Jacobi-Identität und der Invarianz von $A^i\wedge A^j\wedge
A^k$ unter zyklischer Vertauschung.\map
\end{rem}

\begin{prop}
Sei $A$ ein Zusammenhang auf $P$, dann gilt für alle
$\omega\in\Omega_\hor^k(P,V)^{(G,\rho)}$
\begin{align*}
D_A^2 \omega = \rho_*(F^A)\wedge \omega.\fish
\end{align*}
\end{prop}

Während das Quadrat des gewöhnlichen Differentials immer verschwindet $\ddd^2
= 0$, involviert das Quadrat des absoluten Differentials $D_A^2$ die Krümmung
des Zusammenhangs $A$.

\begin{proof}
Nach Satz \ref{prop:Absolutes-Differential} gilt
\begin{align*}
\ddd (D_A \omega) &=
\ddd(\dom + \rho_*(A)\wedge \omega)
= \ddd^2\omega + \ddd(\rho_*(A)\wedge \omega)\\
&= (\ddd\rho_*(A))\wedge \omega -1 \rho_*(A)\wedge \dom. 
\end{align*}
Nach Definition des absoluten Differentials ist $D_A^2 \omega = (\ddd(
D_A\omega))\circ\pr_h$, wobei der Zusammenhang  auf horizontalen
Formen verschwindet, d.h. $A\circ\pr_h = 0$. Folglich gilt
\begin{align*}
D_A^2\omega &= [(\ddd\rho_*(A))\wedge \omega]\circ\pr_h
= (\rho_*(\ddd A)\wedge \omega)\circ\pr_h\\
&= \rho_*(D_A A)\wedge \omega
= \rho_*(F^A)\wedge \omega,
\end{align*}
denn der Push-forward vertauscht mit dem gewöhnlichen Differential, $\omega$ ist
horizontal und $(\ddd A)\circ\pr_h = D_A A$.\qed
\end{proof}




\subsection{$\U(1)$-Zusammenhänge}

Die Eichfeldtheorien in der Physik lassen sich mit Hilfe von Hauptfaserbündeln
beschreiben. Die Krümmung des Hauptfaserbündels spielt dabei die Rolle der
wirkenden Kraft und die Strukturgruppe spiegelt die Symmetrie der Wechselwirkung
wieder. Die Symmetrie der elektromagnetischen Wechselwirkung wird von der
$\U(1)$ repräsentiert. Im Folgenden wollen wir $\U(1)$-Hauptfaserbündel
$\pi\colon P\to M$ betrachten. Hier vereinfachen sich einige Konzepte, denn
\begin{align*}
\U(1) \defl \setdef{z\in\C}{\abs{z}=1},\qquad \u(1) = \Lie(\U(1))= 
\ii\R,
\end{align*}
sind abelsch. Insbesondere sind die Konjugation und die adjungierte Darstellung trivial,
d.h. $\alpha_g = \id_{\U(1)}$ und $\Ad(g) = \id_{\u(1)}$ für jedes $g\in \U(1)$.

Wie bereits festgestellt, ist der Raum der Zusammenhänge $\Cs(P)$ über $P$ ein
affiner Raum über dem Vektorraum $\Omega^1_\hor(P,\ii\R)^{(\U(1),\Ad)}$. Gegeben
zwei Zusammenhangsformen $A_1,A_2\in\Cs(P)$ ist $\eta =
A_1-A_2\in\Omega^1_\hor(P,\ii\R)^{(\U(1),\Ad)}$ eine horizontale Form vom Typ
$\Ad$, d.h. für jedes $a\in\U(1)$ ist
\begin{align*}
R_a^*\eta = \Ad(a^{-1})\eta = \eta.
\end{align*}
Also ist $\eta$ horizontal und rechtsinvariant. Man definiert nun eine Form
$\hat{\eta}\in\Omega^1(M,i\R)$ durch,
\begin{align*}
\hat{\eta}_x(V_x) \defl \eta_p(V^*_p),
\end{align*}
für $x\in P_x$ und $V$ ein Vektorfeld in $M$ mit horizontalem Lift $V^*$.
Aufgrund der Rechtsinvarianz ist $\hat{\eta}$ wohldefiniert und es gilt
offenbar $\eta = \pi^*\hat{\eta}$. Folglich lässt sich die Differenz
zweier Zusammenhangsformen immer schreiben als $A_1-A_2 = \pi^* \hat{\eta}$, mit
$\hat{\eta}$ einer gewöhnlichen Differentialform mit Werten in $\ii\R$.

Sei nun $A$ ein Zusammenhang, dann ist die Krümmung $F^A$ eine horizontale
2-Form vom Typ $\Ad$ und daher nach obiger Rechnung ebenfalls rechtsinvariant,
also gilt auch
\begin{align*}
F^A = \pi^*\hat{F}^A,\qquad \hat{F}^A\in\Omega^2(M,\ii\R).
\end{align*}
Da $\u(1)$ abelsch ist, vereinfacht sich die Strukturgleichung zu
\begin{align*}
F^A = \ddd A + \frac{1}{2}[A,A] = \ddd A,
\end{align*}
also ist $F^A$ exakt. Im Gegensatz dazu ist
$\hat{F}^A$ allerdings im Allgemeinen \textit{nicht} exakt. Aber geschlossen, denn
\begin{align*}
\pi^*\ddd \hat{F}^A = \ddd\pi^* \hat{F}^A = \ddd F^A = 0,
\end{align*}
und folglich ist ebenfalls $\ddd \hat{F}^A = 0$, denn die Projektion
$\pi$ ist eine Submersion. Somit ist auch $\ii\hat{F}^A$ geschlossen und
außerdem reellwertig, also
\begin{align*}
[\ii \hat{F}^A]\in H_{dR}^2(M,\R).
\end{align*}

\begin{defn}[Definition und Satz]
\label{defn:Erste-Chern-Klasse}
\index{Chern-Klasse}
Sei $A\in\Cs(P)$ ein Zusammenhang auf $P$, dann definiert man die
\emph{erste reelle Chern-Klasse von $P$} als
\begin{align*}
c_1(P)\defl \left[\frac{-1}{2\pi i} \hat{F}^A \right] \in H^2_{dR}(M,\R),
\end{align*}
und diese Definition ist unabhängig vom gewählten Zusammenhang.\fish
\end{defn}

\begin{proof}
Sei $A'\in\Cs(P)$ ein weiterer Zusammenhang auf $P$, dann ist
$A-A'=\pi^*\hat{\eta}$ für ein $\eta\in\Omega^1(M,\ii\R)$. Somit erhalten wir
\begin{align*}
\pi^*(\hat F^{A'}-\hat F^A) = F^{A'}-F^A = \ddd A' - \ddd A = \ddd
\pi^*\hat{\eta} = \pi^*\ddd\hat{\eta},
\end{align*}
und folglich ist $\hat F^{A'}-\hat F^A$ exakt. Daher stimmen ihre entsprechenden de-Rham
Kohomologieklassen überein. Es gilt also
\begin{align*}
\left[\frac{-1}{2\pi i} \hat{F}^{A'}\right] = 
\left[\frac{-1}{2\pi i} \hat{F}^{A}\right],
\end{align*}
und dies war zu zeigen.\qed
\end{proof}

\begin{ex}
\textit{Die erste reelle Chern Klasse des Hopf-Bündels}. Wir betrachten nochmals das Hopf-Bündel $H=(S^3,\pi,\CP;\U(1))$. 
Dazu schreiben wir $S^3$ als Teilmenge von $\C^2$, dann ist die $\U(1)$-Wirkung beschrieben durch
\begin{align*}
S^3\times \U(1)\to S^3,\qquad ((w_1,w_2),z)\mapsto (w_1 \cdot z,w_2\cdot z).
\end{align*}
Eine Projektion auf den $\CP$ ist nun durch die homogenen Koordinaten gegeben
\begin{align*}
\pi\colon S^3\to \CP^1,\qquad (w_1,w_1)\mapsto [w_1:w_2].
\end{align*}
Die Basismannigfaltigkeit $\CP^1$ ist 2-dimensional geschlossen,
zusammenhängend und orientierbar, also gilt $H_{dR}^2(\CP^1,\R)\cong\R$. Der
Isomorphismus kann explizit angegeben werden,
\begin{align*}
H_{dR}^2(\CP^1,\R)\to\R,\qquad [\omega]\mapsto \int_{\CP^1}\omega.
\end{align*}
Wir wollen nun den Wert der ersten reellen Chern Klasse $c_1(H)$ unter diesem
Isomorphismus berechnen. Wir haben bereits explizit eine Zusammenhangsform für
das Hopfbündel berechnet
\begin{align*}
A : TS^3\to \ii\R,\qquad (w_1,w_2) \mapsto 
\frac{1}{2}\left(\bar{\omega_1}\dom_1 - \omega_1\bar{\dom_1} +
\bar{\omega_2}\dom_2 - \omega_2\bar{\dom_2} \right),
\end{align*}
mit Funktionen $\omega_i \colon S^3\to \C$, $(w_1,w_2)\mapsto w_i$.
Die Krümmung dieses Zusammenhangs ist nach der Strukturgleichung
\begin{align*}
F^A = \ddd A = -(\dom_1\wedge \bar{\dom_1} + \dom_2\wedge
\bar{\dom_2}),
\end{align*}
und der Wert der ersten reellen Chern Klasse ist gegeben durch
\begin{align*}
c_1(H) = -\frac{1}{2\pi i} \int_{\CP^1} \hat{F}^A.
\end{align*}
Um das Integral zu berechnen definiere auf
$U=\CP^1\setminus\setdef{[w_1:w_2]}{w_2=0}$ die Abbildung
\begin{align*}
\psi\colon U\to \C,\qquad [w_1:w_2]\mapsto \frac{w_1}{w_2}.
\end{align*}
Dann ist $\psi$ wohldefiniert und ein Diffeomorphismus. Weiterhin
ist
\begin{align*}
\CP^1\setminus U = \setdef{[w_1:0]}{w_1\in\C\setminus\setd{0}} = \setd{[1:0]}
\end{align*}
ein einziger Punkt. Setzt man für $z\in \C$
\begin{align*}
\tilde{F}_z \defl -\frac{\dz\wedge \ddd\bar{z}}{(1+\abs{z})^2},
\end{align*}
dann ist $\tilde{F}_z$ eine 2-Form auf $\C$. Auf $S^3$ gilt $\abs{w_1}^2 +
\abs{w_2}^2 = 1$ und folglich ist
\newcommand{\dbomega}{\ddd\bar{\omega}}
\begin{align*}
\pi^*\psi^*\tilde{F} &=
-\frac{\ddd\left(\frac{w_1}{w_2} \right)\wedge
\ddd\left(\frac{\bar{w_1}}{\bar{w_2}}\right)}
{\left(1+\abs{\frac{w_1}{w_2}}^2\right)^2}\\
&=
-w_2^{-2}\bar{w_2}^{-2}\frac{
\left(w_2\dom_1 - w_1\dom_2\right)
\wedge
\left(\bar{w}_2\dbomega_1 - \bar{w}_1
\dbomega_2\right)}
{\left(1+\abs{\frac{w_1}{w_2}}^2\right)^2}\\
&= -\left(w_2\dom_1 - w_1\dom_2\right)
\wedge
\left(\bar{w}_2\dbomega_1 - \bar{w}_1
\dbomega_2\right)\\
&= 
 -\biggl(\abs{w_2}^2 \dom_1\wedge \ddd\bar{\omega}_1
 +\abs{w_1}^2 \dom_2\wedge \ddd\bar{\omega}_2\biggr)\\
 &\quad\;
+ w_2\bar{w_1}\dom_1\wedge\dbomega_2
+ w_1\bar{w_2}\dom_2\wedge\dbomega_1\\
&= F_A  + 
w_1\bar{w_1}\dom_1\wedge\dbomega_1
+w_2\bar{w_2}\dom_2\wedge\dbomega_2 
\\
&\quad\;
+ w_2\bar{w_1}\dom_1\wedge\dbomega_2
+ w_1\bar{w_2}\dom_2\wedge\dbomega_1\\
&= F_A + 
\left(\bar{w}_1\dom_1 + \bar{w}_2\dom_2 \right)
\wedge
\left(w_1\dbomega_1 + w_2\dbomega_2\right)\\
&= F_A + 
\left(\bar{w}_1\dom_1 + \bar{w}_2\dom_2 \right)
\wedge
\bar{\left(\bar{w}_1\dom_1 + \bar{w}_2\dom_2 \right)}\\
&= F_A.
\end{align*}
Damit berechnet man schließlich
\begin{align*}
c_1(H) &= -\frac{1}{2\pi i} \int_{\CP^1} \hat{F}^A
 = -\frac{1}{2\pi i} \int_{U} \bar{F^A}\\
  &= -\frac{1}{2\pi i} \int_{U} \psi^* \tilde{F}
  = -\frac{1}{2\pi i} \int_{\C} \tilde{F}\\
  &= \frac{1}{2\pi i} \int_{\C} \frac{\dz\wedge \ddd\bar{z}}{(1+\abs{z})^2}
  = \frac{1}{2\pi} \int_{\R^2} \frac{-\dx\wedge \dy + \dy\wedge \dx 
  }{(1+x^2+y^2)^2}\\
  &= \frac{-1}{\pi} \int_{\R^2} \frac{\dx\wedge \dy  
  }{(1+x^2+y^2)^2} = 
  \frac{-1}{\pi} \int_{r=0}^\infty \int_{\ph=0}^{2\pi} \frac{r\,\dr\wedge\dph  
  }{(1+r^2)^2}\\
  &= -2 \int_{r=0}^\infty \frac{r\,\dr  
  }{(1+r^2)^2} =
  \frac{1}{1+r^2} \bigg|_{r=0}^\infty \\ 
  &= - 1.
\end{align*} 
Im Besonderen folgt, dass $\hat{F}^A$ nicht exakt ist.\boxc
\end{ex}


\section{Lokale Beschreibung der Krümmung und Transformationsverhalten}

Sei $\pi\colon P\to M$ ein $G$-Hauptfaserbündel und $A$ ein Zusammenhang auf $P$.
Weiterhin sei $\setd{U_i}$ eine offene Überdeckung von $M$. Zu einem lokalen
Schnitt $s_i \colon U_i\to P$ in $P$ existiert eine lokale Zusammenhangsform
\begin{align*}
A^{s_i} \defl s_i^* A \in\Omega^1(U,\g).
\end{align*}
Wir suchen nun nach einer lokalen Beschreibung der Krümmung und einem Weg aus
den lokalen Beschreibungen der Krümmung die globale Krümmung zurückzuerhalten.

\begin{defn}
\index{Krümmungsform!lokale}
Die \emph{lokale Krümmungsform von $A$} zum lokalen Schnitt $s_i \colon U_i\to P$ ist
definiert durch,
\begin{align*}
F^{s_i} \defl s_i^* F^A \in \Omega^2(U_i,\g).\fish
\end{align*}
\end{defn}

Sei nun $s_j \colon U_j\to P$ ein weiterer Schnitt, dann existiert auf dem Schnitt $U_i\cap U_j$ ein $G$-Kozyklus
$g_{ij}\colon U_i\cap U_j\to G$, so dass
\begin{align*}
s_j = s_i\cdot g_{ij},\qquad A^{s_j} = \Ad(g_{ij}^{-1})A^{s_i} + \mu_{ij}.
\end{align*}
Ein analoges Transformationsverhalten haben die lokalen Krümmungsformen.

\begin{lem}
\begin{propenum}
\item $F^{s_j} = \Ad(g_{ij}^{-1})F^{s_i}$ auf $U_i\cap U_j$.
\item Für Matrizengruppen gilt
\begin{align*}
F^{s_j} = g_{ij}^{-1}F^{s_i}g_{ij}.\fish
\end{align*}
\end{propenum}
\end{lem}
\begin{proof}
a): Es gilt $s_j = s_i\cdot g_{ij}$ und somit erhalten wir mit der Produktformel
\ref{prop:Produktformel}
\begin{align*}
\ds_j(X) = \dR_{g_{ij}}(\ds_i(X)) + \widetilde{\dL_{g_{ij}^{-1}}(\ddd
g_{ij}(X))},\qquad X\in\chi(M).
\end{align*}
Seien nun $X,Y$ Vektorfelder in $M$, dann gilt
\begin{align*}
F^{s_j}(X,Y) &= F^A(\ds_j(X),\ds_j(Y))
= F^A(\dR_{g_{ij}}(\ds_i(X)),\dR_{g_{ij}}(\ds_i(Y)))\\
&= (R_{g_{ij}}^* F^A)(\ds_i(X),\ds_i(X)) =
\Ad(g_{ij}^{-1})F^{s_i}(X,Y).\qed
\end{align*}
\end{proof}

\begin{rem}
Für Matrizengruppen übertragen sich die Darstellungen der globalen Krümmung auf
die lokalen Krümmungsformen,
\begin{align*}
&F^s = \ddd A^s + A^s\wedge A^s,\\
&\ddd F^s = F^s\wedge A^s - A^s \wedge F^s.\map
\end{align*}
\end{rem}

\begin{ex}
Sei $P$ ein $\U(1)$-Hauptfaserbündel, dann gilt aufgrund der Kommutativität
\begin{align*}
F^s = \ddd A^s,\qquad
F^{s_i} = g_{ji}^{-1}F^{s_j}g_{ji} = F^{s_j}.
\end{align*}
Somit stimmen alle lokalen Krümmungsformen auf den Durchschnitten überein. Das
System $\setd{F^i}$ definiert somit wieder eine globale Krümmungsform auf $M$
mit Werten in $i\R$.

Außerdem gilt aufgrund der Kommutativität, dass $\ddd F^s = 0$, weshalb die
Bianchi Identität in der Physik auch homogene Feldgleichung genannt wird. Sie
entspricht gerade dem ersten Teil der Maxwell Gleichungen.\boxc
\end{ex}


\section{Krümmung und Integrabilität}

\begin{prop}
\label{prop:Krümmung-v-Projektion}
Seien $X,Y$ horizontale Vektorfelder auf $P$. Dann gelten
\begin{propenum}
\item $F^A(X,Y) = -A([X,Y])$.
\item $\pr_v([X,Y]) = -\widetilde{F^A(X,Y)}$.\fish
\end{propenum}
\end{prop}
\begin{proof}
a): Da $X,Y$ horizontal sind, ist $[A,A](X,Y) = 0$ und daher
\begin{align*}
F^A(X,Y) = \ddd A(X,Y) = X(A(Y)) - Y(A(X)) - A([X,Y])
= -A([X,Y]),
\end{align*}
da die Zusammenhangsform auf horizontalen Vektorfeldern verschwindet.

b): Es gilt $\pr_v(X) = \widetilde{A(X)}$ und die Behauptung folgt mit a).\qed
\end{proof}

\begin{rem}[Erinnerung.]
\index{Distribution}
\index{Distribution!integrabel}
\index{Distribution!involutiv}
Sei $N$ eine Mannigfaltigkeit, dann ist eine Distribution ein Unterbündel
$D\subset TN$ des Tangentialbündels an $N$. Eine Distribution $D$ heißt
\emph{involutiv}, falls gilt
\begin{align*}
X,Y\in \Gamma(D) \Rightarrow [X,Y]\in\Gamma(D),
\end{align*} 
und sie heißt \emph{integrabel}, wenn in jedem Punkt $x\in M$ eine
Untermannigfaltigkeit $N_x$ von $N$ existiert, so dass
\begin{align*}
T_p N_x = D_p,\qquad p\in P.
\end{align*}
Die größte zusammenhängende Untermannigfaltigkeit mit dieser Eigenschaft heißt
dann \emph{maximale Integralmannigfaltigkeit} von $D$ durch $x$.

Der Satz von Frobenius besagt nun, dass eine Distribution genau dann integrabel
ist, wenn sie involutiv ist.\map
\end{rem}

\begin{prop}
\label{prop:Integrabilität-Krümmung}
\begin{propenum}
\item Das vertikale Tangentialbündel $T^vP\subset TP$ ist eine involutive
Distribution auf $P$.
\item Das horizontale Tangentialbündel $T^hP\subset TP$ ist integrabel genau
dann, wenn $F^A \equiv 0$ ist.\fish
\end{propenum}
\end{prop}
\begin{proof}
a): Es gilt $T^v_gP = \setdef{\tilde{X}(g)}{X\in\G}$ und 
$[\tilde{X},\tilde{Y}] = \widetilde{[X,Y]}$ für $X,Y\in\g$. Somit ist der
Kommutator vertikaler Vektorfelder wieder vertikal und daher $T^vP$ involutiv.

b): Seien $X,Y$ horizontale Vektorfelder auf $P$, dann ist ihr Kommutator genau
dann horizontal, wenn $\pr_v([X,Y]) = 0$. Nach dem vorangegangenen Satz ist dies
jedoch äquivalent dazu, dass $F^A(X,Y) = 0$ ist. Somit muss die Krümmung
auf horizontalen Vektorfeldern verschwinden und als horizontale Form
verschwindet sie auch auf den vertikalen Vektorfeldern. Also ist $T^hP$
involutiv genau dann, wenn $F^A\equiv 0$.\qed
\end{proof}

\begin{defn}
\index{Zusammenhang!flach}
Ein Zusammenhang $A$ auf $P$ heißt \emph{flach}, falls $F^A\equiv 0$.\fish
\end{defn}

\begin{ex}
Der kanonische flache Zusammenhang auf dem trivialen $G$"=Hauptfaserbündel
ist tatsächlich flach.\boxc
\end{ex}

\begin{prop}
Sei $A$ ein Zusammenhang auf $P$. Dann sind folgende Bedingungen äquivalent:
\begin{equivenum}
\item $A$ ist flach.
\item $T^hP\subset TP$ ist integrabel.
\item Es existiert eine offene Überdeckung $\setd{U_i}$ von $M$, so dass
$(P_{U_i},A)$ isomorph ist zum trivialen $G$-Hauptfaserbündel  über $U_i$ mit
dem kanonischen flachen Zusammenhang.\fish
\end{equivenum}
\end{prop}

\begin{figure}
\centering
\begin{pspicture}(0,-1.92)(4.54,1.92)

\psline(0.52,-1.58)(3.9,-1.58)
\psline{->}(4.32,1.44)(4.32,-1.12)
\psframe(3.9,1.54)(0.52,-1.1)
\psbezier(3.02,1.52)(3.64,0.9)(3.6,-0.26)(3.46,-1.08)
\psbezier[linecolor=purple](2.28,1.52)(3.0,0.92)(2.86,-0.26)(2.72,-1.08)
\psbezier(1.48,1.52)(2.3,0.98)(2.18,-0.26)(1.92,-1.08)
\psbezier(0.68,1.52)(1.6,0.94)(1.4,-0.38)(1.12,-1.08)

\psbezier[linecolor=darkblue](0.54,0.12)(1.4,0.5)(3.2,-0.18)(3.88,0.56)
\psbezier(3.88,-0.52)(3.12,0.14)(1.18,-0.86)(0.54,-0.3)
\psbezier(0.54,0.84)(1.58,1.16)(2.46,0.26)(3.9,1.42)

\rput(4.36,1.705){\color{darkgray}$P$}
\rput(4.34,-1.595){\color{darkgray}$M$}
\rput(2.73,-1.775){\color{darkgray}$x$}
\rput(2.29,1.745){\color{purple}$P_x$}
\rput(0.25,0.145){\color{darkblue}$N_p$}
\psdots(2.84,0.2)
\psdots(2.7,-1.58)
\rput(3.02,0.065){\color{darkgray}$p$}
\end{pspicture}
\caption{Der Totalraum $P$ zerfällt in Fasern $P_x$ und maximale
Integralmannigfaltigkeiten $N_p$.}
\end{figure}

\begin{prop}[Ergänzung]
Sei $M$ einfach zusammenhängend, dann ist $(P,\pi,M;G)$ genau dann trivial,
wenn $F^A \equiv 0$.\fish
\end{prop}

\section{Der klassische Krümmungstensor}

Sei $\pi\colon P\to M$ ein $G$-Hauptfaserbündel und $E= P\times_\rho V$ ein
Vektorbündel. Für eine kovariante Ableitung $\nabla$ auf $E$ kennen wir bereits
aus dem vergangenen Semester den klassischen Krümmungsbegriff, nämlich den
Krümmungstensor
\begin{align*}
R_{X,Y}^\nabla = \nabla_X\nabla_Y - \nabla_Y\nabla_X - \nabla_{[X,Y]} \;\in\;
\Omega^2(M,\End E).
\end{align*}
Wir wollen nun untersuchen, wie die 2-Form $R$ mit $F^A$ zusammenhängt.

\begin{prop}
Sei $A$ ein Zusammenhang auf $P$. Dann gilt für einen Schnitt $e\in \Gamma(E)$
und Vektorfelder $X,Y$, dass
\begin{align*}
R_{X,Y}^{\nabla^A}e = [p,\rho_*(F^A(X^*,Y^*))v],
\end{align*}
wobei $X^*$ und $Y^*$ horizontale Lifts von $X$ und $Y$ sind.\fish
\end{prop}
\begin{proof}
Es gilt $e=[p,\bar{e}]$ mit einer Funktion $\bar{e}\in
\Cs^\infty(P,V)^{(G,\rho)}$. Somit berechnet sich
\begin{align*}
R_{X,Y}^{\nabla^A}e &=
[p,X^*(Y^*(\bar{e}))-Y^*(X^*(\bar{e}))-[X,Y]^*(\bar{e})]\\ &= [p,[X^*,Y^*](\bar{e})-[X,Y]^*(\bar{e})]
= [p,\pr_v([X^*,Y^*](\bar{e}))]\\
&= [p,-\widetilde{F^A(X^*,Y^*)}(\bar{e})],
\end{align*}
nach Satz \ref{prop:Krümmung-v-Projektion}. Nun gilt nach Definition des fundamentalen Vektorfeldes und
der $G$-äquivarianz von $\bar{e}$,
\begin{align*}
\widetilde{F^A(X^*,Y^*)}(\bar{e}) &= 
\frac{\ddd}{\dt}\bigg|_{t=0}\bar{e}\left(p\cdot \exp(t\, F^A(X^*,Y^*))
\right)\\
&=\frac{\ddd}{\dt}\bigg|_{t=0}
\rho\left(\exp(-t\, F^A(X^*,Y^*)\right)
\bar{e}(p) )\\
&=-\ddd\rho(F^A(X^*,Y^*))(\bar{e}(p)). 
\end{align*}
Mit $\bar{e}(p) = v$ folgt somit die Behauptung.\qed
\end{proof}

Über den Faserisomorphismus $[p]$ werden somit $R_{X,Y}$ und
$\rho_*(F^A(X^*,Y*))$ identifiziert. 

\begin{lem}
Sei $M$ eine Riemannsche Mannigfaltigkeit und $\nabla$ der Levi-Civita
Zusammenhang auf $TM$. Sei $P=P_{\GL_n}$ das Rahmenbündel über $M$ und $A$ der
durch $\nabla$ induzierte Zusammenhang.
Dann entspricht die Bianchi-Identität der Krümmung
\begin{align*}
D_AF^A = 0
\end{align*}
gerade der 2. Bianchi-Identität des Krümmungstensors
\begin{align*}
(\nabla_X R)_{Y,Z} + 
(\nabla_Y R)_{Z,X}  +
(\nabla_Z R)_{X,Y} = 0,
\end{align*}
für Vektorfelder $X,Y,Z$ in $M$.\fish
\end{lem}
\begin{proof}
Seien also Vektorfelder $X,Y,Z$ in $M$ gegeben, dann gilt
\begin{align*}
&D_A F^A(X^*,Y^*,Z^*)\\
&\qquad= \ddd F^A(X^*,Y^*,Z^*)\\
&\qquad= X^* (F^A(Y^*,Z^*))
+ Y^* (F^A(Z^*,X^*))
+ Z^* (F^A(X^*,Y^*))\\
&\qquad\quad - F^A([X^*,Y^*],Z^*)
+ F^A([X^*,Z^*],Y^*)
- F^A([Y^*,Z^*],X^*)\\
&\qquad= X^* (F^A(Y^*,Z^*))
+ Y^* (F^A(Z^*,X^*))
+ Z^* (F^A(X^*,Y^*))\\
&\qquad\quad - F^A([X,Y]^*,Z^*)
+ F^A([X,Z]^*,Y^*)
- F^A([Y,Z]^*,X^*),\tag{*}
\end{align*}
denn es gilt $[X,Y]^*= \pr_h([X^*,Y^*])$ und  $F^A$ ist horizontal. Betrachten
wir $R_{Y,Z}$ als Schnitt in $\End TM$, dann gilt
\begin{align*}
\nabla_X (R_{YZ}) = [p]\, X^*(F^*(Y^*,Z^*)).
\end{align*} 
Wenden wir den Faserisomorphismus auf den Ausdruck (*) an, ergibt sich
\begin{align*}
&[p]\, D_A F^A(X^*,Y^*,Z^*) \\
&\qquad = 
\nabla_X (R_{Y,Z}) - R_{[X,Y],Z}
- \nabla_Y (R_{Z,X}) + R_{[X,Z],Y}
+ \nabla_Z (R_{X,Y}) - R_{[Y,Z],X}\\
&\qquad = 
\nabla_X (R_{Y,Z}) - R_{\nabla_X Y,Z} + R_{\nabla_Y X,Z}\\
&\qquad\quad - \nabla_Y (R_{Z,X}) + R_{\nabla_X Z,Y} - R_{\nabla_Z X,Y}\\
&\qquad \quad+ \nabla_Z (R_{X,Y}) - R_{\nabla_Y Z,X} + R_{\nabla_Z Y,X}.
\end{align*}
Weiterhin gilt für die kovariante Ableitung
\begin{align*}
(\nabla_X R)_{Y,Z} = \nabla_X (R_{Y,Z}) - R_{\nabla_X Y,Z} - R_{Y,\nabla_X Z},
\end{align*}
so dass wir schließlich erhalten
\begin{align*}
[p]\,D_A F^A(X^*,Y^*,Z^*) = 
(\nabla_X R)_{Y,Z} + (\nabla_Y R)_{Z,X} + (\nabla_Z R)_{X,Y}.
\end{align*}
Weil $[p]$ ein Isomorphismus ist, impliziert $D_A F^A \equiv = 0$, dass der
rechte Ausdruck für alle Vektorfelder $X,Y$ und $Z$ verschwindet.\qed
\end{proof}


\section{Torsion}

Sei $P=P_{\GL_n}$ das Rahmenbündel über einer Mannigfaltigkeit $M$. Das
Tangentialbündel über $M$
\begin{align*}
TM = P\times_\rho \R^n,\qquad \rho\colon \GL_n\to \Aut(\R^n) \cong \GL_n
\end{align*}
lässt sich als zu $P$ assoziiertes Vektorbündel auffassen, wobei die Darstellung
$\rho$ gerade der Identität entspricht. Eine 1-Form $\omega\in \Omega^1(M,TM)$
in $M$ mit Werten in $TM$ lässt sich mit einer horizontalen 1-Form
$\bar{\omega}\in \Omega_\hor^1(P,\R^n)^\id$ vom Typ $\Id$ identifizieren. Im
Folgenden wollen wir eine spezielle Form dieser Art untersuchen.

\begin{defn}
\index{1-Form!kanonische}
Die \emph{kanonische 1-Form $\theta\in\Omega^1(P,\R^n)$} ist definiert als
\begin{align*}
\theta_p(X) \defl [p]^{-1}(\dpi(X)),\qquad X\in T_pP,
\end{align*}
wobei $[p]\colon \R^n\to T_pP$ den Faserisomorphismus bezeichnet.\fish
\end{defn}

Die kanonische 1-Form entspricht im Wesentlichen der Identit\"at, modulo Identifikation.

\begin{lem}
\begin{propenum}
\item Die kanonische 1-Form ist horizontal und vom Typ $\Id$, d.h.
\begin{align*}
\theta\in \Omega^1_\hor(P,\R^n)^{\GL_n}\cong \Omega^1(M,TM) =
\Gamma(T^*M\otimes TM) = \Gamma(\End TM).
\end{align*}
\item Sei $X$ ein Vektorfeld in $M$ und $\bar{X}\in \Cs^\infty(P,\R^n)^{\GL_n}$
die zugehörige Funktion über $\R^n$, dann gilt
\begin{align*}
\theta(X^*) = \bar{X}.
\end{align*}
\item Für die zu $\theta$ gehörige 1-Form $\hat{\theta}\in\Omega^1(M,TM)$
in $M$ gilt $\hat{\theta} = \Id_{TM}$.\fish
\end{propenum}
\end{lem}
\begin{proof}
a): Da der vertikale Tangentialraum gerade als der Kern von $\dpi$ definiert
ist, gilt $\pr_h = \dpi$ und somit ist $\theta=[p]^{-1}\circ\dpi$ horizontal.
Weiterhin gilt für $g\in \GL_n$ und $X\in T_pP$,
\begin{align*}
(R_g^*\theta)_p(X) &= \theta_{pg}(\dR_g(X)) = [p\cdot g]^{-1}\dpi(\dR_g(X))\\
&= g^{-1}[p]^{-1}\dpi(X)
= g^{-1}\theta_p(X).
\end{align*}
Also ist $\theta$ auch vom Typ $\id$.

b): Sei nun $X$ ein Vektorfeld in $M$, dann gilt für $p\in P$
\begin{align*}
\theta(X^*_p) = [p]^{-1}\dpi(X^*) = [p]^{-1} X = \bar{X}.
\end{align*}

c): Sei $x\in M$ und $X$ ein Vektorfeld in $M$, dann gilt
\begin{align*}
\hat{\theta}_x(X) = [p,\theta(X^*_p)] = 
[p,\bar{X}_p] = X_x,
\end{align*} 
und folglich ist $\hat{\theta}=\Id_{TM}$.\qed
\end{proof}

\begin{defn}
\index{Torsionsform}
Die \emph{Torsionsform} des Zusammenhangs $A$ ist definiert durch
\begin{align*}
\Theta^A \defl D_A \theta.\fish
\end{align*}
\end{defn}

Für einen Zusammenhang $\nabla$ auf $TM$ kennen wir bereits die Torsion
\begin{align*}
T(X,Y) = \nabla_X Y - \nabla_Y X - [X,Y],\qquad X,Y\in \Gamma(TM).
\end{align*}
Durch obige Definition der Torsionsform $\Theta$ ist tatsächlich eine
Verallgemeinerung der Torsion $T$ auf $TM$ gegeben.

\begin{lem}
\begin{propenum}
\item Seien $X$ und $Y$ Vektorfelder in $P$, dann gilt
\begin{align*}
\Theta(X,Y) = \ddd\theta(X,Y) + A(X)\theta(Y) - \theta(Y)A(X).\fish
\end{align*}
\item Für Vektorfelder $X$ und $Y$ in $M$ und $m\in M$ gilt
\begin{align*}
T_x(X,Y) = [p](\Theta_p(X^*,Y^*)),\qquad p\in P_x.\fish
\end{align*}
\end{propenum}
\end{lem}
\begin{proof}
a): Dies folgt direkt aus der Formel für das absolute Differential
\begin{align*}
\Theta(X,Y) &= D_A \theta(X,Y) = 
\ddd \theta(X,Y) + \rho_*(A)\wedge \theta(X,Y)\\
&= \ddd \theta(X,Y) + A(X)\theta(Y) - A(Y)\theta(X), 
\end{align*}
denn die Darstellung $\rho$ entspricht der Identität.

b): Seien $X,Y\in\Gamma(TM)$ mit zugehörigen Funktionen $\bar{X},\bar{Y}\in
\Cs^\infty(P,\R^n)^{\GL_n}$. Die kovariante Ableitung lässt sich dann schreiben
als
\begin{align*}
(\nabla_X Y)_x = [p,\ddd \bar{Y}_p(X^*)] = [p,X^*(\bar{Y})_p]
= [p,X^*(\theta(Y))_p].
\end{align*}
Da $\theta$ horizontal ist, folgt $\theta([X^*,Y^*]) = \theta([X,Y]^*)$ und
somit ist
\begin{align*}
[p](\Theta_p(X^*,Y^*)) &= [p,\ddd\theta_p(X^*,Y^*)]\\
&= [p,X^*\theta(Y^*)-Y^*\theta(X^*)-\theta([X,Y]^*)]\\
&= (\nabla_X Y - \nabla_Y X - [X,Y])_x.\qed
\end{align*}
\end{proof}

Während das Quadrat des gewöhnlichen Differentials verschwindet $\ddd^2 = 0$,
involviert das Quadrat des absoluten Differentials die Krümmung, so dass
\begin{align*}
D_A \Theta = D_A^2 \theta = F^A\wedge \theta.
\end{align*}

\begin{lem}
Sei $M$ eine Riemannsche Mannigfaltigkeit und $\nabla$ der Levi-Civita
Zusammenhang auf $TM$. Sei $P=P_{\GL_n}$ das Rahmenbündel über $M$ und $A$ der
durch $\nabla$ induzierte Zusammenhang. Dann entspricht die Identität
\begin{align*}
D_A \Theta = F^A \wedge \theta
\end{align*}
gerade der 1. Bianchi-Identität für den Krümmungstensor $R$.\fish
\end{lem}
\begin{proof}
Der Levi-Civita Zusammenhang ist torsionsfrei. Somit verschwindet die
Torsionsform $\Theta \equiv 0$ und folglich ist auch $D_A \Theta \equiv 0$.
Seien also $X,Y,Z$ Vektorfelder in $M$. Dann gilt
\begin{align*}
&(F^A\wedge \theta)(X^*,Y^*,Z^*) \\
&\qquad=
F^A(X^*,Y^*)\theta(Z^*) + 
F^A(Y^*,Z^*)\theta(X^*) + 
F^A(Z^*,X^*)\theta(Y^*)\\
&\qquad =  
F^A(X^*,Y^*)\bar{Z} + 
F^A(Y^*,Z^*)\bar{X} + 
F^A(Z^*,X^*)\bar{Y}.
\end{align*}
Weiterhin gilt $[p]F^A(X^*,Y^*)\bar{Z} = R_{XY}Z$ und somit ist
\begin{align*}
0 =  [p](F^A\wedge \theta)(X^*,Y^*,Z^*) = 
R_{XY}Z
 + R_{YZ}X + R_{ZX}Y.
\end{align*}
Dies ist gerade die 2. Bianchi-Identität für $R$.\qed
\end{proof}


\chapter{Parallelverschiebung, Geodätische und Exponentialabbildung}

\section{Parallelverschiebung in Hauptfaserbündeln}

Sei $\pi\colon P\to M$ ein $G$-Hauptfaserbündel. Ein Zusammenhang auf $P$ ist eine Zerlegung des Tangentialbündels
\begin{align*}
TP = T^hP\oplus T^vP
\end{align*}
in rechtsinvariante Distributionen $T^hP$ und $T^vP$. Das vertikale Tangentialbündel ist eindeutig durch $T^vP=\ker \dpi$ bestimmt, während sein Komplement, das horizontale Tangentialbündel, nicht eindeutig festgelegt ist.

Äquivalent zur Wahl eines Komplements von $T^vP$ ist die Wahl einer Zusammenhangsform $A\in\Omega^1(P,\g)$ . Diese bestimmt das horizontale Tangentialbündel durch $T^hP=\ker A$.

Wir wollen nun ein weiteres äquivalentes Konzept untersuchen, nämlich die
Parallelverschiebung.

\begin{defn}
\index{horizontaler Lift!von Vektorfeldern}
Sei $X$ ein Vektorfeld auf $M$. Ein Vektorfeld $X^*$ auf $P$ heißt
\emph{horizontaler Lift von $X$}, falls
\begin{defnenum}
\item $\dpi(X^*) = X$, und
\item $X_p^* \in T_p^hP$ für alle $p\in P$.\fish
\end{defnenum}
\end{defn}

\begin{figure}[h]
\centering
\begin{pspicture}(0,-1.9)(4.87,1.9)

\rput(4.34,-1.555){\color{darkgray}$M$}

\rput(4.36,1.585){\color{darkgray}$P$}

\rput(4.7,0.105){\color{darkgray}$\pi$}
\psframe(3.84,1.56)(0.0,-1.04)
\psbezier[linecolor=purple](1.82,1.54)(2.38,0.94)(2.16,-0.46)(2.02,-1.02)

\rput(1.63,1.725){\color{purple}$P_x$}
\psline{->}(4.32,1.38)(4.32,-1.26)

\rput(1.99,0.125){\color{purple}$p$}
\psline[linecolor=darkyellow]{->}(1.98,1.32)(2.24,1.44)
\psline[linecolor=darkyellow]{->}(2.1,1.04)(2.38,1.12)
\psline[linecolor=darkyellow]{->}(2.16,0.78)(2.44,0.82)
\psline[linecolor=darkyellow]{->}(2.18,0.52)(2.46,0.54)
\psline[linecolor=darkyellow]{->}(2.22,0.24)(2.52,0.24)
\psline[linecolor=darkyellow]{->}(2.18,-0.04)(2.48,-0.04)
\psline[linecolor=darkyellow]{->}(2.16,-0.28)(2.46,-0.3)
\psline[linecolor=darkyellow]{->}(2.12,-0.56)(2.42,-0.6)
\psline[linecolor=darkyellow]{->}(2.06,-0.84)(2.34,-0.9)
\psdots[linecolor=darkblue](1.92,-1.58)
\psdots[linecolor=purple](2.22,0.24)

\rput(2.65,-1.355){\color{darkblue}$X$}

\rput(1.85,-1.755){\color{darkblue}$x$}

\rput(2.77,0.4){\color{darkyellow}$X^*_p$}
\psline[linecolor=darkblue]{->}(1.96,-1.58)(2.3,-1.56)
\psline[linecolor=darkblue]{->}(2.94,-1.56)(3.3,-1.64)
\psline[linecolor=darkblue]{->}(1.16,-1.58)(1.48,-1.64)
\psline[linecolor=darkblue]{->}(0.38,-1.58)(0.7,-1.54)

\psbezier(0.1,1.54)(1.08,0.82)(0.52,-0.58)(0.4,-1.02)
\psline[linecolor=darkyellow]{->}(0.34,1.32)(0.56,1.5)
\psline[linecolor=darkyellow]{->}(0.52,1.06)(0.76,1.18)
\psline[linecolor=darkyellow]{->}(0.62,0.8)(0.88,0.88)
\psline[linecolor=darkyellow]{->}(0.66,0.52)(0.94,0.54)
\psline[linecolor=darkyellow]{->}(0.66,0.24)(0.96,0.24)
\psline[linecolor=darkyellow]{->}(0.64,-0.04)(0.92,-0.1)
\psline[linecolor=darkyellow]{->}(0.6,-0.32)(0.88,-0.4)
\psline[linecolor=darkyellow]{->}(0.54,-0.52)(0.82,-0.62)
\psline[linecolor=darkyellow]{->}(0.48,-0.78)(0.74,-0.88)

\psbezier(0.92,1.56)(1.7,0.78)(1.38,-0.56)(1.22,-1.04)
\psline[linecolor=darkyellow]{->}(1.1,1.34)(1.34,1.5)
\psline[linecolor=darkyellow]{->}(1.24,1.08)(1.52,1.22)
\psline[linecolor=darkyellow]{->}(1.34,0.8)(1.62,0.88)
\psline[linecolor=darkyellow]{->}(1.4,0.52)(1.68,0.56)
\psline[linecolor=darkyellow]{->}(1.42,0.24)(1.72,0.24)
\psline[linecolor=darkyellow]{->}(1.42,-0.06)(1.7,-0.08)
\psline[linecolor=darkyellow]{->}(1.38,-0.32)(1.68,-0.36)
\psline[linecolor=darkyellow]{->}(1.34,-0.58)(1.62,-0.64)
\psline[linecolor=darkyellow]{->}(1.28,-0.82)(1.56,-0.88)

\psbezier(2.92,1.54)(3.4,0.98)(3.16,-0.5)(3.02,-1.02)
\psline[linecolor=darkyellow]{->}(3.06,1.32)(3.34,1.42)
\psline[linecolor=darkyellow]{->}(3.16,1.04)(3.44,1.1)
\psline[linecolor=darkyellow]{->}(3.18,0.78)(3.48,0.82)
\psline[linecolor=darkyellow]{->}(3.2,0.5)(3.48,0.5)
\psline[linecolor=darkyellow]{->}(3.2,0.24)(3.5,0.22)
\psline[linecolor=darkyellow]{->}(3.18,-0.04)(3.48,-0.08)
\psline[linecolor=darkyellow]{->}(3.16,-0.28)(3.44,-0.32)
\psline[linecolor=darkyellow]{->}(3.12,-0.56)(3.42,-0.62)
\psline[linecolor=darkyellow]{->}(3.06,-0.82)(3.36,-0.88)
\end{pspicture} 
% \begin{pspicture}(0,-1.9)(4.86,1.9)
% 
% 
% \rput(4.34,-1.555){\color{darkgray}$M$}
% 
% \rput(4.36,1.585){\color{darkgray}$P$}
% 
% \rput(4.69,0.105){\color{darkgray}$\pi$}
% 
% \psframe(3.84,1.56)(0.0,-1.04)
% \psbezier[linecolor=purple](1.64,1.54)(2.2,0.94)(2.02,-0.48)(1.94,-1.02)
% 
% 
% \rput(1.63,1.725){\color{purple}$P_x$}
% 
% \psline{->}(4.32,1.38)(4.32,-1.26)
% 
% \rput(1.8,0.125){\color{darkgray}$p$}
% 
% 
% \psline(0.02,-1.58)(3.82,-1.58)
% \psline[linecolor=darkyellow]{->}(1.8,1.32)(2.1,1.32)
% \psline[linecolor=darkyellow]{->}(1.92,1.04)(2.22,1.04)
% \psline[linecolor=darkyellow]{->}(1.98,0.78)(2.28,0.78)
% \psline[linecolor=darkyellow]{->}(2.02,0.52)(2.32,0.52)
% \psline[linecolor=darkyellow]{->}(2.04,0.24)(2.34,0.24)
% \psline[linecolor=darkyellow]{->}(2.04,-0.04)(2.34,-0.04)
% \psline[linecolor=darkyellow]{->}(2.02,-0.28)(2.32,-0.28)
% \psline[linecolor=darkyellow]{->}(2.0,-0.56)(2.3,-0.56)
% \psline[linecolor=darkyellow]{->}(1.96,-0.82)(2.26,-0.82)
% \psline[linecolor=darkblue]{->}(1.92,-1.58)(2.22,-1.58)
% 
% \psdots[linecolor=darkblue](1.92,-1.58)
% \psdots[linecolor=purple](2.04,0.24)
% 
% \rput(2.33,-1.395){\color{darkblue}$X$}
% 
% \rput(1.81,-1.755){$x$}
% \rput(2.73,0.2){\color{darkyellow}$X^*_p$}
% 
% \end{pspicture} 
\caption{Vektorfeld $X$ in $M$ mit horizontalem Lift $X^*$}
\end{figure}

Der horizontale Lift eines Vektorfeldes $X$ auf $M$ ist ein horizontales
Vektorfeld auf $P$, das $\pi$-verknüpft ist zu $X$. Der folgende Satz besagt nun,
dass solch ein Vektorfeld immer existiert und sogar eindeutig bestimmt ist.

\begin{prop}
\begin{propenum}
\item Zu jedem Vektorfeld $X$ auf $M$ gibt es einen eindeutig bestimmten
horizontalen Lift $X^*$.
\item Der horizontale Lift $X^*$ ist rechtsinvariant, d.h.
\begin{align*}
\dR_g(X_p^*) = X_{pg}^*,\qquad \text{für alle }p\in P\text{ und } g\in G.
\end{align*}
\item Sei $Z$ ein horizontales und rechtsinvariantes Vektorfeld auf  $P$. Dann
existiert ein Vektorfeld $X$ auf $M$, so dass $X^* = Z$.
\item Seien $X$ und $Y$ Vektorfelder auf $M$ und $f\in\Cs^\infty(M)$, dann gelten
\begin{align*}
&(X+Y)^* = X^* + Y^*,\\
&(fX)^*_p = f(\pi(p)) \,  X^*,\\
&[X,Y]^* = \pr_h([X^*,Y^*]).
\end{align*}
\item Sei $Z$ ein horizontales Vektorfeld auf $P$, und $\tilde{B}$ das
fundamentale Vektorfeld zu $B\in\g$. Dann ist der Kommutator horizontal, d.h.
\begin{align*}
[Z,\tilde{B}] \in \Gamma(T^hP).
\end{align*}
Ist $Z$ sogar ein horizontaler Lift, so verschwindet der Kommutator, d.h. für
jedes Vektorfeld $X$ auf $M$ ist
\begin{align*}
[X^*,\tilde{B}] = 0.\fish
\end{align*} 
\end{propenum}
\end{prop}

\begin{proof}
a): Durch die Wahl eines Zusammenhanges auf $P$ zerfällt das Tangentialbündel in
einen horizontalen und einen vertikalen Teil,
\begin{align*}
TP = T^hP \oplus T^vP.
\end{align*}
Die Projektion $\pi$ ist eine Submersion und, da $T^vP = \ker \dpi$ gilt, ist
die Einschränkung ihres Differentials
\begin{align*}
\dpi\colon T^hP \to TM,
\end{align*}
ein Isomorphismus. Man definiert nun punktweise den horizontalen Lift als
\begin{align*}
X^*_p \defl \left(\dpi\big|_{T^hP}\right)^{-1}(X_{\pi(p)}).
\end{align*}
Falls also ein horizontaler Lift zu $X$ existiert, ist er bereits durch
diese Formel eindeutig festgelegt. Um die Glattheit von $X^*$ zu zeigen,
betrachtet man zu einer offenen Umgebung $U\subset M$ die lokale
Trivialisierung
\begin{align*}
\phi\colon P_U\to U\times G.
\end{align*}
Da die lokale Trivialisierung ein Diffeomorphismus ist, ist ihr Differential ein
Isomorphismus und man definiert lokal für $p\in P_U$,
\begin{align*}
Y_p \defl (\dphi)^{-1}(X_{\pi(p)}\oplus 0).
\end{align*}
Dann ist $Y$ ein glattes Vektorfeld auf $P_U$ und $\dpi(Y) = X$, denn $\phi$ ist
in der ersten Komponente gerade die Projektion auf $M$. Aufgrund der bereits
gezeigten Eindeutigkeit gilt dann lokal $Y = X^*\big|_{P_U}$ und folglich ist
$X^*$ tatsächlich ein glattes Vektorfeld.

b): Der horizontale Tangentialraum ist eine rechtsinvariante Distribution, d.h.
für jedes $g\in G$ und $p\in P$ gilt
\begin{align*}
\dR_g (X^*_p) \in T_{pg}^h P.
\end{align*}
Weiterhin wirkt $G$ fasertreu, d.h. $\pi\circ R_g = \pi$. Damit gilt
\begin{align*}
\dpi( \dR_g (X_p^*)) = \dpi(X_p^*) = X_{\pi(p)} = X_{\pi(p\cdot g)} = \dpi(X^*_{p\cdot g}).
\end{align*}
Aus der Eindeutigkeit von $X^*$ folgt nun, dass $\ddd R_g(X^*_p) = X_{pg}^*$
und somit ist $X^*$ rechtsinvariant.

c): Sei $Z$ ein horizontales und rechtsinvariantes Vektorfeld auf $P$. Dann
definiert man punktweise für $x\in M$
\begin{align*}
X_x \defl \dpi(Z_p),
\end{align*}
wobei $p\in P_x$ beliebig ist. Für ein anderes $q\in P_x$ gilt $q = p\cdot g$
für ein $g\in G$ und somit aufgrund der Rechtsinvarianz,
\begin{align*}
\dpi(Z_q) = \dpi(\dR_g Z_p) = \dpi(Z_p) = X_x,
\end{align*}
also ist $X$ wohldefiniert und $Z$ ist ein horizontaler Lift von $X$. Aus der Eindeutigkeit des horizontalen Lifts
folgt schließlich $X^* = Z$.

d): Die ersten beiden Formeln folgen sofort aus der Eindeutigkeit des
horizontalen Lifts. Weiterhin sind $X$ und $X^*$ $\pi$-verknüpft und daher sind
auch die Kommutatoren $\pi$-verknüpft,
\begin{align*}
\dpi(\pr_h([X^*,Y^*])) = 
\dpi([X^*,Y^*]) = 
[\dpi(X^*),\dpi(Y^*)] = [X,Y].
\end{align*}
Somit ist $\pr_h([X^*,Y^*])$ ein horizontaler Lift von $[X,Y]$ und aufgrund der
Eindeutigkeit stimmt er mit $[X,Y]^*$ überein.

e): Sei $Z$ ein horizontales Vektorfeld in $P$ und $B\in\g$. Der Fluss des
fundamentalen Vektorfeldes $\tilde{B}$ ist gegeben durch
\begin{align*}
\ph_t : P\to P,\qquad p\mapsto p\cdot \exp(tB), 
\end{align*}
d.h. $\ph_t = R_{\exp(tB)}$. Somit gilt für den Kommutator
\begin{align*}
[\tilde{B},Z]_p  &= \frac{\ddd}{\dt}\bigg|_{t=0} \ph_t^*(Z_{\ph(t)})\\
&= \frac{\ddd}{\dt}\bigg|_{t=0} \dR_{\exp(-tB)}(Z_{p\exp(tB)})\\
&= \frac{\ddd}{\dt}\bigg|_{t=0} \gamma(t).
\end{align*}
Nach Voraussetzung ist das Vektorfeld $Z$ horizontal und der horizontale Tangentialraum ist
rechtsinvariant. Also ist $\gamma$ ist eine Kurve in $T^h_pP$ und daher ist
$[\tilde{B},Z]$ horizontal.

Ist $Z$ außerdem rechts\-invariant, dann vereinfacht sich $\gamma$ zu
\begin{align*}
\gamma(t) = \dR_{\exp(-tB)}(Z_{p\exp(tB)}) = Z_p,
\end{align*}
ist also konstant und folglich verschwindet der Kommutator.\qed
\end{proof}

Ein Vektorfeld auf $M$ lässt sich somit in eindeutiger Weise zu einem
horizontalen, rechtsinvarianten Vektorfeld auf $P$ liften. Als nächstes wollen
wir das Liften von Kurven in $M$ betrachten. 


\begin{rem}[Bemerkung zur Notation.]
Sofern nicht anders angegeben bezeichne $I=[a,b]$ ein nichtentartetes endliches
Intervall und $\gamma\colon I\to M$ eine Kurve in $M$.

Eine solche Kurve heißt \textit{glatt}, wenn sie stückweise stetig
differenzierbar ist. Für eine solche Kurve existiert dann ohnehin eine
Parametrisierung, so dass die Kurve auf dem ganzen Intervall $I$ stetig
differenzierbar ist. Im Folgenden seien alle Kurven als glatt vorausgesetzt.
\map
\end{rem}

\begin{defn}
\index{horizontaler Lift!von Kurven}
Sei $\gamma\colon I\to M$ eine Kurve in $M$. Eine Kurve $\gamma^* : I\to P$ heißt
\emph{horizontaler Lift von $\gamma$}, falls
\begin{defnenum}
\item $\pi(\gamma^*) = \gamma$, und
\item $\dot \gamma^*(t) \in T_{\gamma^*(t)}^h P$ für alle $t\in I$.\fish
\end{defnenum}
\end{defn}

\begin{figure}[H]
\centering
% \begin{pspicture}(0,-2.02)(4.86,2.02)
% 
% \psbezier[linecolor=darkblue](0.64,-1.38)(1.92,-1.08)(1.78,-1.94)(3.2,-1.32)
% 
% 
% \psframe(3.84,1.68)(0.0,-0.92)
% \psdots[linecolor=darkblue](1.9,-1.48)
% 
% 
% \psbezier[linecolor=purple](1.64,1.66)(2.2,1.06)(2.02,-0.36)(1.94,-0.9)
% 
% \psdots[linecolor=darkblue](2.04,0.36)
% \psbezier[linecolor=darkblue](0.64,0.5)(1.92,0.8)(1.78,-0.06)(3.2,0.56)
% 
% \psline{->}(4.32,1.5)(4.32,-1.14)
% 
% \psline[linecolor=darkyellow]{->}(1.08,0.56)(1.34,0.58)
% \psline[linecolor=darkyellow]{->}(2.5,0.34)(2.76,0.4)
% \psline[linecolor=darkyellow]{->}(2.04,0.36)(2.28,0.28)
% \psline[linecolor=darkyellow]{->}(1.52,0.52)(1.78,0.44)
% \psline[linecolor=darkyellow]{->}(2.94,0.46)(3.18,0.54)
% \psline[linecolor=darkyellow]{->}(0.66,0.5)(0.92,0.56)
% \psline[linecolor=darkyellow]{->}(1.12,-1.32)(1.38,-1.3)
% \psline[linecolor=darkyellow]{->}(1.48,-1.34)(1.74,-1.42)
% \psline[linecolor=darkyellow]{->}(1.9,-1.48)(2.14,-1.56)
% \psline[linecolor=darkyellow]{->}(0.66,-1.38)(0.92,-1.32)
% \psline[linecolor=darkyellow]{->}(2.5,-1.54)(2.76,-1.48)
% \psline[linecolor=darkyellow]{->}(2.96,-1.42)(3.2,-1.34)
% 
% 
% \rput(3.19,-1.65){\color{darkblue}$\gamma$}
% 
% \rput(4.34,-1.435){\color{darkgray}$M$}
% 
% \rput(4.36,1.705){\color{darkgray}$P$}
% 
% \rput(4.69,0.225){\color{darkgray}$\pi$}
% 
% \rput(1.89,-1.795){\color{darkblue}$\gamma(t_0)$}
% 
% \rput(1.63,1.845){\color{purple}$P_{\gamma(t_0)}$}
% 
% \rput(3.23,0.345){\color{darkblue}$\gamma^*$}
% 
% \rput(1.85,0.245){\color{darkblue}$u$}
% \end{pspicture} 
\begin{pspicture}(0,-2.02)(4.93,2.02)
\psbezier(0.7,-1.33)(1.98,-1.03)(2.04,-1.85)(3.44,-1.25)
\psframe(3.9,1.68)(0.06,-0.92)
\psdots[linecolor=darkblue](1.88,-1.37)
\psbezier[linecolor=purple](1.7,1.66)(2.26,1.06)(2.0,-0.4)(1.94,-0.9)
\psline[linecolor=darkblue]{->}(1.34,-1.27)(1.62,-1.27)
\psline[linecolor=darkblue]{->}(1.88,-1.37)(2.12,-1.45)
\psline[linecolor=darkblue]{->}(0.7,-1.33)(0.96,-1.27)
\psline[linecolor=darkblue]{->}(2.68,-1.47)(2.96,-1.45)
\psline[linecolor=darkblue]{->}(3.34,-1.29)(3.56,-1.19)

\rput(3.25,-1.65){\color{darkgray}$\gamma$}

\rput(4.4,-1.435){\color{darkgray}$M$}
\rput(4.42,1.705){\color{darkgray}$P$}
\rput(4.76,0.225){\color{darkgray}$\pi$}
\psline{->}(4.38,1.5)(4.38,-1.14)


\rput(1.95,-1.795){\color{darkblue}$\gamma(t_0)$}

\rput(1.7,1.845){\color{purple}$P_{\gamma(t_0)}$}
\psbezier(2.38,1.68)(3.2,0.98)(2.6,-0.26)(2.66,-0.9)
\psbezier(3.04,1.68)(3.8,1.04)(3.42,-0.36)(3.34,-0.9)
\psbezier(0.28,1.68)(0.72,1.0)(0.0,-0.26)(0.58,-0.9)
\psbezier(0.96,1.68)(1.56,0.88)(1.02,-0.24)(1.3,-0.9)

\psbezier(0.4,0.6)(1.58,0.28)(1.38,0.6)(2.04,0.6)(2.7,0.6)(1.94,-0.08)(3.46,-0.22)
\psline[linecolor=darkyellow]{->}(2.74,-0.06)(3.0,-0.16)
\psline[linecolor=darkyellow]{->}(2.04,0.6)(2.34,0.6)
\psline[linecolor=darkyellow]{->}(3.44,-0.22)(3.74,-0.26)
\psline[linecolor=darkyellow]{->}(0.4,0.6)(0.66,0.54)
\psline[linecolor=darkyellow]{->}(1.26,0.46)(1.52,0.46)
\psdots[linecolor=purple](2.04,0.6)

\rput(1.87,0.445){\color{purple}$u$}
\rput(3.21,0.105){\color{darkgray}$\gamma^*$}
\end{pspicture} 
\caption{Kurve $\gamma$ mit horizontalem Lift $\gamma^*$.}
\end{figure}


\begin{prop}
Sei $\gamma\colon I\to M$ eine Kurve in $M$, $t_0\in I$ und $u\in P_{\gamma(t_0)}$.
Dann existiert genau ein horizontaler Lift
\begin{align*}
\gamma_u^* : I\to P
\end{align*}
von $\gamma$ mit $\gamma_u^*(t_0) =u$.\fish
\end{prop}

Die Idee des Beweises beruht darin, das Problem auf eine gewöhnliche
Differentialgleichung zurückzuführen und anschließend einen allgemeinen
Existenz- und Eindeutigkeitssatz anzuwenden.

\begin{lem}[Allgemeiner \textsc{Ee}-Satz]
\index{\textsc{Ee}-Satz}
Sei $G$ eine Lie-Gruppe und $\g$ die zugehörige Lie-Algebra. Weiterhin sei $v\colon
I \to \g$ eine stetige Kurve und $0\in I$. Dann existiert eine eindeutig
bestimmte $C^1$-Kurve $g\colon I\to G$ mit
\begin{align*}
g(0) = e,\qquad \dot{g}(t) = \dR_{g(t)}(v(t)).\fish
\end{align*} 
\end{lem}
\begin{proof}
Wir beweisen nur den Fall, dass $I=[0,1]$ ist. Dazu betrachten wir das
Vektorfeld $Z$ auf $G\times I$ definiert durch
\begin{align*}
Z_{(g,s)} \defl \left(\dR_g(v(s)),\frac{\partial}{\partial s}\bigg|_s\right).
\end{align*}
Sei $z\in G\times I$, dann existiert für $\abs{t}$ hinreichend klein eine
Integralkurve $\phi_t$ von $Z$ mit
\begin{align*}
\phi_t(e,s) = (g(t),s+t),
\end{align*} 
wobei $g(0) = e$ und $\dot{g}(t) = \dR_{g}(v(t))$. Weiterhin ist
$\setd{e}\times[0,1]$ kompakt, also existiert ein $\ep > 0$, so dass
$\phi_t$ für alle Anfangswerte aus $\setd{e}\times [0,1]$ und für alle
$t$ mit $\abs{t} < \ep$ existiert. Zerlege nun $[0,1]$ so, dass
\begin{align*}
0 = t_0 < t_1 < \ldots < t_r = 1,\qquad \abs{t_k-t_{k-1}} \le \ep.
\end{align*}
Dann existiert $\phi_t = (g_1(t),t)$ auf $[0,t_1]$. Weiterhin existiert auf
$[0,t_2-t_1]$ die Lösung
\begin{align*}
\phi_t(e,t_1) = (g_2(t),t_1+t),
\end{align*}
wobei $\dot{g}_2(t) = \dR_{g_2(t)}(v(t+t_1))$ und $g_2(0) = e$. Wir setzen nun
\begin{align*}
g(t) = \begin{cases}
g_1(t), & 0\le t \le t_1,\\
g_2(t-t_1)g_1(t_1), & t_1< t\le t_2. 
\end{cases}
\end{align*}
Dann ist $g$ auf $[0,t_2]$ definiert und stückweise differenzierbar. Weiterhin
gilt für $t_1< t \le t_2$,
\begin{align*}
\dot{g}(t) &= \dR_{g_1(t_1)}\dot{g}_2(t-t_1)\\
&=\dR_{g_1(t_1)}\dR_{g_2(t-t_1)}v(t)\\
&=\dR_{g_2(t-t_1)g_1(t_1)}v(t)\\
&=\dR_{g(t)}v(t).
\end{align*}
Analog setzt man $g$ auf ganz $[0,1]$ fort.\qed
\end{proof}

\begin{figure}[h]
\centering
\begin{pspicture}(-0.2,-1.9)(5.0,1.9)
\psline{->}(0.14,-1.36)(0.16,1.44)
\psline{->}(0.14,-1.34)(4.88,-1.34)
\rput(4.9,-1.575){\color{darkgray}$I$}
\rput(0.13,1.705){\color{darkgray}$G$}
\psline(0.36,-1.22)(0.36,-1.48)
\psline(1.16,-1.22)(1.16,-1.48)
\psline(4.58,-1.22)(4.58,-1.48)
\psline(1.96,-1.22)(1.96,-1.48)

\rput(1.14,-1.695){\color{darkgray}$t_1$}
\rput(1.96,-1.695){\color{darkgray}$t_2$}
\rput(0.38,-1.675){\color{darkgray}$t_0$}
\rput(4.62,-1.695){\color{darkgray}$t_r$}
\psbezier[linecolor=darkblue](0.36,-0.78)(0.66,-0.22)(0.88,-0.84)(1.16,-0.1)
\psbezier(1.16,-0.78)(1.4,-0.5)(1.36,-1.64)(1.96,-0.9)
\psbezier[linestyle=dashed,dash=0.06cm
0.06cm](1.96,-0.78)(2.24,-0.6)(2.32,-1.14)(2.56,-0.52)
\psbezier(3.98,-0.68)(4.2,-0.86)(4.22,-1.16)(4.58,-0.88) \psbezier[linewidth=0.04,linestyle=dashed,dash=0.16cm 0.16cm](3.36,-0.78)(3.58,-0.58)(3.58,-1.1)(3.94,-1.14)
\psbezier[linecolor=darkblue](1.16,-0.1)(1.4,0.18)(1.36,-0.96)(1.96,-0.22)
\psbezier[linecolor=darkblue,linestyle=dashed,dash=0.06cm
0.06cm](1.96,-0.22)(2.24,-0.04)(2.32,-0.58)(2.56,0.04)
\psbezier[linecolor=darkblue](3.92,0.9)(4.14,0.72)(4.16,0.42)(4.52,0.7)
\psbezier[linecolor=darkblue,linestyle=dashed,dash=0.06cm
0.06cm](3.34,1.26)(3.56,1.46)(3.56,0.94)(3.92,0.9)

\rput(3.96,-1.735){\color{darkgray}$t_{r-1}$}
\psline(3.96,-1.24)(3.96,-1.5)

\rput(0.76,-0.675){\color{darkgray}$g_1$}
\rput(1.64,-0.775){\color{darkgray}$g_2$}
\rput(2.49,-0.995){\color{darkgray}$g_3$}
\rput(2.29,0.165){\color{darkblue}$g$}

\psline(0.05,-0.78)(0.25,-0.78)
\rput(-0.05,-0.755){\color{darkgray}$e$}
\end{pspicture} 
\caption{Zur Konstruktion der Kurvenstücke $g_i$ auf $[0,t_i-t_{i-1}]$ und dem
Verkleben zu $g$.}
\end{figure}


\begin{proof}[Beweis des Satzes.]
Wir beschränken uns auf den Fall, dass $I=[0,1]$ und $t_0 = 0$ ist. Da $I$
kompakt ist, ist auch die Spur von $\gamma$ kompakt. Sie liegt daher
in endlich vielen offenen Mengen $U_i$ über denen $P$ lokal trivial ist. Mit
Hilfe der lokalen Trivialisierungen
\begin{align*}
\phi_i : \pi^{-1}(U_i)\to U_i\times G,
\end{align*}
können wir die Kurve $\gamma$ lokal in $U_i$ zu einer Kurve auf $P$ liften und
anschließend mit der $G$-Wirkung die einzelnen Lifts zu einer differenzierbaren
Kurve
\begin{align*}
\delta\colon I\to P,\qquad \pi(\delta) = \gamma,
\end{align*}
zusammensetzen, so dass $\delta(0) = u$ ist.

Gesucht ist nun eine Kurve $g\colon I\to G$ so, dass
\begin{align*}
\gamma_u^*(t) = \delta(t)g(t)
\end{align*} 
horizontal ist. Die Kurve $\gamma_u^*$ ist genau dann horizontal, wenn die Spur
ihrer Ableitung ganz im Kern der Zusammenhangsform $A$ liegt. Mit der
Produktformel berechnen wir
\begin{align*}
A(\dot{\gamma}_u^*(t)) &= 
A(\dR_{g(t)}\dot{\delta}(t) +
\widetilde{\dL_{g(t)^{-1}}\dot{g}(t)}(\gamma_u^*(t))\\
&=  \Ad(g(t)^{-1})A(\dot{\delta}(t)) + \dL_{g(t)^{-1}}(\dot{g}(t))\\
&= \dL_{g(t)^{-1}}\left(\dR_{g(t)}A(\dot{\delta}(t)) + \dot{g}(t)\right).
\end{align*}
Da $\dL_{g(t)^{-1}}$ ein Isomorphismus ist, verschwindet
$A(\dot{\gamma}_u^*(t))$ genau dann, wenn
\begin{align*}
0 = \dR_{g(t)}A(\dot{\delta}(t)) + \dot{g}(t).
\end{align*}
Setzen wir $v : I\to \g$ mit $v(t) = -A(\dot{\delta}(t))$, dann ist also eine
Funktion $g\colon I\to G$ gesucht mit
\begin{align*}
g(0) = e,\qquad \dot{g}(t) = \dR_{g(t)}(v(t)).
\end{align*}
Diese existiert nach dem allgemeinen \textsc{Ee}-Satz.\qed
\end{proof}

\begin{defn}
\index{Parallelverschiebung!im Hauptfaserbündel}
Sei $\gamma\colon [a,b]\to M$ eine Kurve in $M$. Die Abbildung
\begin{align*}
P_\gamma^A : P_{\gamma(a)}\to P_{\gamma(b)},\qquad u\mapsto \gamma_u^*(b),
\end{align*}
heißt \emph{Parallelverschiebung bezüglich $A$ in $P$ entlang $\gamma$}. \fish
\end{defn}

Später können wir mit Hilfe dieser Definition ganz einfach eine
Parallelverschiebung in Vektorbündeln, zum Beispiel dem Tangentialbündel, definieren.

\begin{rem}
$P^A$ ist unabhängig von der Parametrisierung von $\gamma$.
\end{rem}

\begin{defn}
\index{Hintereinanderausführung!von Kurven}
\index{Inverse!von Kurven}
\begin{defnenum}
\item
Seien $\alpha\colon [a,b]\to M$ und $\beta\colon [c,d]\to M$ zwei Kurven in $M$ mit
$\alpha(b) = \beta(c)$. Die \emph{Hintereinanderausführung} $\beta\star\alpha$
von $\alpha$ und $\beta$ ist definiert durch
\begin{align*}
\beta\star\alpha(t) \defl 
\begin{cases}
\alpha(a + 2t(b-a)), & 0\le t \le 1/2,\\
\beta(c + (2t-1)(d-c)), & 1/2 < t \le 1.
\end{cases}
\end{align*}
\item Die zu $\alpha\colon [a,b]\to M$ \emph{inverse} Kurve ist definiert durch,
\begin{align*}
\alpha^- : [0,1]\to M,\qquad t\mapsto \alpha(b-t(b-a)).\fish
\end{align*} 
\end{defnenum}
\end{defn}

Die Hintereinanderausführung ist im Allgemeinen nur stückweise stetig
differenzierbar, aber nach unserer Definition dennoch eine glatte Kurve.

\begin{prop}
Seien $\alpha$ und $\beta$ Kurven in $M$. Dann gelten:
\begin{propenum}
\item $P_{\beta\star\alpha}^A = P_\beta^A \circ P_\alpha^A$.
\item $P_\alpha^A$ ist ein Diffeomorphismus und $(P_\alpha^A)^{-1} =
P_{\alpha^-}^A$.
\item $P_\alpha^A$ ist $G$-äquivariant, d.h. für jedes $g\in G$ gilt,
\begin{align*}
R_g \circ P_\alpha^A = P_\alpha^A \circ R_g.\fish
\end{align*}
\end{propenum}
\end{prop}

\begin{proof}[Beweisidee.]
Eigenschaft a) folgt sofort aus der Definition der Parallelverschiebung. 
Weiterhin ist der horizontale Lift einer Kurve eindeutig und $T^hP$ ist
rechtsinvariant, also folgt auch Eigenschaft c).
Für Eigenschaft b) genügt es die Glattheit der Abbildung
\begin{align*}
P_\gamma^A : \pi^{-1}(\gamma(a)) \to \pi^{-1}(\gamma(b))
\end{align*}
zu zeigen. Fixiere dazu $p_0 \in \pi^{-1}(\gamma(a))$ und betrachte den
Faserdiffomorphismus
\begin{align*}
[p_0]: G\to \pi^{-1}(\gamma(a)),\qquad g\mapsto p_0 \cdot g.
\end{align*}
Dann ist $p_1 \defl P_\gamma^A(p_0)\in P$ und aufgrund der $G$-Äquivarianz gilt
\begin{align*}
P_\gamma^A(p_0\cdot g) = P_\gamma^A(p_0)\cdot g,
\end{align*}
also gilt auch $P_\gamma^A\circ [p_0](g) = [p_1](g)$ und damit
\begin{align*}
P_\gamma^A = [p_1]\circ [p_0]^{-1}.
\end{align*}
Da die Faserdiffeomorphismen glatte Abbildungen sind, folgt sofort die Glattheit
der Parallelverschiebung.\qed
\end{proof}

\begin{ex}
\label{ex:Parallelverschiebung-S1-Flach}
\begin{exenum}
\item Wir betrachten die zweifache Überlagerung der $S^1$. Zu zwei Punkten $x,y$
in $S^1$ existieren je zwei Lifts $u,u'$ und $v,v'$ in der zweifachen
Überlagerung. Weiterhin gibt es in $S^1$ zwei mögliche Kurven, die $x$ und $y$ verbinden.
Die Kürzere benennen wir $\gamma$ und die Längere $\delta$. Offenbar gilt
\begin{align*}
&\gamma(0) = x, \qquad \gamma(1) = y,\\
&\delta(0) = x, \qquad \delta(1) = y.
\end{align*}
Weiterhin sind die Fasern der Überlagerung zweielementig, also diskret und
folglich hat die Überlagerung dieselbe Dimension wie die $S^1$. Somit sind auch
ihre Tangentialräume isomorph und aus Dimensionsgründen gilt $T^hP = TP$.
Insbesondere ist der gewöhnliche Lift einer Kurve automatisch horizontal. Liften
wir die Kurven $\gamma$ und $\delta$, zu $\gamma_u^*$ und
$\delta_u^*$, so gilt (siehe Abbildung \ref{abb:S1-2x-lifts})
\begin{align*}
&\gamma_u^*(0) = u, \qquad \gamma_u^*(1) = v',\\
&\delta_u^*(0) = u, \qquad \delta_u^*(1) = v.
\end{align*}
Also ist $\gamma_u^*(1) = v' \neq v = \delta_u^*(1)$. Somit hängt die
Parallelverschiebung in $S^1$ vom gewählten Weg ab!

\begin{figure}[h]
\label{abb:S1-2x-lifts}
\centering
\begin{pspicture}(0,-1.72)(3.24,1.7)
\psbezier(1.58,0.98)(0.64,0.94)(0.66,0.12)(1.28,0.14)(1.9,0.16)(3.16,1.68)(1.58,1.52)(0.0,1.36)(1.3,0.46)(1.94,0.46)(2.58,0.46)(2.52,1.02)(1.58,0.98)
\psellipse(1.59,-0.85)(0.79,0.47)

\rput(3.04,1.365){\color{darkgray}$P$}
\rput(3.04,-0.955){\color{darkgray}$S_1$}

\psline{->}(3.0,1.18)(3.0,-0.7)

\psdots(1.22,-1.24)
\psdots(2.14,-1.16)
\psdots(1.2,0.14)
\psdots(1.2,0.72)
\psdots(2.16,0.74)
\psdots(2.16,0.48)


\rput(2.17,-1.5){\color{darkgray}$y$}

\rput(2.17,0.225){\color{darkgray}$v$}

\rput(2.17,1.15){\color{darkgray}$v'$}

\rput(1.07,-1.5){\color{darkgray}$x$}

\rput(1.07,-0.02){\color{darkgray}$u$}

\rput(1.07,1.125){\color{darkgray}$u'$}

\end{pspicture}\qquad 
\begin{pspicture}(0,-1.72)(3.24,1.7)
\psbezier(1.58,0.98)(0.64,0.94)(0.66,0.12)(1.28,0.14)(1.9,0.16)(3.16,1.68)(1.58,1.52)(0.0,1.36)(1.3,0.46)(1.94,0.46)(2.58,0.46)(2.52,1.02)(1.58,0.98)
\psellipse(1.59,-0.85)(0.79,0.47)

\rput(3.04,1.365){\color{darkgray}$P$}
\rput(3.04,-0.955){\color{darkgray}$S_1$}

\psline{->}(3.0,1.18)(3.0,-0.7)

\psdots(1.22,-1.24)
\psdots(2.14,-1.16)
\psdots(1.2,0.14)
\psdots(1.2,0.72)
\psdots(2.16,0.74)
\psdots(2.16,0.48)


\rput(2.17,-1.5){\color{darkgray}$y$}

\rput(2.17,0.225){\color{darkgray}$v$}

\rput(2.17,1.15){\color{darkgray}$v'$}

\rput(1.07,-1.5){\color{darkgray}$x$}

\rput(1.07,-0.02){\color{darkgray}$u$}

\rput(1.07,1.125){\color{darkgray}$u'$}

\psbezier[linecolor=darkblue]{->}(1.22,-1.32)(1.5,-1.4)(1.88,-1.38)(2.16,-1.22)
\psbezier[linecolor=purple]{->}(1.2,-1.3)(0.52,-1.12)(0.6,-0.34)(1.58,-0.34)(2.56,-0.34)(2.54,-1.02)(2.24,-1.16)
\psbezier[linecolor=darkblue]{->}(1.2,0.06)(1.56,0.1)(1.8,0.24)(2.2,0.66)
\psbezier[linecolor=purple]{->}(1.14,0.08)(0.94,0.12)(0.66,0.38)(0.84,0.68)(1.02,0.98)(1.48,1.12)(1.98,1.02)(2.48,0.92)(2.6,0.62)(2.18,0.4)

\rput(0.55,0.685){\color{purple}$\delta_u^*$}
\rput(1.81,0.0){\color{darkblue}$\gamma_u^*$}
\rput(1.63,-1.55){\color{darkblue}$\gamma$}
\rput(0.66,-0.555){\color{purple}$\delta$}
\end{pspicture} 
\caption{Zweifache Überlagerung der $S^1$ (links), sowie mit den $x$ und $y$
verbinden Kurven samt Lifts (rechts)}
\end{figure}

% Vorlesung vom 22. Juni 2011

\item
\label{ex:Parallelverschiebung-unabhängig-kanonisch-flach}
Sei $\pi\colon P_0 = M\times G\to M$ das triviale $G$-Hauptfaserbündel versehen mit
dem kanonischen flachen Zusammenhang $A_0=\ddd \pr_2$. Dieser Zusammenhang ist
äquivalent dazu, für einen Punkt $(x,g)$ in $P$ den horizontalen Tangentialraum
als
\begin{align*}
T_{(x,g)}^hP \cong T_xM
\end{align*}
festzulegen. Sei nun $\gamma$ eine Kurve in $M$ mit Anfangspunkt $x$ und
Endpunkt $y$. Der horizontale Lift von $\gamma$ ist nun einfach gegeben durch
\begin{align*}
\gamma_{(x,g)}^*(t) = (\gamma(t),g).
\end{align*}
Die Parallelverschiebung entlang $\gamma$ bezüglich $A_0$ hat somit die Form
\begin{align*}
P_\gamma^{A_0} \colon (P_0)_x = \setd{x}\times G\to \setd{y}\times G = (P_0)_y,\qquad
(x,g) \mapsto (y,g).
\end{align*}
Sie hängt damit offenbar nicht von der Wahl des Weges $\gamma$ ab, der $x$ und
$y$ verbindet.\boxc
\end{exenum}
\end{ex}

Im Allgemeinen hängt die Parallelverschiebung von der Wahl des Weges ab. Im
Extremfall des trivialen Hauptfaserbündels versehen mit dem kanonischen flachen
Zusammenhang jedoch nicht. Wir werden sehen, dass die Unabhängigkeit der
Parallelverschiebung von der Wahl des Weges bemerkenswerte Konsequenzen hat.
Im Fall einer einfach zusammenhängenden Basismannigfaltigkeit ist die
Unabhängigkeit außerdem äquivalent dazu, dass die Krümmung des Zusammenhangs
verschwindet.

\begin{prop}
Sei die Parallelverschiebung auf $P$ bezüglich $A$ unabhängig von der Wahl des
Weges. Dann existiert ein $G$-Hauptfaserbündelisomorphismus
\begin{align*}
\phi\colon P_0\to P,\qquad \text{mit}\qquad \phi^* A = A_0,
\end{align*}
wobei $P_0$ das triviale $G$-Hauptfaserbündel versehen mit dem kanonischen
flachen Zusammenhang $A_0$ bezeichnet. Man schreibt dafür auch $(P,A)$ ist
\emph{isomorph} zu $(P_0,A_0)$.\fish
\end{prop}

Ist also die Parallelverschiebung unabhängig von der Wahl des Weges, dann ist
das Hauptfaserbündel trivial und der Zusammenhang ist ein Pull-back des
kanonischen flachen Zusammenhangs, also ebenfalls trivial. Wir haben bereits
gezeigt, dass ein Hauptfaserbündel $P$ genau dann trivial ist, wenn ein
globaler Schnitt
\begin{align*}
s: M\to P.
\end{align*}
existiert. Wir suchen nun nach Bedingungen an den Schnitt $s$, so dass der
Zusammenhang $A$ auf $P$ trivial wird.

\begin{lem}
Das $G$-Hauptfaserbündel $(P,A)$ ist isomorph zu $(P_0,A_0)$ genau dann, wenn
ein globaler, horizontaler Schnitt existiert, d.h. ein Schnitt
\begin{align*}
s: M\to P,\qquad \ds(TM) = T^hP.\fish
\end{align*}
\end{lem}
\begin{proof}
$\Rightarrow$: Sei $\phi\colon P_0=M\times G\to P$ ein Isomorphismus mit $\phi^* A =
A_0$. Dann gilt $\ker A = \dphi(\ker A_0)$ und somit gilt für jeden Punkt
$(x,g)$ in $P$,
\begin{align*}
\dphi(T_{(x,g)} (M\times \setd{g})) = T^h_{\phi(x,g)}P.
\end{align*}
Ein globaler Schnitt von $P$ ist gegeben durch
\begin{align*}
s: M\to P,\qquad s(x) = \phi(x,e).
\end{align*}
Weiterhin ist $T_xM \cong T_{(x,g)} (M\times \setd{g})$ für jedes $g\in G$ und
folglich
\begin{align*}
\ds(T_xM) = \dphi(T_{(x,e)} (M\times \setd{e})) = T^h_{\phi(x,e)}P =
T^h_{s(x)}P.
\end{align*}
Also ist der Schnitt $s$ auch horizontal.

$\Leftarrow$: Sei $s\colon M\to P$ ein $A$-horizontaler Schnitt, d.h. $\ds(TM) =
T^hP$. Als Trivialisierung von $P$ wählen wir
\begin{align*}
\phi\colon P_0\to P,\qquad (x,g) \mapsto s(x)\cdot g = R_g s(x).
\end{align*}
Dann ist $\phi$ ein Diffeomorphismus, denn $s$ ist ein globaler Schnitt und $G$
wirkt einfach transitiv und frei auf $P$. Weiterhin gilt
\begin{align*}
\dphi(T_{(x,g)}(M\times \setd{g})) &= \dR_g\circ \ds(T_xM) = 
\dR_g(T_{s(x)}^hP) \\ &= T_{s(x)g}^hP = T_{\phi(x,g)}P,
\end{align*}
denn $T^hP$ ist eine rechtsinvariante Distribution.\qed
\end{proof}

Um den Satz zu beweisen, genügt es also einen horizontalen, globalen Schnitt zu
konstruieren.

\begin{proof}[Beweis des Satzes.]
Sei die Parallelverschiebung unabhängig von der Wahl des Weges. Wir fixieren
$x_0\in M$ und $u_0\in P_{x_0}$, und definieren einen Schnitt durch
\begin{align*}
s: M\to P,\qquad x\mapsto s(x) \defl P_\gamma^A(u_0) = \gamma_{u_0}^*(1),
\end{align*}
wobei $\gamma\colon [0,1]\to M$ ein beliebiger Weg von $x_0$ nach $x$ sei. Aufgrund
der Unabhängigkeit der Parallelverschiebung von der Wahl des Weges ist $s$
wohldefiniert. Weiterhin hängt $\gamma_{u_0}^*(1)$ als Lösung einer gewöhnlichen
Differentialgleichung differenzierbar vom Anfangswert $x$ ab, also ist $s$
differenzierbar. Sei nun $\alpha\colon [0,1]\to M$ eine Kurve durch $x$ mit
$\dot{\alpha}(0) = X\in T_xM$, dann gilt
\begin{align*}
\ds(X) = \frac{\ddd}{\dt}\bigg|_{t=0} s\circ\alpha(t)
= \frac{\ddd}{\dt}\bigg|_{t=0} P_{\gamma\star\alpha(t)}^A(u_0)
= \frac{\ddd}{\dt}\bigg|_{t=0} (\gamma\star\alpha)^*_{u_0}(t) \in T^hP,
\end{align*}
wobei $\gamma$ eine Kurve von $x_0$ nach $x$ bezeichnet. Somit ist $s$ ein
horizontaler, globaler Schnitt in $P$ und nach dem vorangegangen Lemma ist daher
$P$ trivial mit trivialem Zusammenhang.\qed 
\end{proof}

Der Zusammenhang $A$ auf $P$ heißt flach, wenn seine Krümmung verschwindet, d.h.
$F^A \equiv 0$. Aus dem Satz von Frobenius folgt, dass dies genau dann der Fall
ist, wenn die horizontale Distribution $T^hP$ integrabel ist. Mit Hilfe des
vorangegangen Satzes können wir die Aussage wie folgt erweitern.

\begin{prop}
Sei $A$ ein Zusammenhang auf $P$. Dann sind folgende Aussagen äquivalent.
\begin{equivenum}
\item $A$ ist flach.
\item $T^hP \subset TP$ ist integrabel.
\item Es existiert eine offene Überdeckung $\setd{U_i}$ von $M$, so dass
$(P_{U_i},A)$ isomorph ist zum trivialen $G$-Hauptfaserbündel über $U_i$ mit dem
kanonischen flachen Zusammenhang.
\item Es existiert eine offene Überdeckung $\setd{U_i}$ von $M$, so dass die
Parallelverschiebung auf jedem $P_{U_i}$ unabhängig von der Wahl des Weges
ist.
\item Es existiert eine offene Überdeckung $\setd{U_i}$ von $M$ und lokale,
horizontale Schnitte
\begin{align*}
s_i : U_i \to P,\qquad \ds_i(TM) = T^hP.\fish
\end{align*}
\end{equivenum}
\end{prop}
\begin{proof}
(i)$\iff$(ii) war eine Konsequenz aus dem Satz von Frobenius, die wir bereits in
Satz \ref{prop:Integrabilität-Krümmung} gezeigt haben.

(iii)$\iff$(v) ist die Aussage des vorangegangen
Lemmas.


(iii)$\Rightarrow$(iv) haben wir in Beispiel \ref{ex:Parallelverschiebung-S1-Flach} \ref{ex:Parallelverschiebung-unabhängig-kanonisch-flach} gezeigt.

(iv)$\Rightarrow$(v): Sei $\setd{U_i}$ eine offene Überdeckung auf der die Parallelverschiebung unabhängig vom gewählten Weg ist. Fixiere eine $x\in U_i$ und ein $u\in P_{x}\subset P_{U_i}$. So ist
\begin{align*}
s: U_i \to P,\qquad y \mapsto \gamma_{xy}^*(u),
\end{align*}
für eine beliebige Kurve in $U_i$ von $x$ nach $y$ wohldefiniert und ein horizontaler Schnitt.

(ii)$\Rightarrow$(v): Sei also $T^hP$ integrabel, d.h. für
jeden Punkt $p\in P$ existiert eine maximale zusammenhängende
Integralmannigfaltigkeit $N_p$ so, dass für $q\in N_p$ gilt, $T_qN_p = T_q^hP$.
Somit entspricht der Tangentialraum an $N_p$ gerade dem horizontalen
Tangentialraum an $P$, welcher wiederum zu $TM$ isomorph ist. Also haben $N_p$
und $M$ dieselbe Dimension. Weiterhin ist die Projektion
\begin{align*}
\pi\colon P\to M
\end{align*}
eine Submersion und folglich ihre Einschränkung
\begin{align*}
\pi\big|_{N_p} \colon N_p\to M
\end{align*}
eine Überlagerung, d.h. für eine hinreichend kleine Umgebung $U\subset M$ von
$\pi(p)$ ist
\begin{align*}
\pi^{-1}(U)\cap N_p = \dot{\bigcup}\, N_i,
\end{align*}
die disjunkte Vereinigung offener Mengen $N_i$ so, dass
\begin{align*}
\pi\big|_{N_i} \colon N_i\to M
\end{align*}
ein Diffeomorphismus ist. Man kann somit eine Überdeckung von $M$ und  lokale
horizontale Schnitte
\begin{align*}
s : U\to N_i\subset P,\qquad s_i \defl (\pi\big|_{N_i})^{-1}
\end{align*}
definieren.

(v)$\Rightarrow$(i): Sei eine Überdeckung von $P$ mit horizontalen Schnitten gegeben.
Die Krümmung ist eine horizontale 2-Form und verschwindet somit bereits auf
vertikalen Vektoren. Wir müssen daher nur zeigen, dass sie auch auf horizontalen
Vektoren verschwindet. Wähle dazu $x\in M$ und $p\in P_x$. Dann existiert ein
horizontaler, lokaler Schnitt $s_i : U_i\to P$, so dass $x\in U_i$ und
\begin{align*}
\ds_i : T_x U_i\to T^hP_{U_i}
\end{align*}
ein Isomorphismus ist. Für horizontale Vektoren $X,Y\in T_p^hP$ existieren
dann Urbilder $V,W\in T_xM$ so, dass
\begin{align*}
\ds_i(V) = X,\qquad \ds_i(W) = Y.
\end{align*}
Nach dem vorangegangen Satz ist $(P,A)$ lokal isomorph zum trivialen Bündel mit
dem kanonischen flachen Zusammenhang, also ist
\begin{align*}
F^{s_i} = s_i^* F^A \equiv 0.
\end{align*}
Dann gilt aber auch
\begin{align*}
F^A(X,Y) = F^A(\ds_i(V),\ds_i(W)) = s_i^*F(V,W) = 0.
\end{align*}
Also verschwindet die Krümmung auch auf den horizontalen Vektorfeldern und ist
somit identisch Null.\qed
\end{proof}

% \begin{rem}[Bemerkungen.]
% \begin{remenum}
% \item Die Krümmung des Zusammenhangs $F^A$ verschwindet genau dann, wenn
% das Hauptfaserbündel mit lokalen, horizontalen Schnitten überdeckt werden kann,
% d.h.
% \begin{align*}
% F^A \equiv 0 \quad \iff\quad  \exists s_i :U_i \to P\; \text{horizontal}.\tag{*}
% \end{align*}
% Nach dem vorigen Satz ist dies dann äquivalent dazu, dass $P$ lokal isomorph
% ist zum trivialen Hauptfaserbündel mit dem kanonischen flachen Zusammenhang.
% 
% \begin{proof}
% $\Rightarrow$: Sei also $F^A \equiv 0$. Dann ist $T^hP$ integrabel, d.h. für
% jeden Punkt $p\in P$ existiert eine maximale zusammenhängende
% Integralmannigfaltigkeit $N_p$ so, dass für $q\in N_p$ gilt, $T_qN_p = T_q^hP$.
% Somit entspricht der Tangentialraum an $N_p$ gerade dem horizontalen
% Tangentialraum an $P$, welcher wiederum zu $TM$ isomorph ist. Also haben $N_p$
% und $M$ dieselbe Dimension. Weiterhin ist die Projektion
% \begin{align*}
% \pi\colon P\to M
% \end{align*}
% eine Submersion und folglich ihre Einschränkung
% \begin{align*}
% \pi\big|_{N_p} \colon N_p\to M
% \end{align*}
% eine Überlagerung, d.h. für eine hinreichend kleine Umgebung $U\subset M$ von
% $\pi(p)$ ist
% \begin{align*}
% \pi^{-1}(U)\cap N_p = \dot{\bigcup}\, N_i,
% \end{align*}
% die disjunkte Vereinigung offener Mengen $N_i$ so, dass
% \begin{align*}
% \pi\big|_{N_i} \colon N_i\to M
% \end{align*}
% ein Diffeomorphismus ist. Man kann somit eine Überdeckung von $M$ durch lokale
% Schnitte
% \begin{align*}
% s : U\to N_i\subset P,\qquad s_i \defl (\pi\big|_{N_i})^{-1}
% \end{align*}
% definieren.
% 
% $\Leftarrow$: Sei eine Überdeckung von $P$ mit horizontalen Schnitten gegeben.
% Die Krümmung ist eine horizontale 2-Form und verschwindet somit bereits auf
% vertikalen Vektoren. Wir müssen daher nur zeigen, dass sie auch auf horizontalen
% Vektoren verschwindet. Wähle dazu $x\in M$ und $p\in P_x$. Dann existiert ein
% horizontaler, lokaler Schnitt $s_i : U_i\to P$, so dass $x\in U_i$ und
% \begin{align*}
% \ds_i : T_x U_i\to T^hP_{U_i}
% \end{align*}
% ein Isomorphismus ist. Für horizontale Vektoren $X,Y\in T_p^hP$ existieren
% dann Urbilder $V,W\in T_xM$ so, dass
% \begin{align*}
% \ds_i(V) = X,\qquad \ds_i(W) = Y.
% \end{align*}
% Nach dem vorangegangen Satz ist $(P,A)$ lokal isomorph zum trivialen Bündel mit
% dem kanonischen flachen Zusammenhang, also ist
% \begin{align*}
% F^{s_i} = s_i^* F^A \equiv 0.
% \end{align*}
% Dann gilt aber auch
% \begin{align*}
% F^A(X,Y) = F^A(\ds_i(V),\ds_i(W)) = s_i^*F(V,W) = 0.
% \end{align*}
% Also verschwindet die Krümmung auch auf den horizontalen Vektorfeldern und ist
% somit identisch Null.
% \qed
% \end{proof}

\begin{rem}
Alternativ kann man die lokalen Schnitte auch mittels Geodätischen definieren.
Zu $x_0\in M$ existiert eine Normalenumgebung $U$ von $x_0$, so dass zwischen
$x_0$ und einem anderen Punkt $x\in U$ eine eindeutig bestimmte Geod\"atische $\gamma$ existiert. Man definiert dann für ein beliebiges $u\in
P_{x_0}$,
\begin{align*}
s: U\to P,\qquad s(x)\defl P_\gamma^A(u). 
\end{align*}
\end{rem}

Ist die Basismannigfaltigkeit $M$ einfach zusammenhängend, dann lässt sich
die Aussage des vorangegangen Satzes wie folgt globalisieren.

\begin{prop}
Sei die Fundamentalgruppe von $M$ trivial. Dann verschwindet die Krümmung des Zusammenhangs genau dann, wenn die
Parallelverschiebung nicht vom gewählten Weg abhängt,  d.h.
\begin{align*}
F^A\equiv 0 \quad \iff \quad P_\gamma^A \text{ ist unabhängig vom gewählten
Weg}.\fish
\end{align*}
\end{prop}
\begin{proof}
Wir müssen nur zeigen, dass $F^A\equiv 0$ die Unabhängigkeit von der Wahl des
Weges impliziert.

Seien $x,y\in M$ und $\gamma$ bzw. $\delta$ Wege, die $x$ und $y$ verbinden. Da
$M$ einfach zusammenhängend ist, sind $\gamma$ und $\delta$ homotop. Es
existiert also eine Homotopie $F$ mit
\begin{align*}
&F: I\times I\to M,\qquad F_0(t) = \gamma(t),\qquad F_1(t) = \delta(t).
\end{align*}
Da die Krümmung des Zusammenhangs verschwindet, ist $P$ lokal trivial.
Man zerlegt nun $I\times I$ in kleine Kästchen $K_i$ so, dass $P_{F(K_i)}$
trivial ist. Wir betrachten nur den Fall von vier
Kästchen (siehe Abbildung \ref{fig:Unabhangigkeit-Wegewahl}), der allgemeine
Fall folgt analog. Für jedes Kästchen $K_i$ ist die Parallelverschiebung auf $P_{F(K_i)}$ unabhängig vom
gewählten Weg. Somit gilt dann
\begin{align*}
P_{\delta_1} &= P_{\tau_2}\circ P_{\ep_1}  = 
P_{\tau_2}\circ P_{\tau_1}\circ P_{\gamma_1},\\
P_{\delta_2} &= P_{\ep_2}\circ P_{\tau_2^{-}}
= P_{\gamma_2}\circ P_{\tau_1^{-1}}\circ  P_{\tau_2^{-}}.
\end{align*} 
Also ist
\begin{align*}
P_{\delta} &= P_{\delta_2}\circ P_{\delta_1}
= P_{\gamma_2}\circ P_{\tau_1^{-1}}\circ  P_{\tau_2^{-}} \circ
P_{\tau_2}\circ P_{\tau_1}\circ P_{\gamma_1}\\
&= P_{\gamma_2}\circ P_{\gamma_1} = P_{\gamma}.\qed\map
\end{align*}

\begin{figure}[h]
\centering
\begin{pspicture}(0,-2.08)(7.54,2.1)
\psbezier[linecolor=darkblue](4.14,-0.02)(4.74,-2.06)(7.52,-0.82)(7.38,0.68)
\psbezier[linecolor=purple](4.14,-0.04)(4.7,1.58)(6.0,-0.02)(7.4,0.68)
\psbezier(4.14,-0.02)(5.4,-0.64)(6.98,-0.26)(7.38,0.66)
\psbezier(5.54,0.64)(5.54,-0.16)(5.52,-0.56)(6.24,-0.92)
\psline[linecolor=purple](4.6,0.7)(4.74,0.68)(4.7,0.54)
\psline[linecolor=purple](6.34,0.62)(6.44,0.5)(6.34,0.4)
\psline(6.44,-0.02)(6.62,-0.08)(6.54,-0.24)
\psline[linecolor=darkblue](6.76,-0.48)(6.94,-0.44)(6.9,-0.62)
\psline(5.46,0.18)(5.54,0.34)(5.64,0.18)
\psline(5.8,-0.74)(5.78,-0.58)(5.96,-0.64)
\psline(4.86,-0.18)(5.02,-0.3)(4.86,-0.38)
\psline[linecolor=darkblue](4.8,-0.82)(4.86,-0.96)(4.72,-1.0)

\rput(4.56,0.92){\color{purple}$\delta_1$}

\rput(6.4,0.785){\color{purple}$\delta_2$}

\rput(5.8,0.185){\color{darkgray}$\tau_2$}

\rput(6.1,-0.575){\color{darkgray}$\tau_1$}

\rput(6.53,0.13){\color{darkgray}$\ep_2$}

\rput(7.0,-0.8){\color{darkblue}$\gamma_2$}

\rput(4.82,-1.175){\color{darkblue}$\gamma_1$}

\rput(4.77,-0.035){\color{darkgray}$\ep_1$}

%\psframe(3.44,0.7)(0.16,-1.3)
\psline(3.44,0.7)(3.44,-1.3)(0.16,-1.3)(0.16,0.7)
\psline{->}(3.18,-1.3)(3.82,-1.3)
\psline{->}(0.16,0.44)(0.16,1.14)
\psline(0.16,-0.3)(3.44,-0.3)
\psline(1.78,0.7)(1.78,-1.3)
\psline[linecolor=purple](3.44,0.7)(0.16,0.7)
\psline[linecolor=darkblue](3.44,-1.3)(0.16,-1.3)

\rput(0.07,1.265){\color{darkgray}$s$}

\rput(3.84,-1.495){\color{darkgray}$t$}
\psbezier{->}(2.18,1.12)(3.02,1.96)(5.04,1.78)(5.66,1.08)

\rput(3.92,1.925){\color{darkgray}$F$}
\end{pspicture} 
\label{fig:Unabhangigkeit-Wegewahl}
\caption{Zum Beweis der Unabhängigkeit der Wahl des Weges.}
\end{figure}
\end{proof}

\section{Parallelverschiebung in Vektorbündeln}

Sei $\pi\colon P\to M$ ein $G$-Hauptfaserbündel mit Zusammenhang $A$ und $G$ wirke
von links auf einer Mannigfaltigkeit $F$. Das assoziierte Faserbündel bezeichnen
wir mit
\begin{align*}
E = P\times_G F,\qquad [p,v] \sim [p\cdot g,g^{-1}v].
\end{align*}

\begin{defn}
\index{Parallelverschiebung!im Vektorbündel}
Sei $I=[a,b]$ ein Intervall und $\gamma\colon I\to M$ eine Kurve in $M$.
Die \emph{Parallelverschiebung in $E$ entlang $\gamma$} ist definiert durch
\begin{align*}
P_\gamma^E : E_{\gamma(a)} \to E_{\gamma(b)},\qquad
[p,v] \mapsto [P_\gamma^Ap,v].\fish
\end{align*}
\end{defn}

\begin{rem}[Bemerkungen.]
\begin{remenum}
\item
Durch diese Definition ist die Parallelverschiebung in $E$
tatsächlich wohldefiniert. Sei $[p,v]\in E$, dann gilt für $g\in E$ aufgrund der
$G$-Äquivarianz von $P_\gamma^A$,
\begin{align*}
P_\gamma^E [p\cdot g,g^{-1}\cdot v]
&= \left[P_\gamma^A (p\cdot g),g^{-1}\cdot v \right]
= \left[R_g P_\gamma^A p,g^{-1}\cdot v \right]
= \left[P_\gamma^A p,v \right]\\
&= P_\gamma^E [g,v].
\end{align*}
\item Sei $\gamma^*$ ein horizontaler Lift von $\gamma$, dann gilt
\begin{align*}
P_\gamma^E = [\gamma^*(b)]\circ [\gamma^*(a)]^{-1},
\end{align*}
wobei $[p]\colon F\to E$, $v\mapsto [p,v]$.
\begin{proof}
Setze $p = \gamma^*(a)$, dann genügt es die Aussage für die Äquivalenzklasse
$[p,v]$ zu zeigen. Dann ist
\begin{align*}
P_\gamma^A p = \gamma^*(b),
\end{align*}
und folglich
\begin{align*}
[\gamma^*(b)][\gamma^*(a)]^{-1}[p,v]
= [\gamma^*(b)]v = 
[\gamma^*(b),v] = P_\gamma^E [p,v].\qed 
\end{align*}
\end{proof}
\item Ist $E$ ein Vektorbündel, dann ist $P_\gamma^E$ ein
Vektorraumisomorphismus.\map
\end{remenum}
\end{rem}

\begin{ex}
Sei $E= TM$ das Tangentialbündel, dann ist $E= P_{\GL_n}\times_{\GL_n} \R^n$.
Weiterhin sei $x\in M$ und $p = (e_1,\ldots,e_r)$ eine Basis von $T_xM$. Ein
Tangentialvektor $v\in T_xM$ lässt sich schreiben als
\begin{align*}
v = \sum_{i=1}^r v_i e_i.
\end{align*} 
Zu einer Kurve $\gamma\colon I\to M$ mit $\gamma(a) = x$, ist die
Parallelverschiebung
\begin{align*}
P_\gamma^A (p) = \gamma_p^*(t) = (e_1(t),\ldots,e_r(t))
\end{align*}
eine Kurve im Rahmenbündel. Die Parallelverschiebung in $TM$ ist nun gegeben
durch
\begin{align*}
P^{TM}_\gamma : T_xM\to T_{\gamma(t)}M,\qquad v\mapsto \sum_{i=1}^r v_i e_i(t).\boxc
\end{align*}
\end{ex}

Ein Zusammenhang auf dem Hauptfaserbündel induziert eine kovariante Ableitung
auf den assoziierten Faserbündeln. Diese kovariante Ableitung können wir nun
auch mittels der Parallelverschiebung beschreiben.

\begin{prop}
Sei $E$ ein zu $P$ assoziiertes Vektorbündel versehen mit der von $A$ induzierten kovarianten Ableitung $\nabla^A$. 
Dann gilt für jedes
Vektorfeld $X$ auf $M$ und jeden Schnitt $e$ in $E$ 
\begin{align*}
(\nabla_X^A e)_x = \frac{\ddd}{\dt}\bigg|_{t=0} P^E e(\gamma(t)),
\end{align*}
wobei $P^E\colon E_{\gamma(t)}\to E_{\gamma(0)}$ die Parallelverschiebung entlang
$\gamma$ beschreibt und $\gamma$ eine Kurve in $M$ ist mit $\gamma(0) = x$ und
$\dot{\gamma}(0) = X_x$.\fish
\end{prop}

\begin{proof}
Sei $\gamma^*$ ein horizontaler Lift von $\gamma$, dann ist
\begin{align*}
P^E = [\gamma^*(0)]\circ [\gamma^*(t)]^{-1},
\end{align*}
und $\dot{\gamma}^*(0) = X_p^*$. Weiterhin gilt $e(\pi(p)) = [p,\bar{e}(p)]$,
also ist $e(\gamma(t)) = [\gamma^*(t),\bar{e}(\gamma^*(t))]$ und folglich
\begin{align*}
\frac{\ddd}{\dt}\bigg|_{t=0} P^E e(\gamma(t)) &= 
[\gamma^*(0)] \left(\frac{\ddd}{\dt}\bigg|_{t=0}
[\gamma^*(t)]^{-1}(e(\gamma(t)) \right)\\
&= [\gamma^*(0)] \left(
\ddd\bar{e}(\dot{\gamma}^*(0)) \right)\\
&= [p,\ddd \bar{e}(X_p^*)]
= (\ddd_A e)(X_x) \\
&= (\nabla^A_X e)_x.\qed
\end{align*}
\end{proof}

% Vorlesung vom 27.06.2011

\section{Parallelverschiebung im Tangentialbündel}

Für Vektorbündel wie das Tangentialbündel $TM$ über einer Mannigfaltigkeit $M$
gibt es einen direkteren und weniger abstrakten Weg, die Parallelverschiebung zu
definieren. Wir wollen dies im Folgenden am Beispiel des Tangentialbündels
erarbeiten und zeigen, dass dieser Zugang äquivalent zur Definition von
Parallelverschiebung in assoziierten Faserbündeln ist. Allerdings fehlt einem
bei diesem Zugang das >>big picture<<, dass sich nämlich zahlreiche 
Objekte, einmal für das $G$-Hauptfaserbündel $P$ definiert, auf die assoziierten
Vektorbündel
\begin{align*}
E = P\times_\rho V,\qquad \rho : G\to \GL(V),
\end{align*}
übertragen, wobei man bereits viele interessante Vektorbündel erhält, wenn man
den Totalraum $P$  festhält und lediglich die Darstellung
variiert.
Nichts desto trotz bietet  der folgende, elementare Zugang ebenfalls seine
Vorzüge, weshalb wir ihn auch präsentieren. 

\begin{defn}
\index{Vektorfeld!entlang einer Kurve}
Ein \emph{Vektorfeld entlang einer Kurve}  $\gamma\colon I\to M$ ist eine glatte
Abbildung
\begin{align*}
X: I\to TM,
\end{align*}
mit $X(t)\in T_{\gamma(t)}M$ für alle $t\in I$. Den \emph{Raum aller
Vektorfelder entlang $\gamma$} bezeichnen wir mit $\chi(\gamma)$.\fish
\end{defn}

\begin{rem}
Der Raum $\chi(\gamma)$ ist ein unendlichdimensionaler Vektorraum über $\R$. Für
ein Vektorfeld $X$ entlang $\gamma$ und eine Funktion $f\colon I\to \R$ ist auch
wiederum
\begin{align*}
fX : I\to TM,\qquad t\mapsto f(t)X(t)
\end{align*}
ein Vektorfeld entlang $\gamma$. Somit ist $\chi(\gamma)$ außerdem ein
$\Cs^\infty(I)$-Modul.\map
\end{rem}

\begin{ex}
\begin{exenum}
\item Betrachten wir zu einer Kurve $\gamma$ in $M$ die erste Ableitung
\begin{align*}
\dot{\gamma}\colon I\to TM,
\end{align*} 
so ist dies offenbar ein Vektorfeld entlang $\gamma$.
\item Schränken wir ein gegebenes Vektorfeld $X\in\chi(M)$ auf die Spur der
Kurve $\gamma$ ein,
\begin{align*}
X_\gamma \defl X\circ \gamma 
\end{align*}
so ist die Einschränkung $X_\gamma$ ein Vektorfeld entlang $\gamma$.\boxc
\end{exenum}
\end{ex}

\begin{prop}
Sei $M$ eine Mannigfaltigkeit mit einer kovarianten Ableitung $\nabla$ auf $TM$.
Dann existiert eine eindeutig bestimmte Abbildung
\begin{align*}
\frac{\nabla}{\dt} \colon \chi(\gamma)\to \chi(\gamma),
\end{align*}
mit den folgenden Eigenschaften:
\begin{propenum}
\item $\frac{\nabla}{\dt}$ ist $\R$-linear.
\item Für alle Vektorfelder $X\in\chi(\gamma)$ und $f\in\Cs^\infty(I)$ gilt
\begin{align*}
\frac{\nabla}{\dt}(f\cdot X) = \frac{\ddd f}{\dt}X + f \frac{\nabla}{\dt}X.
\end{align*}
\item Falls $X_\gamma = X\circ \gamma$ für ein Vektorfeld $X\in\chi(M)$, dann
gilt
\begin{align*}
\frac{\nabla}{\dt}(X_\gamma) = \nabla_{\!\dot{\gamma}}\,X.\fish
\end{align*}
\end{propenum}
\end{prop}

Die Beweisidee beruht darauf, in lokalen Koordinaten aus den Eigenschaften
a)-c) eine explizite, eindeutige und von der Wahl der Koordinaten
unabhängige Formel für $\frac{\nabla}{\dt}$ herzuleiten. Dann definiert man
$\frac{\nabla}{\dt}$ mit Hilfe dieser Formel und zeigt wiederum, dass mit
dieser Definition die Eigenschaften a)-c) erfüllt sind.

\begin{proof}
Sei $(U,x)$ eine Karte in $M$ und $\gamma\colon I\to M$ eine Kurve. Da die Spur
von $\gamma$ kompakt ist, können wir ohne Einschränkung annehmen, dass $\gamma$
ganz in $U$ liegt. Wir schreiben dann
\begin{align*}
x\circ\gamma = (\gamma_1,\ldots,\gamma_n)
\end{align*}
für die Komponenten von $\gamma$. Sei $X$ ein Vektorfeld entlang $\gamma$, dann
besitzt es die lokale Darstellung
\begin{align*}
X(t) = \sum_{i=1}^n f_i(t) \frac{\partial}{\partial x_i}\bigg|_{\gamma(t)}.
\end{align*}
Angenommen es gibt solch eine Abbildung $\frac{\nabla}{\dt}$. Dann man berechnet
für $X$ mit Hilfe der $\R$-Linearität (a) und der Derivationsregel (b),
\begin{align*}
\frac{\nabla}{\dt} X(t) &= \sum_{i=1}^n \frac{\nabla}{\dt}\left(f_i
\frac{\partial}{\partial x_i}\bigg|_{\gamma(t)}\right)\\
&= \sum_{i=1}^n  \frac{\ddd f_i }{\ddd t} \frac{\partial}{\partial
x_i}\bigg|_{\gamma(t)} + f_i \frac{\nabla}{\dt}\left(\frac{\partial}{\partial
x_i}\bigg|_{\gamma(t)} \right).
\end{align*}
Weiterhin gilt nach Eigenschaft (c), dass
\begin{align*}
\frac{\nabla}{\dt}\left(\frac{\partial}{\partial
x_i}\bigg|_{\gamma(t)} \right) &= 
\nabla_{\!\dot{\gamma}(t)} \frac{\partial}{\partial
x_i}\Bigg|_{\gamma(t)}
=
\sum_{j=1}^n \dot{\gamma}_j(t)
\left(\nabla_{\!\!\frac{\partial}{\partial
x_j}} \frac{\partial}{\partial
x_i}\right)\Bigg|_{\gamma(t)}\\
&=
\sum_{j,k=1}^n \dot{\gamma}_j(t)\,
\Gamma_{ji}^k(\gamma(t)) \frac{\partial}{\partial x_k} \bigg|_{\gamma(t)},
\end{align*}
wobei $\Gamma_{ji}^k$ die Christoffelsymbole der kovarianten Ableitung $\nabla$
bezeichnen. Somit erhalten wir für $\frac{\nabla}{\dt}$ die lokale Darstellung
\begin{align*}
\frac{\nabla}{\dt} X = 
\sum_{i=1}^n \left(\frac{\ddd f_i}{\dt}\frac{\partial}{\partial x_i} + f_i
\sum_{j,k=1}^n \dot{\gamma}_j\, \Gamma_{ji}^k \frac{\partial}{\partial x_k}
\right).\tag{*}
\end{align*}
Die Bedingungen (a)-(c) legen somit die Abbildung $\frac{\nabla}{\dt}$ in
lokalen Koordinaten eindeutig fest. Aus Eigenschaft b) folgt, unter Verwendung 
von Abschneidefunktionen, dass es keinen Unterschied macht, ob man $X$
zuerst lokalisiert und anschließend ableitet, oder ob man zuerst ableitet und
dann lokalisiert. Somit ist es gerechtfertigt, die Definition von $\frac{\nabla}{\dt}$
zunächst nur lokal anzugeben.
Es verbleibt zu zeigen, dass diese
Definition nicht von der Wahl der Koordinaten abhängt, und dass durch (*) 
tatsächlich die Eigenschaften a)-c) erfüllt sind.\qed
\end{proof}

\begin{rem}[Bemerkung zur Notation.]
Anstatt $\frac{\nabla}{\dt}$ schreibt man auch
\begin{align*}
\frac{\nabla}{\dt} X \equiv \dot{X} \equiv \nabla_{\!\dot{\gamma}}X.
\end{align*}
Es gilt dabei zu beachten, dass mit $\dot{X}$ oder $\nabla_{\!\dot{\gamma}}X$
im Allgemeinen immer $\frac{\nabla}{\dt}$ gemeint ist und nicht die
Zeitableitung oder die kovariante Ableitung im üblichen Sinne. Eigenschaft
(iii) besagt ja gerade, dass $\frac{\nabla}{\dt} X = \nabla_{\dot{\gamma}}X$ nur
für Vektorfelder $X=X_\gamma$, die durch Einschränkung eines Vektorfeldes
$M$ hervorgegangen sind, gilt.

Ist $X$ gerade die erste Ableitung der Kurve $\gamma$, d.h. $X=\dot{\gamma}$,
dann schreibt man auch
\begin{align*}
\frac{\nabla}{\dt} X = \dot{X} = \ddot{\gamma}.\map
\end{align*}
\end{rem}

Ein Zusammenhang $\nabla$ auf einer Riemannschen Mannigfaltigkeit $(M,g)$ heißt
metrisch, falls
\begin{align*}
X g(Y,Z) = g(\nabla_X Y,Z) + g(Y,\nabla_X Z),
\end{align*}
d.h. die Lie-Ableitung nach $X$ ist kompatibel mit der Metrik $g$ und erfüllt
eine Art Produktregel.
Eine analoge Aussage gilt auch für die Abbildung $\frac{\nabla}{\dt}$.

\begin{cor}
Sei $(M,g)$ eine Riemannsche Mannigfaltigkeit und $\nabla$ ein metrischer
Zusammenhang auf $TM$. Für Vektorfelder $X_1,X_2\in\chi(\gamma)$ gilt dann
\begin{align*}
\frac{\ddd}{\dt} g(X_1,X_2) = g(\dot{X}_1,X_2) + g(X_1,\dot{X}_2).\fish
\end{align*} 
\end{cor}

Im letzten Semester haben wir gesehen, dass der Zusammenhang $\nabla$ genau dann
metrisch ist, wenn die Metrik $g$ parallel ist, d.h.
\begin{align*}
\nabla g \equiv 0.
\end{align*}
Parallele Objekte haben zahlreiche angenehme Eigenschaften und sind in vielen
Situationen leichter zu handhaben. Mit Hilfe des eben erarbeiteten
Ableitungsbegriffes, lässt sich Parallelität sehr elementar für Vektorfelder
entlang einer Kurve definieren.

\begin{defn}
\index{Vektorfeld!parallel}
Ein Vektorfeld $X\in\chi(\gamma)$ heißt \emph{parallel}, falls $\dot{X} =
0$.\fish
\end{defn}

\begin{ex}
Betrachten wir $M=\R^n$ versehen mit der Standardmetrik und dem flachen
Zusammenhang. Dann verschwinden alle Christoffelsymbole, d.h. $\Gamma_{ji}^k
\equiv 0$ und für ein Vektorfeld entlang $\gamma$,
\begin{align*}
X = \sum_{i=1}^n f_i \frac{\partial}{\partial x_i},
\end{align*}
nimmt die Formel (*) die besonders einfache Form
\begin{align*}
\dot{X} = \sum_{i=1}^n \frac{\ddd f_i}{\dt} \frac{\partial}{\partial x_i}
\end{align*}
an. Somit ist $X$ genau dann parallel, also $\dot{X}  =0$, wenn die
Koeffizientenfunktionen konstant sind, d.h. $\dot{f}_i \equiv 0$. Dies
entspricht auch der Vorstellung, denn das Vektorfeld $X$ ändert beim Verlauf
entlang $\gamma$ seine Richtung nicht.\boxc
\end{ex}

\begin{lem}
Sei $\gamma\colon I\to M$ eine glatte Kurve und $Z_0\in T_{\gamma(t_0)}M$ für ein
$t_0\in I$. Dann existiert genau ein paralleles Vektorfeld $Z\in\chi(\gamma)$
mit
\begin{align*}
Z(t_0) = Z_0.\fish 
\end{align*}
\end{lem}
\begin{proof}
Die Bedingung $\dot{X}\equiv 0$ ist nach Gleichung (*) äquivalent zu $n$
gekoppelten, gewöhnlichen Differentialgleichungen
\begin{align*}
\dot{f}_i = - \sum_{j,k=1}^n \dot{\gamma}_j\, \Gamma_{jk}^i\, f_k ,\qquad i =
1,\ldots,n.
\end{align*}
Diese besitzen nach dem allgemeinen Existenz und Eindeutigkeitssatz eine
eindeutige lokale Lösung
\begin{align*}
(-\ep,\ep)\to \R^n,\qquad t\mapsto (f_1(t),\ldots,f_n(t)).
\end{align*}
Außerdem hängen die Lösungen differenzierbar vom Anfangswert ab. Die $f_i$
liefern nun die Koeffizienten des gesuchten Vektorfeldes $Z$.\qed
\end{proof}

\begin{figure}[h]
\centering
\begin{pspicture}(-0.2,-1.31)(5,1.33)
\psbezier[linecolor=darkblue](0.08,-0.31)(1.14,0.83)(2.78,-1.29)(3.7,0.03)
\psline[linecolor=darkyellow]{->}(0.08,-0.29)(0.52,0.63)

\rput(0.09,-0.465){\color{darkgray}$p$}
\rput(4.3,-0.205){\color{darkgray}$q=\gamma(t)$}
\rput(1.97,-0.445){\color{darkblue}$\gamma$}
\psline[linecolor=darkyellow]{->}(1.34,-0.03)(1.78,0.89)
\psline[linecolor=darkyellow]{->}(2.4,-0.35)(2.84,0.57)
\psline[linecolor=darkyellow]{->}(3.68,0.01)(4.12,0.93)
\psdots[linecolor=darkblue](0.08,-0.29)
\psdots[linecolor=darkblue](3.68,0.01)

\rput(0.54,0.835){\color{darkgray}$Z_0$}
\rput(4.22,1.135){\color{darkgray}$Z(t)$}
\end{pspicture}
\caption{Parallelverschiebung eines Tangentialvektors $Z_0$ entlang $\gamma$.} 
\end{figure}

Mit Hilfe des vorangegangen Lemmas können wir nun ganz einfach eine
Parallelverschiebung im Tangentialbündel definieren. Gegeben sei ein
Tangentialvektor $Z_0\in T_xM$ an einen Punkt $p\in M$. Für einen weiteren Punkt
$q\in M$ wählt man nun eine Kurve $\gamma$, die $x$ und $y$ verbindet und für
die $\dot{\gamma}(0) = Z_0$ gilt. Nun existiert ein eindeutig
bestimmtes, paralleles Vektorfeld entlang dieser Kurve $\gamma$. Mit diesem
Vektorfeld kann man $Z_0$ von $T_pM$ parallel nach $T_qM$ verschieben.

\begin{defn}
\index{Parallelverschiebung!im Tangentialbündel}
Sei $\gamma$ eine Kurve in $M$, dann heißt die Abbildung
\begin{align*}
\tau\colon T_{\gamma(t_0)}M \to T_{\gamma(t)}M,\qquad Z_0\mapsto Z(t),
\end{align*}
mit $Z\in \chi(\gamma)$ parallel und $Z(t_0) = Z_0$, \emph{Parallelverschiebung
entlang $\gamma$}.\fish
\end{defn}

\begin{ex}
Betrachten wir wieder $M=\R^n$ mit der Standardmetrik und dem flachen
Zusammenhang. Dann ist die Parallelverschiebung gerade durch die gewöhnliche
Translation von Vektoren gegeben.\boxc
\end{ex}

Im Allgemeinen hängt die Parallelverschiebung vom Zusammenhang ab und kann
beliebig kompliziert werden. Die Menge aller möglichen Parallelverschiebungen
lässt sich durch die Holonomiegruppe (siehe Kapitel \ref{chap:Holonimietheore})
beschreiben.

\begin{rem}[Bemerkung zur Notation.]
Sei $\gamma$ eine Kurve in $M$, die $x$ und $y$ verbindet, so bezeichnen wir die
Parallelverschiebung entlang $\gamma$ auch mit
\begin{align*}
\tau \equiv \tau(\gamma) \equiv \tau_{xy} \colon T_xM\to T_yM.\map 
\end{align*}
\end{rem}

Die Parallelverschiebung im Hauptfaserbündel $P_\gamma^A$ beschreibt einen
Diffeomorphismus der Fasern. Ein analoges Resultat gilt auch für die so
definierte Parallelverschiebung im Tangentialbündel, nur dass $\tau$ aufgrund
der linearen Struktur sogar ein Isomorphismus ist.

\begin{lem}
\begin{propenum}
\item Die Parallelverschiebung ist ein linearer Isomorphismus
\begin{align*}
\tau_{xy}\colon T_xM\to T_yM,\qquad x,y\in M.
\end{align*}
\item Sei $g$ eine Metrik auf $M$. Der Zusammenhang $\nabla$ ist genau dann
metrisch, wenn $\tau_{xy}$ eine Isometrie ist.\fish
\end{propenum}
\end{lem}
\begin{proof}
a): Aufgrund der Eindeutigkeit des parallelen Vektorfeldes an $\gamma$ folgt
sofort die Linearität von $\tau$. Außerdem hängen die Lösungen der gewöhnlichen
Differentialgleichung differenzierbar vom Anfangswert ab, also ist $\tau$ auch
differenzierbar. Letztlich folgt aus der Eindeutigkeit, dass $\tau_{yx}$ invers
ist zu $\tau_{xy}$ und folglich ist $\tau$ ein linearer, differenzierbarer
Isomorphismus zwischen den Tangentialräumen an $x$ und $y$.

b): $\Rightarrow$: Sei $\nabla$ ein metrischer Zusammenhang auf $TM$, dann gilt
für Vektorfelder $X$ und $Y$ entlang $\gamma$, dass
\begin{align*}
\frac{\ddd}{\dt} g(X,Y) = g(\dot{X},Y) + g(X,\dot{Y}).
\end{align*}
Sind nun $X$ und $Y$ parallel, so verschwindet die rechte Seite und $g(X,Y)$ ist
konstant. Wähle also $X_0,Y_0\in T_xM$, so gilt
\begin{align*}
(\tau_{xy}^* g)(X_0,Y_0)
= g(\tau_{xy}X_0,\tau_{xy}Y_0) = g(X(t),Y(t)) = g(X_0,Y_0),
\end{align*}
denn $X(t)$ und $Y(t)$ sind parallele Vektorfelder entlang $\gamma$ mit
Anfangswert $X_0$ bzw. $Y_0$. Somit ist $\tau_{xy}$ eine Isometrie.

$\Leftarrow$: Sei $\tau_{xy}$ eine Isometrie, dann gilt für parallele
Vektorfelder $X$, $Y$ entlang $\gamma$,
\begin{align*}
g(X(t),Y(t)) = (\tau_{pq}^*g)(X_0,Y_0) = g(X_0,Y_0).
\end{align*}
Somit ist $g(X(t),Y(t))$ konstant. Wähle nun in $p$ eine Orthonormalbasis
$(\partial_1,\ldots,\partial_n)$ von $T_pM$ und definiere
\begin{align*}
\partial_i(t) \defl \tau_{xy}(\partial_i),\qquad i=1,\ldots,n.
\end{align*}
Dann ist nach obiger Rechnung auch $(\partial_1(t),\ldots,\partial_n(t))$
eine Orthonormalbasis, d.h.
\begin{align*}
g(\partial_i(t),\partial_j(t)) = \delta_{ij},\qquad t\in I.
\end{align*}
Somit hängt die Entwicklung von $g(X(t),Y(t))$ nur von den
Koeffizientenfunktionen ab. Genauer seien
\begin{align*}
X(t) = \sum_{i=1}^n a_i(t) \partial_i(t),\qquad
Y(t) = \sum_{j=1}^n b_j(t) \partial_j(t),
\end{align*}
dann gilt für die Metrik
\begin{align*}
\frac{\ddd}{\dt}\, g(X(t),Y(t)) &= 
 \frac{\ddd}{\dt} \sum_{i,j=1}^n a_i(t) b_j(t)\,
 g(\partial_i(t),\partial_j(t))\\ 
 &= \sum_{i,j=1}^n \left(\dot{f}_i(t)g_j(t) + f_i(t)\dot
 g_j(t)\right)\,\delta_{ij}\\ 
 &= g(\dot{X}(t),Y(t)) + g(X(t),\dot Y(t)).
\end{align*}
Seien nun $Z\in T_xM$ und $X,Y$ Vektorfelder auf $M$. Weiterhin sei $\gamma
: I\to M$ eine Kurve durch $x$ mit $\dot{\gamma}(0) = Z$. Dann gilt nach dem
bisher gezeigten,
\begin{align*}
Z g(X,Y) &= \frac{\ddd}{\dt}\bigg|_{t=0} g(X,Y)\circ \gamma(t)\\
&= \frac{\ddd}{\dt}\bigg|_{t=0} g_{\gamma(t)}(X_\gamma(t),
Y_\gamma(t))\\
&= g(\dot{X}_\gamma(0),Y_p) + 
g(X_p,\dot{Y}_\gamma(0))\\
&= g(\nabla_{\!Z} X,Y_p) + 
g(X_p,\nabla_{\!Z} Y),
\end{align*}
denn $\dot{X}_\gamma(0) = \nabla_{\dot{\gamma}(0)}X = \nabla_Z X$. Also ist
$\nabla$ metrisch und dies war zu zeigen.\qed
\end{proof}

Sei $\nabla$ ein Zusammenhang auf $TM$. Wir schreiben das Tangentialbündel
\begin{align*}
TM = P_{\GL_n}\times_{\GL_n} \R^n \defr E
\end{align*}
als assoziiertes Vektorbündel zum Rahmenbündel $P_{\GL_n}$ und versehen
$P_{\GL_n}$ mit dem durch $\nabla$ induzierten Zusammenhang $A$. Dann verfügen
wir über zwei Parallelverschiebungen, nämlich einerseits die induzierte
Parallelverschiebung
\begin{align*}
P^E_\gamma\colon E_x\to E_y,\qquad [p,v]\mapsto [P_\gamma^A p,v], 
\end{align*}
und die soeben definierte Parallelverschiebung
\begin{align*}
\tau_{xy}\colon T_xM\to T_yM,\qquad Z_0\mapsto Z(b). 
\end{align*}
Zum Abschluss zeigen wir, diese zwei Definitionen tatsächlich äquivalent sind.

\begin{prop}
Sei $\gamma\colon I\to M$ eine Kurve in $M$. Dann gilt
\begin{align*}
\tau_{\gamma} = P^E_{\gamma}.\fish
\end{align*}
\end{prop}
\begin{proof}
Sei $x\in M$ und $X\in T_xM$. Weiterhin sei $\gamma$ eine Kurve in $M$ durch $x$
mit $\dot{\gamma}(0) = X$. Dann ist die durch den Zusammenhang $A$ induzierte
Parallelverschiebung gegeben durch
\begin{align*}
X(t) = P^E_{\gamma(t)}X = [\gamma^*(0)]\circ [\gamma^*(t)]^{-1}X,
\end{align*}
wobei $\gamma^*$ den horizontalen Lift von $\gamma$ nach $P$ bezeichnet.
Offenbar ist $X(t)$ ein Vektorfeld entlang $\gamma$. Weiterhin gilt die Formel
\begin{align*}
\frac{\nabla}{\dt} X(t) = \frac{\ddd}{\ds}
\left(P_{\gamma|_{[t,t+s]}}^E\right)^{-1}(X(t+s)).
\end{align*}
Nun gilt jedoch
\begin{align*}
\left(P_{\gamma|_{[t,t+s]}}^E\right)^{-1}(X(t+s)) &=
[\gamma^*(t)]\circ [\gamma^*(t+s)]^{-1} X(t+s)\\
&= [\gamma^*(t)]\circ
[\gamma^*(t+s)]^{-1} \circ [\gamma^*(t+s)]\circ [\gamma^*(0)](X)\\
&= [\gamma^*(t)]\circ [\gamma^*(0)](X).
\end{align*}
Also hängt der Ausdruck nicht von $s$ ab, d.h. die Ableitung verschwindet
und $X(t)$ ist parallel. Somit ist $X(t)$ das eindeutig bestimmte parallele
Vektorfeld mit $X(0) = X$ und somit ist $X(t) = \tau_{x\gamma(t)}(X)$.\qed
\end{proof}


\section{Parallele Schnitte und Parallelverschiebung}

Sei $\pi\colon E\to M$ ein Vektorbündel mit einer kovarianten Ableitung $\nabla^E$.

\begin{defn}
\index{Schnitt!paralleler}
\index{Unterbündel!$\nabla^E$-invariant}
\begin{defnenum}
\item
Der \emph{Raum der parallelen Schnitte in $E$} ist definiert als
\begin{align*}
\Par(E,\nabla^E) \defl \setdef{e\in \Gamma(E)}{\nabla^E e = 0}.
\end{align*}
\item Ein Unterbündel $F\subset E$ heißt \emph{$\nabla^E$-invariant} (oder
\emph{parallel}), falls
\begin{align*}
\nabla_X^E f \in \Gamma(F),\qquad \text{für alle }f\in \Gamma(F)\text{ und }
X\in\chi(M).\fish
\end{align*}
\end{defnenum}
\end{defn}

\begin{rem}
Sei $D\subset TM$ eine $\nabla$-invariante Distribution. Ist der Zusammenhang
$\nabla^E$ torsionsfrei, dann ist $D$ integrabel, denn
\begin{align*}
[X,Y] = \nabla_X Y - \nabla_Y X \in D.
\end{align*}
Im Fall eines torsionsfreien Zusammenhangs sind also insbesondere die parallelen Unterbündel integrabel.\map
\end{rem}

\begin{prop}
Sei $M$ einfach zusammenhängend und $F\subset E$ ein paralleles Unterbündel vom
Rang $r$. Ist $R^{\nabla^F} = 0$, dann gilt
\begin{align*}
\dim \Par(E,\nabla^E) \ge r.\fish
\end{align*}
\end{prop}
\begin{proof}
Sei $x\in M$ und $(v_1,\ldots,v_r)$ eine Basis der Faser $F_x$. Man definiert
nun Schnitte in $F$ durch
\begin{align*}
f_i(y) \defl P_{\gamma_{xy}}^F(v_i) \in F_y,\qquad 1\le i\le r,
\end{align*} 
wobei $\gamma_{xy}$ eine Kurve von $x$ nach $y$ bezeichnet. 

\textit{$f_i$ ist wohldefiniert}. Wir schreiben $F= P\times_{\GL_r} \K^r$, wobei
$P$ das Rahmenbündel von $F$ bezeichnet. Der Zusammenhang auf $P$ sei durch
$\nabla^F$ induziert, dann gilt
\begin{align*}
R_{XY}^{\nabla^F}f = [p,F^A(X^*,Y^*)v],
\end{align*}
falls $f=[p,v]$. Somit ist $R^{\nabla^F} = 0$ genau dann, wenn $F^A = 0$ gilt.
Da $M$ nach Voraussetzung einfach zusammenhängend ist, hängt daher die
Parallelverschiebung $P_\gamma^A$ in $P$ nicht vom gewählten Weg ab. Somit
hängt auch die Parallelverschiebung in $F$, gegeben durch
\begin{align*}
P^{\nabla^F}_\gamma = [P_\gamma^A(p)]\circ[p]^{-1}
\end{align*}
nicht vom gewählten Weg ab und der Schnitt $f_i$ ist wohldefiniert. Schließlich hängt $f_{i}$ als Lösung einer gewöhnlichen Differentialgleichung differenzierbar vom Anfangswert ab und ist folglich ein glatter Schnitt in $F$.

\textit{$f_i$ ist parallel}. Wir müssen zeigen, dass $\nabla^F f_i = 0$ ist.
Sei dazu $y\in M$ und $X$ ein Vektorfeld auf $M$. Dann gilt
\begin{align*}
(\nabla_X^F f_i)_y  &= 
\frac{\ddd}{\dt}\bigg|_{t=0} P_{\gamma(t)y}^{\nabla^F}(f_i(\gamma(t)))
= 
\frac{\ddd}{\dt}\bigg|_{t=0}
P_{\gamma(t)y}^{\nabla^F}P_{x\gamma(t)}^{\nabla^F}(v_i)\\
&
= 
\frac{\ddd}{\dt}\bigg|_{t=0}
P_{xy}^{\nabla^F}(v_i) = 0,
\end{align*}
denn $P^{\nabla^F}$ hängt nicht vom gewählten Weg ab.\qed 
\end{proof}

\begin{figure}[h]
\centering
\begin{pspicture}(0,-2.18)(4.98,2.18)
\psbezier[linecolor=darkblue](0.6,0.2)(1.22,0.82)(2.74,0.06)(3.44,1.16)
\psbezier(2.9,-1.88)(3.74,-1.28)(4.1,0.16)(3.44,1.16)
\psbezier(2.9,-1.88)(2.68,-0.78)(0.66,-1.5)(0.6,0.2)

\rput(2.19,0.36){\color{darkblue}$\gamma$}
\rput(0.42,0.165){\color{darkblue}$x$}
\rput(3.07,-2.1){\color{darkblue}$y$}
\rput(3.88,1.185){\color{darkblue}$\gamma(t)$}
\psline[linecolor=darkyellow]{->}(0.6,0.18)(0.0,1.28)
\psline[linecolor=darkyellow]{->}(3.44,1.14)(3.9,2.04)

\rput(0.24,1.425){\color{darkyellow}$v_i$}
\rput(4.55,1.985){\color{darkyellow}$f_i(\gamma(t))$}
\psbezier[linecolor=darkyellow]{->}(3.28,1.68)(2.32,1.76)(1.32,1.58)(0.56,0.96)
\rput(1.99,1.865){\color{darkyellow}$P_\gamma^\nabla$}
\psline[linecolor=darkblue](1.72,0.6)(1.56,0.48)(1.74,0.38)
\psline(3.64,-0.34)(3.78,-0.16)(3.88,-0.38)
\psline(1.44,-0.84)(1.18,-0.86)(1.3,-1.06)

\psdots[linecolor=darkblue](0.6,0.2)
\psdots[linecolor=darkblue](3.44,1.14)
\psdots[linecolor=darkblue](2.9,-1.86)
\end{pspicture} 
\caption{Zur Parallelverschiebung von $f_i$.}
\end{figure}

\begin{rem}
Sei $e\in\Gamma(E)$ ein paralleler Schnitt, dann gilt
\begin{align*}
R_{XY}^E e = 0
\end{align*} 
für alle Vektorfelder $X$ und $Y$ in $M$. Somit ist $F= \mathrm{span}\setd{e}$
parallel.
Unter den Voraussetzungen des vorangegangenen Satzes ist dann der Schnitt
vollständig durch Parallelverschiebung beschrieben, d.h. für $x,y\in M$ gilt
\begin{align*}
e(y) = P_{\gamma_{xy}}^{\nabla}e(x).\map
\end{align*}
\end{rem}


\begin{prop}
Ein Unterbündel $F\subset E$ ist genau dann parallel, wenn $F$ von der Parallelverschiebung in $E$ invariant gelassen wird.\fish
\end{prop}
\begin{proof}
$\Rightarrow$: Sei $F\subset E$ parallel, dann gilt für jeden Schnitt $f$ in $F$ und jedes Vektorfeld $X$ auf $M$, dass
\begin{align*}
\nabla_{X} f \in \Gamma(F).
\end{align*}
Somit ist die kovariante Ableitung $\nabla^F$ in $F$ gerade durch Einschränkung der kovarianten Ableitung $\nabla^E$ auf das Unterbündel $F$ gegeben. Folglich erhält man die Parallelverschiebung in $F$ gerade durch Einschränkung der Parallelverschiebung in $E$,
\begin{align*}
\tau^F = \tau^E\bigg|_{F}.
\end{align*}

$\Leftarrow$: Sei umgekehrt $F$ invariant unter Parallelverschiebung, dann verläuft die Parallelverschiebung eines Schnittes in $f$ in $F$ ganz in $F$ und somit ist auch $\nabla_{X}f$ ein Schnitt in $F$.\qed
\end{proof}

\section{Geodätische}

Wir suchen nun zu zwei Punkten $x$ und $y$ in $M$ die \textit{kürzeste
Verbindung}, also eine Kurve in $M$, die $x$ und $y$ verbindet und deren Länge
minimal wird. Wie solche Kurven aussehen, hängt sicher von der Topologie der
Mannigfaltigkeit ab, aber auch von der gewählten Metrik, denn diese misst ja
die >>Länge<<. Wie wir noch sehen werden, ist auch die Wahl eines Zusammenhangs wesentlich, um solche Verbindungen überhaupt sinnvoll definieren zu können. 

\begin{defn}
\index{Geodätische}
Eine \emph{Geodätische} ist eine Kurve $\gamma\colon I\to M$ mit $\ddot{\gamma} = 0$,
d.h.
\begin{align*}
\frac{\nabla}{\dt}\dot{\gamma} = 0.\fish
\end{align*}
\end{defn}

Die Existenz von Geodätischen sichert die kanonische Wahl für eine
Verbindung von $x$ und $y$. Viele Objekte, die von der Wahl eines Weges
abhängen, sind somit automatisch wohldefiniert. Später werden wir mit Hilfe der
Geodätischen die von Lie-Gruppen bekannte Exponentialabbildung auf Mannigfaltigkeiten
verallgemeinern.


\begin{rem}
In lokalen Koordinaten $(U,x)$ schreibt sich für eine
Kurve $\gamma\colon I\to M$ die Geodätengleichung $\ddot\gamma=0$  als
\begin{align*}
\ddot{\gamma}_k(t) + \sum_{i,j=1}^n \Gamma_{ji}^k (\gamma(t))\,
\dot{\gamma}_i(t)\, \dot{\gamma}_j(t) = 0,\qquad k=1,\ldots,n.
\end{align*}

Dies ist wiederum eine gewöhnliche, nichtlineare Differentialgleichung für
$\dot{\gamma}_{k}$. Aus dem allgemeinen Existenz- und Eindeutigkeitssatz folgt nun,
dass für jeden Punkt $p$ und jeden Anfangsgeschwindigkeitsvektor $X\in T_{p}M$ eine Geodätische durch $p$ für kurze Zeit existiert und eindeutig bestimmt ist.\map 
\end{rem}

\begin{ex}
\begin{exenum}
\item Sei $M=\R^n$ versehen mit der Standardmetrik und dem flachen
Zusammenhang, d.h. $\Gamma_{ji}^k \equiv 0$. Dann gilt
\begin{align*}
\frac{\nabla}{\dt} \dot\gamma = 0 \quad\iff\quad \ddot\gamma = 0 \quad\iff\quad
\gamma(t) = p+ t\cdot v,
\end{align*}
für $p\in \R^n$ und $v\in T_{p}\R^n \cong \R^n$.
\item Sei $M=S^n\subset\R^{n+1}$ die $n$-dimensionale Sphäre versehen mit der
von $\R^{n+1}$ induzierten Metrik und dem induzierten Zusammenhang. Dann liegen
die Geodätischen von $S^n$ auf Großkreisen. Zu zwei Punkten $x$, $y$ gibt es hier
also zwei verschiedene Geodätische eine >>kürzere<< und eine >>längere<<.\boxc
\end{exenum}
\end{ex}

Die Geodätischen hängen nach unserer Definition vom gewählten Zusammenhang ab. Damit diese also überhaupt ``kurz'' seien können, muss der Zusammenhang metrisch sein, und ``kurz'' ist dann bezüglich der entsprechenden Metrik zu verstehen.
Außerdem ist es zunächst nicht klar, warum diese Geodätischen überhaupt >>kürzeste<< Verbindungen
beschreiben. Tatsächlich minimieren diese Kurven auch nur lokal den Abstand,
aber nicht notwendigerweise global, wie man bereits am Beispiel der Sphäre
sieht.

\begin{lem}
Sei $(M,g)$ eine Riemannsche Mannigfaltigkeit mit dem Levi-Civita
Zusammenhang $\nabla$. Für eine Geodätische $\gamma$ ist die Funktion
\begin{align*}
f(t) = g(\dot{\gamma}(t),\dot{\gamma}(t))
\end{align*}
konstant.\fish
\end{lem}
\begin{proof}
Der Levi-Civita Zusammenhang ist metrisch und folglich gilt
\begin{align*}
\dot{f} = 2g(\ddot{\gamma},\dot\gamma) = 0.\qed
\end{align*}
\end{proof}

Die Länge der Tangentialvektoren einer Geodätischen oder Geschwindigkeitsvektoren, wie sie auch in
der Physik genannt werden,  ist also konstant. Man kann Geodätische
daher als gleichförmige Bewegung interpretieren. Insbesondere hängt die
Eigenschaft >>geodätisch<< zu sein von der Parametrisierung der Kurve ab. Eine
nichtgleichförmige Umparametrisierung wird diese Eigenschaft zerstören.

\begin{lem}
Sei $\gamma\colon I\to M$ eine nichtkonstante Geodätische. Eine Umparametrisierung $\gamma\circ h \colon J\to M$ ist genau dann eine Geodätische, wenn $h(t) = at + b$ gilt.\fish
\end{lem}
\begin{proof}
Aus der Geodätengleichung in lokalen Koordinaten erhalten wir,
\begin{align*}
\ddot{\gamma\circ h} &= \frac{\ddd^2 h}{\dt^2} \dot{\gamma}\circ h + 
\left( \frac{\ddd h}{\dt}\right)^2 \ddot{\gamma}\circ h\\
&=\frac{\ddd^2 h}{\dt^2} \dot{\gamma}\circ h,
\end{align*}
denn $\gamma$ ist geodätisch. Ist nun auch $\gamma\circ h$ eine Geodäte, dann verschwindet die linke Seite und da $\abs{\dot\gamma}\neq 0$ ist folglich $h(t) = at+b$. Umgekehrt folgt aus dieser Gestalt von $h$, dass die rechte Seite verschwindet und somit $\gamma\circ h$ eine Geodäte ist.\qed
\end{proof}

Wir wollen nun weiter untersuchen, inwiefern Geodätische >>kurz<< sind.

\begin{defn}
\index{Wegeraum}
\index{Energiefunktional}
Der \emph{Wegeraum} der glatten Wege von $x$ nach $y$ in $M$ ist definiert als
\begin{align*}
\Omega_{xy}(M) \defl \setdef{\gamma\colon[0,1]\to M}{\gamma(0)=x,\quad \gamma(1)=y}.
\end{align*} 
Das \emph{Energiefunktional} auf $\Omega_{xy}$ ist die Abbildung
\begin{align*}
E: \Omega_{xy}(M) \to \R,\qquad
E[\gamma] = \frac{1}{2} \int_0^1 g(\dot{\gamma}(t),\dot{\gamma}(t))\dt.\fish
\end{align*}
\end{defn}

\begin{prop}
Sei $(M,g)$ eine Riemannsche Mannigfaltigkeit versehen mit dem Levi-Civita-Zusammenhang.
Eine Kurve $\gamma\in \Omega_{xy}(M)$ ist ein kritischer Punkt von $E$ genau
dann, wenn $\gamma$ geodätisch ist.\fish
\end{prop}


Die Geodätischen sind also genau die kritischen Punkte des
Energiefunktionals. Allerdings schließt dies nicht aus, dass es sich um Maxima
handelt. Definiert man den Abstand zwischen zwei Punkten als
\begin{align*}
\dist(x,y) = \inf_{\gamma\in \Omega_{xy}} L(\gamma) = 2 \inf_{\gamma\in
\Omega_{xy}} E(\gamma),
\end{align*}
dann kann eine Geodätische auch die \textit{längstmögliche} Verbindung zwischen
$x$ und $y$ beschreiben. Schlimmer noch, eine Geodätische kann auch schlicht nur
ein kritischer Punkt von $E$ sein aber kein Extremum so wie es bei einem
Sattelpunkt der Fall ist. Dann minimiert $\gamma$ den Abstand auf einigen
Untermannigfaltigkeiten von $M$, während $\gamma$ den Abstand auf anderen
Untermannigfaltigkeiten maximiert.

\begin{rem}[Bemerkungen.]
\begin{remenum}
\item
Der Wegeraum $\Omega_{xy}(M)$ ist eine unendlichdimensionale
Untermannigfaltigkeit des Raumes der Kurven in $M$. Somit ist $E$ Funktional auf
einer unendlichdimensionalen Mannigfaltigkeit. Um besser zu verstehen, wie
kritische Punkte von $E$ zu interpretieren sind, betrachten wir zunächst den
endlichdimensionalen Fall. Sei eine Abbildung
\begin{align*}
f: N\to \R
\end{align*}
gegeben. Ein Punkt $z\in N$ ist ein kritischer Punkt von $f$ genau dann, wenn das Differential
$\df_z$ verschwindet. Dies ist äquivalent dazu, dass für jede Kurve $\eta\colon I\to N$ mit
$\eta(0) = z$ gilt,
\begin{align*}
\frac{\ddd}{\dt} f\circ\eta(t) = 0.
\end{align*}

In unseren Fall ist nun $N=\Omega_{xy}(M)$ und ein Punkt $z\in N$ entspricht
einer Kurve $\gamma$ von $x$ nach $y$. Wir betrachten jetzt \emph{Variationen}
von $\gamma$, das sind Abbildungen
\begin{align*}
(-\ep,\ep) \times I\to M,\qquad (s,t)\mapsto \gamma_s(t), 
\end{align*}
mit $\gamma_0\equiv \gamma$ und $\gamma_s\in\Omega_{xy}$. So ist $\gamma$ genau dann
ein kritischer Punkt von $E$, wenn
\begin{align*}
\frac{\ddd}{\ds} E[\gamma_s] = 0
\end{align*}
für alle Variationen $\gamma_s$ gilt.
\item 
Viele geometrische Strukturen lassen sich als kritische Punkt eines gewissen
Funktionals interpretieren, wie hier die Geodätischen. Im vergangen Semester
haben wir beispielsweise gesehen, dass die Einsteinmetriken gerade kritische
Punkte des Hilbertfunktionals sind.\map
\end{remenum}
\end{rem}

\begin{figure}[h]
\centering
\begin{pspicture}(0,-1.4)(3.6,1.4)
\psbezier[linecolor=darkblue](0.26,-0.04)(0.9,-1.26)(2.58,-0.6)(3.3,-0.02)
\psbezier(0.26,-0.04)(1.38,-0.34)(2.54,0.4)(3.3,-0.04)
\psbezier(0.28,-0.04)(2.04,-1.38)(1.74,1.38)(3.28,-0.04)
\psbezier(0.26,-0.04)(1.16,1.06)(2.64,1.14)(3.28,-0.04)
\psdots[linecolor=darkblue](0.28,-0.06)
\psdots[linecolor=darkblue](3.28,-0.04)
\rput(0.12,-0.175){\color{darkgray}$p$}
\rput(3.47,-0.135){\color{darkgray}$q$}

\rput(2.3,-0.835){\color{darkblue}$\gamma_0=\gamma$}
\rput(2.33,0.985){\color{darkgray}$\gamma_s$}
\end{pspicture} 
\caption{Eine Kurve $\gamma$ und einige Variationen.} 
\end{figure}

\begin{prop}[Variationsformel]
Sei $\gamma_{s}$ eine Variation von $\gamma$ und $\delta = \frac{\partial}{\partial s}\big|_{0}$. Dann gilt
\begin{align*}
\delta E[\gamma_{s}] = -\int g(\delta \gamma_{s},\ddot\gamma)\dt.\fish
\end{align*}
\end{prop}

Mit Hilfe der Variationsformel ergibt sich sofort, dass die Geodätischen genau die kritischen Punkte von $E$ sind.

\begin{cor}
Kürzeste Verbindungen sind Geodätische.\fish
\end{cor}
\begin{proof}
Die kürzeste Verbindung zwischen zwei Punkten minimiert das Energiefunktional und ist folglich ein kritischer Punkt von $E$, also eine Geodätische.\qed
\end{proof}

Die Umkehrung, also mit Geodätischen tatsächlich kürzeste Verbindungen zwischen zwei Punkten zu definieren, wird allerdings noch etwas Vorbereitung erfordern. Wir bemerken noch, dass Isometrien auch Geodätische erhalten, was nicht verwunderlich ist, denn nahezu alle Objekte, die über die Metrik definiert werden, werden auch durch Isometrien erhalten.

\begin{cor}
Isometrien überführen Geodätische in Geodätische.\fish
\end{cor}
\begin{proof}
Sei $f$ eine Isometrie und $\gamma$ eine Geodätische, dann gilt $\dot{f\circ \gamma} = \df(\dot{\gamma})$ und somit ist
\begin{align*}
E[f\circ \gamma] &= \frac{1}{2}\int_{0}^1 g(\df(\dot\gamma),\df(\dot\gamma))\dt
= \frac{1}{2}\int_{0}^1 (f^*g)(\dot\gamma,\dot\gamma)\dt\\
&= \frac{1}{2}\int_{0}^1 g(\dot\gamma,\dot\gamma)\dt
= E[\gamma].
\end{align*}
Somit ist $\gamma$ ein kritischer Punkt genau dann, wenn $f\circ\gamma$ ein kritischer Punkt ist.\qed
\end{proof}


\section{Die Exponentialabbildung}

Für eine Lie-Gruppe $G$ mit zugehöriger Lie-Algebra $\g$ kennen wir bereits die
Exponentialabbildung
\begin{align*}
\exp\colon \g \to G,\qquad X\mapsto \exp(X) = c_X(1),
\end{align*}
die $X$ mit der entsprechenden Integralkurve existiert. Letztere existieren auf Lie-Gruppen für
alle Zeiten und daher ist $\exp$ auf ganz $\g$ wohldefiniert. Die Exponentialabbildung ist
sogar ein lokaler Diffeomorphismus um Null und ermöglicht die Definition
besonders handlicher Kartenabbildungen, sogenannten Normalkoordinaten.

Wir wollen die Exponentialabbildung nun für allgemeine Mannigfaltigkeiten
definieren. Dazu verwenden wir die Geodätischen, denn lokal lassen sich zwei Punkte
eindeutig durch eine Geodätische verbinden und diese Verbindung hängt
differenzierbar von den Anfangswerten ab.

\begin{lem}
Sei $x\in M$ und $v\in T_xM$. Dann existiert eine Umgebung $V$ von $v$ und ein
Intervall $I$ um Null, so dass
\begin{align*}
 V\times I\to TM,\qquad (w,s)\mapsto \gamma_w(s)
\end{align*}
eine wohldefinierte, glatte Abbildung ist, wobei $\gamma_w$ die eindeutig
bestimmte Geodätische zu $w$ ist, d.h. $\gamma_w(0) = x$ und $\dot{\gamma}_w(0)
= w$.\fish
\end{lem}

\begin{proof}
Die Geodätische zu $w$ ist eindeutig durch eine gewöhnliche
Differentialgleichung bestimmt. Sie hängt damit differenzierbar
vom Anfangswert $w$ ab und wächst höchstens exponentiell schnell, d.h.
zu $v\in T_xM$ existiert eine Umgebung $V$ sowie ein Intervall $I$ um Null, so
dass $\gamma_w(s)$ auf $V\times I$ definiert und differenzierbar ist.\qed
\end{proof}

\begin{defn}
\index{Exponentialabbildung}
Sei $x\in M$ und $V_x\subset T_xM$ die Menge der Vektoren $v$, für die die
Geodätische $\gamma_v$ auf $I=[0,1]$ definiert ist. Die Abbildung
\begin{align*}
\exp_x : V_x\subset T_xM\to M,\qquad v\mapsto \exp_x(v) = \gamma_v(1)
\end{align*}
heißt \emph{Exponentialabbildung}.\fish
\end{defn}


\begin{rem}
Sei $x\in M$ und $v\in T_xM$. Dann gilt für
$t\ge 0$ hinreichend klein, dass
\begin{align*}
\exp_x(tv) = \gamma_v(t),
\end{align*}
denn $s\mapsto \gamma_v(st)$ ist die Geodätische zum Anfangsvektor $tv$ und
somit folgt aus der Eindeutigkeit, dass
\begin{align*}
\gamma_v(t) = \gamma_{tv}(1) = \exp_p(tv). 
\end{align*}

Im Gegensatz zur Exponentialabbildung auf Lie-Gruppen ist $\exp_p(tv)$ jedoch
im Allgemeinen nur für kleine $t$ definiert.\map 
\end{rem}

\begin{ex}
Sei $p\in S^n$ und $v\in T_pS^n = p^\bot$ mit $\abs{v} = 1$. Die Geodätische zu diesem
Anfangsvektor ist der Großkreis
\begin{align*}
\gamma_v(t) = \cos t\,\cdot p+ \sin t\, \cdot v. 
\end{align*}
Die Exponentialabbildung hat daher die Form
\begin{align*}
\exp_p : T_pS^n \to S^n,\qquad \exp(tv) = \cos t\cdot p + \sin t\cdot v.
\end{align*}
Somit ist $\exp_p$ ein Diffeomorphismus von
\begin{align*}
\setdef{v\in T_pS^n}{\abs{v}< \pi} \to S^n\setminus\setd{-p}.\boxc
\end{align*}
\end{ex}

Die Exponentialabbildung ist überall differenzierbar und wie ihr Pedant für Lie-Gruppen ein lokaler Diffeomorphismus um Null.

%TODO: Bild, Großkreise Sphäre

\begin{prop}
Zu jedem Punkt $x\in M$ existiert eine Umgebung von Null in $T_xM$, auf
der die Exponentialabbildung ein Diffeomorphismus auf ihr Bild ist.\fish
\end{prop}
\begin{proof}
Sei $V\subset T_xM$ hinreichend klein, dann ist
\begin{align*}
\exp_x : V\to M,
\end{align*}
wohldefiniert und glatt. Das Differential der Exponentialabbildung ist eine
Abbildung
\begin{align*}
\ddd \exp_p\colon T_0(T_xV)\cong T_xM \to T_xM,
\end{align*}
und für $v\in T_xM$ gilt
\begin{align*}
\ddd \exp_x(v) = \frac{\ddd}{\dt}\bigg|_0 \exp_x(tv)
= \frac{\ddd}{\dt}\bigg|_0 \gamma_v(t)
= v.
\end{align*}
Somit ist das Differential ein Isomorphismus und $\exp_x$ nach dem Umkehrsatz
ein lokaler Diffeomorphismus um Null.\qed
\end{proof}

\begin{figure}[h]
\centering
\begin{pspicture}(0,-1.4936931)(7.2658916,1.5136931)
\psbezier(4.5658913,-0.54630697)(5.4258914,-0.38630694)(6.2058916,-0.64630693)(6.625891,-1.266307)
\psbezier(5.0258913,0.35369304)(5.7658916,0.69369304)(6.9458914,0.19369306)(7.1658916,-0.42630696)
\psbezier(4.5858912,-0.56630695)(4.6658916,-0.14630695)(4.7658916,0.11369305)(5.0458913,0.35369304)
\psbezier(7.1858916,-0.42630696)(6.9258914,-0.42630696)(6.7458916,-0.68630695)(6.625891,-1.266307)
\psline(1.3258914,1.1736931)(1.3258914,-1.3463069)
\psline(0.22589143,-0.22630695)(2.5658915,-0.22630695)
\psbezier[linecolor=darkblue](1.6658914,0.41369304)(0.8658914,0.9136931)(1.3258914,0.09369305)(1.1058915,-0.18630695)(0.88589144,-0.46630695)(0.0,-0.45892084)(0.9258914,-0.9663069)(1.8517828,-1.473693)(2.4658914,-0.08630695)(1.6658914,0.41369304)

\rput(2.0158913,0.63869303){\color{darkblue}$\hat{V}$}
\psdots[linecolor=darkblue](1.3258914,-0.22630695)

\rput(2.5258913,-1.061307){\color{darkgray}$T_xM$}
\psbezier[linecolor=purple](5.4258914,0.09369305)(5.3258915,-0.40630695)(6.3858914,-0.88630694)(6.4258914,-0.26630694)(6.4658914,0.35369304)(5.5258913,0.5936931)(5.4258914,0.09369305)
\psdots[linecolor=purple](6.0858912,-0.12630695)

\rput(5.9858913,0.11869305){\color{darkgray}$x$}

\rput(6.5758915,-0.58130693){\color{purple}$U$}

\rput(7.0658913,-1.321307){\color{darkgray}$M$}
\psbezier[linecolor=darkyellow]{->}(2.1858914,1.053693)(3.3658915,1.2736931)(4.4058914,1.1536931)(5.0858912,0.81369305)

\rput(3.6458914,1.3986931){\color{darkyellow}$\exp$}
\psline[linecolor=darkblue](1.3458915,-0.22630695)(1.8058914,-0.846307)
\psbezier[linecolor=purple](6.0658913,-0.14630695)(5.6858916,-0.20630695)(5.7258916,0.15369305)(5.4258914,-0.00630695)
\end{pspicture} 
\caption{Die Exponentialabbildung auf sternförmigen Mengen.}
\end{figure}

Im Allgemeinen kann man nicht erwarten, dass die Exponentialabbildung ein globaler Diffeomorphismus ist, geschweige denn surjektiv oder auch nur auf ganz $T_{x}M$ definiert. Man muss sich zunächst mit kleinen Umgebungen der Null begnügen, die sich aber beliebig verkleinern lassen, zum Beispiel zu besonders handlichen,  sternförmigen Umgebungen.

\begin{defn}
\index{normale Umgebung}
Sei $p\in P$ und $\hat{V}\subset T_pM$ eine sternförmige Umgebung der Null. Ist
\begin{align*}
\exp_x : \hat{V}\to U = \exp_x(\hat{V})
\end{align*}
ein Diffeomorphismus, dann nennt man $U$ eine \emph{normale} Umgebung von
$x$.\fish
\end{defn}

Ist die Urbildmenge $\hat{U}$ sternförmig, dann ist auch die Bildmenge $U$
>>sternförmig<< in Bezug auf die Geodätischen.

\begin{lem} 
\index{Geodätische!radiale}
Sei $U\subset M$ eine normale Umgebung von $x\in M$. Dann existiert für jeden
Punkt $y\in U$ eine eindeutig bestimmte Geodätische
\begin{align*}
\sigma\colon I\to M,\qquad \dot{\sigma}(0) = \exp_x^{-1}(y),
\end{align*}
die ganz in $U$ verläuft und $x$ und $y$ verbindet. Sie heißt \emph{radiale Geodätische}.\fish
\end{lem}
\begin{proof}
Wir zeigen zunächst die \textit{Existenz}. Nach Voraussetzung ist
\begin{align*}
\exp_x \colon \hat{V}\to U
\end{align*}
ein Diffeomorphismus und $\hat{V}$ sternförmig. Zu $v= \exp_x^{-1}(y)$
setzen wir
\begin{align*}
\sigma(t) \defl \exp_x(tv).
\end{align*}
Dann ist $\sigma$ für $0\le t\le 1$ wohldefiniert und eine Geodätische, die ganz
in $U$ verläuft, mit $\dot{\sigma}(0) = v$.

Es verbleibt die \textit{Eindeutigkeit} zu zeigen. Sei $\tau\colon I\to M$ eine
weitere Geodätische, die $x$ und $y$ verbindet und sei $w = \dot{\tau}(0)$. Die
Geodätische $t\mapsto \exp_x(tw)$ hat denselben Anfangsvektor wie $\tau$ und
folglich gilt
\begin{align*}
\tau(t) = \exp_x(tw),\qquad t\in I.
\end{align*}
Falls $tw$ in $\hat{V}$ liegt für $0\le t\le 1$, dann gilt
\begin{align*}
\exp_x(w) = \tau(1) = y = \exp_x(v),
\end{align*}
und da die Exponentialabbildung auf $\hat{V}$ injektiv ist, folgt $w = v$. 
Andernfalls existiert ein $t_0 > 0$, so dass $tw$ in $\hat{V}$ liegt für $0\le
t\le t_0$ und
\begin{align*}
\exp_x(t_0 w) = \tau(t_0) = y = \exp_x(v).
\end{align*} 
Somit gilt $t_0w = v = \exp_x^{-1}(y)$. Parametrisieren wir $\tau$
gleichförmig so um, dass $\dot{\tau}(0) = v$ ist, dann gilt auch $\tau\equiv
\sigma$.\qed
\end{proof}

Wählt man als sternförmige Umgebung eine Kugel um Null, dann werden die radialen Geodätischen tatsächlich zu kürzesten Verbindungen.

\begin{prop}
\index{Injektivitätsradius}
Sei $\ep > 0$ so, dass $\exp_{x}$ auf der Kugel $B_{\ep}\subset T_{x}M$ vom Radius $\ep$ um Null ein Diffeomorphismus ist. Dann existiert für jedes $y\in \exp_{x}(B_{\ep})$ genau eine kürzeste Verbindung von $x$ nach $y$, nämlich die radiale Geodätische. 
Das größtmögliche $\ep$ nennt man den \emph{Injektivitätsradius} der Exponentialabbildung.\fish
\end{prop}


Nun stellt sich die Frage, ob zwei beliebige Punkte auf $M$ durch eine Geodätische verbunden werden können.

\begin{prop}
Sei $(M,g)$ eine zusammenhängende Riemannsche Mannigfaltigkeit. Dann können je zwei Punkte durch eine gebrochene Geodätische, also Kurve, die stückweise geodätisch ist, verbunden werden.\qed
\end{prop}
\begin{proof}
Zu einem beliebigen Punkt $x\in M$ setzen wir
\begin{align*}
\Cs \defl \setdef{y\in M}{\text{es existiert eine gebrochene Geodätische }\gamma_{{xy}}}.
\end{align*}
Dann ist $\Cs$ offen, denn falls $z\in \Cs$, dann lassen sich $x$ und $z$ durch eine gebrochene Geodätische verbinden. Sei $U(z)$ eine normale Umgebung von $z$, dann lässt sich diese gebrochene Geodätische zu jedem Punkt in $U(z)$ fortsetzen.
Umgekehrt ist aus genau demselben Argument auch das Komplement von $\Cs$ offen. Aber $\Cs$ ist nichtleer und $M$ zusammenhängend, also gilt $\Cs = M$.\qed
\end{proof}

Die Aussage ist natürlich höchst unbefriedigend, denn schon im $\R^n$ können gebrochene Geodätische komplizierte Gebilde werden, obwohl wir im $\R^n$ die kürzesten Verbindungen genau kennen, es sind gerade Strecken. Ist die Exponentialabbildung jedoch an einem Punkt global definiert, dann lässt sich die Aussage erheblich verbessern.

\begin{prop}
Sei $M$ eine zusammenhängende Riemannsche Mannigfaltigkeit und $x\in M$. Sei $\exp_{x}$ auf ganz $T_{x}M$ definiert, dann gibt es für jedes $y\in M$ eine kürzeste Geodätische, die $x$ und $y$ verbindet.\fish
\end{prop}

\begin{rem}
In diesem Fall ist die Exponentialabbildung surjektiv. Die für Lie"=Gruppen definierte ``klassische'' Exponentialabbildung ist immer auf ganz $T_{e}M = \g$ definiert, jedoch gibt es Lie-Gruppen auf denen die Exponentialabbildung nicht surjektiv ist. Somit stimmen die mittels Geodätischen definierte Exponentialabbildung und die ``klassische'' Exponentialabbildung für Lie-Gruppen im Allgemeinen nicht überein! Ist die Lie-Gruppe jedoch zusammenhängend und trägt eine bi-invariante Metrik, dann ist wieder alles in Ordnung, wie wir noch sehen werden.\map 
\end{rem}

Der folgende Satz liefert nun Kriterien dafür, dass die Exponentialabbildung global definiert ist.

\begin{prop}[Satz von Hopf-Rinow]
Sei $(M,g)$ eine zusammenhängende Riemannsche Mannigfaltigkeit. Dann sind folgende Aussagen äquivalent:
\begin{equivenum}
\item Für ein $x\in M$ ist die Exponentialabbildung $\exp_{x}$ auf ganz $T_{x}M$ definiert.
\item Der metrische Raum $(M,d)$ ist vollständig, wobei
\begin{align*}
d(x,y) = \inf_{\gamma\in\Omega_{xy}} L(\gamma).
\end{align*}
\item Die Mannigfaltigkeit $(M,g)$ ist geodätisch vollständig, d.h. jede Geodätische ist auf ganz $\R$ definiert.\fish
\end{equivenum}
\end{prop}

\begin{rem}
Die Vollständigkeit der Mannigfaltigkeit ist tatsächlich notwendig. Betrachtet man beispielsweise $M=\R^n\setminus\setd{0}$, dann ist $M$ nicht vollständig und offenbar existieren keine Geodätischen zwischen zwei Punkten, deren Verbindungsstrecke durch  den Nullpunkt gehen würde.\map
\end{rem}

\section{Normalkoordinaten}

Sei $(M,g)$ eine Riemannsche Mannigfaltigkeit. Mit Hilfe der
Exponentialabbildung wollen wir nun spezielle Karten von $M$ definieren.

\begin{defn}
\index{Normalkoordinaten}
Sei $x\in M$ und $U$ eine normale Umgebung von $x$. Wählen wir eine
Orthonormalbasis $\setd{e_i}$ von $T_xM$, dann nennt man die Kartenabbildung
\begin{align*}
\ph\colon U\to \R^n,\qquad
y \mapsto (y_1,\ldots,y_n),\quad \exp_x^{-1}(y) = \sum_{i=1}^n y_i e_i,
\end{align*}
\emph{Normalkoordinaten} auf $U$.\fish
\end{defn}

Wählen wir Normalkoordinaten um $x$, dann nimmt in $x$ die Metrik eine besonders
einfache Form an und die Christoffel-Symbole verschwinden. Somit sind
Normalkoordinaten für viele Rechnungen besonders angenehm.

\begin{prop}
Seien $(x_1,\ldots,x_n)$ Normalkoordinaten von $x$. Dann gelten
\begin{propenum}
\item $g_{ij}(x) = \delta_{ij}$, und
\item $\Gamma_{ij}^k(x) = 0$.\fish
\end{propenum} 
\end{prop}
\begin{proof}
a): Zunächst ist nach Definition,
\begin{align*}
g_{ij}(x) = g_p\left(\frac{\partial}{\partial
x_i}\bigg|_p,\frac{\partial}{\partial x_j}\bigg|_x \right).
\end{align*}
Weiterhin berechnet man die Koordinatenvektorfelder zu
\begin{align*}
\frac{\partial}{\partial x_i}\bigg|_x = \frac{\ddd}{\dt}\bigg|_0 \ph^{-1}(t e_i)
= \frac{\partial}{\partial x_i}\bigg|_0 \exp_x(t e_i) = e_i.
\end{align*}
Somit ist $g_{ij}(p) = g_x(e_i,e_j) = \delta_{ij}$, denn $\setd{e_i}$ ist
orthonormal.

b): Sei $\gamma_v(t) = \exp_x(tv)$ die Geodätische zum Anfangsvektor $v\in
T_pM$. Dann schreibt sich 
\begin{align*}
v = \sum_{i=1}^n a_i e_i
\end{align*}
und folglich ist
\begin{align*}
x(\gamma_v(t)) = (a_1t,\ldots,a_n t).
\end{align*}
Die $i$-te Koordinate von $\gamma_v$ ist somit gegeben als $(\gamma_v)_i(t) =
a_i t$ und daher gilt
\begin{align*}
(\dot{\gamma}_v)_i = a_i,\qquad (\ddot{\gamma}_v)_i = 0.
\end{align*}
Aus der Geodätengleichung folgt nun, dass 
\begin{align*}
0 = \sum_{i,j=1}^n \Gamma_{ij}^k(x) a_i a_j,\qquad 1\le k\le n,
\end{align*}
für jede Wahl von $v$. Also gilt $\Gamma_{ij}^k(x) = 0$ für $1\le i,j,k\le
n$.\qed
\end{proof}

\section{Der Geodätische Fluss}

Eine Kurve $\gamma\colon I\to M$ ist eine Geodätische genau dann, wenn in jeder Karte $(U,x)$ die Geodätengleichungen erfüllt sind
\begin{align*}
\ddot\gamma_{k} + \sum_{i,j=1}^n \dot\gamma_{i}\dot\gamma_{j} \Gamma_{ij}^k = 0,\qquad k=1,\ldots,n.\tag{*}
\end{align*}
Wir wollen nun ein Vektorfeld auf dem Tangentialbündel definieren, dessen Integralkurven gerade die Geodätischen sind.

Jede Karte $(U,x)$ auf $M$ induziert eine Karte auf $\pi^{-1}(U)\subset TM$ durch
\begin{align*}
(x_{1},\ldots,x_{n},y_{1},\ldots,y_{n})\quad \hat{=}\quad \sum_{i=1}^n y_{i}\frac{\partial}{\partial x_{i}}\bigg|_{x}.
\end{align*} 
Setzen wir $x_{k} \defl \gamma_{k}$ und $y_{k}\defl \dot\gamma_{k}$, dann gilt (*) genau dann, wenn für $k=1,\ldots,n$ die Differentialgleichungen
\begin{align*}
\dot x_{k} &= y_{k},\\
\dot y_{k} &= -\sum_{i,j=1}^n y_{i} y_{j} \Gamma_{ij}^k,
\end{align*}
erfüllt sind. Definieren wir also auf $TM$ ein Vektorfeld
\begin{align*}
G_{(x_{1},\ldots,x_{n},y_{1},\ldots,y_{n})} \defl \sum_{k=1}^n y_{k} \frac{\partial}{\partial x_{k}} - 
\sum_{k,i,j=1}^n  y_{i}y_{j}\Gamma_{ij}^k \frac{\partial}{\partial y_{k}},
\end{align*}
dann sind die Integralkurven zu $G$ gerade die Geodätischen in $M$.

\begin{rem}
Auf der symplektischen Mannigfaltigkeit $T^*M$ betrachtet man die Hamiltonfunktion
\begin{align*}
H(q,p) = \frac{1}{2}\abs{p}^2,\qquad p\in T_{q}^*M.
\end{align*}
Dann sind die Geodätischen auf $M$ die Lösungen der Hamiltonschen Gleichungen
\begin{align*}
\dot p_{j} = -\frac{\partial H}{\partial q_{j}},\qquad \dot q_{j} = \frac{\partial H}{\partial p_{j}},
\end{align*}
mit zugehörigem Hamilton Vektorfeld
\begin{align*}
X_{H} = \sum_{i=1}^n \left( -\frac{\partial H}{\partial q_{i}}\frac{\partial}{\partial p_{i}}
+ \frac{\partial H}{\partial p_{i}}\frac{\partial}{\partial q_{i}}\right).
\end{align*}
Die Metrik definiert einen Diffeomorphismus zwischen dem Tangential- und dem Cotangentialbündel
\begin{align*}
\psi : TM\to T^*M,\qquad X\mapsto g(X,\cdot).
\end{align*}
Dieser Diffeomorphismus identifiziert $X_{H}$ mit $G$,
\begin{align*}
\psi_{*}G = X_{H}.\map
\end{align*}
\end{rem}

\section{Geodätische und Exponentialabbildung auf Lie-Gruppen}

Sei $G$ eine Lie-Gruppe mit Lie-Algebra $\g$. Im vergangenen Semester haben wir bereits eine Exponentialabbildung für Lie-Gruppen definiert,
\begin{align*}
\exp_{G}\colon \g\to G,\qquad X\mapsto \exp_{G}(X),
\end{align*}
die ein linksinvariantes Vektorfeld $X$ auf den Wert der zugehörigen Integralkurve zum Zeitpunkt $t=1$ abbildet. Für die Lie-Gruppen sind die Integralkurven für alle Zeiten definiert und somit ist auch die Exponentialabbildung auf ganz $\g$ wohldefiniert.
Das Ziel dieses Abschnittes ist es zu untersuchen, wie diese Exponentialabbildung mit der Exponentialabbildung, die wir mittels Geodätischen definiert haben, zusammenhängt.

\begin{defn}
Eine stetige Abbildung $T\colon \R\to G$ heißt \emph{1-Parameter Untergruppe von $G$}, falls
\begin{align*}
T(t+s) = T(t)T(s)\quad\text{ für alle }t,s\in\R.\fish
\end{align*}
\end{defn}

Für eine 1-Parameter Untergruppe $T$ von $G$ gilt offenbar $T(0)=e$. Weiterhin ist $T$ automatisch differenzierbar und es gilt
\begin{align*}
T(t) = \exp_{G}(tX),\qquad X=\dot{T}(0)\in\g,
\end{align*}
wobei $\exp_{G}$ die klassische Exponentialabbildung bezeichnet. Die 1-Parameter Untergruppen entsprechen also der ``klassischen'' Exponentialabbildung für Lie-Gruppen. Um zu untersuchen, wann diese mit der allgemeinen Exponentialabbildung zusammenfällt, benötigen wir noch ein Hilfsmittel, um den Levi-Civita Zusammenhang mittels Isometrien zu übertragen.

\begin{lem}
Seien $(M,g_{M})$ und $(N,g_{N})$ Riemannsche Mannigfaltigkeiten versehen mit dem Levi-Civita-Zusammenhang.
Sei $\phi\colon M\to N$ eine Isometrie, dann gilt für Vektorfelder $X$ und $Y$ in $M$,
\begin{align*}
\dphi(\nabla_{X}^M Y) = \nabla_{\dphi(X)}^N(\dphi(Y)),
\end{align*}
wobei $\dphi(X)_{q} = \dphi(X_{p})$ für alle $q = \phi(p)\in N$ gilt.\fish
\end{lem}

Es stellt sich nun heraus, dass für zusammenhängende Lie-Gruppen mit bi-invarianter Metrik die Geodätischen und die 1-Parameteruntergruppen zusammenfallen. 

\begin{prop}
Sei $G$ eine zusammenhängende Lie-Gruppe mit einer linksinvarianten Metrik $\lin{\cdot,\cdot}$. Dann sind folgende Aussagen äquivalent:
\begin{equivenum}
\item Die Metrik $\lin{\cdot,\cdot}$ ist rechts- also bi-invariant.
\item Die Metrik $\lin{\cdot,\cdot}$ ist $\Ad(G)$-invariant.
\item Die Inversion $g\mapsto g^{-1}$ ist eine Isometrie.
\item $\lin{X,[Y,Z]} + \lin{[Y,X],Z} = 0$ für alle $X,Y$ und $Z\in \g$.
\item $\nabla_{X} Y = \frac{1}{2}[X,Y]$ für alle $X$ und $Y\in \g$.
\item $\nabla_{X} X = 0$ für alle $X\in \g$.
\item Die Geodätischen durch das Einselement von $G$ sind genau die 1-Parameter Untergruppen von $G$.\fish
\end{equivenum}
\end{prop}

\begin{proof}
Die Äquivalenzen (i)$\Leftrightarrow$(ii)$\Leftrightarrow$(iii)$\Leftrightarrow$(iv)$\Leftrightarrow$(v)$\Leftrightarrow$(vi) wurden bereits gezeigt.

(vi)$\Rightarrow$(vii): 
Sei $\alpha$ eine 1-Parameteruntergruppe von $G$. Dann existiert ein $X\in \g$ so, dass $\alpha(t) = \exp(tX)$. Folglich ist
$\alpha(0) = e$ und $\dot{\alpha}(t) = X_{\alpha(t)}$ und daher ist $\dot{\alpha}(t)$ ein Vektorfeld entlang $\alpha$. Somit gilt
\begin{align*}
\frac{\nabla}{\dt} \dot{\alpha}(t) = \nabla_{X}X\bigg|_{\alpha} = 0,
\end{align*}
und daher ist $\alpha$ eine Geodätische. Dies sind aber auch schon alle möglichen Geodätischen, denn diese sind durch $\gamma(0) = e$ und $\dot{\gamma}(0) = X$ bereits ohne Ausnahme eindeutig bestimmt.

(vi)$\Leftarrow$(vii): Seien die Geodätischen genau die 1-Parametergruppen. Zu $X\in \g$ existiert daher eine Geodätische durch $e$ der Form
\begin{align*}
\gamma(t) = \exp_{G}(tX).
\end{align*}
Dann ist $\dot{\gamma}(t) = X_{\gamma(t)}$ ein Vektorfeld entlang $\gamma$ und somit nach der Geodätengleichung
\begin{align*}
0 = \frac{\nabla}{\dt} \dot{\gamma}(t) = 
\frac{\nabla}{\dt} X_{\gamma(t)} = 
\nabla_{X}X\bigg|_{\gamma(t)}.
\end{align*}
Also verschwindet $\nabla_{X}X$ entlang $\gamma$. Sei nun $q=\gamma(t)$ und $p\in G$ beliebig. Dann existiert ein $g\in \G$ so, dass $p = g\cdot q = L_{g}(q)$. Somit ist
\begin{align*}
(\nabla_{X}X)_{p} = (\dL_{g}\nabla_{X}X)_{q}
= (\nabla_{\dL_{g}X}\dL_{g}X)_{q} = 0,
\end{align*}
denn die Linkstranslation ist eine Isometrie der Metrik und $X$ ist linksinvariant. Also gilt $\nabla_{X}X\equiv 0$, was zu zeigen war.\qed
\end{proof}

\section{Geodätische auf Untermannigfaltigkeiten}

Sei $(M,g)$ eine Riemannsche Untermannigfaltigkeit von $(\bar{M},\bar{g})$, d.h. das Tangentialbündel an $\bar{M}$ spaltet sich auf in das Tangentialbündel an $M$ und das Normalenbündel an $M$, und die Metrik auf $M$ ist durch Einschränkung gegeben,
\begin{align*}
T\bar{M} = TM\oplus NM,\qquad g = \bar{g}\bigg|_{TM\times TM}.
\end{align*}

Der Levi-Civita Zusammenhang auf $M$ lässt sich mit Hilfe der 2. Fundamentalform über den Levi-Civita Zusammenhang auf $\bar{M}$ beschreiben,
\begin{align*}
\bar{\nabla}_{X}Y = \nabla_{X}Y + \II(X,Y),\qquad X,Y\in \chi(M).\tag{*}
\end{align*}
Dabei sind die Vektorfelder $X$ und $Y$ zunächst nur auf $M$ definiert, obiger Ausdruck hängt aber nicht von der Fortsetzung zu Vektorfeldern auf $\bar{M}$ ab und ist daher wohldefiniert.

Sei $\gamma\colon I\to M$ eine Kurve in $M$ und $Y$ ein Vektorfeld entlang $\gamma$. Wir bezeichnen die kovarianten Ableitungen entlang $\gamma$ wie folgt,
\begin{align*}
\frac{\bar{\nabla}}{\dt} Y = \dot{Y},\qquad
\frac{\nabla}{\dt} Y = Y'.
\end{align*}

Nun suchen wir nach einer Möglichkeit, analog zu (*), auch die kovariante Ableitung entlang $\gamma$  in $M$ durch die kovariante Ableitung entlang $\gamma$ in $\bar{M}$ auszudrücken.

\begin{lem}
Sei $Y$ ein Vektorfeld in $M$ entlang $\gamma$. Dann gilt
\begin{align*}
\dot{Y} = Y' + \II(\dot{\gamma},Y).
\end{align*}
\end{lem}
\begin{proof}
Sei $(U,x)$ eine Karte von $M$, dann schreibt sich $Y$ lokal in $U$ als
\begin{align*}
Y(t) = \sum_{i=1}^n Y_{i}(t) \frac{\partial}{\partial x_{i}}\bigg|_{\gamma(t)}.
\end{align*}
Für die eingeschränkte kovariante Ableitung ergibt sich dort,
\begin{align*}
\frac{\bar{\nabla}}{\dt} Y &= 
\sum_{i=1}^n \left( \dot{Y}_{i}\frac{\partial}{\partial x_{i}}
+
Y_{i} \bar{\nabla}_{\dot{\gamma}} \frac{\partial}{\partial x_{i}}\right)\bigg|_{\gamma(t)}\\
&= 
\sum_{i=1}^n \left( \dot{Y}_{i}\frac{\partial}{\partial x_{i}}
+
Y_{i} \left(\nabla_{\dot{\gamma}} \frac{\partial}{\partial x_{i}} + \II\left(\dot{\gamma},\frac{\partial}{\partial x_{i}}\right)\right)\right)\bigg|_{\gamma(t)}\\
&= 
\sum_{i=1}^n \left( \dot{Y}_{i}\frac{\partial}{\partial x_{i}}
+
Y_{i} \nabla_{\dot{\gamma}}\right) + \II \left(\dot{\gamma},\sum_{i=1}^n Y_{i}\frac{\partial}{\partial x_{i}}\right)\bigg|_{\gamma(t)}\\
&= 
\frac{\nabla}{\dt}Y + \II \left(\dot{\gamma},Y\right).\qed
\end{align*}
\end{proof}

\begin{cor}
Sei $\gamma\colon I\to M$ eine Kurve in $M$, dann gilt
\begin{align*}
\ddot{\gamma} = \gamma'' + \II(\dot{\gamma},\dot{\gamma}).
\end{align*}
Somit ist $\gamma$ genau dann eine Geodätische in $M$, wenn $\ddot{\gamma}$ normal ist.\fish
\end{cor}

\begin{ex}
Nach dieser Vorbereitung können wir zeigen, dass die Geodätischen auf der $n$-dimensionalen Sphäre $S^n$ exakt die Großkreise sind.

Ein \emph{Großkreis} ist eine geschlossene, doppelpunktfreie und nach Bogenlänge parametrisierte Kurve in $S^n$, deren Spur durch $E\cap S^n$ gegeben ist, wobei $E\subset\R^{n+1}$ eine 2-dimensionale Ebene durch Null bezeichnet.

Sei $\alpha$ ein Großkreis, dann ist aufgrund der Parametrisierung nach Bogenlänge $\abs{\dot{\alpha}}\equiv 1$. Die Beschleunigung kann demnach an keinem Punkt einen Betrag in Richtung des Geschwindigkeitsvektors liefern, d.h. $\dot{\alpha}\,\bot\, \ddot{\alpha}$. Andererseits ist $T_{p}S^n = p^\bot$ für jedes $p\in S^n$, also gilt auch $\alpha\,\bot\, \dot\alpha$. Nun verläuft die Kurve ganz in der 2-dimensionalen Ebene $E$, also sind $\alpha$, $\dot\alpha$ und $\ddot\alpha$ tangential an $E$ und folglich gilt
\begin{align*}
\ddot\alpha \parallel \alpha.
\end{align*}
Somit ist $\ddot\alpha$ normal und $\alpha$ nach dem vorangegangen Korollar eine Geodätische.

Umgekehrt sei $\gamma$ eine nach Bogenlänge parametrisierte Geodätische, d.h. $\abs{\dot\gamma}\equiv 1$. Setzen wir
\begin{align*}
E = \span\setd{\gamma(0),\dot\gamma(0)},
\end{align*}
dann ist $E$ eine 2-dimensionale Ebene durch Null und die Kurve
\begin{align*}
\sigma(t) = \cos t \cdot\gamma(0) + \sin t \cdot\dot\gamma(0)
\end{align*}
verläuft ganz in $E$, ist also ein Großkreis. Weiterhin ist $\sigma(0) = \gamma(0)$ und $\dot\sigma(0) = \dot\gamma(0)$, also gilt aufgrund der Eindeutigkeit der Geodätischen
\begin{align*}
\gamma = \sigma,
\end{align*}
und $\gamma$ selbst ist ein Großkreis.\boxc
\end{ex}

\chapter{Holonomietheorie}

\section{Die Holonomie-Gruppe}

Sei $P\to M$ ein $G$-Hauptfaserbündel mit einem fixierten Zusammenhang $A$. Eine
geschlossene Kurve  $\gamma\colon I\to M$ durch den Punkt $x\in M$ besitzt für jedes $p\in
P_x$ einen eindeutigen horizontalen Lift $\gamma^*_p$ durch $p$. Mit diesem Lift haben wir die Parallelverschiebung in $P$ definiert,
\begin{align*}
P_{\gamma}^A : P\to P,\qquad p\mapsto \gamma_{p}^*(b).
\end{align*}
Wie bereits festgestellt, ist $\gamma^*_p$ nicht notwendigerweise geschlossen und die >>Differenz<< von Anfangs- und Endwert der Kurve $\gamma_p^*$ hängt von der Kurve $\gamma$ selbst ab. Dies führt dazu, dass im Allgemeinen die Parallelverschiebung von der Wahl des Weges abhängt. Wir wollen die Menge der möglichen Parallelverschiebungen eines Punktes in $P$ nun genauer untersuchen.

\begin{defn}
Sei $x$ ein Punkt in $M$. Die Menge der \emph{geschlossenen Kurven durch $x$}
bezeichnen wir mit \emph{$\Omega(x)$} und die Menge der \emph{nullhomotopen,
geschlossenen Kurven durch $x$} mit \emph{$\Omega^\o(x)$}.\fish
\end{defn}

Geschlossene Kurven durch einen Punkt $x$ nennt man auch \emph{Schleifen} an $x$.

\begin{defn}
\index{Holonomiegruppe}
Sei $p\in P_x$ für ein $x\in M$.
\begin{defnenum}
\item
Die \emph{Holonomiegruppe} von $A$ an $p$ ist definiert als
\begin{align*}
\Hol_p (A) \defl \setdef{g\in G}{
P_\gamma^A(p) = p\cdot g\text{ für ein }\gamma\in \Omega(x)} \subset G
\end{align*}
\item
Die \emph{reduzierte Holonomiegruppe} an $p$ ist definiert als
\begin{align*}
\Hol_p^\o (A) \defl \setdef{g\in G}{
P_\gamma^A(p) = p\cdot g\text{ für ein }\gamma\in \Omega_\o(x)} \subset
\Hol_p(A).\fish
\end{align*}
\end{defnenum}
\end{defn}


\begin{figure}[h]
\centering
\begin{pspicture}(0,-2.19)(3.5531259,2.21)

\rput(3.0531259,-1.845){\color{darkgray}$M$}
\rput(3.0731258,2.035){\color{darkgray}$P$}
\rput(3.3831258,0.135){\color{darkgray}$\pi$}
\psline{->}(3.0331259,1.73)(3.0331259,-1.47)

\psbezier[linecolor=darkblue](0.7731259,0.51)(1.2731259,0.47)(2.273126,0.45)(2.273126,0.77)
\psbezier[linecolor=darkblue,linestyle=dashed,dash=0.06cm 0.06cm]%
(2.273126,0.75)(2.153126,0.95)(0.3331259,0.79)(0.19312589,1.05)
\psbezier[linecolor=darkblue](0.19312589,1.09)(0.3331259,1.39)(0.61312586,1.29)(0.8131259,1.29)

\psellipse(1.2431259,1.71)(1.07,0.24)
\psbezier(0.19312589,-0.27)(0.4131259,-0.55)(2.193126,-0.53)(2.273126,-0.25)
\psline(0.19312589,1.69)(0.19312589,-0.29)
\psline(2.289126,1.69)(2.269126,-0.27)
\psbezier[linecolor=purple](0.7731259,1.51)(0.9131259,0.79)(0.61312586,0.19)(0.79312587,-0.43)
\psbezier[linecolor=darkblue](0.1731259,-1.71)(0.0,-2.1656544)(0.73312587,-1.73)(1.4531258,-1.95)(2.173126,-2.17)(2.673126,-1.61)(2.1131258,-1.47)(1.5531259,-1.33)(0.3462518,-1.2543457)(0.1731259,-1.71)

\psdots[linecolor=darkyellow](0.7731259,0.51)
\psdots[linecolor=darkyellow](0.8131259,1.29)


\rput(0.5731259,0.535){\color{darkyellow}$p$}
\rput(1.33,1.315){\color{darkyellow}$p\cdot g$}
\psdots[linecolor=darkyellow](0.79312587,-1.89)

\rput(0.9031259,-2.045){\color{darkyellow}$x$}

\rput(1.5631258,-1.145){\color{darkblue}$\gamma$}

\rput(0.86312586,1.695){\color{purple}$P_x$}
\end{pspicture} 
\caption{Zur Holonomiegruppe.}
\end{figure}

% Vorlesung vom 04. Juni 2011

In analoger Weise lässt sich die Holonomiegruppe auch in Vektorbündeln
definieren.

\begin{defn}
\index{Holonomiegruppe!in Vektorbündeln}
Sei $E\to M$ ein Vektorbündel über $M$ mit kovarianter Ableitung $\nabla$.
\begin{defnenum}
\item
Die \emph{Holonomiegruppe von $\nabla$ an $x\in M$}
definiert als
\begin{align*}
\Hol_x(\nabla) \defl \setdef{P_\gamma^\nabla}{\gamma\in
\Omega(x)}\subset \Aut(E_x)
\end{align*}
\item Die \emph{reduzierte  Holonomiegruppe} von $\nabla$ in $x$ ist definiert
als
\begin{align*}
\Hol_x^\o(\nabla) \defl \setdef{P_\gamma^\nabla}{\gamma\in
\Omega_\o(x)}.
\end{align*}
\end{defnenum}
\end{defn}

Die Definition der Holonomiegruppe für Vektorbündel und für Hauptfaserbündel sind äquivalent und lassen sich wie folgt ineinander überführen.

\begin{prop}
Sei $E= P\times_\rho V$ ein zu $P$ assoziiertes Vektorbündel mit induzierter kovarianter Ableitung $\nabla$. Dann gilt für $x\in M$,
\begin{align*}
\Hol_x(\nabla) = [p] \rho(\Hol_p(A))[p]^{-1},\qquad p\in P_x,
\end{align*}
wobei $[p]\colon V\to E_x$ den Faserisomorphismus und $\rho\colon G\to \GL(V)$ eine
$G$-Darstellung bezeichnet. Eine analoge Aussage gilt f\"ur $\Hol_{x}^{\o}(\nabla)$.\fish
\end{prop}
\begin{proof}
Sei $\gamma\in \Omega(x)$ und $[p,v]\in E$. Da $\gamma$ geschlossen ist, bildet
$P_\gamma^A$ die Faser $P_x$ auf die Faser $P_x$ ab und es gilt $P_\gamma^A(p)
= p\cdot g_\gamma$ mit einem Element $g_\gamma\in G$. Nach Definition der
induzierten Parallelverschiebung gilt
\begin{align*}
P_\gamma^\nabla([p,v]) &= [P_\gamma^A(p),v]
= [p\cdot g_\gamma, v]
= [p\cdot g_\gamma, \rho(g_\gamma)^{-1} \rho(g_\gamma)v]\\
&= [p, \rho(g_\gamma)v].
\end{align*}
Somit ist $P_{\gamma}^{\nabla}  = [p] \circ \rho(g_{\gamma}) \circ [p]^{-1}$, was zu zeigen war.\qed
\end{proof}

Bisher haben wir noch nicht gezeigt, dass es sich bei der Holonomiegruppe auch tats\"achlich um eine Gruppe handelt. Die algebraischen Gruppeneigenschaften folgen sofort aus den Rechenregeln f\"ur die Parallelverschiebung. Allerdings ist die Holonomiegruppe im Allgemeinen nicht abgeschlossen, daher erfordert es etwas mehr Anstrengung, um zu zeigen, dass es sich tats\"achlich um eine Lie-Gruppe von $G$ handelt.

\begin{lem}
\begin{propenum}
\item F\"ur jedes $p\in P$ sind $\Hol_{p}(A)$ und $\Hol_{p}^{\o}(A)$ algebraische Untergruppen von $G$ und es gilt,
\begin{align*}
\Hol_{p}^{\o}(A)\subset \Hol_{p}(A) \subset G.
\end{align*}
\item Für jedes $p\in P$ ist $\Hol_{p}^\o(A)$ ein Normalteiler von $\Hol_{p}(A)$.
\item Seien $p\in P$ und $g\in \G$, dann gilt
\begin{align*}
\Hol_{p\cdot g}(A) = g^{-1}\Hol_{p}(A)g,\qquad
\Hol_{p\cdot g}^\o(A) = g^{-1}\Hol_{p}^\o(A)g.\fish
\end{align*}
\end{propenum}
\end{lem}

Aus Eigenschaft c) folgt unmittelbar, dass die Holonomiegruppe bis auf Konjugation nicht vom gewählten Punkt in der Faser $P_{x}$ abhängt. Sie sieht also im Wesentlichen auf der gesamten Faser gleich aus.

\begin{proof}
a): Die Inklusionsfolge ist klar. Seien nun $\gamma$ und $\delta$ geschlossene Wege durch $x$ und $p\in P_{x}$. Dann gilt
\begin{align*}
p\cdot g_{\gamma\star\delta} &= P_{\gamma\star \delta}^{A}(p) = P_{\gamma}^{A}\circ P_{\delta}^{A}(p) = 
P_{\gamma}^A(p\cdot g_{\delta}) \\
 &= P_{\gamma}^A(p)\cdot g_{\delta} = p\cdot g_\gamma \cdot g_\delta.
\end{align*}
denn der horizontale Lift ist rechtsinvariant. Somit ist $g_{\gamma\star\delta} = g_{\gamma}\cdot g_{\delta}$, und folglich ist die Holonomiegruppe abgeschlossen unter die Multiplikation von $G$. Analog berechnet man, dass die Inverse zu $g_{\gamma}$ gerade durch $g_{\gamma^-}$ gegeben ist. Also ist die Holonomiegruppe tatsächlich eine algebraische Untergruppe von $G$.

b): Sei $\gamma$ eine geschlossene Kurve durch $x$ und $\gamma_{\o}$ eine geschlossene, nullhomotope Kurve durch $x$. Dann ist $\gamma^-\star\gamma_{\o}\star\gamma$ ebenfalls geschlossen und nullhomotop. Somit gilt
\begin{align*}
g_{\gamma}^{-1}\cdot g_{\gamma_{\o}}\cdot g_{\gamma} = 
g_{\gamma^-}\cdot g_{\gamma_{\o}}\cdot g_{\gamma} = 
g_{\gamma^-\star \gamma_{\o}\star \gamma} \in \Hol_{x}^{\o}(A).
\end{align*}
Also ist $\Hol_{x}^{\o}(A)$ ein Normalteiler von $\Hol_{x}(A)$.

c): Die Parallelverschiebung vertauscht mit der Rechtstranslation. Seien also $p\in P$ und $g\in G$, dann gilt
\begin{align*}
\Hol_{p\cdot g}(A) &= \setdef{g_{\gamma}}{P_{\gamma}(p\cdot g) = (p\cdot g)\cdot g_{\gamma},\quad \gamma\in \Omega(x)}\\
&= \setdef{g_{\gamma}}{P_{\gamma}(p) = p\cdot (g\cdot g_{\gamma}\cdot g^{-1}),\quad \gamma\in \Omega(x)}\\
&= \setdef{g^{-1}\cdot g_{\gamma}\cdot g}{P_{\gamma}(p) = p\cdot g_{\gamma},\quad \gamma\in \Omega(x)}.\qed
\end{align*}
\end{proof}

Um einzusehen, dass die Holonomiegruppe darüber hinaus auch eine Lie-Untergruppe ist, bedarf es noch einiger topologischer Feinheiten. 

\begin{prop}[Satz von Freudenthal-Yamabe]
Sei $H\subset G$ eine algebraische Untergruppe einer Lie-Gruppe $G$. Falls sich jedes Element von $H$ durch eine, ganz in $H$ verlaufende Kurve mit dem Einselement verbinden lässt, so ist $H$ eine Lie-Untergruppe von $G$.\fish
\end{prop}
\begin{proof}[Beweisskizze.] Ein vollständiger Beweis findet sich in \cite[Satz 1.22]{Baum:2009wk}.
Die Idee des Beweises beruht darauf, durch geeignete Konstruktion einer integrablen Distribution $S$, mit dem Satz von Frobenius die Untergruppe $H$ als Integralmannigfaltigkeit zu $S$ zu gewinnen. Man definiert dazu
\begin{align*}
S_{e}\defl \setdef{X\in T_{e}G}{X = \dot{\gamma}(0),\text{ wobei }\gamma\text{ durch }e\text{ geht und ganz in $H$ verläuft}},
\end{align*}
und setzt $S_{g} \defl \dL_{g}(S_{e})$. Der schwierige Teil ist nun zu zeigen, dass $S$ tatsächlich eine involutive Distribution ist. Es folgt dann, dass $H$ die maximale Integralmannigfaltigkeit durch $e$ ist, und dass die Gruppenoperationen glatt sind. Damit ist $H$ eine Untermannigfaltigkeit von $G$ im schwachen Sinne und folglich eine Lie-Untergruppe von $G$.\qed
\end{proof}

Als Anwendung des Satzes von Freudenthal-Yamabe erhalten wir die gewünschte Aussage über die Differenzierbarkeit der Holonomiegruppe.

\begin{prop}
Sei $p$ ein Punkt in $P$. Dann gelten:
\begin{propenum}
\item Die algebraischen Untergruppen $\Hol_{p}^{\o}(A)$ und $\Hol_{p}(A)$ sind Lie-Untergruppen von $G$.
\item Die Untergruppe $\Hol_{p}^{\o}(A)\subset \Hol_{p}(A)$ ist die Zusammenhangskomponente der Eins.\fish
\end{propenum}
\end{prop}

\begin{proof}[Beweisskizze.] Ein vollständiger Beweis findet sich in \cite[Satz 4.3]{Baum:2009wk}.
Wir zeigen zunächst, dass $\Hol_{p}^{\o}(A)$ eine Lie-Untergruppe ist. Sei dazu $g\in \Hol_{p}^{\o}(A)$, dann ist $P_{\gamma}^A(p) = p\cdot g$ für ein $\gamma\in \Omega_{\o}(x)$. Sei $F\colon I\times I\to M$ die Homotopie zwischen der Kurve $\gamma$ und der konstanten Kurve $\gamma_{\o}\equiv x$. Setzen wir $\gamma_{s} = F_{s} = F(\cdot,s)$, dann ist
\begin{align*}
P_{\gamma_{s}}^A(p) =p\cdot g_{s},
\end{align*}
mit $g_{s}\in \Hol_{p}^\o(A)$ für $0\le s \le 1$, wobei $g_{0} = e$ und $g_{1} = g$ ist. Somit ist $g_{s}$ eine Kurve in $\Hol_{p}^\o(A)$, die $g$ mit dem Einselement verbindet. Nach dem Satz von Freudenthal-Yamabe ist daher $\Hol_{p}^\o(A)$ eine Lie-Untergruppe von $G$.

Um die Lie-Gruppen Eigenschaft auch für $\Hol_{p}(A)$ einzusehen, verwenden wir das Resultat der Topologie, dass die Fundamentalgruppe $\pi_{1}(M)$
jeder Mannigfaltigkeit abzählbar ist. Weiterhin wird $\pi_{1}(M)$ bis auf Homotopie von glatten Kurven definiert, d.h. jedes Element lässt sich als glatte Kurve realisieren. Somit ist die Abbildung
\begin{align*}
f: \pi_{1}(M)\to \Hol_{p}(A)/\Hol_{p}^\o(A),\qquad \gamma\mapsto [g_{\gamma}],
\end{align*}
ein surjektiver Gruppenhomomorphismus und der Quotient $\Hol_{p}(A)/\Hol_{p}^\o(A)$ ist abzählbar. Weiterhin ist die Holonomiegruppe als abzählbare Vereinigung ihrer Nebenklassen gegeben,
\begin{align*}
\Hol_{p}(A)  = \bigcup_{n} \,g_{n}\,\Hol_{p}^\o(A).
\end{align*}
Wir haben schon gezeigt, dass $\Hol_{p}^\o(A)$ glatt ist, also trägt jede Nebenklasse eine Mannigfaltigkeitsstruktur und damit ist auch die abzählbare Vereinigung $\Hol_{p}(A)$ eine glatte Mannigfaltigkeit mit glatten Gruppenoperationen.\qed
\end{proof}

Als nächstes möchten wir untersuchen, wie die Reduktion der Strukturgruppe $G$ und die Holonomiegruppe zusammenhängen. Es wird sich herausstellen, dass die Holonomiegruppe die ``kleinste'' Gruppe darstellt, auf die sich die Strukturgruppe reduzieren lässt. Es gilt aber mehr, denn sogar der Zusammenhang lässt sich auf diese Gruppe reduzieren.

\begin{defn}
Sei $Q$ eine $H$-Reduktion von $P$ und $A^\star$ ein Zusammenhang auf $Q$ so, dass
\begin{align*}
\ker(A^\star) = T^hQ \subset T^hP = \ker(A).
\end{align*}
Dann ist $(Q,A^\star)$ ein \emph{$H$-Reduktion von $(P,A)$} und man sagt, der Zusammenhang \emph{$A$ reduziert sich auf $H$}.\fish
\end{defn}

Als passendes Unterbündel für die Reduktion des Zusammenhangs $A$ definieren wir nun das Holonomiebündel.

\begin{defn}
Sei $u\in P$, dann ist das \emph{Holonomiebündel von $A$ durch $u$} definiert als
\begin{align*}
P^A(u) \defl \setdef{p\in P}{\text{es existiert eine horizontale Kurve in $P$ von $u$ nach $p$}}.\fish
\end{align*}
\end{defn}

\begin{prop}[Reduktionssatz der Holonomietheorie]
Für jedes $u\in P$ gelten:
\begin{propenum}
\item Durch $\pi\colon P^A(u)\to M$ ist ein $\Hol_{u}(A)$-Hauptfaserbündel gegeben.
\item Der Zusammenhang $A$ reduziert sich auf $\Hol_{u}(A)$, d.h. $P^A(u)\subset P$ ist eine $\Hol_{u}(A)$-Reduktion und
\begin{align*}
T^hP^A(u) \subset T^hP.\fish
\end{align*}
\end{propenum}
\end{prop}

\begin{rem}[Bemerkungen.]
\begin{remenum}
\item Eigenschaft b) beschreibt beschreibt eine besonders ``starke'' Reduktion, denn es reduziert sich nicht nur die Strukturgruppe auf die Holonomiegruppe, sondern auch der Zusammenhang reduziert sich.
\item Der Totalraum $P$ lässt sich als disjunkte Vereinigung von Holonomiebündeln $P^A(u)$ beschreiben, denn durch
\begin{align*}
p\sim q \iff \text{es existiert eine horizontale Kurve von $p$ nach $q$}
\end{align*}
ist eine Äquivalenzrelation auf $P$ gegeben, und aus $p\sim q$ folgt, dass die entsprechenden Holonomiegruppen übereinstimmen, d.h. $\Hol_{p}(A) = \Hol_{q}(A)$.\map
\end{remenum}
\end{rem}

\begin{defn}
Ein Zusammenhang $A$ heißt \emph{irreduzibel}, falls sich $A$ auf keine echte Lie-Untergruppe reduzieren lässt.\fish
\end{defn}

Dass die Holonomiegruppe tatsächlich die ``kleinste'' Untergruppe ist auf die sich $P$ samt dem Zusammenhang $A$ reduziert, schreibt sich nun wie folgt.

\begin{prop}
Das Holonomiebündel $P^A(u)$ zusammen mit dem induzierten Zusammenhang ist irreduzibel.\fish
\end{prop}


\begin{rem}
Sei $Q\subset P$ eine $H$-Reduktion und $E$ ein assoziiertes Vektorbündel. Dann gilt
\begin{align*}
E = P\times_{\rho} V = Q\times_{\rho\big|_{H}}V.
\end{align*}
Durch die Einschränkung von $\rho$ auf $H$ lässt sich $E$ als Summe von Bündeln schreiben. Reduziert sich nicht nur die Strukturgruppe, sondern auch der Zusammenhang, so besteht die Bündelsumme aus parallelen Unterbündeln.\map
\end{rem}

Die Holonomiegruppe beinhaltet viele interessante Informationen über das Hauptfaserbündel und den Zusammenhang. Jedoch stellt es sich im Allgemeinen als äußerst schwierig heraus, diese Gruppe explizit zu berechnen. Die Krümmung $F^A$ ist eine 2-Form auf $P$ mit Werten in der Lie-Algebra $\g$. Mit Hilfe der Krümmung lässt sich die Lie-Algebra der Holonomiegruppe angeben, welche die Holonomiegruppe in erster Näherung beschreibt. Allerdings ist auch die Krümmung ein kompliziertes und manchmal unzugängliches Objekt\ldots

\begin{prop}[Holonomietheorem von Ambrose-Singer]
Für jedes $u\in P$ gilt,
\begin{align*}
\Lie(\Hol_{u}(A)) = \span\setdef{F^A(X,Y)}{p\in P^A(u),\quad X,Y\in T_{p}^hP}.\fish
\end{align*}
\end{prop}

Bevor wir den Beweis des Holonomietheorems angehen, betrachten wir noch den Fall einer einfach zusammenhängenden Mannigfaltigkeit. 
Hier erhalten wir folgendes globales Resultat.

\begin{cor}
Sei $G$ zusammenhängend und $M$ einfach zusammenhängend. Dann gilt
\begin{align*}
A\text{ ist irreduzibel } \quad\iff\quad \g = \span\setdef{F^A(X,Y)}{p\in P^A(u),\quad X,Y\in T_{p}^hP}.\fish
\end{align*}
\end{cor}

\begin{proof}
Da $M$ einfach zusammenhängend ist, sind alle Schleifen nullhomotop. Also ist $\Omega_{\o}(x) = \Omega(x)$. Folglich gilt für jedes $p\in P$ auch $\Hol_{p}^\o(A) = \Hol_{p}(A)$ und somit ist $\Hol_{p}(A)$ zusammenhängend. Daher ist $\Hol_{p}(A)$ genau dann eine echte Untergruppe von $G$, wenn auch die Lie-Algebra von $\Hol_{p}(A)$ eine echte Unteralgebra von $\g$ ist.\qed
\end{proof}

\begin{proof}[Beweis des Holonomietheorems]
Wir können ohne Einschränkung annehmen, dass $G=\Hol_{u}(A)$ und $P=P^A(u)$. Andernfalls gehen wir durch Reduktion zum Holonomiebündel über. Weiterhin schreiben wir $\g = \Lie(\Hol_{u}(A))$ und setzen
\begin{align*}
\m \defl \span\setdef{F^A(X,Y)}{p\in P^A(u),\quad X,Y\in T_{p}^hP}.
\end{align*}

Zu zeigen ist also, dass $\m = \g$. Seien dazu $p\in P$ sowie $X,Y\in T^h_{p}P$ und $W\in \g$. Wir betrachten die Kurve
\begin{align*}
\alpha\colon [0,1] \to \m,\qquad t\mapsto \alpha(t) \defl \left(R_{\exp(tW)}^* F^A\right)_{p}(X,Y).
\end{align*}
Die Krümmung ist eine 2-Form vom Typ $\Ad$, d.h.
\begin{align*}
\dot{\alpha}(0) &=  \frac{\ddd}{\dt}\bigg|_{0} F_{\exp(tW)}(\dR_{\exp(tW)}(X),\dR_{\exp(tW)}(Y))\\
&= \frac{\ddd}{\dt}\bigg|_{0} \Ad(\exp(-tW))F_{p}(X,Y)\\
&= -\ad(W) F_{p}(X,Y)\\
&= [F_{p}^A(X,Y),W].
\end{align*}
Da $\alpha$ eine Kurve in $\m$ beschreibt, ist auch $\dot{\alpha}(0)=[F_{p}^A(X,Y),W]\in\m$ für jedes $W\in\g$. Also ist $\m$ ein Ideal von $\g$ und damit insbesondere eine Lie-Unteralgebra.

Man definiert nun folgende Distribution
\begin{align*}
%E : P\to TP = T^hP\oplus T^vP,\qquad  ???
E \subset TP = T^hP\oplus T^vP,\qquad 
p\mapsto E_{p}\defl T_{p}^hP\oplus \setdef{\tilde{W}_{p}}{W\in \m},
\end{align*}
und verifiziert, dass $E$ involutiv ist. Aufgrund der Linearität müssen wir dazu nur folgende drei Fälle untersuchen.

\textit{1. Fall}, $V,W\in \m$. Dann gilt
\begin{align*}
[\tilde{V},\tilde{W}] = \widetilde{[V,W]} \in \Gamma(E).
\end{align*}

\textit{2. Fall}, $X\in \Gamma(T^hP)$ und $V\in \m$. Dann ist $[X,\tilde{V}]$ horizontal und somit ein Schnitt in $E$.

\textit{3. Fall}, $X,Y\in \Gamma(T^hP)$. Der horizontale Teil des Kommutators liegt ohnehin in $E$ und für den vertikalen gilt,
\begin{align*}
\pr_{v}[X,Y] = - \widetilde{F^A(X,Y)},
\end{align*}
also liegt dieser auch in $E$.

Somit ist $E$ involutiv und nach dem Satz von Frobenius existiert eine maximale Integralmannigfaltigkeit $Q\subset P$ von $E$ durch $u\in P$.
Ein Punkt $q$ liegt in $Q$ genau dann, wenn eine Kurve $\gamma$ von $u$ nach $q$ existiert, die ganz in $Q$  verläuft und für die gilt,
\begin{align*}
\dot{\gamma}(t) \in E_{\gamma(t)}.
\end{align*}
Somit ist $P=P^A(u)\subset Q$, denn $T^hP\subset TQ=E$. Also gilt $P=Q$ und somit auch $E=TQ=TP$. Insbesondere ist $\m = \g$.\qed
\end{proof}

\section{Flache Zusammenhänge und Holonomie}

Ein Zusammenhang $A$ auf $P$ heisst flach, wenn seine Krümmung verschwindet, d.h. $F^A\equiv 0$. 
Für den Fall einer einfach zusammenhängenden Mannigfaltigkeit $M$ haben wir folgende Äquivalenz postuliert:
\begin{equivenum}
\item $A$ ist flach,
\item $(P,A)$ ist isomorph zu trivialen $G$-Hauptfaserbündel  mit dem kanonischen flachen Zusammenhang,
\item Die Parallelverschiebung hängt nicht vom Weg ab.
\end{equivenum}

(ii)$\iff$(iii) haben wir bereits eingesehen. Mit Hilfe der Holonomietheorie können wir die verbleibende Behauptung nun ganz leicht beweisen.
\begin{proof}
Verschwindet die Krümmung, dann ist nach dem vorangegangen Satz die Lie-Algebra der Holonomiegruppe trivial. Da aber $M$ einfach zusammenhängend ist, folgt dass auch die Holonomiegruppe trivial ist. Dann gilt für jedes $p\in P$ und jede geschlossene Kurve $\gamma$ in $M$, 
\begin{align*}
P_{\gamma}^A(p) = p.
\end{align*}
Seien nun $\alpha$ und $\beta$ zwei Kurven in $M$ von $x$ nach $y$, dann gilt
\begin{align*}
q = P_{\alpha}^A(p) = P_{\alpha}^A\circ P_{\beta^-}^A\circ P_{\beta}^A(p)
= P_{\alpha\star\beta^-}\circ P_{\beta}^A(p)
= P_{\beta}^A(p).
\end{align*}
Also hängt die Parallelverschiebung nicht vom gewählten Weg ab.

Hängt umgekehrt die Parallelverschiebung nicht vom gewählten Weg ab, dann gilt
\begin{align*}
P_{\gamma}^A(p) = p,\qquad  \text{für } p\in P_{x},\; \gamma\in\Omega(x) = \Omega_{\o}(x).
\end{align*}
Somit ist die Holonomiegruppe trivial, also ist ihre Lie-Algebra diskret und folglich muss die Krümmung verschwinden.\qed
\end{proof}

Als unmittelbare Konsequenz aus dem Reduktionssatz erhalten wir.

\begin{cor}
Der Zusammenhang $A$ reduziert sich genau dann auf die triviale Gruppe, wenn $(P,A)$ trivial ist.\fish
\end{cor}

\begin{rem}
Die Fundamentalgruppe von $M$ sei nicht trivial und die Krümmung verschwinde, also
\begin{align*}
\pi_{1}(M)\neq 1,\qquad F^A\equiv 0.
\end{align*}
So ist $\Hol_{p}(A)$ im Allgemeinen nicht zusammenhängend und daher nur diskret aber nicht trivial. Somit ist $M^\star = P^A(u)\to M$ eine Überlagerung von $M$. Weiterhin ist $\Hol_{p}^\o(A)$ trivial und daher ist die Darstellung
\begin{align*}
\rho\colon \pi_{1}(M)\to \Hol_{u}(A)
\end{align*}
ein surjektiver Gruppenhomomorphismus. Die Holonomiegruppe ist folglich eine Quotientengruppe der Fundamentalgruppe,
\begin{align*}
\Hol_{p}(A) = \pi_{1}(M)/\Gamma.
\end{align*}
Die flachen Vektorbündel sind nun assoziiert zu entsprechenden Darstellungen der Fundamentalgruppe. Die Modulr\"aume
flacher Vektorb\"undel  spielen eine fundamentale Rolle in der konformen Quantenfeldtheorie. Für 2-dimensionale Flächen sind die Darstellungen vollständig untersucht.

Die universelle Überlagerung  $\tilde M \to M$ ist ein $\pi_{1}(M)$-Hauptfaserbündel. Das Holonomiebündel ergibt sich
als dazu via $\rho$ assoziiertes Faserbündel:
\begin{align*}
P\defl \tilde M \times_{\rho}\Hol_{p}(A) \ .\map
\end{align*}

\end{rem}


% Vorlesung vom 06. Juni 2011

\section{Parallele Schnitte und Holonomie}

Sei $\pi\colon E\to M$ ein assoziiertes Vektorbündel
\begin{align*}
E = P\times_\rho V,
\end{align*}
mit kovarianter Ableitung $\nabla$, die durch den Zusammenhang $A$ auf $P$ induziert wird. Wir wollen nun genauer untersuchen, wie parallele Schnitte, das sind Schnitte $e\in\Gamma(E)$ mit $\nabla e = 0$, mit der Holonomie zusammenhängen.

\begin{rem}[Bemerkungen.]
\begin{remenum}
\item Die parallelen Schnitte sind invariant unter Parallelverschiebung, d.h. für zwei Punkte $x$ und $y$ in $M$ gilt
\begin{align*}
\e(y) = \tau_{xy}\e(x).
\end{align*}
Ein paralleler Schnitt ist daher bereits eindeutig durch den Wert an einer einzigen Stelle festgelegt.
\item
Verschwindet ein paralleler Schnitt an einer Stelle, so folgt dass er identisch Null ist. Nichttriviale parallele Schnitte haben also keine Nullstellen.
\item Der Vektorraum der parallelen Schnitte $\Par(E,\nabla)$ ist ein endlichdimensionaler Unterraum von $\Gamma(E)$.\map
\end{remenum}
\end{rem}

\begin{prop}
Der Raum der parallelen Schnitte in $E$ ist durch die Holonomie-invarianten Vektoren in $V$ gegeben,
\begin{align*}
\Par(E,\nabla) &\cong \setdef{v\in V}{\rho(g)v = v\quad \text{für alle }g\in \Hol_{p}(A)}.
\end{align*}
\end{prop}
\begin{proof}
$\supset$:
Sei $v\in V$ ein Fixpunkt der Darstellung, es gelte also $\rho(g)v=v$ für alle $g\in \Hol_{p}(A)$. Wir fixieren ein beliebiges $x\in M$, und definieren
\begin{align*}
\ph_{v}(y) \defl [P_{\gamma}^A(p),v],\qquad y\in M,
\end{align*}
wobei $\gamma$ eine Kurve von $x$ nach $y$ ist und $\pi(p)=x$. Es ist nun zu zeigen, dass $\ph_{v}$ ein wohldefinierter, glatter und paralleler Schnitt in $E$ ist.

Wir zeigen zuerst, dass die Definition von $\ph_{v}(y)$ nicht von der Wahl des Weges $\gamma$ abhängt. Sei dazu $\mu$ eine weitere Kurve von $x$ nach $y$, dann ist $\gamma\star \mu^-$ ein geschlossener Weg durch $x$.
Somit gibt es ein $h_{\gamma\star \mu^-}\in \Hol_{p}(A)$ so, dass 
\begin{align*}
p\cdot h_{\gamma\star \mu^-} = P_{\gamma\star \mu^-}^A(p) = P_{\mu^-}^A\circ P_{\gamma}^A(p).
\end{align*}
Andererseits gilt aufgrund der $G$-äquivarianz
\begin{align*}
P_{\gamma}^A(p) =P_{\mu}^A(p\cdot h_{\gamma\star \mu^-}) =  P_{\mu}^A(p)\cdot h_{\gamma\star \mu^-}.
\end{align*}
Schließlich folgt, da $v$ ein Fixpunkt ist,
\begin{align*}
[P_{\gamma}^A(p),v] = [P_{\mu}^A(p)\cdot h_{\gamma\star \mu^-},v] = 
[P_{\mu}^A(p),\rho(h_{\gamma\star \mu^-})v] = 
[P_{\mu}^A(p),v].
\end{align*}
Also ist $\ph_{v}$ wohldefiniert. Aufgrund der differenzierbaren Abhängigkeit der Lösung gewöhnlicher Differentialgleichungen von den Anfangswerten, ist $\ph_{v}$ auch glatt.

Weiterhin gilt für ein Vektorfeld $X$ auf $M$, dass
\begin{align*}
(\nabla_{X}\ph_{v})(y) &= \frac{\ddd}{\dt}\bigg|_{0} P_{\alpha(t)y}^E\ph_{v}(\alpha(t))
= \frac{\ddd}{\dt}\bigg|_{0} [P_{\alpha(t)y}^AP^A_{x\alpha(t)}(p),v]\\
&= \frac{\ddd}{\dt}\bigg|_{0} [P_{xy}(p),v] = 0,
\end{align*}
denn die Parallelverschiebung hängt nicht vom Weg ab.
Somit ist $\ph_{v}$ tatsächlich parallel.

$\subset$: Sei $\ph$ ein glatter, paralleler Schnitt in $E$. Schreiben wir das Vektorbündel assoziiert zum Holonomiebündel,
\begin{align*}
E = P\times_{G} V = P^A(u) \times_{\Hol_{u}(A)}V,
\end{align*}
so entspricht $\ph$ einerseits einer Funktion $\bar{\ph} \in\Cs^\infty(P,V)^G$ und andererseits einer Funktion $\bar{\psi}\in \Cs^\infty(P(u)^A,V)^{\Hol_{u}(A)}$, wobei $\bar{\psi} = \bar{\ph}\big|_{P^A(u)}$. Nach Voraussetzung ist $\ph$ parallel, also gilt
\begin{align*}
0 = \bar{\nabla_{X}\ph} = X^*(\bar{\ph}),
\end{align*}
und $\bar{\ph}$ ist konstant entlang horizontaler Kurven. Da das Holonomiebündel $P^A(u)$ gerade von horizontalen Kurven erzeugt wird, existiert also ein Vektor $v\in V$ so, dass
\begin{align*}
v \equiv \bar{\ph}\big|_{P^A(u)} = \bar{\psi}.
\end{align*}
Nun ist die Funktion $\bar{\psi}$ vom Typ $\Hol_{u}(A)$, d.h. für $h\in\Hol_{u}(A)$ gilt,
\begin{align*}
v = \bar{\psi}(q\cdot h) = \rho(h)^{-1}\bar{\psi}(q) = \rho(h)^{-1}v.
\end{align*}
Somit ist $\rho(h)v = v$ für alle $h\in \Hol_{u}(A)$ und $v$ ist Holonomie-invariant.\qed
\end{proof}

Fixpunkte der Holonomiedarstellung entsprechen demnach parallelen Schnitten.

\begin{prop}[Zusatz]
Ist die Basismannigfaltigkeit $M$ außerdem einfach zusammenhängend, dann gilt sogar,
\begin{align*}
\Par(E,\nabla) &\cong \setdef{v\in V}{\rho_{*}(X)v = 0\quad \text{für alle }X\in \Lie(\Hol_{p}(A))}.\fish
\end{align*}
\end{prop}
\begin{proof}
Sei $M$ einfach zusammenhängend, dann ist $\Hol_{p}(A) = \Hol_{p}^\o(A)$ eine zusammenhängende Lie-Gruppe und folglich gilt
\begin{align*}
\rho(g)v = v \text{ für alle } g\in \Hol_{p}(A)
\end{align*}
genau dann, wenn
\begin{align*}
\rho_{*}(X)v = 0\text{ für alle } X\in \Lie(\Hol_{p}(A)).\qed
\end{align*}
\end{proof}

\section{Holonomiegruppen von Riemannschen Mannigfaltigkeiten}

Sei $(M,g)$ eine Riemannsche Mannigfaltigkeit und $\nabla$ der Levi-Civita-Zusammenhang auf dem Tangentialbündel. Da der Levi-Civita-Zusammenhang der eindeutig bestimmte, metrische und torsionsfreie Zusammenhang auf $TM$ ist, ist durch $\nabla$ in kanonischer Weise eine Holonomiegruppe auf $M$ gegeben. Wir bezeichnen diese mit
\begin{align*}
\Hol_{x}(M,g) \defl \Hol_{x}(\nabla) = 
\setdef{P_{\gamma} \colon T_{x}M\to T_{x}M}{\gamma\in\Omega(x)},\qquad x\in M.
\end{align*}

Die Elemente der Holonomiegruppe können wir mit speziellen Automorphismen von $T_{x}M$ identifizieren. Dies liefert eine Darstellung der Holonomiegruppe auf den Automorphismen von $T_{x}M$,
\begin{align*}
\Hol_{x}(M,g) \subset \GL(T_{x}M) \cong \GL_{n}.
\end{align*}

\begin{rem}
Die Holonomiegruppe hängt bis auf Konjugation nicht von der Wahl des Punktes ab. Genauer gilt für Punkte $x$ und $y$ in $M$, dass
\begin{align*}
\Hol_{y}(M,g) = P_{\sigma}\circ \Hol_{x}(M,g) \circ P_{\sigma}^{-1},
\end{align*}
wobei $\sigma$ einen Weg bezeichnet, der $x$ und $y$ verbindet.\map
\end{rem}

\begin{ex}
\begin{exenum}
 \item Auf dem $\R^n$ ist die Parallelverschiebung unabhängig von der Wahl des Weges. Somit ist die Holonomiegruppe trivial,
 \begin{align*}
\Hol_{x}^\o(\R^n) = \Hol_{x}(\R^n) = \setd{\Id}.
\end{align*}
\item In Kapitel 1-G haben wir festgestellt, dass auf einer Mannigfaltigkeit genau dann eine Riemannsche Metrik existiert, wenn das Rahmenbündel eine $\O(n)$-Reduktion besitzt, d.h. wenn sich die Strukturgruppe $\GL_{n}$ auf $\O_{n}$ reduzieren lässt. Analog dazu existiert auf einer Mannigfaltigkeiten genau dann eine Riemannsche Metrik, wenn die Holonomiegruppe eine Untergruppe der orthogonalen Gruppe ist, $\Hol_{x}(M) \subset \O_{n}$.

Für orientierbare Riemannsche Mannigfaltigkeiten lässt sich die Strukturgruppe sogar auf $\SO_{n}$ reduzieren. Außerdem ist $(M,g)$ genau dann orientierbar, wenn $\Hol_{x}(M)\subset \SO_{n}$.
\item Die $n$-dimensionale Sphäre $S^n$ ist eine orientierbare Riemannsche Mannigfaltigkeit und es gilt $\Hol_{x}(S^n) = \SO_{n}$.\boxc
\end{exenum}
\end{ex}

\begin{rem}[Bemerkungen.]
\begin{remenum}
\item
Das Holonomietheorem von Ambrose-Singer erlaubt einer Beschreibung der Lie-Algebra der Holonomiegruppe mit Hilfe der Krümmung. Im Fall des Tangentialbündels mit Levi-Civita Zusammenhang vereinfacht sich diese Beschreibung zu
\begin{align*}
&\Lie(\Hol_{x}(M,g)) \\
&\quad= \setdef{P_{\gamma^-} R_{y}\bigl(P_{\gamma}(V),P_{\gamma}(W)\bigr) P_{\gamma}}{\atop{\gamma \text{ eine Kurve von }x\text{ nach }y}{ V,W\in T_{x}M}}.
\end{align*}
Insbesondere ist $R_{x}(V,W)$ stets ein Element der Holonomiegruppe $\Hol_{x}(M,g)$. Die ersten beiden Symmetrien des Krümmungstensors lassen sich besonders kurz schreiben als
\begin{align*}
R_{x}\in \Sym^2(\Lambda^2 T_{x}^*M).
\end{align*}
Nach Ambrose-Singer gilt darüber hinaus
\begin{align*}
R_{x}\in \Sym^2(\Lie(\Hol_{x}(M,g))).
\end{align*}
\item Die Holonomiegruppe ist ein globales Objekt und lässt sich im Allgemeinen nicht berechnen, wenn man nur eine kleine Umgebung der Mannigfaltigkeit betrachtet. Allerdings erhält man die Holonomiegruppe einer reell analytischen Mannigfaltigkeiten bereits durch das Betrachten von kleinen Umgebungen, hier ist die Holonomie dann tatsächlich bereits lokal bestimmt.\map
\end{remenum}
\end{rem}


\begin{prop}[Holonomie-Prinzip]
Sei $\Tc\to M$ ein Tensorbündel auf $M$. Dann gelten:
\begin{propenum}
\item Sei $T\in \Gamma(\Tc)$ mit $\nabla T \equiv 0$. Dann ist $T(x) \in\Tc_{x}$ invariant unter $\Hol_{x}(M,g)$, d.h.
\begin{align*}
g T(x) = T(x),\quad \text{für alle }g\in \Hol_{x}(M,g).
\end{align*}
\item Sei $T_{x}\in\Tc_{x}$ ein $\Hol_{x}(M,g)$-invarianter Tensor. Dann existiert genau ein Schnitt $T\in \Gamma(\Tc)$ mit $\nabla T \equiv 0$ und $T(x) = T_x$.\fish
\end{propenum}
\end{prop}

Parallele Schnitte sind also punktweise invariant unter der Holonomiegruppe und umgekehrt existiert zu jedem invarianten Tensor ein paralleler Schnitt.

\begin{ex}
\begin{exenum}
\item Sei $\Hol_{x}(M,g) \subset \O(T_{x}M,g_{x})$, wobei
$\O(T_{x}M)$ nach Definition das Skalarprodukt $g_{x}\in \Sym^2(T_{x}M)$ erhält. Das Holonomie-Prinzip liefert nun die Existenz einer parallelen Metrik $g\in \Gamma(\Sym^2(TM))$ mit $g(x) = g_{x}$. Für beliebige Vektorfelder $X,Y,Z$ auf $M$ gilt also
\begin{align*}
0 = (\nabla_{X}g)(Y,Z) = X(g(Y,Z)) - g(\nabla_{X}Y,Z) - g(Y,\nabla_{X}Z),
\end{align*}
d.h. der Zusammenhang ist metrisch. Dies deckt sich mit der Beobachtung, dass eine $\O_{n}$-Reduktion genau dann vorliegt, wenn auf $M$ eine Riemannsche Metrik existiert.
\item Eine Riemannsche Mannigfaltigkeit $(M,g)$ ist genau dann orientierbar, wenn
\begin{align*}
\Hol_{x}(M,g) \subset \SO(T_{x}M).
\end{align*}

$\Rightarrow$: Sei $M$ orientierbar, dann existiert eine parallele Volumenform. Da die Parallelverschiebung $P_{\gamma}^\nabla$ die Orientierung erhält, folgt, dass $\Hol_{x}(M,g)\subset \SO(T_{x}M)$ gilt.

$\Leftarrow$: Sei $\Hol_{x}(M,g) \subset \SO(T_{x}M) = \setdef{A \in \O(T_{x}M)}{A^*\vol_{\o} = \vol_{\o}}$, wobei $\vol_{\o} = e_{1}\wedge \ldots \wedge e_{n}$ das kanonische Volumenelement von $T_{x}M$ bezeichnet. Nach Voraussetzung ist $\vol_{\o}\in \Lambda^n T_{x}^*M$ invariant unter $\Hol_{x}(M,g)$, also existiert ein paralleler Schnitt $\vol \in \Gamma(\Lambda^n T^*M)$. Somit ist $\vol$ eine $n$-Form auf $M$ ohne Nullstellen, und definiert daher eine Orieniterung.
\item Auf $M$ existiert ein paralleles Vektorfeld genau dann, wenn ein $\Hol_{x}$ invarianter Vektor $v$ existiert, d.h. wenn
\begin{align*}
\Hol_{x}(M,g) \subset \O_{n-1}\subset \O_{n}.
\end{align*}
\item Sei $M$ eine $2n$-dimensionale Riemmannsche Mannigfaltigkeit und $\Tc = \End(TM)$ das Endomorphismenbündel. Reduziert sich die Holonomiegruppe wie folgt,
\begin{align*}
\Hol_{x}(M^{2n},g) \subset U(T_{x}M,g_{x},J_{x}) = \setdef{A\in \SO(T_{x}M,g)}{AJ_{x} = J_{x}A},
\end{align*}
wobei $J_{x}\in \End(T_{x}M)$ mit $J_{x}^*g_{x} = g_{x}$ und $J_{x}^2 = -\Id$ eine fast komplexe Struktur auf $T_{x}M$ bezeichnet, so ist $M$ eine Kählermannigfaltigkeit, d.h. es existiert ein Endomorphismus $J$ von $TM$ mit
\begin{align*}
J^*g = g,\qquad J^2 = -\Id,\qquad \nabla J = 0.
\end{align*}
Existiert umgekehrt eine fastkomplexe Struktur $J\in \End(TM)$, so reduziert sich die Strukturgruppe auf $\U_{n}$. Ist $J$ außerdem parallel, also $\nabla J \equiv 0$, so reduziert sich sogar die Holonomiegruppe auf $U_{n}$.\boxc
\end{exenum}
\end{ex}

\section{Riemannsche Produkte und Holonomie}

Sei wieder $(M,g)$ eine Riemannsche Mannigfaltigkeit. Im Folgenden wollen wir untersuchen, wie die Holonomiegruppe eines Riemannschen Produktes aussieht und umgekehrt, welche Kriterien für die Holonomiegruppe garantieren, dass $M$ ein Riemannsches Produkt ist.

\begin{prop}
\label{prop:Paralleles-Unterbündel-Invarianz-Holonomiedarstellung}
Folgende Aussagen sind äquivalent.
\begin{propenum}
\item Es existiert paralleles Unterbünndel des Tangentialbündels vom Rang $k$.
\item Die Holonomiedarstellung $\Hol_{x}(M,g) \to \GL(T_{x}M)$ lässt einen $k$-dimensionalen Unterraum von $T_{x}M$ invariant.\fish
\end{propenum}
\end{prop}
\begin{proof}
$\Rightarrow$: Sei $E$ ein paralleles Unterbündel, d.h. eine $k$-dimensionale Distribution, die von der Parallelverschiebung erhalten wird.
Also gilt für jedes $x\in M$ und jede Schleife $\gamma$ an $x$, dass
\begin{align*}
P_{\gamma} E_{x} = E_{x}.
\end{align*}
Somit ist $E_{x}$ ein $k$-dimensionaler Holonomie-invarianter Unterraum von $T_{x}M$.

$\Leftarrow$: Sei umgekehrt $E_{x}$ ein $k$-dimensionaler $\Hol_{x}(M,g)$-invarianter Unterraum von $T_{x}M$.
Sei $y\in M$ und $\gamma$ eine Kurve von $x$ nach $y$, dann definiert man
\begin{align*}
E_{y} = P_{\gamma} E_{x}.
\end{align*}
Dann ist $E_{y}$ wohldefiniert, denn für einen weiteren Weg $\delta$ von $x$ nach $y$ und ein $v\in E_{x}$ gilt
\begin{align*}
P_{\delta^-} P_{\gamma} (v) = w\in E_{x},
\end{align*}
da die Parallelverschiebung $E_{x}$ invariant lässt. Somit ist $P_{\delta}(w) = P_{\gamma}(v)$ und folglich gilt auch $P_{\delta}(E_{x}) = P_{\gamma}(E_{x})$. Man erhält somit eine glatte Distribution $E$, denn die  Parallelverschiebung hängt differenzierbar von den Anfangswerten ab. Außerdem ist $E$ nach Konstruktion invariant unter Parallelverschiebung, also parallel.\qed
\end{proof}

\begin{rem}[Zur Erinnerung:]
Ein paralleles Unterbündel $E$ vom Rang $r$ ist integrabel, denn für $X,Y\in E$ gilt
\begin{align*}
[X,Y] = \nabla_{X}Y - \nabla_{Y}X \in E,
\end{align*}
aufgrund der Torsionsfreiheit des Levi-Civita Zusammenhangs.\map
\end{rem}

\begin{prop}
Sei $(M,g) = (M_{1},g_{1})\times (M_{2},g_{2})$ ein Riemannsches Produkt. Dann gilt in jedem Punkt $x=(x_{1},x_{2})\in M$,
\begin{align*}
\Hol_{(x_{1},x_{2})}(M,g) \cong \Hol_{x_{1}}(M_{1},g_{1})\times \Hol_{x_{2}}(M_{2},g_{2}).
\end{align*}
Insbesondere ist $TM=TM_{1}\oplus TM_{2}$ eine Aufspaltung in $\Hol_{(x_1,x_2)}(M,g)$-invariante Unterräume.\fish
\end{prop}

\begin{proof}
Für $(p,q)\in M$ setzen wir für $a=1$ oder $2$,
\begin{align*}
E_{(p,q)}^a = 
\begin{cases}
T_{p}M_{1}, & a=1,\\
T_{q}M_{2}, & a=2.
\end{cases}
\end{align*}
Dann ist $E^a$ eine involutive Distribution und somit integrabel. Weiterhin gilt für die kovariante Ableitung auf $M$,
\begin{align*}
\nabla_{X_{1}\oplus X_{2}}^{g_{1}\times g_{2}}(Y_{1}\oplus Y_{2}) = 
\left(\nabla_{X_{1}}^{g_{1}}Y_{1}\right)\oplus \left(\nabla_{X_{2}}^{g_{2}}Y_{2}\right).
\end{align*}
Somit schreibt sich die Parallelverschiebung entlang einer Kurve $\gamma$ als Produkt,
\begin{align*}
P_{\gamma}^\nabla =
\begin{pmatrix}
P_{\gamma_{1}}^{\nabla_{1}} \\
& P_{\gamma_{2}}^{\nabla_{2}}
\end{pmatrix},
\end{align*}
bezüglich $TM=TM_{1}\oplus TM_{2}$. Somit sind $TM_{1}$ und $TM_{2}$ Holonomie-invariant.\qed
\end{proof}

Unser Ziel ist es nun eine Umkehrung für diese Aussage zu finden. Wann ist $M$ ein Riemannsches Produkt, wenn das Tangentialbündel $TM$ in Holonomie-invariante Unterräume zerfällt? Im Allgemeinen wird dies nur lokal der Fall sein.

\begin{defn}
\index{Mannigfaltigkeit!irreduzibel}
Eine Mannigfaltigkeit $(M,g)$ mit irreduzibler Holonomiedarstellung nennt man \emph{irreduzibel}.\fish
\end{defn}

Irreduzible Mannigfaltigkeiten bilden quasi die Bausteine für Mannigfaltigkeiten. Lokal sieht jede Mannigfaltigkeit wie ein Riemannsches Produkt aus solchen Bausteinen aus.

\begin{prop}[Zerlegungssatz]
Besitze die Holonomiedarstellung einen $k$-dimensionalen invarianten Unterraum. Dann existiert zu jedem $x\in M$ eine offene Umgebung $U$ von $x$ und Riemannsche Mannigfaltigkeiten $(U_{1},g_{1})$ und $(U_{2},g_{2})$ mit
\begin{align*}
(U,g) \cong (U_{1},g_{1})\times (U_{2},g_{2}).\fish
\end{align*}
\end{prop}
\begin{proof}[Beweisidee.]
Sei $E_{x}\subset T_{x}M$ ein Holonomie-invarianter Unterraum von $T_{x}M$. Dann ist auch $E_{x}^\bot$ Holonomie-invariant, denn $\Hol_{x}(M,g) \subset \O_{n}$. Durch $E_{x}$ und $E_{x}^\bot$ werden parallele Distributionen definiert. Da der gewählte Zusammenhang metrisch ist, sind diese auch integrabel. Nach dem Satz von Frobenius existieren maximale Integralmannigfaltigkeiten $M_{1}$ und $M_{2}$ zu $E$ bzw. $E^\bot$. Zu jedem Punkt $p\in M$ liefert der Satz von Frobenius außerdem Karten $(U_{i},x_{i})$ von $M_{i}$, so dass sich $M_{i}$ lokal als Graph einer Funktion schreiben lässt. Wir definieren nun Metriken
\begin{align*}
g_{1} \defl g\bigg|_{E\times E},\qquad
g_{2} \defl g\bigg|_{E^\bot\times E^\bot},
\end{align*}
mit denen $(U_{i},g_{i})$ zu einer Riemannschen Mannigfaltigkeit wird und $(U_{1}\times U_{2},g) = (U_{1},g_{1})\times (U_{2},g_{2})$ gilt.\qed
\end{proof}

\begin{prop}[Zerlegung der Holonomiegruppe]
Sei $E\subset T_{x}M$ ein Holonomie-invarianter Unterraum. Dann sind die Gruppen
\begin{align*}
H_{1} \defl \setdef{h\in \Hol_{x}^\o(M,g)}{h\big|_{E^\bot} = \Id_{E^\bot}},\qquad
H_{2} \defl \setdef{h\in \Hol_{x}^\o(M,g)}{h\big|_{E} = \Id_{E}},
\end{align*}
Normalteiler von $\Hol_{x}(M,g)$ und es gilt
\begin{align*}
\Hol_{x}^\o(M,g) = H_{1}\times H_{2}.\fish
\end{align*}
\end{prop}
\begin{proof}[Beweisidee.]
Sei $\gamma\in\Omega(x)$ eine geschlossene Kurve durch $x$. Nach Voraussetzung lässt die Parallelverschiebung $E$ und $E^\bot$ invariant, d.h.
\begin{align*}
P_{\gamma}E = E,\qquad P_{\gamma}E^\bot = E^\bot.
\end{align*}
Somit gilt aber auch, dass
\begin{align*}
P_{\gamma} H_{j} P_{\gamma}^{-1} \subset H_{j},\qquad j=1,2,
\end{align*}
also sind die $H_{j}$ tatsächlich Normalteiler von $\Hol_{x}(M,g)$. Nach Definition kommutieren Elemente von $H_{1}$ mit Elementen von $H_{2}$, d.h. $H_{1}\cap H_{2} = \setd{\Id_{T_{x}M}}$. Folglich ist
\begin{align*}
\psi\colon H_{1}\times H_{2}\to \Hol_{x}^\o(M,g),\qquad (a,b)\mapsto a\cdot b
\end{align*}
ein injektiver Gruppenhomomorphismus. Nun ist noch zu zeigen, dass $\psi$ auch surjektiv ist, d.h. dass sich jedes Element in $\Hol_{x}^\o(M,g)$ als Produkt aus Elementen von $H_{1}$ und $H_{2}$ schreiben lässt.

Sei wieder $\gamma$ eine beliebige geschlossene Kurve an $x$. 
Da ihre Spur kompakt ist, kann man mittels dem Lasso Lemma endlich viele Kurven $\delta_{j}$ konstruieren, deren Hintereinanderausführung wieder $\gamma$ ergeben, und für die gilt
\begin{align*}
\delta_{j} = \alpha_{j}^- \star \sigma_{j}\star \alpha,
\end{align*}
wobei $\sigma_{j}$ eine Schleife an einen Punkt $x_{j}$ bezeichnet, die vollständig in einer hinreichend kleinen Umgebung $U_{j}$ verläuft so, dass $M$ lokal in $U$ ein Riemannsches Produkt ist, und $\alpha_{j}$ eine Kurve von $x$ nach $x_{j}$ ist.
Schreibt man $\sigma_{j}=(\sigma_{j}^1,\sigma_{j}^2)$ gemäß der Riemannschen Produktstruktur und setzt $\delta_{j}^i = \alpha_{j}^-\star \sigma_{j}^i\star \alpha_{j}$, so sieht man ein, dass
\begin{align*}
P_{\delta_{j}^i} = P_{\alpha_{j}^{-}}P_{\sigma_{j}^i}P_{\alpha_{j}} \in H^i,
\end{align*}
denn $P_{\sigma_{j}^i}$ ist gerade die Translation einer Schleife an $x$. Somit
ist $P_{\gamma}$ als Hintereinanderausführung aller $P_{\delta_{j}^i}$ gerade ein Produkt von Elementen aus $H_1$ und $H_2$.\qed
\end{proof}

\begin{figure}[h]
\centering
\begin{pspicture}(0,-1.846467)(3.5031214,1.8464673)
\psframe[linecolor=darkblue](2.6431212,1.0535328)(0.6231213,-0.9664672)
\psdots[linecolor=darkblue](0.64312136,-0.9464672)

\rput(2.5731213,1.2785327){\color{darkblue}$\gamma$}
\psbezier[fillstyle=solid,fillcolor=darkyellow,opacity=0.2](0.40312135,0.3735328)(0.03941433,1.032808)(0.38637483,1.3129327)(0.9431214,1.3735328)(1.4998679,1.4341328)(2.2804093,1.0411184)(1.7231213,0.073532805)(1.1658335,-0.8940528)(0.76682836,-0.28574228)(0.40312135,0.3735328)
\psbezier[fillstyle=solid,fillcolor=darkblue,opacity=0.2](1.9831214,1.3135328)(2.9845467,1.8264672)(3.1337223,1.458873)(2.9631214,0.4735328)(2.7925205,-0.5118074)(2.576356,-0.17938672)(2.0231214,0.113532804)(1.4698867,0.40645233)(0.9816959,0.8005984)(1.9831214,1.3135328)
\psbezier[fillstyle=solid,fillcolor=purple,opacity=0.2](0.42312133,-0.26646718)(0.84624267,0.6396058)(1.4574132,0.01974501)(1.5031214,-0.4464672)(1.5488294,-0.91267943)(2.1231213,-0.8264672)(1.5431213,-1.2264673)(0.96312135,-1.6264672)(0.0,-1.1725402)(0.42312133,-0.26646718)
\psbezier[fillstyle=solid,fillcolor=darkgray,opacity=0.2](1.4631213,-0.9664672)(0.12312134,-0.106467195)(1.4031214,-0.3664672)(1.5231214,0.0535328)(1.6431214,0.4735328)(2.38003,0.91662866)(2.6631215,-0.0864672)(2.9462128,-1.089563)(2.8031213,-1.8264672)(1.4631213,-0.9664672)

\rput(0.7531213,-1.08){\color{darkblue}$x$}

\rput(0.70312136,-1.6214672){\color{purple}$U_{1}$}
\rput(2.8031213,-1.6014673){\color{darkgray}$U_{2}$}
\rput(3.2531214,1.4185328){\color{darkblue}$U_{3}$}
\rput(0.54312134,1.52){\color{darkyellow}$U_{4}$}
\end{pspicture} 
\begin{pspicture}(0,-1.53)(9.2,1.51)
\psframe(1.3,0.21)(0.1,-0.99)
\psframe(4.34,0.21)(3.14,-0.99)
\psline(1.98,-0.99)(1.98,0.13)(3.16,0.13)
\psline(3.16,0.19)(1.92,0.19)(1.92,-0.99)
\psframe(7.34,1.33)(6.14,0.13)
\psline(4.98,-0.97)(4.98,0.15)(6.16,0.15)
\psline(6.16,0.21)(4.92,0.21)(4.92,-0.97)
\psframe(9.08,1.39)(7.88,0.19)

\psline(7.9,0.21)(7.9,-1.19)
\psline(7.96,-1.19)(7.96,0.21)

\psline(0.0,-0.25)(0.12,-0.43)(0.24,-0.25)
\psline(3.04,-0.31)(3.16,-0.49)(3.28,-0.31)
\psline(1.8,-0.03)(1.92,-0.21)(2.04,-0.03)
\psline(6.04,0.95)(6.16,0.77)(6.28,0.95)
\psline(4.8,-0.25)(4.92,-0.43)(5.04,-0.25)
\psline(7.78,1.01)(7.9,0.83)(8.02,1.01)
\psline(7.78,-0.29)(7.9,-0.47)(8.02,-0.29)

\psline(1.16,-0.45)(1.28,-0.27)(1.4,-0.45)
\psline(1.86,-0.69)(1.98,-0.51)(2.1,-0.69)
\psline(4.2,-0.53)(4.32,-0.35)(4.44,-0.53)
\psline(4.86,-0.71)(4.98,-0.53)(5.1,-0.71)
\psline(7.2,0.61)(7.32,0.79)(7.44,0.61)
\psline(7.84,-0.87)(7.96,-0.69)(8.08,-0.87)
\psline(8.94,0.57)(9.06,0.75)(9.18,0.57)

\psline(5.52,0.27)(5.7,0.15)(5.52,0.03)
\psline(3.86,0.31)(3.68,0.19)(3.86,0.07)
\psline(6.86,1.43)(6.68,1.31)(6.86,1.19)
\psline(8.6,1.49)(8.42,1.37)(8.6,1.25)

\psline(0.82,0.31)(0.64,0.19)(0.82,0.07)
\psline(2.56,0.31)(2.38,0.19)(2.56,0.07)
\psline(5.44,0.33)(5.26,0.21)(5.44,0.09)

\psline(0.54,-0.85)(0.72,-0.97)(0.54,-1.09)
\psline(2.6,0.25)(2.78,0.13)(2.6,0.01)
\psline(6.6,0.27)(6.78,0.15)(6.6,0.03)

\psline(3.62,-0.85)(3.8,-0.97)(3.62,-1.09)
\psline(8.42,0.33)(8.6,0.21)(8.42,0.09)

\rput(0.67,-1.345){\color{purple}1.}
\rput(0.7,-0.4){\color{purple}$\sigma_{1}$}

\rput(3.21,-1.345){\color{darkgray}2.}

\rput(3.8,-0.4){\color{darkgray}$\sigma_{2}$}
\rput(2.6,0.5){\color{darkgray}$\alpha_{2}$}

\rput(6.51,-1.345){\color{darkblue}3.}
\rput(6.7,0.7){\color{darkblue}$\sigma_{3}$}
\rput(5.5,-0.1){\color{darkblue}$\alpha_{3}$}

\rput(8.53,-1.365){\color{darkyellow}4.}
\rput(8.53,0.7){\color{darkyellow}$\sigma_{4}$}
\rput(8.2,-0.5){\color{darkyellow}$\alpha_{3}$}

\end{pspicture}
\caption{Zur Konstruktion der Kurven des Lasso Lemmas.}
\end{figure}

Zerfällt der Tangentialraum an einen Punkt in irreduzible Unterräume, dann
besagt der Zerlegungssatz, dass die Mannigfaltigkeit lokal die Gestalt eines
Riemannschen Produktes hat und außerdem die Holonomiegruppe zerfällt. Für
einfach zusammenhängende Mannigfaltigkeiten gilt folgendes globales Resultat.

\begin{prop}[Zerlegungssatz von de-Rham]
Sei $(M,g)$ eine einfach zusammenhängende, vollständige, Riemannsche
Mannigfaltigkeit. Dann ist $(M,g)$ isometrisch zu einem Produkt einfach
zusammenhängender, vollständiger, Riemannscher Mannigfaltigkeiten
\begin{align*}
(M,g) \cong (M_0,g_0)\times \ldots \times (M_k,g_k).
\end{align*}
Dabei ist $(M_0,g_0)$ ein euklidischer Raum und für $i\neq 0$ sind die
Mannigfaltigkeiten $(M_i,g_i)$ irreduzibel und nicht flach. Weiterhin gilt
\begin{align*}
\Hol_x(M,g) \cong \Hol_x(M_1,g_1)\times \ldots \times \Hol_x(M_k,g_k)\fish.
\end{align*}
\end{prop}

\begin{proof}[Beweisidee.]
Für ein $x\in M$ fest betrachten wir die Holonomiedarstellung
\begin{align*}
\rho\colon \Hol_x(M,g)\to \Aut(T_xM).
\end{align*}
Nach Voraussetzung zerfällt $T_xM$ in
irreduzible, invariante Unterräume,
\begin{align*}
T_xM = E_0\oplus \ldots \oplus E_k.
\end{align*}
Folglich lässt sich nach Satz
\ref{prop:Paralleles-Unterbündel-Invarianz-Holonomiedarstellung}
das Tangentialbündel als Summe aus parallelen Distributionen $E_i$ beschreiben.
Diese sind integrabel, also existieren nach dem Satz von Frobenius maximale
Integralmannigfaltigkeiten $M_i$ zu $E_i$. Man zeigt nun, dass $M$ tatsächlich
das Produkt dieser Mannigfaltigkeiten ist und, dass deren Metrik durch
Einschränkung gegeben ist,
\begin{align*}
M = M_0\times \ldots \times M_k,\qquad g_i = g\bigg|_{E_i\times E_i}.\qed
\end{align*}
\end{proof}

\section{Klassifikation der Holonomiegruppen}

Lassen wir beliebige Zusammenhänge auf dem Hauptfaserbündel $P\to M$ zu, so
können wir keine Einschränkung an die Holonomiegruppe machen. Erstaunlicherweise
ist für den Levi-Civita Zusammenhang nur eine sehr kurze Liste von
Holonomiegruppen möglich.

\begin{prop}[Klassifikationstheorem von Berger]
Sei $(M,g)$ einfach zusammenhängend, orientierbar, irreduzibel und nicht
symmetrisch. Dann sind für die Holonomiegruppe $\Hol_x(M,g)$ nur folgende
Gruppen möglich:\\

\vspace{2mm}

\begin{tabular}{c|l|l}
Dimension  & Gruppe & \\\hline
$n$ & $\SO(n)$ & der generische Fall\\
$n=2m$ & $\U(m)$ & Kähler-Mannigfaltigkeit\\
$n=2m$ & $\SU(m)$ & Calabi-Yau Mannigfaltigkeit\\
$n=4m$ & $\Sp(m)$ & Hyperkähler-Mannigfaltigkeit\\
$n=4m$ & $\Sp(m)\cdot \Sp(1)$ & quaternionische Kähler-Mannigfaltigkeit\\
$n=7$ & $G_2$\\
$n=8$ & $\Spin(7)$.\fish
\end{tabular}
\end{prop}

\begin{rem}[Bemerkungen.]
\begin{remenum}
\item
Für eine nicht einfach zusammenhängende Mannigfaltigkeit betrachte die
universelle Überlagerung
\begin{align*}
\pi\colon \tilde M \overset{\pi_1 M}{\longrightarrow} M.
\end{align*}
Diese ist einfach zusammenhängend und es gilt $\Hol_x(\tilde{M})
=\Hol_x^\o(\tilde{M}) = \Hol_x^\o(M)$. Somit ist im Klassifikationstheorem für
nicht einfach zusammenhängende Mannigfaltigkeiten die Holonomiegruppe durch die
reduzierte Holonomiegruppe $\Hol_x^\o(M)$ zu ersetzen.
\item Die möglichen Holonomiegruppen sind genau die Gruppen, die transitiv auf
Sphären operieren. Es ist zum Beispiel,
\begin{align*}
&S^n  =\SO(n+1)/\SO(n),&& \S^{2n+1} = \SU(n+1)/\SU(n),\\
&S^6 = G_2/\SU(3), && S^7 = \Spin(7)/G_2.
\end{align*}
\item Falls die Holonomiegruppe aus $\setd{\SU(m),\Sp(m),\Spin(7),\G_2}$ stammt,
verschwindet die Ricci-Krümmung und folglich ist die Mannigfaltigkeit Einstein.
Auch für $\Hol_x\subset \Sp(m)\cdot \Sp(1)$ ist $M$ Einstein.

Daher spielen diese Gruppen eine fundamentale Rolle in der Physik.\map  
\end{remenum}
\end{rem}

\chapter{Charakteristische Klassen}

Charakteristische Klassen bilden einen Mechanismus, der einem gegebenen Vektor-
bzw. Hauptfaserbündel eine de-Rham Kohomologieklasse zuordnet,
welche die Nichttrivialität der Mannigfaltigkeit misst,
\begin{align*}
M\mapsto c(M)\in H_\dR(M),
\end{align*}
d.h. für das triviale Bündel verschwinden alle charakteristischen Klassen.
Weiterhin charakterisieren diese Klassen insofern, dass für isomorphe Bündel
alle Klassen übereinstimmen und umgekehrt zwei Bündel, bei denen eine Klasse
nicht übereinstimmt, auch nicht isomorph sind.

\begin{ex}
Ein prominentes Beispiel für charakteristische Klassen sind die \emph{Chern
Klassen} für ein komplexes Vektorbündel $E\to M$ (siehe
\ref{defn:Erste-Chern-Klasse}). Ganz allgemein ist die $i$-te
Chern-Klasse von $E$ definiert als die Kohomologieklasse
\begin{align*}
c_i(E)\in H_\dR^{2i}(M),\qquad i =0,1,\ldots 
\end{align*}
so dass für die \emph{totale Chern-Klasse} $c(E) = c_0(E) + c_1(E) + \ldots$
gilt:
\begin{defnenum}
\item $f^*c(E) = c(f^*E)$,
\item $c(E\oplus F) = c(E)c(F)$,
\item $c_1(L) [\CP^1] = 1$,\quad für $L\to \CP^1$ das tautologische
Geradenbündel.
\end{defnenum}

Weiterhin definiert man für einen aufsteigenden Index $i_1 < \ldots < i_k$
die \emph{Chern Zahl} als
\begin{align*}
\lin{c_{i_1}(E)\cdot \ldots \cdot c_{i_k}(E),[M]}
\defl
\int_M \alpha_{i_1}\wedge \ldots \wedge \alpha_{i_k}\, \ddd M,
\end{align*}
wobei $\alpha_{i_r}$ eine geschlossene $2i_r$-Form auf $M$ bezeichnet mit
$[\alpha_{i_r}] = c_{i_r}$.

Die Chern Zahlen nennt man auch topologische Quantenzahlen oder, in der
$\U(1)$-Theorie, Ladungen.\boxc
\end{ex}

\section{Konstruktion}

Wir wollen nun einen allgemeinen Konstruktionsmechanismus für charakteristische
Klassen erarbeiten. Sei dazu $\pi\colon P\to M$ ein $G$-Hauptfaserbündel und $\g$ die
Lie-Algebra der Lie-Gruppe $G$.

\begin{defn}
Eine $k$-lineare, symmetrische Abbildung
\begin{align*}
f : \g\times \ldots \times \g \to \C
\end{align*}
heißt \emph{$G$-invariant}, falls für alle $X_1,\ldots,X_k\in \g$ und $g\in G$
gilt
\begin{align*}
f(\Ad(g)X_1,\ldots,\Ad(g)X_k) = f(X_1,\ldots,X_k).
\end{align*}
Den \emph{Raum der $k$-linearen, symmetrischen und $G$-invarianten Abbildungen
auf $\g$} bezeichnen wir mit \emph{$\Sym^k(\g)^G$}.\fish
\end{defn}

Durch punktweise Addition und skalare Multiplikation wird $\Sym^k(\g)^G$ zu
einem Vektorraum über den komplexen Zahlen.

\begin{lem}
Die Menge $\Sym(\g)^G = \sum_{k\ge0} \Sym^k(\g)^G$ ist eine kommutative
Algebra bezüglich der Multiplikation
\begin{align*}
(f\cdot h)(X_1,\ldots,X_{k+l}) \defl
\frac{1}{(k+l)!}\sum_{\sigma\in S_{k+l}} f(X_{\sigma_1},\ldots,X_{\sigma_k})
h(X_{\sigma_{k+1}},\ldots,X_{\sigma_{k+l}}).\fish
\end{align*}
\end{lem}

\begin{rem}
Die Algebra $\Sym(\g)$ ist isomorph zur Algebra der symmetrischen polynomialen
Funktionen auf $\g$. Sei $(e_1,\ldots,e_r)$ eine Basis von $\g$. Eine
symmetrische, homogene polynomiale Funktion vom Grad $k$ lässt sich beschreiben
durch
\begin{align*}
p: \g\to \C,\qquad p(X)
 = \sum_{1\le i_1,\ldots,i_k \le r}
\!\!\!\!
a_{i_1\ldots i_k}\, X_{i_1}\cdot \ldots \cdot X_{i_k},
\qquad
X=\sum_{i=1}^r X_i e_i,
\end{align*}
wobei die Koeffizienten $a_{i_1\ldots i_k}$ symmetrisch in den Indizes sind.

Zu einer Abbildung $f\in \Sym^k(\g)$ definiert man eine polynomiale Funktion
\begin{align*}
P_f(X)  = f(X,\ldots,X).
\end{align*}
Diese ist offenbar symmetrisch und homogen vom Grad $k$.

Umgekehrt definiert man für eine homogene, polynomiale Funktion vom Grad $k$
\begin{align*}
p: \g\to \C,\qquad p(X) = \sum_{1\le i_1,\ldots,i_k \le r}
\!\!\!\!
a_{i_1\ldots i_k}\, X_{i_1}\cdot \ldots \cdot X_{i_k},
\end{align*}
eine Abbildung $f\in \Sym^k(\g)$ durch
\begin{align*}
f(e_{i_1},\ldots,e_{i_k}) = a_{i_1\ldots i_k},\qquad 1\le i_1,\ldots,i_k \le r
\end{align*}
und setzt $f$ anschließend multilinear fort.\map
\end{rem}

Die $G$-invarianten Abbildungen aus $\Sym^k(\g)$ entsprechen nun polynomialen
Funktionen, die invariant unter der $\Ad$-Wirkung sind.

\begin{defn}
\index{Polynom!invariantes}
Ein \emph{invariantes Polynom} ist eine polynomiale Funktion $p\colon\g\to \C$ mit
\begin{align*}
p(\Ad(g)X) = p(X)
\end{align*}
für alle $X\in\g$ und $g\in G$.\fish
\end{defn}

Determinante und Spur sind einfache Beispiele für invariante Polynome.

\begin{rem}
\index{Maximaler Torus}
\index{Weyl-Gruppe}
Eine Menge $t\subset\g$, so dass jedes Element in $\g$ unter $G$ konjugiert ist zu einem Element aus $t$, also
\begin{align*}
GtG^{-1} = \g,
\end{align*}
heißt \emph{maximaler Torus}. Die Gruppe $W$ der Elemente, die den maximalen Torus in sich überführen,
\begin{align*}
WtW^{-1} \subset t,
\end{align*}
heißt \emph{Weyl-Gruppe}. Damit lassen sich die invarianten Polynome darstellen als
\begin{align*}
\Sym^k(\g)^G = \Sym^k(t)^W = \C[\sigma_{1},\ldots,\sigma_{r}],
\end{align*}
d.h. jedes invariante Polynome ist bereits eindeutig durch die Werte auf dem maximalen Torus $t$ bestimmt und invariant unter der Wirkung der Weyl Gruppe.\map
\end{rem}

\begin{ex}
Sei $G=\GL_n^\C$ und $\g = \gl_n$, der Matrizenring.  Der maximale Torus zu $t$ sind die Diagonalmatrizen
\begin{align*}
t = \setdef{\diag(\lambda_{1},\ldots,\lambda_{n})}{\lambda_{i}\in \C},
\end{align*}
die wir mit ihren Eigenwerten, also $\C^n$, identifizieren können. Die Weyl-Gruppe, also die Gruppe, die $t$ erhält, sind gerade die Permutationen. Die invarianten Polynome sind also symmetrische polynomiale Funktionen, die nur von den
Eigenwerten abhängen.~\boxc
\end{ex}

\begin{prop}
Jedes symmetrische Polynom in $\lambda_1,\ldots,\lambda_n$ lässt sich als
Polynom in den elementarsymmetrischen Funktionen schreiben.\fish
\end{prop}

Unser Ziel ist nun, für jedes invariante Polynom eine charakteristische Klasse
zu definieren.

\begin{defn}
Seien $\omega_1 \in \Omega^{i_1}(N,\g)$, \ldots, $\omega_k\in\Omega^{i_k}(N,\g)$
und $r=i_1+\ldots+i_k$. Zu einer Abbildung $f\in \Sym^k(\g)^G$ definiert man 
die \emph{$r$-Form $f(\omega_1\wedge \ldots \wedge
\omega_k)$} $\in\Omega^r(N,\C)$ durch
\begin{align*}
&f(\omega_1\wedge \ldots \wedge \omega_k)(Y_1,\ldots,Y_r) \\
&\qquad\defl
\frac{1}{i_1!\cdot \ldots \cdot i_k!} \sum_{\sigma\in S_r}
\sign(\sigma) f(\omega_1(Y_{\sigma_1},\ldots,Y_{\sigma_{i_1}}), \ldots,
\omega_k(\ldots,Y_{\sigma_r})).\fish
\end{align*}
\end{defn}

Unter Verwendung der Krümmung $F^A$ des Zusammenhanges $A$ können wir jetzt
jedem invarianten Polynom eine Form auf dem Hauptfaserbündel $P$ mit Werten in
$\C$ zuordnen.

\begin{defn}
Sei $A$ ein Zusammenhang auf $P$ mit Krümmung $F^A\in\Omega_\hor^2(P,\g)^\Ad$.
Für $f\in \Sym^k(\g)^G$ definiert man nun
\begin{align*}
f(F^A) \defl f(F^A\wedge \ldots \wedge F^A)\in \Omega^{2k}(P,\C).\fish
\end{align*}
\end{defn}

\begin{ex}
Sei $e_1,\ldots,e_r$ eine Basis von $\g$, dann schreibt sich die Krümmung in
dieser Basis als
\begin{align*}
F^A = \sum_{i=1}^r F_i e_i,\qquad F_i\in\Omega_\hor^2(P,\C)^\Ad \cong
\Omega^2(M,\C).
\end{align*}
Ein invariantes Polynom $p$ entspricht einer Abbildung $f\in \Sym^k(\g)^\G$  und
wir schreiben
\begin{align*}
p(F^A) = f(F^A) = \sum_{1\le i_1,\ldots,i_k\le r} a_{i_1\ldots i_k}
F_{i_1}\wedge \ldots \wedge F_{i_k},
\end{align*}
wobei die $a_{i_1\ldots i_k}$ die Koeffizienten von $p$ in der gewählten Basis
von $\g$ bezeichnen. Aufgrund der Invarianz von $p$ unter $\Ad$ ist diese
Definition unabhängig von den gewählten Koordinaten und daher ist $p(F^A)
= f(F^A)$ wohldefiniert.

Ist zum Beispiel $\g = \gl_n$, der Matrizenring, dann ist eine Basis durch die
Standardeinheitsmatrizen $E_{ij}$ gegeben, so dass
\begin{align*}
F^A = \sum_{i,j=1}^n F_{ij}E_{ij},\qquad \tr(F^A) = \sum_{i=1}^n F_{ii}.\boxc
\end{align*}
\end{ex}

\begin{lem}
Sei $f\in \Sym^k(\g)^G$ und $A$ ein Zusammenhang auf $P$. Dann gelten:
\begin{propenum}
\item $f(F^A) \in \Omega_\hor^{2k}(P,\C)^{\Ad}\cong \Omega^{2k}(M,\C)$.
\item Die Differentialform $f(F^A)$ ist geschlossen, d.h. $\df(F^A) = 0$.
\item Seien $A_0$ und $A_1$ Zusammenhänge auf $P$, dann ist die
Differenz $f(F^{A_0})-f(F^{A_1})$ exakt. Genauer gilt
\begin{align*}
f(F^{A_0}) = f(F^{A_1}) + \dom,
\end{align*}
wobei $\omega$ eine horizontale $(2k-1)$-Form auf $P$ vom Typ $\Ad$
bezeichnet.~\fish
\end{propenum}
\end{lem}

Somit ist $f(F^A)$ eine geschlossene Form auf $M$, liegt also in der de Rham
Kohomologie, und ist unabhängig vom gewählten Zusammenhang.

\begin{proof}
a): Die Krümmung ist eine horizontale Form und $f(F^A)$ involviert nur zyklische
Permutationen von $F^A$, angewandt auf die Argumente von $f(F^A)$, ist also
selbst wieder horizontal. Weiterhin vertauscht der Pullback mit $f$, so dass für
jedes $g\in G$ gilt
\begin{align*}
R_g^*f(F^A) = f(R_g^* F^A) = f(\Ad(g^{-1})F^A) = f(F^A), 
\end{align*}
denn die Krümmung ist vom Typ $\Ad$ und $f$ ist $\Ad$-invariant. Also ist
$f(F^A)$ ein Element von $\Omega^{2k}_\hor(P,\C)^{\Ad}\cong \Omega^{2k}(M,\C)$.

b): Für horizontale Formen vom Typ $\Ad$ stimmt das gewöhnliche Differential mit
dem absoluten Differential überein. Wählen wir eine Basis $(e_1,\ldots,e_r)$ von
$\g$, dann schreibt sich
\begin{align*}
F^A = \sum_{i=1}^r F_i e_i,\qquad F_i\in \Omega_\hor^2(P,\C)^{\Ad},
\end{align*}
und somit gilt
\begin{align*}
\ddd f(F^A) &= D_A f(F^A)
=  \sum_{1\le i_1,\ldots,i_k \le r} D_A(F_{i_1}\wedge \ldots \wedge F_{i_k})\;
f(e_{i_1},\ldots,e_{i_k})\\
&= 
\sum_{1\le i_1,\ldots,i_k \le r} \sum_{j=1}^k F_{i_1}\wedge \ldots\wedge (D_A
F_{i_j}) \wedge \ldots \wedge F_{i_k}\; f(e_{i_1},\ldots,e_{i_k})\\
&=0,
\end{align*}
denn aufgrund der Bianchi-Identität ist $D_A F^A = 0$ und folglich verschwinden
auch alle $D_A F_{i_j}$.

c): Seien $A_0$ und $A_1$ Zusammenhänge auf $P$. Setzen wir $X= M\times \R$ und
bezeichnen die Projektion mit
$\phi\colon X\to M$, $(x,t)\mapsto x$,
so ist das Pullback Bündel gegeben durch
\begin{align*}
\tilde{P} = \phi^*P = \setdef{(x,p)\in X\times P}{\phi(x)=\pi(p)},
\end{align*}
und das folgende Diagramm kommutiert:

{

\centering

\begin{tikzpicture}[description/.style={fill=white,inner sep=2pt}]
\matrix (m) [matrix of math nodes, row sep=3em,
column sep=2.5em, text height=1.5ex, text depth=0.25ex]
{ \phi^*P & & P \\
N &  & M \\ };
\path[->,font=\scriptsize]
(m-1-1) edge node[auto] {$ \bar{\phi} $} (m-1-3)
	    edge node[auto,swap] {$ \bar{\pi} $} (m-2-1)
(m-2-1) edge node[auto] {$ \phi $} (m-2-3)
(m-1-3) edge node[auto] {$ \pi $} (m-2-3);
\end{tikzpicture}

}

\noindent
Durch $\tilde{A}_0 = \bar{\phi}^*A_0$ und  $\tilde{A}_1 = \bar{\phi}^*A_1$ sind
Zusammenhänge auf $\tilde{P}$ gegeben.
Setzen wir für $x=(m,\lambda)\in X$ und $X\in T_{(x,p)}\tilde{P}$,
\begin{align*}
\tilde{A}(X) \defl (1-\lambda)\tilde{A}_0(X) + \lambda\tilde{A}_1(X),
\end{align*}
dann ist auch $\tilde{A}$ ein Zusammenhang auf $\tilde{P}$. Die Krümmung
 können wir als 2-Form auf $X$ mit Werten im adjungierten Bündel
identifizieren $F^{\tilde{A}}\in\Omega^2(X,\Ad(\tilde{P}))$. Schreiben wir für
die Inklusion
\begin{align*}
i_\lambda\colon M\to X,\qquad m\mapsto (m,\lambda),
\end{align*}
dann gilt für Vektorfelder $X$ und $Y$ auf $M$,
\begin{align*}
(i_\lambda^* F^{\tilde{A}})(X,Y) &= 
F^{\tilde{A}}(\ddd i_\lambda(X),\ddd i_\lambda(Y))\\
&= 
[(p,x),\bar{F}^{\tilde{A}}(\ddd i_\lambda(\tilde{X}),\ddd
i_\lambda(\tilde{Y}))]\\
&= 
[(p,x),(1-\lambda)\bar{F}^A_0(\ddd i_\lambda(\tilde{X}),\ddd
i_\lambda(\tilde{Y})) + 
\lambda \bar{F}^A_1(\ddd i_\lambda(\tilde{X}),\ddd i_\lambda(\tilde{Y}))
]\\
&=
(1-\lambda)[p,\bar{F}^A_0(\tilde{X},\tilde{Y})] +
\lambda[p,\bar{F}^A_1(\tilde{X},\tilde{Y})]\\
&=
(1-\lambda)F^A_0(X,Y) + \lambda F^{A}_1(X,Y).
\end{align*}
Die Abbildungen $i_0$ und $i_1$ sind homotop, daher stimmen $i_0^*$ und $i_1^*$
auf der Kohomologie überein, also gilt
\begin{align*}
[f(F^{A_0})] = [i_0^*f(F^{\tilde{A}})] = [i_1^*f(F^{\tilde{A}})] = 
[f(F^{A_1})].
\end{align*}
Somit unterschieden sich $f(F^{A_0})$ und $f(F^{A_1})$ nur um eine exakte Form
auf $M$ und diese entspricht einer exakten, horizontalen Form vom Typ
$\Ad$ auf $P$.
%
%c): Sei $A$ ein Zusammenhang auf $P$, dann ist $F^A$ eine horizontale, $\g$ wertige 2-Form vom Typ $\Ad$. Wählen wir eine Basis $e_1,\ldots,e_n$ von $\g$, dann schreib sich
%\begin{algin*}
%F^A = \sum F_i e_i,\qquad F_i \in\Omega_\hor^2(P,\C)^\Ad.
%\end{align*}
%Die $F_i$ lassen sich somit mit gewöhnlichen Formen $\Omega^2(M,\C)$ identifizieren.
%
%
%
\qed
\end{proof}

\begin{defn}
Die Abbildung
\begin{align*}
\Ws_P : \Sym(\g)^G \to H_{\dR}(M,\C),\qquad f \mapsto [f(F^A)]
\end{align*}
heißt \emph{Weil-Homomorphismus} des $G$-Hauptfaserbündels $P$.\fish
\end{defn}

In folgendem Satz fassen wir die Eigenschaften charakteristischer Klassen
zusammen.

\begin{prop}
\begin{propenum}
\item Sei $\phi\colon N\to M$ differenzierbar, dann gilt
\begin{align*}
\Ws_{\phi^* P} = \phi^*\Ws_P.
\end{align*}
\item Sind $P_1$ und $P_2$ isomorphe Hauptfaserbündel, so gilt $\Ws_{P_1} =
\Ws_{P_2}$.
\item Ist $P$ isomorph zum trivialen Bündel, dann gilt $\Ws_P\equiv 0$.\fish
\end{propenum}


\end{prop}
\begin{proof}
a): 
Das zurückgezogene Bündel ist gegeben durch
\begin{align*}
\phi^*P \defl \setdef{(x,p)\in N\times P}{\phi(x) = \pi(p)},
\end{align*}
so dass folgendes Diagramm kommutiert

{

\centering

\begin{tikzpicture}[description/.style={fill=white,inner sep=2pt}]
\matrix (m) [matrix of math nodes, row sep=3em,
column sep=2.5em, text height=1.5ex, text depth=0.25ex]
{ \phi^*P & & P \\
N &  & M \\ };
\path[->,font=\scriptsize]
(m-1-1) edge node[auto] {$ \pr_2 $} (m-1-3)
	    edge node[auto,swap] {$ \bar{\pi} $} (m-2-1)
(m-2-1) edge node[auto] {$ \phi $} (m-2-3)
(m-1-3) edge node[auto] {$ \pi $} (m-2-3);
\end{tikzpicture}

}

Sei $A$ ein Zusammenhang auf $P$, dann ist $\pr_2^*A$ ein
Zusammenhang auf $\phi^*P$ und für ein horizontales Vektorfeld $X^*$ auf $P$
ist auch $\ddd \pr_2(X^*)$ ein horizontales Vektorfeld auf $\phi^*P$. Die
Krümmung auf $\phi^*P$ ist gegeben durch $F^{\pr_2^* A} = \pr_2^*
F^A$. Somit gilt für Vektorfelder $X_1,\ldots,X_{2k}$ auf $N$ und ein
$f\in \Sym^k(\g)^G$,
\begin{align*}
f(F^{\pr_2^*A})_x(X_1,\ldots,X_{2k})
&= 
f(\pr_2^* F^A \wedge \ldots\wedge \pr_2^* F^A)_{(x,p)}(X_1^*,\ldots,X_{2k}^*)\\
&= 
f(F^A \wedge \ldots\wedge
F^A)_{p}(\ddd\pr_2(X_1^*),\ldots,\ddd\pr_2(X_{2k}^*)) \\
&= f(F^A \wedge
\ldots\wedge F^A)_{p}(\dphi(X_1^*),\ldots,\dphi(X_{2k}^*)) \\
&= \phi^*
f(F^A)_x(X_1,\ldots,X_{2k}),
\end{align*}
denn aufgrund der Kommutativität des Diagramms und der Eindeutigkeit des
horizontalen Lifts gilt $\dpi(\ddd\pr_2(X^*)) = \dphi(\ddd\bar{\pr}(X^*)) =
\dphi(X)$. Also gilt auch
\begin{align*}
\Ws_{\phi^*P}(f) = [f(F^{\pr_2 A})] = \phi^*[f(F^A)]
= \phi^* \Ws_{P}(f).
\end{align*}

b): Sei $\psi\colon P_1\to P_2$ ein Hauptfaserbündelisomorphismus und $A_2$ ein
Zusammenhang auf $P_2$, dann ist $A_1=\psi^* A_2$ ein Zusammenhang auf $P_1$.
Für ein Vektorfeld $X$ auf $M$ mit $A_1$-horizontalem Lift $X^*$ auf $P_1$, ist
$\dps(X^*)$ ein $A_2$-horizontaler Lift auf $P_2$. Somit gilt für Vektorfelder
$X_1,\ldots,X_{2k}$ auf $M$ und eine Abbildung $f\in \Sym^k(\g)^G$,
\begin{align*}
f(F^{A_1})_x(X_1,\ldots,X_{2k})_p
&= f(F^{A_2}\wedge \ldots \wedge
F^{A_2})_p(\dps(X_1^*),\ldots,\dps(X_{2k}^*))\\ 
&= f(F^{A_2})_{x}(X_1,\ldots,X_{2k}).
\end{align*}

c): Sei $P$ trivial, also isomorph zum trivialen Bündel $P_0$. Der
kanonisch flache Zusammenhang $A_0$ auf $P_0$ ist flach, d.h. $F^{A_0}\equiv 0$
und folglich ist nach b)
\begin{align*}
\Ws_P = \Ws_{P_0} = 0.\qed
\end{align*}

\end{proof}

\section{Charakteristische Klassen in Vektorbündeln}

Sei $E\to M$ ein Vektorbündel vom Rang $n$ mit kovarianter Ableitung $\nabla$.
Außerdem sei $G = \GL(n,\C)$, und $\g = \gl(n,\C)$ bezeichne die
komplexwertigen $n\times n$ Matrizen. Die Krümmung auf $E$ ist eine 2-Form mit
Werten in dem Endomorphismenbündel,
\begin{align*}
R_{XY}s = \nabla_X\nabla_Y s - \nabla_Y \nabla X s - \nabla_{[X,Y]}s,\qquad
s\in \Gamma(E),\; X,Y\in \chi(M).
\end{align*}
Wählen wir eine lokale Basis von $E$ also Schnitte $s_1,\ldots,s_n$ definiert
über $U\subset M$, dann schreibt sich die Krümmung lokal als
\begin{align*}
R_{XY}s_i = \sum_{j=1}^n F_{ij}(X,Y)s_j,
\end{align*}
mit gewöhnlichen 2-Formen $F_{ij}\in\Omega^2(U)$. Also können wir die Krümmung
mit
\begin{align*}
F \defl (F_{ij}), 
\end{align*}
einer $\g$-wertigen 2-Form auf $U$, identifizieren. Diese 2-Form hängt jedoch von
der Wahl der Schnitte $s_i$ ab und ist zunächst nur lokal in $U$ definiert. Hier
kommen die invarianten Polynome ins Spiel\ldots

\begin{lem}
Seien $s_1,\ldots,s_n$ und $s_1',\ldots,s_n'$ zwei lokale Basen von $E$ über
$U\subset M$, und sei $p$ ein invariantes Polynom. Dann gilt
\begin{align*}
p(F) = p(F').
\end{align*}
Insbesondere ist $p(F)\in\Omega^{even}(M,\C)$ global definiert und unabhängig
von der Wahl der $s_i$.\fish
\end{lem}

\begin{proof}
Sei $g\colon U\to \GL(n,\C)$ die Transformation, die $s_1,\ldots,s_n$ in
$s_1',\ldots,s_n'$ überführt. Dann schreibt sich
\begin{align*}
F'(X,Y) = gF(X,Y)g^{-1} = \Ad(g)F(X,Y).
\end{align*}
Da das Polynom $\Ad$-invariant ist, folgt $p(F') = p(F)$. Somit ist $p(F)$
unabhängig von der gewählten Lokalisierung $s_1,\ldots,s_n$. Also sind alle
Lokalisierungen kompatibel und lassen sich zu einem globalen $p(F)$
verkleben.\qed
\end{proof}


\begin{ex}
Die Spur und die Determinante sind invariante Polynome und man erhält,
\begin{align*}
\tr(F) &= \sum_{i=1}^n F_{ii} \in \Omega^2(M,\C),\\
\det(F) &= \sum_{\sigma\in S_n} \sign(\sigma) F_{1\sigma(1)}\wedge \ldots
\wedge F_{n\sigma(n)} \in \Omega^{2n}(M,\C).\boxc
\end{align*} 
\end{ex}


Interpretiert man $E\to M$ als zum
$G$-Hauptfaserbündel $P\to M$  assoziiertes Vektorbündel und $\nabla$ als die
durch den Zusammenhang $A$ induzierte kovariante Ableitung, dann entspricht  $p(F^\nabla)$ der auf dem Hauptfaserbündel definierten
Form $p(F^A)$. Insbesondere ist $p(F^\nabla)$ geschlossen und hängt bis auf
eine exakte Form nicht von der gewählten kovarianten Ableitung $\nabla$ ab, ist also ein Element
der de-Rham Kohomologie von $M$ und es gilt
\begin{align*}
[p(F^\nabla)] = \Ws_P(p) = [p(F^A)].
\end{align*}
Somit übertragen sich die Eigenschaften von $\Ws_p$ auch auf $[p(F^\nabla)]$.

\begin{lem}
Sei $E$ ein Vektorbündel über $M$ und $f\colon N\to M$ differenzierbar. Dann gelten
\begin{propenum}
\item $f^*[p(F^\nabla)] = [p(f^*F^\nabla)]$,
\item Falls $E$ trivial ist, gilt $[p(F^\nabla)]\equiv 0$.
\item Sind $E_1$ und $E_2$ isomorph, dann gilt $[p(F^{E_1})] \equiv
[p(F^{E_2})]$.\fish
\end{propenum}
\end{lem}

\section{Invariante Polynome auf $\gl(n,\C)$}

Die Abbildungen aus $\Sym^k(\g)^G$ sind äquivalent zu den symmetrischen,
homogenen, polynomialen Funktionen auf $\g$ vom Grad $k$, die invariant unter
der $\Ad$-Wirkung sind --- kurz den invarianten Polynomen. Für den Spezialfall
$\g=\gl(n,\C)$ sind dies gerade die symmetrischen polynomialen
Funktionen in den Eigenwerten. Um diese genauer untersuchen zu können,
definieren wir die elementarsymmetrischen polynomialen Funktionen $\sigma_k$
und $s_k$ für $k=0,1,\ldots$ implizit durch die Gleichungen
\begin{align*}
\det(\Id + tX) &= \sum_{k=0}^n \sigma_k(t) t^k,\\
\tr(\e^{tX}) &= \sum_{k\ge 0} s_k(t)\frac{t^k}{k!},
\end{align*}
für jedes $X\in \gl(n,\C)$. Da sowohl Determinante als auch Spur $\Ad$-invariant
sind, gilt dies auch für die $\sigma_k$ und $s_k$.

\begin{ex}
Entwickeln wir die Determinante $\det(\Id+tX)$, ergibt sich 
\begin{align*}
\sigma_0(X) = 1,\qquad \sigma_1(X) = \tr(X),\qquad \sigma_n(X) = \det(X).
\end{align*}
Analog erhält man beim Entwickeln der Spur,
\begin{align*}
s_0(X) = n, \qquad s_1(X) = \sigma_1(X).\boxc
\end{align*}
\end{ex}

Ganz allgemein erhält man für die elementarsymmetrischen Funktionen die folgende
Darstellung.

\begin{lem}
Sei $X$ diagonalisierbar mit Eigenwerten $\lambda_1,\ldots,\lambda_n$, dann
gelten
\begin{align*}
\sigma_k(X) &= \sigma_k(\lambda_1,\ldots,\lambda_n) = \sum_{i_1 < \ldots < i_k}
\lambda_{i_1}\cdot \ldots\cdot \lambda_{ik},\\
s_k(X) &= s_k(\lambda_1,\ldots,\lambda_n) = \sum_{i=1}^n \lambda_i^k.\fish
\end{align*}
\end{lem}
\begin{proof}
Nach Annahme ist $X$ diagonalisierbar, also existiert eine invertierbare Matrix
$g\in \GL(n,\C)$ so, dass
\begin{align*}
gXg^{-1} = \diag(\lambda_1,\ldots,\lambda_n).
\end{align*}
Aufgrund der $\Ad$-Invarianz der Determinante folgt somit
\begin{align*}
\det(\Id + tX) &= \det(\Id + tgXg^{-1}) =
\det(\diag(1+t\lambda_1,\ldots,1+t\lambda_n))\\
&= \prod_{i=1}^n (1+t\lambda_i) = 
\sum_{k=1}^n \sigma_k(\lambda_1,\ldots,\lambda_n)t^k.
\end{align*}
Analog erhält man bei Anwendung auf die Spur,
\begin{align*}
\tr(\e^{tX})&= \tr(\e^{tgXg^{-1}}) = 
\tr(\diag(\e^{t\lambda_1},\ldots,\e^{t\lambda_n}))\\
&= \sum_{i=1}^n \e^{t\lambda_i} = 
\sum_{\atop{i=1,\ldots,n}{k\ge 0}} \lambda_i^k \frac{t^k}{k!}
= \sum_{k\ge 0} s_k(\lambda_1,\ldots,\lambda_n)\frac{t^k}{k!}.\qed
\end{align*}
\end{proof}

\begin{rem}[Bemerkungen.]
\begin{remenum}
\item Die Polynome $\setd{\sigma_k}$ bzw. $\setd{s_k}$ bilden eine Basis des
Raumes der invarianten Polynome.
\item Die Newtonformel besagt,
\begin{align*}
s_k - \sigma_1 s_{k-1} \pm \ldots \pm k s_1 = 0,\qquad k\ge 0.
\end{align*}
Somit lassen sich die Polynome $s_k$ als Kombination der Polynome
$\sigma_1,\ldots,\sigma_k$ ausdrücken. Es gilt zum Beispiel
\begin{align*}
s_2 &= \sigma_1^2 - 2\sigma_2\\
s_3 &= \sigma_1^3 - 3\sigma_1\sigma_2 + 3\sigma_3,\\
s_i &= Q_i(\sigma_1,\ldots,\sigma_i),
\end{align*}
wobei $Q_i$ das $i$-te Newton-Polynom bezeichnet.
\item Jedes Element $X\in\gl(n,\C)$ aufgefasst als Abbildung $X\colon \C^n\to \C^n$
induziert für jedes $k\ge 0$ eine Abbildung
\begin{align*}
\Lambda^k X : \Lambda^k \C^n\to \Lambda^k \C^n.
\end{align*}
Dazu betrachtet man zunächst wieder den Fall, dass $X$ diagonalisierbar ist. Sei
$(e_1,\ldots,e_n)$ eine Orthonormalbasis, dann setzt man z.B. für $k=2$
\begin{align*}
\Lambda^kX(e_i\wedge e_j) = (Xe_i) \wedge (Xe_j) = \lambda_i \lambda_j
(e_i\wedge e_j).
\end{align*}
Folglich ist  $\tr(\Lambda^2 X) = \sigma_2(X)$. Da die diagonalisierbaren
Matrizen dicht in $\gl(n,\C)$ liegen, erhält man ganz allgemein
\begin{align*}
\sigma_k(X) = \tr(\Lambda^k X),\qquad 0 \le k \le n.
\end{align*}
\item Nach obiger Konstruktion erhält man die explizite Formel
\begin{align*}
\sigma_k(X) = \frac{1}{k!} \sum_{\sigma} \sign(\sigma) X_{i_1 j_1}\cdot \ldots
\cdot X_{i_k j_k},
\end{align*}
wobei $\sigma$ die Permutationen bezeichnet, die $(i_1,\ldots,i_k)$ auf
$(j_1,\ldots,j_k)$ abbilden.~\map
\end{remenum}
\end{rem}

\section{Chern-Klassen und Chern-Charakter}

\begin{defn}
Sei $E\to M$ ein komplexes Vektorbündel mit kovarianter Ableitung $\nabla$.
\begin{defnenum}
\item 
Die \emph{$k$-te reelle Chern-Klasse} von $E$ ist definiert als
\begin{align*}
c_k(E)\defl \left[\sigma_k\left(\frac{i}{2\pi} F^\nabla \right) \right]
\in H^{2k}_{\dR}(M,\C),
\end{align*}
und die Summe dieser Klassen wird als \emph{totale reelle Chern-Klasse}
bezeichnet
\begin{align*}
c(E) \defl \sum_{k\ge 0} c_k(E) = \left[\det\left(\Id + \frac{i}{2\pi}
F^\nabla\right) \right]\in H^{even}_{\dR}(M,\C).
\end{align*}
\item Der \emph{Chern-Charakter} von $E$ ist definiert als
\begin{align*}
\ch(E) \defl \left[\tr(\e^{\frac{i}{2\pi}F^\nabla})\right],
\end{align*}
und die Koeffizienten der Entwicklung bezeichnet man mit
\begin{align*}
\ch_k(E) = \left[s_k\left(\frac{i}{2\pi} F^\nabla\right) \right].\fish
\end{align*}
\end{defnenum}
\end{defn}

\newcommand{\rank}{\mathrm{rank}}

\begin{ex}
\begin{exenum}
\item Die erste reelle Chern-Klasse ist gegeben durch
\begin{align*}
c_1(E) = \frac{i}{2\pi}[\tr(F^\nabla)] \in H_{\dR}^2(M,\C).
\end{align*}
\item Die zweite reelle Chern-Klasse berechnet sich zu
\begin{align*}
c_2(E) &= \frac{1}{2}\left(\frac{i}{2\pi}\right)^2
\left(\sum_{1\le i,j\le n} F_{ii}\wedge F_{jj} - F_{ij} \wedge F_{ji} \right)\\
&= 
\frac{1}{8\pi^2}\left(\tr(F^\nabla\wedge F^\nabla) - \tr(F^\nabla)^2 \right)
\end{align*}
\item Für den Chern-Charakter evaluiert man mit Hilfe der Newtonformel,
\begin{align*}
\ch(E) = \rank(E) + c_1(E)  + \frac{1}{2}(c_1(E)^2 - 2c_2(E)) + \ldots\boxc
\end{align*} 
\end{exenum}
\end{ex}

\begin{rem}
Für $k> \rank(E)$ verschwinden alle Chern-Klassen, d.h. $c_k(E) = 0$. Der
Chern-Charakter dagegen ist als Potenzreihe gegeben und bricht im Allgemeinen
nicht ab.\map
\end{rem}

\begin{lem}
Seien $f\colon N\to M$ eine differenzierbare Abbildung, sowie $E$, $E_1$ und $E_2$
Vektorbündel über $M$. Dann gelten:
\begin{propenum}
\item Die totale reelle Chern-Klasse ist natürlich und multiplikativ unter Whitneysummenbildung,
\begin{align*}
f^*c(E) = c(f^*E),\qquad\qquad c(E_1\oplus E_2) = c(E_1)\cdot \ch(E_2).
\end{align*}
\item Der Chern-Charakter ist additiv und multiplikativ, d.h.
\begin{align*}
\ch(E_1\oplus E_2 ) = \ch(E_1) + \ch(E_2),\qquad
\ch(E_1\otimes E_2) = \ch(E_1)\cdot\ch(E_2).\fish
\end{align*}
\end{propenum}
\end{lem}
\begin{proof}
a): Die Natürlichkeit folgt direkt aus der allgemeinen Natürlichkeit
charakteristischer Klassen.
Seien $s_1$ und $s_2$ Schnitte in $E_1$ bzw. $E_2$ und sei $X$ ein Vektorfeld auf $M$. Dann gilt für den Levi-Civita-Zusammenhang
\begin{align*}
\nabla_{X}(s_{1}\oplus s_{2}) = (\nabla_{X}s_{1})\oplus (\nabla_{X}s_{2}),
\end{align*}
und folglich hat dessen Krümmung die Gestalt
\begin{align*}
R_{XY}^{E_{1}\oplus E_{2}}(s_{1}\oplus s_{2}) = R_{XY}^{E_{1}}s_{1}\oplus R_{XY}^{E_{2}}s_{2}.
\end{align*}
Fassen wir die Krümmung als Endomorphismus auf und stellen diesen auf $\gl(n,\C)$ in einer lokalen Basis $s_1,\ldots,s_n$ von $E$ als $F^{E_{1}\oplus E_{2}}$ dar, so hat diese Matrix Blockdiagonalgestalt. Ganz allgemein gilt für Matrizen $A$ und $B\in\gl(n,\C)$, dass
\begin{align*}
\det\left(
\Id +
t\begin{pmatrix}
A \\ & B
\end{pmatrix}
\right) 
&=
\det
\begin{pmatrix}
\Id + tA \\ & \Id + tB
\end{pmatrix}
= 
\det(\Id + tA)\det(\Id+tB)
\end{align*}
Folglich gilt
\begin{align*}
c(E_{1}\oplus E_{2}) &= 
\left[\det\left(\Id + \frac{i}{2\pi}F^{E_{1}\oplus E_{2}}\right)\right]
=
\left[\det\left(\Id + \frac{i}{2\pi}
\begin{pmatrix}
F^{E_{1}}\\
& F^{E_{2}}
\end{pmatrix}
\right)\right]\\
&=
\left[\det\left(\Id + \frac{i}{2\pi}
F^{E_{1}}
\right)\right]
\left[\det\left(\Id + \frac{i}{2\pi}
F^{E_{2}}
\right)\right]
= 
c(E_{1})c(E_{2}).
\end{align*}

b): Die Krümmung der Whitneysumme hat Blockdiagonalgestalt, diese bleibt unter dem Matrixexponential erhalten, d.h. es gilt
\begin{align*}
\exp\left(\frac{i}{2\pi} F^{E_{1}\oplus E_{2}}\right)
=
\exp\left(\frac{i}{2\pi} 
\begin{pmatrix}
F^{E_{1}}\\
&F^{E_{2}}
\end{pmatrix}
\right)
= 
\begin{pmatrix}
\exp\left(\frac{i}{2\pi} F^{E_{1}}\right)\\
&
\exp\left(\frac{i}{2\pi} F^{E_{2}}\right)
\end{pmatrix}.
\end{align*}
Somit folgt gerade
\begin{align*}
\ch(E_{1}\oplus E_{2}) &= 
\left[\tr\left(\e^{\frac{i}{2\pi} F^{E_{1}\oplus E_{2}}}\right)\right]
=
\left[\tr\left(\e^{\frac{i}{2\pi} F^{E_{1}}}\right)\right]
+
\left[\tr\left(\e^{\frac{i}{2\pi} F^{E_{2}}}\right)\right]\\
&=
\ch(E_{1})+\ch(E_{2}).
\end{align*}

Als nächstes müssen wir die Krümmung des Tensorproduktes analysieren.
Seien dazu wieder $s_{1}$ und $s_{2}$ Schnitte in $E_{1}$ bzw. $E_{2}$ und $X$ sei ein Vektorfeld auf $M$. Dann gilt
\begin{align*}
\nabla_{X}(s_{1}\otimes s_{2}) = (\nabla_{X}s_{1})\otimes s_{2} + s_{1}\otimes(\nabla_{X} s_{2}).
\end{align*}
Man prüft ohne Umwege nach, dass dann für die Krümmung gilt,
\begin{align*}
R_{XY}^{E_{1}\otimes E_{2}}(s_{1}\otimes s_{2}) = (R_{XY}^{E_{1}}s_{1})\otimes s_{2} + s_{1}\otimes(R_{XY}^{E_{2}}s_{2}) = 
(R_{{XY}}^{E_{1}}\otimes R_{XY}^{E_{2}})(s_{1}\otimes s_{2}).
\end{align*}

Seien nun $A,B\in \gl(n,\C)$ diagonalisierbar und $(e_{1},\ldots,e_{n})$ sowie $(f_{1},\ldots,f_{n})$ Basen in denen die Matrizen Diagonalgestalt annehmen. Dann gilt
\begin{align*}
(A\otimes B)(e_{i}\otimes f_{j}) = (A e_{i})\otimes f_{j} + e_{i}\otimes (B f_{j})
= (\lambda_{i} + \mu_{j}) (e_{i}\otimes f_{j}),
\end{align*}
also ist $A\otimes B$ diagonal bezüglich der Basis $\setd{e_{i}\otimes f_{j}}_{i,j=1}^n$ mit Eigenwerten $(\lambda_{i}+\mu_{j})$.   Somit gilt für die Funktionen $s_{k}$, dass
\begin{align*}
s_{k}(A\otimes B) = \sum_{i,j=1}^n (\lambda_{i}+\mu_{j})^k
=
\sum_{i,j=1}^n\sum_{r=1}^k \binom{k}{r} \lambda_{i}^r\mu_{j}^{r-k}
=
\sum_{i,j=1}^n\sum_{r=1}^k \frac{k!}{(k-r)!r!} \lambda_{i}^r\mu_{j}^{r-k}.
\end{align*}
Damit erhalten wir nach Anwendung des Cauchy-Produktsatzes,
\begin{align*}
\sum_{k\ge 0} s_{k}(A\otimes B) \frac{t^k}{k!} &= 
\sum_{i,j=1}^n \sum_{k\ge 0} \sum_{r=1}^k
\frac{k!}{k!}  \frac{\lambda_{i}^rt^r}{r!}\frac{\mu_{j}^{r-k}t^{r-k}}{(r-k)!} = 
\sum_{i,j=1}^n \left(\sum_{r\ge 0} \frac{\lambda_{i}^rt^r}{r!}\right)
 \left(\sum_{k\ge 0} \frac{\mu_{j}^kt^k}{k!}\right)\\
&=
\left(\sum_{r\ge 0} \sum_{i=1}^n\frac{\lambda_{i}^rt^r}{r!}\right)
 \left(\sum_{k\ge 0} \sum_{j=1}^n\frac{\mu_{j}^kt^k}{k!}\right)\\
&= 
\left(\sum_{r\ge 0} s_{r}(A) \frac{t^r}{r!}\right)
 \left(\sum_{k\ge 0} s_{k}(B)\frac{t^k}{k!}\right).
\end{align*}
Da die diagonalisierbaren Matrizen dicht in $\gl(n,\C)$ liegen und obiger Ausdruck stetig in $A$ und $B$ ist, gilt diese Beziehung für allgemeine Matrizen. Somit erhalten wir schließlich,
\begin{align*}
\ch(E_{1}\otimes E_{2}) &= \sum_{k\ge0} s_{k}(F^{E_{1}}\otimes F^{E_{2}}) \frac{t^k}{k!}
=
\left(\sum_{r\ge 0} s_{r}(F^{E_{1}}) \frac{t^r}{r!}\right)\tag{*}
 \left(\sum_{k\ge 0} s_{k}(F^{E_{2}})\frac{t^k}{k!}\right)\\
& = \ch(E_{1})\cdot \ch(E_{2}).\qed
\end{align*}
\end{proof}


\begin{lem}
Sei $E\to M$ ein komplexes Vektorbündel. Dann gelten:
\begin{propenum}
\item $c_{k}(E^*) = c_{k}(\bar{E}) = (-1)^k c_{k}(E)$.
\item Die $k$-te reelle Chern-Klasse ist eine reelle Kohomologieklasse, d.h.
\begin{align*}
c_{k}(E) \in H_{\dR}^{2k}(M,\R).
\end{align*}
\item Existieren $k$ überall linear unabhängige Schnitte von $E$, dann gilt
\begin{align*}
c_{j}(E) = 0,\quad \text{für } j > \rank(E)-k.\fish
\end{align*}
\end{propenum}
\end{lem}

\begin{proof}
a): Sei $\lambda$ ein Schnitt in $E^*$, sowie $e$ ein Schnitt in $E$ und $X$ ein Vektorfeld auf $M$, dann gilt
\begin{align*}
(\nabla_{X}\lambda)e = X(\lambda( e)) - \lambda(\nabla_{X}e).
\end{align*}
Wählen wir eine lokale Basis $(s_{1},\ldots,s_{n})$ von $E$ über $U\subset M$ und bezeichnen die dazu duale Basis mit $(s_{1}^*,\ldots,s_{n}^*)$, so verifiziert man, dass
\begin{align*}
F^\nabla = -F^{\nabla^*}.
\end{align*}
Somit gilt für die elementarsymmetrischen Funktionen
\begin{align*}
\sigma_{k}\left(\frac{i}{2\pi}F^{\nabla^*}\right)
= 
\sigma_{k}\left(\frac{-i}{2\pi}F^{\nabla}\right)
= (-1)^k
\sigma_{k}\left(\frac{i}{2\pi}F^{\nabla}\right),
\end{align*}
und da $E^*\cong \bar{E}$ folgt die Behauptung.

b): Nach Voraussetzung ist $E$ ein komplexes Vektorbündel, daher existiert auf $E$ ein hermitisches Skalarprodukt. Wählen wir einen metrischen Zusammenhang, dann gilt bezüglich einer lokalen Orthonormalbasis $(s_{1},\ldots,s_{n})$ von $E$ über $U\subset M$, dass
\begin{align*}
F_{XY}^\nabla \in \u(n) = \setdef{A\in\gl(n,\C)}{A^* = -A}.
\end{align*}
Die elementarsymmetrischen Funktionen hängen nur von den Eigenwerten ab, daher gilt
\begin{align*}
\bar{\sigma_{k}\left(\frac{i}{2\pi}F\right)} = 
\sigma_{k}\left(\frac{-i}{2\pi}\bar{F}\right)
= 
\sigma_{k}\left(\frac{i}{2\pi}F^\top\right) = 
\sigma_{k}\left(\frac{i}{2\pi}F^\top\right),
\end{align*}
und folglich ist $c_{k}(E)$ reell. Analog dazu rechnet man
\begin{align*}
\bar{\det\left(\Id + \frac{i}{2\pi}F\right)}
&= 
\det\left(\Id - \frac{i}{2\pi}\bar{F}\right)
=
\det\left(\Id + \frac{i}{2\pi}F^\top\right)
= 
\det\left(\Id + \frac{i}{2\pi}F\right),
\end{align*}
also ist $c(E)$ reell.

c): Sei $E= \Es_{k}+F$, wobei $\Es_k$ das triviale Bündel vom Rang $k$ bezeichnet. Dann gilt für die totale Chern-Klasse
\begin{align*}
c(E) = c(\Es_{k})\cdot c(F) = c(F).
\end{align*}
Der Rang von $F$ ist $n-k$ und folglich verschwinden alle Chern-Klassen $c_{i}(F)=c_{i}(E)$ für $i>n-k$.\qed
\end{proof}

\begin{rem}
Die Chern-Klassen sind nicht nur reelle Kohomologieklassen, sondern sogar ganzzahlig. Bezeichnen wir den Homomorphimsmus, der die ganzzahligen Homologieklassen in die de-Rham Kohomologie einbettet, mit
\begin{align*}
\Hs_{\Z}^k : H^{2k}(M,\Z) \to H_{\dR}^{2k}(M,\R),
\end{align*}
so gilt, dass
\begin{align*}
c_{k}(E) \in \im \Hs_{\Z}^k.
\end{align*}
Integrieren wir also $c_{k}(E)$ über einen glatten $2k$-Zykel $N$, so gilt
\begin{align*}
\int_{N} c_{k}(E) \in \Z.\map
\end{align*}
\end{rem}

\begin{lem}
Seien $E_{1}$ und $E_{2}$ Vektorbündel über $M$. Dann gilt
\begin{align*}
c_{1}(E_{1}\oplus E_{2}) = \rank(E_{1})c_{1}(E_{2}) + \rank(E_{2})c_{1}(E_{1}).\fish
\end{align*}
\end{lem}
\begin{proof}
Nach Definition gilt für den Chern-Charakter,
\begin{align*}
\ch(E_{1}\otimes E_{2}) = \rank(E_{1}\otimes E_{2}) + c_{1}(E_{1}\otimes E_{2}) + O(4),
\end{align*}
wobei $O(4)$ mindestens Terme der 4. Kohomologieklasse $H_{\dR}^4(M,\C)$ beschreibt. Andererseits ist der Chern-Charakter multiplikativ, d.h.
\begin{align*}
\ch(E_{1}\otimes E_{2}) &= \ch(E_{1})\cdot \ch(E_{2})\\
&= 
\left( \rank(E_{1}) + c_{1}(E_{1}) + O(4)\right)
\left( \rank(E_{2}) + c_{1}(E_{2}) + O(4)\right)\\
&= \rank(E_{1})\rank(E_{2}) + 
\rank(E_{1})c_{1}(E_{2}) + \rank(E_{2})c_{1}(E_{1}) + O(4).
\end{align*}
Vergleichen wir nun die Ordnungen, ergibt sich die Behauptung.\qed
\end{proof}

Wir haben die Chern-Klassen bisher mit elementarsymmetrischen Funktionen beschrieben. Es gibt allerdings auch eine geometrische Interpretation. Sei ein Vektorbündel
\begin{align*}
E = E_{1}\oplus \ldots \oplus E_{n}
\end{align*}
gegeben, wobei die $E_{i}$ Linienbündel vom Rang 1 sind. Für diese Linienbündel sind dann nur die nullte und die erste Chern-Klasse von Null verschieden. Schreiben wir $c_{1}(E_{i}) = x_{i}$, dann lässt sich die totale reelle Chern-Klasse von $E$ wie folgt berechnen
\begin{align*}
c(E) &= \prod_{i=1}^n c(E_{i}) = 
\prod_{i=1}^n (1+x_{i})\\
&= 1 + \sigma_{1}(x_{1},\ldots,x_{n}) + \ldots + \sigma_{n}(x_{1},\ldots,x_{n})
= 1 + c_{1}(E) + \ldots
\end{align*}
Es stellt sich nun heraus, dass man immer so verfahren kann.

\begin{prop}[Zerf\"allungsprinzip]
Zu jedem komplexen Vektorbündel $E\to M$ existiert eine Mannigfaltigkeit $X$ und eine Abbildung $f\colon X\to M$, so dass
\begin{align*}
f^*E = E_{1}\oplus \ldots \oplus E_{n},
\end{align*}
wobei die $E_{i}$ Linienbündel vom Rang 1 bezeichnen.\fish
\end{prop}

Aus der Natürlichkeit der Chern-Klassen folgt, dass man f\"ur Rechnungen  immer annehmen kann, dass sich
$E$ als Summe von Linienb\"undeln schreibt.

\section{Charakteristische Klassen reeller Vektorbündel}

Sei nun $E\to M$ ein \textit{reelles} Vektorbündel, ausgestattet mit einem euklidischen Skalarprodukt und einem metrischen Zusammenhang $\nabla$. Wählen wir eine lokale Orthonormalbasis von $E$ über $U\subset M$, dann ist
\begin{align*}
F_{XY}^\nabla \in \so(n) = \setdef{A\in\gl(n,\R)}{A^\top = - A}.
\end{align*}
Komplexifizieren wir $E$ zu $E^\C$, indem wir faserweise komplexifizieren, d.h.
\begin{align*}
E_{x}^\C \defl E_{x}\otimes \C,\qquad x\in M, 
\end{align*}
dann können wir für $E^\C$ Chern-Klassen definieren.

\begin{lem}
Sei $E$ ein reelles Vektorbündel, dann verschwinden die ungeraden Chern-Klassen der Komplexifizierung, d.h.
\begin{align*}
c_{{2k+1}}(E^\C) = 0,\qquad k = 0,1,\ldots\fish
\end{align*}
\end{lem}
\begin{proof}
Als reelles Bündel ist $E$ isomorph zu seinem dualen Bündel $E^*$, dann gilt auch für die Komplexifizierung
\begin{align*}
E^\C \cong (E^*)^\C = (E^\C)^*.
\end{align*}
Nach den Rechenregeln für Chern-Klassen ist daher
\begin{align*}
c_{k}((E^\C)^*) = (-1)^kc_{k}(E^\C),\qquad k=0,1,\ldots,
\end{align*}
und folglich verschwinden alle ungeraden Klassen.\qed
\end{proof}

Durch Komplexifizierung lassen sich also einem reellen Vektorbündel in kanonischer Weise Chern-Klassen zuordnen, wobei nur die geraden Chern-Klassen nichttrivial sind. Diese haben einen speziellen Namen.

\begin{defn}
\index{Pontrjagin-Klasse}
Sei $E\to M$ ein reelles Vektorbündel, dann heißt
\begin{align*}
p_{k}(E) &\defl (-1)^k c_{2k}(E^\C) \in H_{\dR}^{4k}(M,\R).
\end{align*}
die $k$-te \emph{Pontrjagin-Klasse} von $E$. Die \emph{totale Pontrjagin-Klasse} ist definiert als
\begin{align*}
p(E) = \sum_{k\ge 0} p_{k}(E).\fish
\end{align*}
\end{defn}

Die Pontrjagin-Klassen haben analoge Eigenschaften zur den Chern-Klassen.

\begin{prop}
\begin{propenum}
\item Sind $E_{1}$ und $E_{2}$ isomorphe reelle Vektorbündel, so gilt
\begin{align*}
p(E_{1}) = p(E_{2}).
\end{align*}
\item Ist $\phi\colon N\to M$ eine glatte Abbildung und $E$ ein reelles Vektorbündel über $M$, dann gilt
\begin{align*}
p(\phi^* E) = \phi^* p(E).
\end{align*}
\item Sind $E_{1}$ und $E_{2}$ zwei Vektorbündel über $M$, so gilt
\begin{align*}
p(E_{1}\oplus E_{2}) = p(E_{1})\cdot p(E_{2}).
\end{align*}
\end{propenum}
\end{prop}

\begin{rem}
Im Fall eines reellen Vektorbündels ist $G=\SO(n)$ und $\g=\so(n)$, d.h. die Krümmung $F_{XY}\in \so(n)$ ist schiefsymmetrisch. Schiefsymmetrische Matrizen haben je zwei komplex konjugierte und rein imaginäre Eigenwerte. Ihre Normalform ist
\begin{align*}
&A=
\diag\left(
\begin{pmatrix}
0 & \mu_{1}\\
- \mu_{1} & 0
\end{pmatrix}
,\ldots,
\begin{pmatrix}
0 & \mu_{n}\\
- \mu_{n} & 0
\end{pmatrix}
: \mu_{i}\in\R
\right),&& \text{für }\so_{2n},\\
&A=\diag\left(
\begin{pmatrix}
0 & \mu_{1}\\
- \mu_{1} & 0
\end{pmatrix}
,\ldots,
\begin{pmatrix}
0 & \mu_{n}\\
- \mu_{n} & 0
\end{pmatrix},(0) : \mu_{i}\in \R\right),
&& \text{für }A\in\so_{2n+1}.
\end{align*}
\end{rem}
Diese Blockmatrizen bilden einen maximalen Torus $t$ und die zugehörige Weyl-Gruppe ist
\begin{align*}
W = S_{n}\times\Z_{2},
\end{align*}
denn die Normalform ist $\SO(n)$-invariant unter Vertauschung von Blöcken und unter Transponieren einer geraden Anzahl von 2er-Blöcken.


Für ungerade Dimension ist daher
\begin{align*}
\Sym(\so(2n+1))^{\SO(2n+1)} = 
\setdef{\sigma_{k}(\mu_{1}^2,\ldots,\mu_{n}^2)}{k=0,1,\ldots}.
\end{align*}
In gerader Dimension gibt es noch ein zusätzliches invariantes Polynom, so dass
\begin{align*}
\Sym(\so(2n))^{\SO(2n)} = 
\setdef{\sigma_{k}(\mu_{1}^2,\ldots,\mu_{n}^2}{k=0,1,\ldots}\cup
\setd{\Pf}.
\end{align*}
Dabei ist die  \emph{Pfaffsche} $\Pf$ gegeben durch
\begin{align*}
\Pf\left(\diag
\begin{pmatrix}
0 & \mu_{i}\\
-\mu_{i} & 0
\end{pmatrix}
\right) = \prod_{i=1}^n \mu_{i}.
\end{align*}
In ungerader Dimension tritt immer ein Eigenwert Null auf und daher ist die Pfaffsche stets identisch Null und liefert somit nichts neues. In gerader Dimension verschwindet die Pfaffsche im Allgemeinen nicht identisch und ist invariant unter der Weyl-Gruppe.

Eine reellwertige, lineare und schiefsymmetrische Abbildung $A\in\so(2n)$ hat nach obigen Überlegungen je zwei komplex konjugierte und rein imaginäre Eigenwerte. Die Normalform von $A$ ist
\begin{align*}
\diag
A=\begin{pmatrix}
0 & \mu_{i}\\
-\mu_{i} & 0
\end{pmatrix},
\end{align*}
und daher ist die Determinante nichtnegativ. Man kann die Pfaffsche daher auch definieren als
\begin{align*}
\Pf \, A \defl \sqrt{\det A}.
\end{align*}

Identifiziert man $\so(2n)\cong \Lambda^2\R^{2n}$, dann ist für $A\in\so(2n)$,
\begin{align*}
\underbrace{A\wedge \ldots \wedge A}_{n\text{-mal}} \in \Lambda^{2n}\R^{2n} \cong \R.
\end{align*}
Man kann die Pfaffsche dann auch definieren als diejenige Form für die gilt
\begin{align*}
\Pf(A)\,\vol = A\wedge \ldots \wedge A,
\end{align*}
wobei $\vol = e_{1}\wedge \ldots\wedge e_{n}$ für eine Orthonormalbasis $e_{1},\ldots,e_{n}$. Dann schreibt sich die Pfaffsche auch als
\begin{align*}
\Pf(A) = \frac{1}{n!} \lin{A^n,e_{1}\wedge \ldots \wedge e_{n}},
\end{align*}
bzw.
\begin{align*}
\Pf(A) = \frac{1}{n!}\sum_{\sigma\in S_{2n}} \sign(\sigma) A_{\sigma_{1}\sigma_{2}}\cdot A_{\sigma_{3}\sigma_{4}}\cdot \ldots \cdot A_{\sigma_{2n-1}}A_{2n}.\map
\end{align*}

Da die Pfaffsche ein invariantes Polynom beschreibt, lässt sich auch mit ihr eine charakteristische Klasse definieren.

\begin{defn}
\index{Euler-Klasse}
Sei $E\to M$ ein reelles Vektorbündel vom Rang $2n$. Die \emph{Euler-Klasse} von $E$ ist definiert als
\begin{align*}
e(E) \defl \left[\Pf\left(\frac{-1}{2\pi}F\right)\right] \in H_{\dR}^{2n}(M).\fish 
\end{align*}
\end{defn}

\begin{rem}
Wählt man eine lokale Orthonormalbasis von $E$ über $U\subset M$, dann lässt sich die Euler-Klasse mit Hilfe der lokalen Darstellung der Pfaffschen beschreiben als
\begin{align*}
e(E) = \frac{1}{n!}\frac{(-1)^n}{(2\pi)^n}\sum_{\sigma\in S_{2n}} \sign(\sigma) F_{\sigma_{1}\sigma_{2}}\wedge \ldots
\wedge F_{\sigma_{2n-1}\sigma_{2n}}.\map
\end{align*}
\end{rem}

\begin{lem}
Sei $E\to M$ ein reelles Vektorbündel vom Rang $2n$.
\begin{propenum}
\item Es gilt $e(E)^2 = p_{n}(E)$.
\item Sei $E$ außerdem ein komplexes Vektorbündel vom Rang $n$, dann gilt
\begin{align*}
e(E) = c_{n}(E).\fish
\end{align*}
\end{propenum}
\end{lem}
\begin{proof}
a): Das Quadrat der Pfaffschen ist die Determinante, also gilt
\begin{align*}
\Pf\left(\frac{-1}{2\pi}F\right)^2 = 
\det\left(\frac{-1}{2\pi}F\right) = \sigma_{2n}\left(\frac{-1}{2\pi}F\right).
\end{align*}
Andererseits gilt in lokalen Koordinaten,
\begin{align*}
p_{n}(E) = (-1)^n c_{2n}(E) = 
(-1)^n \sigma_{2n}\left( \frac{i}{2\pi}F\right)
= \sigma_{2n}\left( \frac{-1}{2\pi}F\right).
\end{align*}

b): Man betrachtet die Einbettung von $\u(n)$ in $\so(2n)$ gegeben durch
\begin{align*}
\phi\colon \diag(i\mu_{1},\ldots,i\mu_{n}) \mapsto
\diag
\left(
\begin{pmatrix}
& \mu_{1}\\
-\mu_{1}
\end{pmatrix},
\ldots,
\begin{pmatrix}
& \mu_{n}\\
-\mu_{n}
\end{pmatrix}
\right)
\end{align*}
Mit dieser Abbildung können wir den Zusammenhang auf dem reellen Bündel $E$ auf das komplexe Bündel zurückziehen und sehen so die Gleichheit ein.\qed
\end{proof}

\begin{rem}
Sind $M_{1}$ und $M_{2}$ Mannigfaltigkeiten so, dass alle Pontrjagin-Klassen der Tangentialbündel übereinstimmen, dann existiert eine Mannigfaltigkeit $W$ mit $\partial W = M_{1}\cup M_{2}$.\map
\end{rem}
%TODO: Hosenbild.

\section{Multiplikative Folgen}

Sei $f\in \C[x]$ eine formale Potenzreihe in $x$. Dann lässt sich $f$ ein ``Polynom'' $q_{f} \in \C[x_{1},\ldots,x_{n}]$ zuordnen,
\begin{align*}
q_{f}(x_{1},\ldots,x_{n}) = f(x_{1})\cdot \ldots \cdot f(x_{n}).
\end{align*}
Offenbar ist $q_{f}$ symmetrisch und daher lässt sich jeder Grad durch elementarsymmetrische Funktionen ausdrücken, d.h.
\begin{align*}
q_{f} = \sum_{k\ge 0} F_{k}(\sigma_{1},\ldots,\sigma_{k}).
\end{align*}
Die elementarsymmetrischen Funktionen $\sigma_{i}$ entsprechen dann der $i$-ten Chern-Klasse.

\begin{ex}
\begin{exenum}
\item
Sei $f(x) = 1+x$. Dann ist $q_{f}$ tatsächlich ein Polynom gegeben durch
\begin{align*}
q_{f}(x_{1},\ldots,x_{n}) = \prod_{i=1}^n (1+x_{i}) = 1+\sigma_{1}+\sigma_{2}+ \ldots + \sigma_n
\end{align*}
Somit entspricht $q_{f}$ gerade der totalen Chern-Klasse von $E$.
\item Der Chern-Charakter von $E$ wird von der Funktion
\begin{align*}
f(x) = \e^x
\end{align*}
und dem zugehörigen symmetrischen ``Polynom'' $q_{f}$ induziert.
\item Sei $E\to M$ ein reelles Vektorbündel. Betrachten wir
\begin{align*}
f(x) = \frac{\sqrt{x}/2}{\sinh(\sqrt{x}/2)} = 1 - \frac{x}{24} + \frac{7x^2}{5760} \pm \ldots,
\end{align*}
dann entspricht das $f$ zugeordnete Polynom $q_{f}$ der \emph{$\hat{A}$-Klasse von $E$},
\begin{align*}
\hat{A}(E) = \prod_{i=1}^n \frac{x_{i}/2}{\sinh(x_{i}/2)}
=
 1 - \frac{1}{24}p_{1}(E) + \frac{1}{5760}\left(7p_{1}(E)^2 - 4p_{2}(E)\right) \pm \ldots
\end{align*}
Für eine kompakte, $4n$ dimensionale Mannigfaltigkeit nennt man
\begin{align*}
\hat{A}(M) \defl \int_{M} \hat{A}(TM)
\end{align*}
das \emph{$\hat{A}$-Geschlecht} von $M$.
\item Sei $E\to M$ ein reelles Vektorbündel und
\begin{align*}
f(x) = \frac{\sqrt{x}}{\tanh(\sqrt{x})} = 1 + \frac{x}{3} - \frac{x^2}{45} \pm \ldots
\end{align*}
Das ``Polynom'' $q_{f}$ entspricht der \emph{Hirzebruchschen $L$-Klasse}
\begin{align*}
L &= \prod_{i=1}^n \frac{x_{i}}{\tanh(x_{i})}\\
&= 1 + \frac{1}{3}p_{1}(E) - \frac{1}{45}\left(7p_{2}(E) - p_{1}(E)^2\right) \pm \ldots
\end{align*}
Analog ist für eine kompakte, $4n$ dimensionale Mannigfaltigkeit das \emph{Hirzebruchsche $L$-Geschlecht} definiert als
\begin{align*}
L(M) \defl \int_{M} L(TM).\boxc
\end{align*}
\end{exenum}
\end{ex}

\section{Indexsätze}

Sei $M$ eine kompakte Mannigfaltigkeit. Auf dem Raum der Differentialformen $\Omega^*(M)$ betrachten wir den Differentialoperator
\begin{align*}
D = d + d^*.
\end{align*}
Das Quadrat von $D$ entspricht dem gewöhnlichen Laplace-Operator auf Formen
\begin{align*}
D^2 = dd^*+d^*d = \D.
\end{align*}
Bezeichnen wir die Einschränkung auf gerade Formen mit $D_{+}$ und die auf ungerade Formen mit $D_{-}$, also
\begin{align*}
D_{+} \colon \Gamma(\Lambda^{even}TM)\to \Gamma(\Lambda^{odd}TM),\\
D_{-} \colon \Gamma(\Lambda^{odd}TM)\to \Gamma(\Lambda^{even}TM),
\end{align*}
dann ist der \emph{Index} des Operators $D$ definiert als
\begin{align*}
\ind(D)\defl \dim \ker D_{+} - \dim \ker D_{-}.
\end{align*}
Der Index von $D$ stimmt mit dem Index von $\D=D^2$ überein, d.h.
\begin{align*}
\ind(D)
& = \dim \ker \D_{+} - \dim \ker \D_{-}\\ & =
\beta_{0} + \beta_{2} + \ldots - \beta_{1}-\beta_{3}-\ldots\\
&=\chi(M),
\end{align*}
wobei $\beta_{k}(M) = \dim H^k_{dR}(M) = \dim \ker \D\big|_{\Lambda^k TM}$ die $k$-te Bettizahl und $\chi(M)$ die Euler-Charakteristik bezeichnet.
Nach dem Atiyah–Singer Indexsatz lässt sich der Index von $D$ als Integral über die Euler-Klasse ausdrücken, d.h. es gilt
\begin{align*}
\chi(M) = \ind(D) = \int_{M} e(TM).
\end{align*}

Ein weiterer interessanter Differentialoperator ist der Dirac-Operator
\begin{align*}
D: \Gamma(S)\to \Gamma(S),
\end{align*}
der auf Schnitten von sogenannten Spinor-Bündeln wirkt. In diesem Fall besagt der Atiyah-Singer Indexsatz, dass der Index des Dirac-Operators dem $\hat{A}$-Geschlecht der Mannigfaltigkeit entspricht
\begin{align*}
\ind(D) = \int_{M} \hat{A}(TM).
\end{align*}
Dies ist ein erstaunliches Resultat, denn der Index dieses Differentialoperators ist vollständig durch das  $\hat{A}$-Geschlecht, also die Topologie der Mannigfaltigkeit bestimmt. Als erste Anwendung erhalten wir eine Obstruktion gegen die Existenz von Metriken mit positiver Skalarkrümmung.

\begin{prop}[Schrödinger-Lichnerowicz-Weitzenböck Formel]
Sei $M$ eine orientierte Spin Mannigfaltigkeit mit Dirac-Operator $D$. Dann gilt
\begin{align*}
D^2 = \nabla^* \nabla + \frac{\mathrm{scal}}{4}\Id.\fish
\end{align*}
\end{prop}

Ist daher $M$ eine kompakte, orientierte Spin Mannigfaltigkeit und trägt positive Skalarkrümmung, dann ist $D^2$ ein strikt positiver Operator und hat trivialen Kern. Somit ist $\ind(D) = 0$ und folglich verschwindet auch das $\hat{A}$-Geschlecht
\begin{align*}
\int_{M} \hat{A}(TM) = 0.
\end{align*}
Im Umkehrschluss impliziert ein nichtverschwindendes $\hat{A}$-Geschlecht, dass $M$ keine Metrik positiver Skalarkrümmung trägt.

\begin{prop}
Existiert auf $S^n$ eine fastkomplexe Struktur, dann gilt $n=1$ oder $n=3$.\fish
\end{prop}
\begin{proof}
Sei $E\to S^{2n}$ ein komplexes Vektorbündel. Man kann in diesem Fall beweisen, dass der Chern-Charakter von $E$ stets ganzzahlig ist. Da bis auf die Extrempunkte $H_{\dR}^0(S^{2n})$ und $H_{\dR}^{2n}(S^{2n})$ alle Kohomologien von $S^{2n}$ verschwinden, ist $c_{n}(E)$ die einzige nichttriviale Chern-Klasse. Somit gilt für den Chern-Charakter
\begin{align*}
\ch(E) = \frac{c_{n}(E)}{(n-1)!}\in H^{2n}_{\dR}(M,\Z),
\end{align*}
d.h. $(n-1)!$ teilt den ganzzahligen Chern-Charakter.

Nehmen wir also an, dass $E=TS^{2n}$ ein komplexes Vektorbündel ist, dann gilt
\begin{align*}
(n-1)! \quad \bigg|\quad \int_{M} c_n(TS^{2n}) = \chi(S^{2n}) = 2,
\end{align*}
also wird $2$ von $(n-1)!$  geteilt, das heißt $n=1,2$ oder $3$.

Wir müssen also noch den Fall der $S^4$ ausschließen. Dazu betrachten wir ganz allgemein
die Signatur einer Mannigfaltigkeit $M$
\begin{align*}
\sigma\colon H^2(M)\times H^2(M)\to \R,\qquad ([\alpha],[\beta]) \mapsto \int \alpha\wedge \beta.
\end{align*}
Diese ist eine nichtausgeartete, symmetrische Bilinearform und es gilt
\begin{align*}
\sigma(M) = \int_{M} L(TM) = \frac{1}{3}\int_{M} p_{1}(TM) = 
\frac{1}{3}\int_{M} (c_{1}(TM)^2 - 2c_{2}(TM)). 
\end{align*}
Auf $E=TS^4$ verschwindet jedoch die erste Chern-Klasse von $E$, denn $c_{1}(E) \in H_{\dR}^2(S^4)$, und die zweite de-Rham Kohomologie ist trivial. Daher verschwindet auch die Signatur identisch. Nach obiger Rechnung gilt aber
\begin{align*}
0 = \sigma(M) =  \frac{-2}{3}\int_{M} c_{2}(TM),
\end{align*}
ein Widerspruch, denn $\int_Mc_{2}(TM)= \chi(S^4)=2$. Somit kann $E = TS^4$ kein komplexes Vektorbündel sein.\qed
\end{proof}


\chapter{Differentialoperatoren}

\section{Elemente der linearen Algebra}

Ausgangspunkt ist für uns ein $n$-dimensionaler, euklidischer Vektorraum $(V,g)$, d.h. ein reeller Vektorraum $V$ mit positiv definitem Skalarprodukt $g$. Dieses induziert ein Skalarprodukt auf dem Raum der $k$-Formen $\Lambda^kV$ durch die Festlegung, dass
\begin{align*}
\setdef{e_{i_{1}}^*\wedge \ldots \wedge e_{i_{k}}^*}{1\le i_{1}< \ldots < i_{k} \le n}
\end{align*}
eine Orthonormalbasis von $\Lambda^k V$ ist, wann immer $\setd{e_{i}}$ eine Orthonormalbasis von $V$ ist. Dabei ist die zu einem Vektor $X\in V$ duale Linearform definiert als
\begin{align*}
X^* \defl g(X,\cdot) \in\Lambda^1 V^* = V^*.
\end{align*}

\begin{ex}
Seien $X,Y$ und $A,B$ Vektoren in $V$, dann gilt
\begin{align*}
g(X^*\wedge Y^*, A^*\wedge B^*) = g(X,A)g(Y,B) - g(X,B)g(Y,A).\boxc
\end{align*}
\end{ex}

\begin{defn}
\index{Hodge-$*$-Operator}
Der \emph{Hodge-$*$-Operator} ist der eindeutig bestimmte lineare Isomorphismus
\begin{align*}
* : \Lambda^k V^*\to \Lambda^{n-k}V^*,
\end{align*}
der implizit bestimmt ist durch
\begin{align*}
\alpha\wedge * \beta = g(\alpha,\beta)\vol,
\end{align*}
wobei $\vol = e_{1}^*\wedge \ldots \wedge e_{n}^*$ das Standard-Volumenelement für eine Orthonormalbasis $\setd{e_{i}}$ von $V$ bezeichnet.\fish
\end{defn}

\begin{rem}
Für explizite Berechnungen ist es oft geschickt, sich auf Elemente einer Orthogonalbasis zurückzuziehen. Hier gilt
\begin{align*}
*(e_{i_{1}}^*\wedge \ldots \wedge e_{i_{k}}^*) = \sgn(\sigma)\, e_{j_{1}}^*\wedge \ldots \wedge e_{j_{n-k}}^*,
\end{align*}
wobei $I=(i_{1},\ldots,i_{k})$ und $J=(j_{1},\ldots,j_{n-k})$ bezeichnet und $\sigma$ die zu $(i_{1},\ldots,i_{k},j_{1},\ldots,j_{n-k})$ gehörige Permutation.\map
\end{rem}


\begin{lem}[Eigenschaften von $*$]
Seien $\alpha\in\Lambda^k V^*$ und $\beta\in \Lambda^{n-k}V^*$ sowie $X\in V$. Dann gelten
\begin{propenum}
\item $*1 = \vol$ und $*\vol = 1$,
\item $g(\alpha,*\beta) = (-1)^{k(n-k)}g(*\alpha,\beta)$,
\item $*^2 = (-1)^{k(n-k)}\Id_{\Lambda^kV^*}$,
\item $*(X\ic \alpha) = (-1)^{k-1}X^*\wedge *\alpha$,
\item $*(X^*) = X\ic \vol$, und
\item $g(X\ic \alpha,\eta) = g(\alpha, X^*\wedge \eta)$, falls $\eta\in \Lambda^{k-1}V^*$.\fish
\end{propenum}
\end{lem}

\section{Differentialoperatoren auf Riemannschen Mannigfaltigkeiten}

Sei $(M,g)$ eine $n$-dimensionale Riemannsche Mannigfaltigkeit. Der Hodge-$*$-Operator setzt sich fort zu einem Vektorbündelisomorphismus
\begin{align*}
* : \Lambda^k T^*M\to \Lambda^{n-k}T^*M.
\end{align*}
Wir bezeichnen mit $\nabla$ wieder den Levi-Civita Zusammenhang auf $M$.

\begin{lem}
Sei $\alpha$ eine Differentialform und $X$ ein Vektorfeld auf $M$, dann gilt
\begin{align*}
\nabla_{X}*\alpha = * \nabla_{X}\alpha.\fish
\end{align*}
\end{lem}

\begin{defn}
\index{Gradient}
Sei $f\in\Cs^\infty(M)$, dann ist der \emph{Gradient} von $f$ das eindeutig bestimmte Vektorfeld $\grad(f)$ auf $M$ mit
\begin{align*}
g(\grad(f),X) = \df(X) = X(f),\qquad X\in\chi(M),
\end{align*}
d.h. es gilt $\df = (\grad f)^*$.\fish
\end{defn}

\begin{rem}[Vereinbarung.]
Sofern nicht anders angegeben bezeichne für alles Weitere $\setd{e_{i}}$ stets eine Orthonormalbasis von $T_{m}M$ für $m\in M$. Alle Ausdrücke in denen $e_{i}$ auftritt sind dann als punktweise ausgewertet in $m$ zu verstehen.\map
\end{rem}


\begin{defn}
\index{Divergenz}
Die \emph{Divergenz} eines Vektorfeldes $X$ auf $M$ ist definiert als
\begin{align*}
\div(X) = \sum_{i=1}^n g(\nabla_{e_{i}} X,e_{i}) = \tr(\nabla X).\fish
\end{align*}
\end{defn}

Man kann den Gradienten als Differentialoperator erster Ordnung auffassen, nämlich als das gewöhnliche Differential. Wir werden sehen, dass auch die Divergenz als Differentialoperator erster Ordnung aufgefasst werden kann, mit Hilfe des Kodifferentials. Es stellt sich sogar heraus, dass Gradient und Divergenz in einem gewissen Sinne \textit{dual} zu einander sind. Wie für das Differential gilt auch für die Divergenz die Leibniz-Regel, die wiederum den Gradienten involviert.

\begin{lem}
Sei $X$ ein Vektorfeld auf $M$ und $f$ eine glatte Funktion, dann gilt
\begin{align*}
\div (fX ) &= f \div(X) + \df(X).\fish
\end{align*}
\end{lem}
\begin{proof}
In einer lokalen Basis schreibt sich
\begin{align*}
\div(fX) &= \sum_{i=1}^n g(\nabla_{e_{i}}(fX),e_{i})
=
\sum_{i=1}^n \left(g(f\nabla_{e_{i}}X,e_{i})
+ g(\df(e_{i})X,e_{i})\right)\\
&= f\div(X) + \df(X).\qed
\end{align*}
\end{proof}

Auf jeder orientierten Riemannschen Mannigfaltigkeit existiert eine parallele Volumenform. Diese schreibt sich lokal als
\begin{align*}
\vol = e_{1}\wedge \ldots \wedge e_{n} \ .
\end{align*}
Die Lie-Ableitung erhält den Grad von Differentialformen, daher ist die Lie-Ableitung der Standard-Volumenform wieder eine $n$-Form. Diese beiden Volumenformen lassen sich durch die Divergenz ineinander überführen.

\begin{prop}
Sei $X$ ein Vektorfeld auf $M$, dann gilt
\begin{align*}
L_{X}\vol = \ddd(*X^*) = \div(X)\,\vol.\fish
\end{align*}
\end{prop}
\begin{proof}
Mit Hilfe der Cartan-Formel lässt sich die Lie-Ableitung wie folgt ausdrücken
\begin{align*}
L_{X}\vol = X\ic \ddd\vol + \ddd(X\ic \vol) = \ddd(*X^*),
\end{align*}
denn $\ddd\vol = 0$. Weiterhin gilt
\begin{align*}
(L_{X}\vol)(e_{1},\ldots,e_{n}) &= L_{X}\vol(e_{1},\ldots,e_{n}) - \sum_{i=1}^n \vol(\ldots,L_{X}e_{i},\ldots)\\
&=- \sum_{i=1}^n\vol(\ldots,[X,e_{i}],\ldots).
\end{align*}
Schreiben wir $[X,e_{i}] = \sum_{j=1}^n f_{ij} e_{j}$, dann gilt also
\begin{align*}
(L_{X}\vol)(e_{1},\ldots,e_{n}) = - \sum_{i=1}^nf_{ii}.
\end{align*}
Andererseits ist $\nabla$ torsionsfrei und metrisch, so dass
\begin{align*}
\div(X) &= \sum_{i=1}^n g(\nabla_{e_{i}}X,e_{i}) = 
\sum_{i=1}^n \left(g(\nabla_{X}e_{i},e_{i}) -  g([e_{i},X],e_{i})\right)\\
&= \frac{1}{2}  \sum_{i=1}^n Xg(e_{i},e_{i})  - \sum_{i=1}^n g([X,e_{i}],e_{i})\\
&= -\sum_{i=1}^n f_{ii}.\qed
\end{align*}
\end{proof}

Als unmittelbare Konsequenz aus dem Satz von Stokes erhalten wir nun folgenden Integralsatz.

\begin{prop}[Satz von Gauß-Green]
\index{Satz!von Gauß-Green}
Sei $X$ ein Vektorfeld auf $M$, dann gilt
\begin{align*}
\int_{M}\div (X)\dM = 0,
\end{align*}
wobei $\ddd M$ die Integration über $\vol$ bezeichnet.\fish
\end{prop}
\begin{proof}
Alle hier betrachteten Mannigfaltigkeiten sind randlos, d.h. $\partial M  =\varnothing$. Die Behauptung folgt nun mit dem vorangegangen Lemma und dem Satz von Stokes,
\begin{align*}
\int_{M}\div(X)\dM = \int_{M}\div(X)\vol = \int_{M}\ddd(*X^*) = \int_{\partial M} *X^* = 0.\qed
\end{align*}
\end{proof}

\begin{lem}
Sei $\omega$ eine Diffferentialform auf $M$, dann schreibt sich deren Differential lokal als
\begin{align*}
\dom = \sum_{i=1}^n e_{i}^* \wedge \nabla_{e_{i}}\omega.\fish
\end{align*}
\end{lem}

Dieser Ausdruck für das Differential motiviert die Definition eines \textit{Kodifferentials}, das im Wesentlichen durch Anwendung von $*$ auf obigen Formel definiert wird.

\begin{defn}
\index{Kodifferential}
Das \emph{Kodifferential} $\ddd^*\colon \Omega^k(M)\to \Omega^{k-1}(M)$ ist ein Differentialoperator erster Ordnung definiert durch
\begin{align*}
\ddd^*\omega = - \sum_{i=1}^n e_{i}\,\ic \nabla_{e_{i}}\omega,
\end{align*}
für $\omega\in\Omega^k(M)$ und eine lokale Orthonormalbasis $\setd{e_{i}}$.\fish
\end{defn}

Wir überzeugen uns davon, dass obige Definition nicht von der Wahl der Basis abhängt, indem wir folgenden koordinatenunabhängige Definition angeben:

\begin{lem}
\label{lem:Kodifferential-Hodge}
Sei $\omega\in\Omega^k(M)$, dann gilt
\begin{align*}
\ddd^*\omega = -(-1)^{n(k+1)}*\ddd*\omega.
\end{align*}
Insbesondere gilt $\div(X) = -\ddd^*X^*$.\fish
\end{lem}

Während der Gradient einer Funktion durch das gewöhnliche Differential bestimmt ist, ist die Divergenz also durch das Kodifferential gegeben.

\begin{proof}
Wir rechnen in einer lokalen Basis direkt nach, dass
\begin{align*}
*\ddd*\omega &= *\left(
\sum_{i=1}^n e_{i}^* \wedge \nabla_{e_{i}}(*\omega)\right)
= *\left(\sum_{i=1}^n e_{i}^*\wedge *(\nabla_{e_{i}}\omega)\right)\\
&= (-1)^{k+1}*^2
\left(\sum_{i=1}^n e_{i}\ic (\nabla_{e_{i}}\omega)\right)\\
&= (-1)^{k+1}(-1)^{(k+1)(n-k-1)}
\left(\sum_{i=1}^n e_{i}\ic (\nabla_{e_{i}}\omega)\right)\\
&= (-1)^{(k+1)(n-k)}
\left(\sum_{i=1}^n e_{i}\ic (\nabla_{e_{i}}\omega)\right)\\
&=
-(-1)^{n(k+1)}\ddd^* \omega,
\end{align*}
denn $(-1)^{k(k+1)} = 1$.\qed
\end{proof}

Das Kodifferential ist in einem gewissen Sinne dual zum gewöhnlichen Differential. Dazu definieren wir zunächst ein geeignetes Skalarprodukt.

\begin{defn}
Auf dem \emph{Raum $\Omega_{c}^k(M)$ der $k$-Formen auf $M$ mit kompaktem Träger} ist das $L^2$-Skalarprodukt definiert durch
\begin{align*}
(\alpha,\beta)\defl \int_{M} g(\alpha,\beta)\ddd M,\qquad \alpha,\beta \in\Omega_{c}^k(M).\fish
\end{align*}
\end{defn}

Falls die Mannigfaltigkeit $M$ selbst schon kompakt ist, gilt $\Omega_{c}^k(M) = \Omega^k(M)$.

\begin{lem}
Das Kodifferential ist formal $L^2$-adjungiert (oder kurz adjungiert) zum Differential, d.h. für Differentialformen $\alpha\in\Omega^{k-1}_{c(M)}$ und $\beta\in\Omega^{k}_{c}(M)$ gilt
\begin{align*}
(\ddd\alpha,\beta) = (\alpha,\ddd^*\beta).\fish
\end{align*}
\end{lem}


\begin{proof}
Zunächst gilt nach Definition des Hodge-$*$-Operators,
\begin{align*}
(\alpha,\ddd^*\beta) = \int_{M} g(\alpha,\ddd^*\beta)\vol
= \int_{M} \alpha\wedge *\ddd^*\beta.
\end{align*}
Umgekehrt gilt dann natürlich auch
\begin{align*}
(\ddd\alpha,\beta) = \int_{M} g(\ddd\alpha,\beta)\vol = \int_{M} \ddd\alpha\wedge *\beta.
\end{align*}
Weiterhin gilt nach den Rechenregeln für $*$,
\begin{align*}
\alpha\wedge *\ddd^*\beta &= (-1)^{n(k+1)+1}\alpha\wedge *^2\ddd *\beta\\
&= (-1)^{n(k+1)+1}(-1)^{(n-k+1)(k-1)}\alpha\wedge \ddd *\beta\\
&= (-1)^{k^2}\bigl(\ddd(\alpha\wedge *\beta) - \ddd \alpha\wedge *\beta\bigr)(-1)^{k-1}\\
&=-\ddd(\alpha\wedge *\beta) + \ddd \alpha\wedge *\beta.
\end{align*}
Bei Integration verschwindet der erste Term nach dem Satz von Stokes, also gilt
\begin{align*}
(\alpha,\ddd^*\beta) = 
\int_{M} \alpha\wedge *\ddd^*\beta = 
\int_{M} \ddd \alpha\wedge *\beta = (\ddd\alpha,\beta).\qed
\end{align*}
\end{proof}

\begin{defn}
\index{Hessische}
Die \emph{Hessische} einer Funktion $f\in\Cs^\infty(M)$ ist definiert als
\begin{align*}
\Hess(f) = \nabla(\ddd f) \in\Gamma(T^*M\otimes T^*M),
\end{align*}
d.h. für Vektorfelder $X$ und $Y$ auf $M$ gilt
\begin{align*}
\Hess(f)(X,Y) = \nabla_{X}(\df(Y)) - \df(\nabla_{X}Y) = X(Y(f)) - (\nabla_{X}Y)(f).\fish
\end{align*}
\end{defn}

\begin{lem}
Sei $f$ eine glatte Funktion auf $M$, dann gelten
\begin{propenum}
\item $\Hess(f) = \nabla^2 f$, und
\item $\Hess(f)$ ist ein symmetrischer $(0,2)$-Tensor, d.h. $\Hess(f)$ ist $\Cs^\infty(M)$-bilinear und für Vektorfelder $X$ und $Y$ auf $M$ gilt
\begin{align*}
\Hess(f)(X,Y) = \Hess(f)(Y,X).\fish
\end{align*}
\end{propenum}
\end{lem}

\begin{proof}
a): Fassen wir die Funktion $f$ als $(0,0)$-Tensor auf, so gilt nach Definition $\nabla_{X} f = L_{X}f = \df(X)$.

b): Seien also $X$ und $Y$ Vektorfelder, dann gilt
\begin{align*}
&\Hess(f)(X,Y) - \Hess(f)(Y,X) = (\nabla^2 f)_{XY}-(\nabla^2 f)_{YX} \\
&\qquad= 
X(Y(f)) - (\nabla_{X}Y)(f) - Y(X(f)) + (\nabla_{Y}X)(f)\\
&\qquad= [X,Y](f) - (\nabla_{X}Y-\nabla_{Y}X)(f) = 0,
\end{align*}
denn der Zusammenhang ist torsionsfrei.\qed
\end{proof}

\section{Der Laplace-Operator auf Funktionen}

\begin{defn}
\index{Laplace-Operator}
Der \emph{Laplace-Operator} auf $M$ ist ein Differentialoperator 2. Ordnung definiert als
\begin{align*}
\D f = - \div(\grad f),\qquad f\in\Cs^\infty(M).\fish
\end{align*}
\end{defn}

\begin{rem}
Das Vorzeichen in der Definition des Laplace-Operators ist absichtlich so gewählt, dass $\Delta$ ein positiver Operator auf $M$ wird. Wählen wir $M=\Omega\subset \R^n$ offen, dann entspricht der so definierte Laplace-Operator gerade dem dem negativen klassischen Laplace-Operator, d.h
\begin{align*}
\D f = -\D_{\text{klass}}f,\qquad f\in\Cs^\infty(\Omega).\map
\end{align*}
\end{rem}

Wir listen nun einige Eigenschaften des Laplace-Operators auf.

\begin{prop}
Sei $f$ eine glatte Funktion auf $M$ und $\setd{e_{i}}$ eine Orthonormalbasis von $T_{m}M$. Dann gelten
\begin{propenum}
\item $\D f = \ddd^*\ddd f = -*\ddd*\ddd f$,
\item $\D f = -\tr(\Hess(f))$,
\item $(\D f)_{m} = -\sum_{i=1}^n \left[ e_{i}(e_{i}(f)) - (\nabla_{e_{i}}e_{i})f\right]_{m}$,
\item Seien $\setd{\gamma_{i}}$ Geodätische durch $m$ mit $\dot{\gamma}_{i}(0) = e_{i}$, dann gilt
\begin{align*}
(\D f)_{m} = -\sum_{i=1}^n \frac{\ddd}{\dt^2}\bigg|_{t=0} (f\circ\gamma_{i})(t).
\end{align*}
\item In lokalen Koordinaten $(U,x)$ von $M$ gilt
\begin{align*}
\D f\big|_{U} = - \bar{g}^{-1/2} \sum_{i=1}^n \frac{\partial}{\partial x_{i}} \left( g^{ij} \bar{g}^{1/2} \frac{\partial f}{\partial x_{j}}\right),
\end{align*}
wobei $\bar{g} = \det(g_{ij})$ und $g_{ij} = g\left( \frac{\partial}{\partial x_{i}},\frac{\partial}{\partial x_{j}}\right)$.
\item Der Laplace-Operator ist formal selbstadjungiert und positiv, d.h. es gilt
\begin{align*}
(\D f,g) = (f,\D g),\qquad \text{und}\qquad (\D f,f) \ge 0,
\end{align*}
für alle Funktionen $f,g\in\Cs_{c}^\infty(M)$.\fish
\end{propenum}
\end{prop}

\begin{proof}
a): Nach Lemma \ref{lem:Kodifferential-Hodge} gilt $\div(X) = -\ddd^*X^*$. Folglich gilt für eine glatte Funktion $f$, dass
\begin{align*}
-*\ddd * \df = -\ddd^* \df = -\ddd^* (\grad f)^* = -\div(\grad f).
\end{align*}

b) \& c): Man rechnet im Punkt $m$ nach, dass
\begin{align*}
\div(\grad f)_{m} &= \sum_{i=1}^n g\left(\nabla_{e_{i}}\grad(f),e_{i}\right)_{m}\\
&= \sum_{i=1}^n g\left(\grad(f),\nabla_{e_{i}}e_{i}\right)_{m} - g\left(\grad(f),\nabla_{e_{i}}e_{i}\right)_{m}\\
&= \sum_{i=1}^n e_{i}(e_{i}(f))_{m} - (\nabla_{e_{i}}e_{i})(f)_{m}\\
&= \tr(\Hess(f))_{m}.
\end{align*} 

d): Setzen wir die Orthonormalbasis $\setd{e_{i}}$ durch Parallelverschiebung in einer kleinen Umgebung $U$ um $m$ fort, dann ist $(\nabla e_{i})_{m} = 0$ und für die Geodätischen gilt $\dot{\gamma}_{i}(t) = e_{i}\big|_{\gamma_{i}(t)}$. Dann gilt
\begin{align*}
\D f = - \sum_{i=1}^n e_{i}(e_{i}(f)) =  - \sum_{i=1}^n \frac{\ddd^2}{\dt^2} f\circ\gamma_{i}(t).
\end{align*}

f): Das Kodifferential ist adjungiert zum gewöhnlichen Differential, also gilt für glatte Funktionen $f$ und $g$ auf $M$,
\begin{align*}
(\D f,g) = (\ddd^* \ddd f,g) = (\ddd f,\ddd g) = (f,\ddd^*\ddd g) = (f,\D g).
\end{align*}
Insbesondere gilt $(\D f,f) = (\ddd f,\ddd f) = \norm{\ddd f}^2 \ge 0$.\qed
\end{proof}

\begin{prop}[Produktregel]
Seien $f$ und $h$ glatte Funktionen auf $M$, dann gilt
\begin{align*}
\D(f\cdot h) = (\D f)\cdot h + f\cdot (\D h) - 2g(\grad f,\grad h).\fish
\end{align*}
\end{prop}
\begin{proof}
Wählen wir geodätische Normalkoordinaten in $m$, dann gilt
\begin{align*}
\D (f\cdot h)_{m} &= - \sum_{i=1}^n \frac{\partial^2 (f\cdot h)}{\partial x_{i}^2}\bigg|_{,}\\
& = - \sum_{i=1}^n 
\frac{\partial^2 f}{\partial x_{i}^2}\bigg|_{m} h(m)
+
2\frac{\partial f}{\partial x_{i}}\bigg|_{m}
\frac{\partial h}{\partial x_{i}}\bigg|_{m}
+
f(p)\frac{\partial^2 h}{\partial x_{i}^2}\bigg|_{m}\\
&= (\D f)_{m}\cdot h(m)
+ f(m)(\D h)_{m} - 2g(\grad f,\grad g)_{m}.\qed
\end{align*}
\end{proof}

Der Laplace-Operator ist definiert durch $\D =\div(\grad)$  und hängt somit von der Metrik ab. Abbildungen, die die Metrik erhalten erhalten auch den Laplace-Operator.

\begin{prop}
Sei $\ph\colon (M,g)\to (N,h)$ eine Isometrie. Dann gilt
\begin{align*}
\D^M(f\circ\ph) = (\D^Nf)\circ\ph.\fish
\end{align*}
\end{prop}
\begin{proof}
Sei $\eta\in\Cs^\infty(N)$, dann gilt
\begin{align*}
(\D^M(f\circ\ph),\eta\circ\ph)_{M} &= 
(\ddd(\ph^* f),\ddd(\ph^*\eta))_{M}
=
(\ph^*\ddd f,\ph^*\ddd\eta)_{M}\\
&=
(\ddd f,\ddd\eta)_{N}
=
(\nabla^N f,\eta)_{N}\\
&= 
(\ph^*(\D^N f),\eta\circ\ph)_{M}.\qed
\end{align*} 
\end{proof}

\begin{prop}
Sei $\pi\colon (M,g)\to (N,h)$ eine Riemannsche Submersion, d.h. der Tangentialraum an $M$ zerfällt in
\begin{align*}
TM = T^h\oplus T^v,\qquad T^h= \ker\dpi,
\end{align*}
und die Metrik auf $M$ ist gegeben durch
\begin{align*}
g_{M} = \pi^*g_{N} + g_{F},
\end{align*}
mit einer Fasermetrik $g_{F}$ auf $T^h$. Sind die Fasern von $M$ über $N$ totalgeodätisch, so gilt
\begin{align*}
\D^M(f\circ\pi) = (\D^N f)\circ\pi.\fish
\end{align*}
\end{prop}
\begin{proof}
Sei $m\in M$ und der Tangentialraum wie folgt orthogonal zerlegt
\begin{align*}
T_{m} = T_{m}^h \oplus T_{m}^v.
\end{align*}
Wähle Orthonormalbasen $\setd{w_{i}}$ von $T_{m}^h$ und $\{v_{j}\}$ von $T_{m}^v$ und zugehörige Geodätische $\gamma_{i}$ und $\delta_{j}$ durch $m$ mit
\begin{align*}
\dot\gamma_{i}(0) = w_{i},\qquad \dot\delta_{j}(0) = v_{j}.
\end{align*}
Da die Submersion total geodätisch ist, ist $\pi\circ\gamma_{i}$ eine Geodätische von $N$ und $\delta_{j}$ verläuft vollständig in der Faser $F=\pi^{-1}(n)$ mit $m\in F$. Somit gilt
\begin{align*}
\D^M(f\circ\ph)_{m} &= -\sum_{i}\frac{\ddd^2}{\dt^2}\bigg|_{0} (f\circ\pi)\circ \gamma_{i}(t)
 -\sum_{j}\frac{\ddd^2}{\dt^2}\bigg|_{0} (f\circ\pi)\circ \delta_{j}(t)\\
&  = -\sum_{i}\frac{\ddd^2}{\dt^2}\bigg|_{0} f\circ(\pi\circ \gamma_{i})(t)\\
&  = (\Delta^N f)_{\pi(m)}.\qed
\end{align*}
\end{proof}

Seien nun $(M,g)$ und $(N,h)$ Riemannsche Mannigfaltigkeiten und $(M\times N,g\oplus h)$ bezeichne ihr Riemannsches Produkt. Dann sind die Projektionen
\begin{align*}
\pr_{M}\colon M\times N\to M,\quad (m,n)\mapsto m,\qquad
\pr_{N}\colon M\times N\to N,\quad (m,n)\mapsto n,
\end{align*}
totalgeodätische Riemannsche Submersionen. 

\begin{prop}
Seien $a\in\Cs^\infty(M)$ und $b\in\Cs^\infty(N)$, dann ist
\begin{align*}
a\star b\defl
\pr_{M}^*a \cdot \pr_{N}^*b = (a\circ\pr_{M})\cdot (b\circ \pr_{N})\in\Cs^\infty(M\times N),
\end{align*}
mit $a\star b(m,n) = a(m)b(n)$, und es gilt
\begin{align*}
\D^{M\times N}(a\star b) = 
(\D^M a)\star b + 
a\star (\D^N b).\fish
\end{align*}
\end{prop}
\begin{proof}
Mit der Produktregel und dem vorangegangen Satz über Riemannsche Subversionen erhalten wir
\begin{align*}
\D^{M\times N}(a\star b) = { } &
\D^{M\times N}(\pr_{M}^*a \cdot \pr_{N}^*b)\\ = { } & 
(\D^{M\times N}\pr_{M}^*a)\cdot \pr_{N}^*b + 
\pr_{M}^*a\cdot (\D^{M\times N}\pr_{N}^*b)\\
&-2\lin{\grad \pr_{M}^* a ,\grad \pr_{N}^*b}\\
= { } & 
(\D^{M\times N}\pr_{M}^*a)\cdot \pr_{N}^*b + 
\pr_{M}^*a\cdot (\D^{M\times N}\pr_{N}^*b)\\
= { } & 
\pr_{M}^*(\D^Ma)\cdot \pr_{N}^*b
+
\pr_{M}^* a\cdot \pr_{N}^* (\D^N b)\\
{ } = &
(\D^Ma)\star b
+
a\star (\D^N b).\qed
\end{align*}
\end{proof}

\begin{cor}
Sei $a$ eine $\D^M$ Eigenfunktion zum Eigenwert $\lambda$ und sei $b$ eine $\D^N$ Eigenfunktion zum Eigenwert $\mu$. Dann ist $a\star b$ eine $\D^{M\times N}$ Eigenfunktion zum Eigenwert $\lambda+\mu$ und jede Eigenfunktion von $\D^{M\times N}$ ist von dieser Form.\fish
\end{cor}
\begin{proof}
Aus dem vorangegangen Satz folgt, dass $\pr_{M}^* a\cdot \pr_{N}^* b$ eine Eigenfunktion zum Eigenwert $\lambda+\mu$ ist. 
Sowohl $\D^M$ als auch $\D^N$ besitzt ein vollständiges Orthogonalsystem aus Eigenfunktionen $\Ec^M$ bzw. $\Ec^N$.
Da die Metrik auf $M\times N$ durch $g\oplus h$ gegeben ist, erhält man ein Orthonormalsystem aus Eigenfunktionen von $\D^{M\times N}$ durch
\begin{align*}
\Ec^M\star \Ec^N\defl \setdef{a\star b}{a\in \Ec^M\text{ und }b\in\Ec^N}.
\end{align*}
Da $\Ec^M$ dicht in $\Cs^\infty(M)$ und $\Ec^N$ dicht in $\Cs^\infty(N)$ liegt, liegt $\Ec^M\star \Ec^N$ dicht in $\Cs^\infty(M\times N)$ und daher sind durch $\Ec^M\star \Ec^N$ bereits alle möglichen Eigenfunktionen beschrieben.~\qed
\end{proof}

\section{Anwendung: Der Laplace-Operator auf der Sphäre}

Ziel dieses Abschnittes ist es, das Spektrum des Laplace-Operators auf der Sphäre $S^n$ zu bestimmen. Dazu betrachten wir die Sphäre als Untermannigfaltigkeit des $\R^{n+1}$, denn dort hat der Laplace-Operator eine besonders einfache Gestalt.
Wir suchen daher nach einer Möglichkeit den Laplace auf der Sphäre $S^n$ über den Laplace auf dem $\R^{n+1}$ auszudrücken.

\begin{prop}
Sei $f$ eine glatte Funktion auf $\R^{n+1}$, dann gilt
\begin{align*}
\left(\D^{\R^{n+1}}f\right)\big|_{S^n}  = 
\D^{S^n}(f\big|_{S^n}) - \frac{\partial^2 f}{\partial r^2}\bigg|_{S^n} - 
n\frac{\partial f}{\partial r}\bigg|_{S^n}.\fish
\end{align*}
\end{prop}

\begin{proof}
Sei $v=p\in S^n\subset\R^{n+1}$ und $v_{1},\ldots,v_{n}$ eine Orthonormalbasis von $T_{p}S^n = p^\bot$. Die Geodätischen zu $v_{i}$ durch $p$ sind dann gegeben durch
\begin{align*}
\gamma_{i}(t) = \cos t \cdot p + \sin t \cdot v_{i},\qquad 1\le i\le n.
\end{align*}
Mit Hilfe der Geodätischen können wir nun $\D$ auf der Sphäre berechnen . Sei dazu $f\in\Cs^\infty(\R^{n+1})$, dann gilt
\begin{align*}
\frac{\ddd}{\dt} f(\gamma_{i}(t)) &= \df(\dot\gamma_{i}(t)) = 
\df(-\sin t\cdot p + \cos t \cdot v_{i})\\
&= -\sin t \cdot v(f) + \cos t \cdot v_{i}(f)\bigg|_{\gamma_{i}(t)}.
\end{align*}
Analog erhalten wir für die zweite Ableitung
\begin{align*}
\frac{\ddd^2}{\dt^2}\bigg|_{t=0} f(\gamma_{i}(t)) &=
\frac{\ddd}{\dt}
-\sin t \cdot v(f) + \cos t \cdot v_{i}(f)\bigg|_{t=0}\\
&= -v(f) + v_{i}(v_{i}(f))\bigg|_{\gamma_{i}(0)}.
\end{align*}
Somit ist der Laplace-Operator auf der Sphäre gegeben durch,
\begin{align*}
\D(f\big|_{S^n})_{p} &=
-\sum_{i=1}^n 
\frac{\ddd^2}{\dt^2}\bigg|_{t=0} f(\gamma_{i}(t))\\
&= -\sum_{i=1}^n v_{i}(v_{i}(f))\bigg|_{\gamma_{i}(0)} + n v(f)\bigg|_{\gamma_{i}(0)}\\
&= (\D^{\R^n} f)_{p} + v(v(f))_{p} + nv(f)_{p}.
\end{align*}
Da $v(f) = \frac{\partial f}{\partial r}$, folgt schließlich die Behauptung.\qed
\end{proof}

Mit Hilfe dieser Formel wollen wir nun Eigenfunktionen von $\D$ auf der Sphäre $S^n$ berechnen. Dazu betrachten wir die \textit{homogenen Polynome vom Grad $k$ auf $\R^{n+1}$}. Für den Grad $0$ sind dies gerade die konstanten und für den Grad $1$ die linearen Funktionen. Die allgemeine Form eines solchen Polynoms ist
\begin{align*}
H(x) = \sum a_{i_{1}\ldots i_{r}} \; x_{i_{1}}^{n_{i_{1}}} \cdot \ldots \cdot x_{i_{r}}^{n_{i_{r}}}
\end{align*}
mit $n_{i_{1}}+\ldots+n_{i_{r}}=k$. Die Homogenität vom Grad $k$ ist gleichbedeutend mit
\begin{align*}
H = r^k (H\big|_{S^n}),
\end{align*}
d.h. für $x\in S^n$ und $\mu\in\R$ gilt $H(\mu x) = \mu^k H(x)$. Für die ersten beiden Ableitungen ergibt sich somit
\begin{align*}
\frac{\partial H}{\partial r} = kr^{k-1} (H\big|_{S^n}),\qquad
\frac{\partial^2 H}{\partial r^2} = k(k-1)r^{k-2} (H\big|_{S^n}).
\end{align*}
Nach obiger Formel erhalten wir für homogene Polynome
\begin{align*}
\D^{S_{n}}(H\big|_{S^n}) = 
\left(\D^{\R^{n+1}} H\right)_{S^n} + 
k(n+k-1) (H\big|_{S^n}).
\end{align*}
Die harmonischen, homogenen Polynome vom Grad $k$ auf $\R^{n+1}$ sind also Eigenfunktionen von $\D^{S^n}$, mit
\begin{align*}
\D^{S^n}(H\big|_{S^n}) = k(n+k-1) H\big|_{S^n}.\tag{*}
\end{align*}

Wir wollen nun zeigen, dass jede Eigenfunktion als Linearkombination harmonischer, homogener Polynome gegeben ist. Dazu bezeichnen wir den Raum der homogenen Polynome vom Grad $k$ auf $\R^{n+1}$ mit $\Pc_{k}$ und den Raum der harmonischen, homogenen Polynome vom Grad $k$ mit $\Hs_{k}$. Weiterhin bezeichnen wir die Einschränkung einer Funktion $f\in\Cs ^\infty(\R^{n+1})$ auf $S^n$ mit
\begin{align*}
\tilde{f} \defl f\big|_{S^n},
\end{align*}
und schreiben
\begin{align*}
\tilde\Hs_{k} \defl \setdef{\tilde{f}}{f\in\Hs_{k}},\qquad \tilde\Pc_{k} \defl \setdef{\tilde{f}}{f\in\Pc_{k}}.
\end{align*}
Für Funktionen $P$ und $Q\in\Pc_{k}$ definieren wir ein Skalarprodukt durch
\begin{align*}
\lin{P,Q} \defl \int_{S^n} \tilde P \tilde Q\ddd S^n.
\end{align*}
Mit dieser Notation können wir unser Resultat wie folgt postulieren:
\begin{prop}
Das Spektrum von $\D^{S^n}$ ist gegeben durch eine abzählbare Folge von ganzzahligen, positiven Eigenwerten
\begin{align*}
\lambda_{k} = k(n+k-1),\qquad k=0,1,\ldots,
\end{align*}
und der Eigenraum zu $\lambda_{k}$ ist $\tilde\Hs_{k}$.\fish
\end{prop}

\newcommand{\scal}{\mathrm{scal}}
\newcommand{\spec}{\mathrm{spec}}

\begin{rem}
Hier ist $S^n$ mit der Standardmetrik $g$ versehen und daher ist die Skalarkrümmung normiert auf
\begin{align*}
\scal = n(n-1).
\end{align*}
Für eine weitere Metrik $h=c\cdot g$ mit $c\in\R$ gilt
\begin{align*}
\spec(\D,h) = \frac{1}{c}\spec(\D,g).\map
\end{align*}
\end{rem}

Bevor wir den Satz beweisen, benötigen wir ein Resultat über die Struktur der homogenen Polynome.

\begin{lem}
Für $k\ge 0$ lassen sich die Räume homogener Polynome wie folgt orthogonal zerlegen
\begin{align*}
&\Pc_{2k} = \Hs_{2k}\oplus r^2 \Hs_{2k-2} \oplus \ldots \oplus r^{2k} \Hs_{0},\\
&\Pc_{2k+1} = \Hs_{2k+1}\oplus r^2 \Hs_{2k-1} \oplus \ldots \oplus r^{2k}\Hs_{1}.\fish
\end{align*}
\end{lem}
\begin{proof}
Wir zeigen die Behauptung per Induktion. Offenbar liegen die konstanten und die linearen Funktionen im Kern des Laplace Operators und daher gilt 
\begin{align*}
\Pc_{0} = \Hs_{0},\qquad \Pc_{1} = \Hs_{1}.
\end{align*}
Sei also für $k\ge 0$ die Zerlegung $\Pc_{k} = \Hs_{k}\oplus r^2\Pc_{k-2}$ gegeben, dann ist zu zeigen, dass
\begin{align*}
\Pc_{k+2} = \Hs_{k+2}\oplus r^2\Pc_{k}.
\end{align*}
Bezeichne $E(\lambda_{k})$ den Eigenraum von $\D^{S^n}$ zu $\lambda_{k}$, so haben wir bereits gezeigt, dass
\begin{align*}
\tilde\Hs_{k+2} \subset E(\lambda_{k+2}).
\end{align*}
Nach Induktionsvoraussetzung gilt somit
\begin{align*}
\Pc_{k} \subset \bigoplus_{i=1}^k E(\lambda_{i}).
\end{align*}
Da $\D$ selbstadjungiert ist, sind Eigenräume zu verschiedenen Eigenwerten orthogonal und daher gilt insbesondere $\Hs_{k+2}\bot \Pc_{k}$, sodass
\begin{align*}
\Hs_{k+2}\oplus r^2\Pc_{k} \subset \Pc_{k+2}.
\end{align*}
Um die umgekehrte Inklusion einzusehen, betrachte ein Polynom $p\in\Pc_{k+2}$ mit $p\,\bot\, \Pc_{k}$. Dann ist
\begin{align*}
\D p\in\Pc_{k}=\Hs_{k}\oplus r^2\Hs_{k-2}\oplus \ldots,
\end{align*}
d.h. $p$ ist genau dann harmonisch, wenn
\begin{align*}
\D p \,\bot\, r^{2l}\Hs_{k-2l} \text{ für }0\le 2l\le k
\quad
\iff
\quad
\widetilde{\D p} \,\bot\, \tilde\Hs_{k-2l} \text{ für }0\le 2l\le k.
\end{align*}
Nach voriger Rechnung ist
\begin{align*}
\D \tilde p &= \widetilde{\D p} + \widetilde{\frac{\partial^2 p}{\partial r^2}}
+n\widetilde{\frac{\partial p}{\partial r}}\\
&= \widetilde{\D p} + \lambda_{k+2}\tilde{p}.
\end{align*}
Somit gilt für ein harmonisches, homogenes Polynom $H$ vom Grad $k-2l$, dass
\begin{align*}
(\widetilde{\D p},\tilde{H}) 
&= (\D\tilde{p},\tilde{H}) - \lambda_{k+2}(\tilde{p},\tilde{H})\\
&= (\tilde{p},\D\tilde{H}) - \lambda_{k+2}(\tilde{p},\tilde{H})\\
&= (\lambda_{2k-l}-\lambda_{k+2}) (\tilde{p},\tilde{H}) = 0,
\end{align*}
denn $\tilde p\,\bot\, \tilde\Pc_{k}=\bigoplus_{0\le 2l\le k} \tilde\Hs_{k-2l}$. Also ist $p\in\Hs_{k+2}$ und folglich gilt
\begin{align*}
\Hs_{k+2}\oplus r^2\Pc_{k} \supset \Pc_{k+2}.\qed
\end{align*}
\end{proof}

\begin{proof}[Beweis des Satzes.]
Jedes homogene Polynom lässt sich als Linearkombination von harmonischen, homogenen Polynomen darstellen. Weiterhin liegen die Polynome nach dem Satz von Weierstrass dicht in den glatten Funktionen, d.h. auch
\begin{align*}
\bigoplus_{k\ge 0}\tilde\Hs_{k} = \bigoplus_{k\ge 0}\tilde\Pc_{k} \subset \Cs^\infty(S^n)
\end{align*}
liegt dicht. Somit sind die $\lambda_{k}$ die einzigen Eigenwerte von $\D$, denn für ein weiteres $\lambda\in\R$ gäbe es auch einen Eigenraum $E(\lambda)$ der senkrecht auf allen $\tilde\Hs_{k}$ stehen müsste, dann könnten die $\Hs_{k}$ aber nicht dicht liegen. Somit gilt auch $E(\lambda_{k}) = \tilde\Hs_{k}$, denn andernfalls gäbe es eine Eigenfunktion $\ph_{k}$ zu $\lambda_{k}$, die orthogonal auf $\tilde\Hs_{k}$ steht und da $E_{\lambda}$ abgeschlossen ist, ist dies wiederum ein Widerspruch zur Dichtheit.\qed
\end{proof}

\begin{rem}
Die Dimension der Eigenräume lässt sich nun wie folgt berechnen,
\begin{align*}
\dim E(\lambda_{k}) = \dim \Hs_{k} = 
\dim \Pc_{k}-\dim\Pc_{k-2} =
\binom{n+k}{k} - 
\binom{n+k-2}{n-2}.\map
\end{align*}
\end{rem}

\section{Ausblick: Symmetrische Räume}

Sei $M=G/K$ ein symmetrischer Raum. Dann ist $G\to G/K$ ein $K$-Hauptfaserbündel und wir betrachten das Vektorbündel
\begin{align*}
E = G\times_{\rho}V,\qquad \rho\colon K\to \Aut(V).
\end{align*}
Auf $\Gamma(E)$ wirkt die Gruppe $G$ von links, indem man für einen Schnitt $e$ definiert
\begin{align*}
(ge)(x) \defl e(g^{-1}x),\qquad x\in M,\; g\in G.
\end{align*}
Daher definiert $\Gamma(E)$ eine $G$-Darstellung, die sich wie folgt zerlegen lässt
\begin{align*}
\Gamma(E) = \bigoplus_{\lambda\in\hat{G}} V_{\lambda}\otimes \Hom_{k}(V_{\lambda},V),
\end{align*}
wobei die $V_{\lambda}$ irreduzible $G$-Darstellungen bezeichnen. Als $K$-Darstellung ist $V_{\lambda}$ nicht mehr irreduzibel und zerfällt nach dem Lemma von Schur in
\begin{align*}
V_{\lambda}\big|_{K} = V_{1}\oplus \ldots \oplus V_{l} = \C^r,
\end{align*}
wobei $r$ die Anzahl der $V_{i}$ für $1\le i\le l$ bezeichnet mit $V_{i}\cong V$.

Da $M$ symmetrisch ist, entspricht der Levi-Civita Zusammenhang auf $E$ der Richtungsableitung und daher entspricht $\nabla^*\nabla$ dem \textit{Casimir-Operator},
\begin{align*}
(\nabla^*\nabla)^E = \Cas_{V_{\lambda}}\otimes \Id,
\end{align*}
wobei für eine Darstellung $\rho\colon G\to V$ der Casimir-Operator definiert ist durch
\begin{align*}
\Cas_{\rho} \defl \sum_{i} \rho_{*}(x_{i})^2 \in \End(V),
\end{align*}
mit einer Orthonormalbasis $\setd{x_{i}}$  von $\g = \Lie(G)$. Da die Darstellung $V_{\lambda}$ irreduzibel ist, gilt
\begin{align*}
\Cas_{V_{\lambda}} = \lin{v,\lambda + 2\sigma}\Id_{V_{\lambda}}  =(\abs{\lambda}^2 + \lin{\lambda,2\sigma})\Id_{V_{\lambda}},
\end{align*}
wobei $\lambda$ durch die Darstellung $V_{\lambda}$ und $\sigma$ durch die Lie-Algebra bestimmt wird.

\begin{ex}
Die $2n$-dimensionale Sphäre ist ein symmetrischer Raum
\begin{align*}
S^{2n} = \SO(2n+1)/\SO(2n).
\end{align*}
Die Darstellungen von $\SO(2n+1)$ sind parametrisiert durch
\begin{align*}
(\lambda_{1},\ldots,\lambda_{n})\in\Z^n \text{ mit } \lambda_{1}\ge \ldots\ge \lambda_{n}\ge 0.
\end{align*}
Analog sind die Darstellungen von $\SO(2n)$ parametrisiert durch
\begin{align*}
(\lambda_{1},\ldots,\lambda_{n})\in\Z^n \text{ mit } \lambda_{1}\ge \ldots\ge \lambda_{n-1} \ge \abs{\lambda_{n}}.
\end{align*}
Die Einschränkung einer irreduziblen Darstellung von $\SO(2n+1)$ auf $\SO(2n)$ ist nicht irreduzibel, sondern zerfällt
\begin{align*}
\SO(2n+1)\to \SO(2n),\qquad \lambda=(\lambda_{1},\ldots,\lambda_{n})\to \bar\lambda=(\bar\lambda_{1},\ldots,\bar\lambda_{n}).
\end{align*}
Dabei erfüllen die Parametrisierungen die sogenannten \textit{branching rules},
\begin{align*}
\lambda_{1}\ge \bar\lambda_{1}\ge \lambda_{2} \ge \ldots \ge \lambda_{n} \ge|\bar\lambda_{n}|.
\end{align*}
Für $\bar{\lambda}=0$ ist so z.B. $\lambda=(\lambda_{1},0,\ldots,0)$. Daher können wir die glatten Funktionen auf $S^{2n}$ darstellen als
\begin{align*}
\Cs^\infty(S^{2n}) = \bigoplus V_{(\lambda_{1},0,\ldots,0)}\otimes \Hom_{k}(V_{(\lambda_{1},0,\ldots,0)},V).
\end{align*}
Für $M=S^{2n}$ ist $G=\SO(2n+1)$ und auf $\g=\so(2n+1)$ sei das Skalarprodukt durch die Killing-Form induziert. Man berechnet, dass hier
\begin{align*}
\sigma = \frac{1}{2}(2n-1,2n-3,\ldots,1),
\end{align*}
und daher ist der Casimir gegeben durch
\begin{align*}
\Cas_{V_{(\lambda_{k},0,\ldots,0)}} = k^2+k(2n-1) = k(k+2n-1) = \lambda_{k}.
\end{align*}
Die Bestimmung des Spektrums der Sphäre lässt sich also in ein Problem der Darstellungstheorie von Lie-Gruppen übersetzen.\boxc
\end{ex}



% Vorlesung vom 20.07.2011

\section{Der Laplace-Operator auf Differentialformen}

Sei wieder $(M,g)$ eine Riemannsche Mannigfaltigkeit. Wir setzten nun den für Funktionen definierten Laplace-Operator unter Verwendung von Differential und Kodifferential auf Formen fort.

\begin{defn}
\index{Laplace-Operator!Hodge-}
Der \emph{Hodge-Laplace-Operator} auf Formen
\begin{align*}
\D \colon \Omega^k(M)\to \Omega^k(M)
\end{align*}
ist für $\omega\in\Omega^k(M)$ definiert durch
\begin{align*}
\D\omega = \ddd^*\ddd\omega + \ddd\ddd^*\omega.\fish
\end{align*}
\end{defn}

Der Term $\ddd\ddd^*$ verschwindet auf Funktionen und trat daher in der ursprünglichen Definition nicht auf. Viele Eigenschaften des Laplace-Operators auf Funktionen übertragen sich auch auf den Hodge-Laplace-Operator.

\begin{lem}
\begin{propenum}
\item Es gilt $\D = (\ddd + \ddd^*)^2$.
\item Der Laplace-Operator ist selbstadjungiert und positiv, genauer gilt für Formen $\alpha$, $\beta\in \Omega^p_{c}(M)$,
\begin{align*}
(\D \alpha,\beta) = (\alpha,\D\beta),\qquad \text{und} \qquad (\D\alpha,\alpha) = \norm{\ddd\alpha}^2 + \norm{\ddd^*\alpha}^2 \ge 0.
\end{align*}
\item Der Laplace-Operator vertauscht mit dem Differential, dem Kodifferential und dem Hodge-$*$-Operator, d.h. es gilt
\begin{align*}
\D\ddd = \ddd\D,\qquad \D\ddd^* = \ddd^* \D,\qquad\text{und}\qquad \D* = *\D. \fish
\end{align*}
\end{propenum}
\end{lem}
\begin{proof}
a): Eine formale Rechnung zeigt $(\ddd + \ddd^*)^2 = \ddd^2+ \ddd^*\ddd + \ddd\ddd^*  + (\ddd^*)^2=\D$, denn die Quadrate von Differential und Kodifferential verschwinden.

b): Wir haben bereits verifiziert, dass das Kodifferential der zum Differential adjungierte Operator ist. Somit gilt
\begin{align*}
(\D\alpha,\beta)  &=(\ddd^*\ddd\alpha,\beta) + 
(\ddd\ddd^*\alpha,\beta) = 
(\ddd \alpha,\ddd\beta) + (\ddd^*\alpha,\ddd^*\beta) \\ &= 
(\beta,\ddd^*\ddd\alpha) + (\beta,\ddd\ddd^*\beta) = 
(\beta,\D\alpha),
\end{align*}
und insbesondere $(\D\alpha,\alpha) = \norm{\ddd\alpha}^2 + \norm{\ddd^*\alpha}^2$.

c): Vertauschung von $\D$ mit $\ddd$ bzw. $\ddd^*$ zeigt eine formale Rechnung. Sei weiterhin $\omega\in\Omega^k(M)$, dann gilt
\begin{align*}
\D*\omega &= -(-1)^{n(k+1)}\left(*\ddd*\ddd*\omega + 
\ddd*\ddd**\omega\right)\\
&= -(-1)^{n(k+1)}\left(*\ddd*\ddd*\omega + 
**\ddd*\ddd\omega\right)\\
&= -(-1)^{n(k+1)}*\left(\ddd * \ddd*\omega + *\ddd  *\ddd \omega\right)\\
&=*\left(\ddd\ddd^* + \ddd^*\ddd\right)\omega = 
*\D\omega.\qed
\end{align*}
\end{proof}

\begin{rem}
Der Laplace-Operator ist ein elliptischer Differentialoperator. Auf kompakten Mannigfaltigkeiten ist somit das Spektrum diskret und besteht nur aus Eigenwerten endlicher Vielfachheit. Darüber hinaus existiert für $0\le k \le n$ ein Orthonormalsystem aus Eigenformen von $\Omega^k(M)$.\map
\end{rem}

\begin{defn}
Die Formen im Kern des Laplace-Operators heißen \emph{harmonisch} und mit
\begin{align*}
\Hs^k(M) \defl \ker\left(\D\big|_{\Omega^k(M)} \right)
\end{align*}
bezeichnet man den \emph{Raum der harmonischen $k$-Formen}.\fish
\end{defn}

\begin{lem}
Sei $(M,g)$ eine kompakte Riemannsche Mannigfaltigkeit. Dann ist eine $k$-Form $\omega$ genau dann harmonisch, wenn
\begin{align*}
\ddd\omega = 0\qquad \text{und}\qquad \ddd^*\omega = 0.\fish
\end{align*}
\end{lem}
\begin{proof}
Da $\D$ ein positiver Operator ist, ist $\omega$ genau dann harmonisch, wenn
\begin{align*}
(\D\omega,\omega) = \norm{\ddd\omega}^2 + \norm{\ddd^*\omega}^2 = 0.\qed
\end{align*}
\end{proof}

\begin{rem}
Parallele Formen auf $M$ sind geschlossen und kogeschlossen und daher insbesondere harmonisch.\map
\end{rem}

\section{Hodge-Theorie}

Sei nun $(M,g)$ eine kompakte, $n$-dimensionale Riemannsche Mannigfaltigkeit. 

\begin{prop}[Satz von Hodge-de-Rham]
\begin{propenum}
\item Der Raum $\Hs^k(M)$ der harmonischen $k$-Formen auf $M$ ist endlichdimensional.
\item Es gilt die Hodge-Zerlegung
\begin{align*}
\Omega^k(M) = \Hs^k(M)\oplus \im \left(\ddd\big|_{\Omega^{k-1}(M)}\right)
\oplus \im \left(\ddd^*\big|_{\Omega^{k+1}(M)}\right).
\end{align*}
\item $\Hs^k(M) \cong H_{\dR}^k(M)$.\fish
\end{propenum}
\end{prop}

\begin{proof}
a): Der Raum der harmonischen Formen $\Hs^k(M)$ entspricht gerade dem Eigenraum von $\D\big|_{\Omega ^k(M)}$ zum Eigenwert $0$. Da $\D$ elliptisch und $M$ kompakt ist, ist $0$ ein Eigenwert endlicher Vielfachheit.

b): Der Laplace-Operator bildet $\Omega^k(M)$ auf sich selbst ab und ist selbstadjungiert, d.h. es gilt 
\begin{align*}
\Omega^k(M) = \Hs^k(M) \oplus \im\left(\Delta\big|_{\Omega^k(M)}\right).
\end{align*}
Für alle weiteren Details siehe \cite{deRham:1984vp}.

c): Die Abbildung
\begin{align*}
\Hs^k(M) \to H_{\dR}^k(M),\qquad \omega\mapsto [\omega],
\end{align*}
ist injektiv. Angenommen $[\omega]=0$, dann ist $\omega$ exakt, d.h. $\omega = \ddd\eta$ für eine $k-1$-Form $\eta$. Außerdem liegt $\omega$ im Kern von $\D$, d.h. es gilt
\begin{align*}
0 = \norm{\ddd\omega}^2 + \norm{\ddd^*\omega}^2,
\end{align*}
also ist $\omega$ geschlossen und kogeschlossen. Dann gilt aber auch
\begin{align*}
0 = (\ddd^*\omega,\eta) = (\ddd\eta,\ddd\eta) =\norm{\omega}^2,
\end{align*}
also ist $\omega = 0$. Für die Surjektivität siehe \cite{deRham:1984vp}.\qed
\end{proof}



\begin{rem}[Bemerkungen.]
\begin{remenum}
\item Jede de-Rham Kohomologieklasse enthält einen eindeutig bestimmten harmonischen Repräsentanten, nämlich  den mit minimaler $L^2$-Norm.
\item Aus den Rechenregeln für das Differential erhalten wir nun unmittelbar für die de-Rham Kohomologie
\begin{align*}
H^0_{\dR}(M) = H^n_{\dR}(M) = \R.
\end{align*}
\item Die Betti-Zahlen entsprechen  der Dimension des Raums der harmonischen $k$-Formen
\begin{align*}
b_{k}(M) = \dim H_{\dR}^k(M) = \dim \Hs^k(M).
\end{align*}
Somit ist die Euler-Charakteristik gegeben durch
\begin{align*}
\chi(M) = \sum_{k=0}^n (-1)^k b_{k}(M) = \sum_{k=0}^n (-1)^k \dim \Hs^k(M).\map
\end{align*}
\end{remenum}
\end{rem}

\begin{ex}
\begin{exenum}
\item Die Kohomologie der Sphäre ist in den Graden $0$ und $n$ konzentriert
\begin{align*}
b_{1}(S^n)=b_{n}(S^n) = 1,\qquad b_{k}(S^n) = 0,\qquad 1\le k\le n-1.
\end{align*}
\item Auf dem komplex projektiven Raum existiert die sogenannte Kählerform, diese ist gerade und harmonisch und nicht ausgeartet, so dass
\begin{align*}
b_{2k}(\CP^n) = 1,\qquad b_{2k+1}(\CP^n) =0.
\end{align*}
\item Sei $\Sigma_{g}$ eine Riemannsche Fläche vom Geschlecht $g$, dann gilt $b_{1} = 2g$.
\item Die Betti-Zahlen des Torus $T^n = S_{1}\times \ldots \times S_{1}$ sind
\begin{align*}
b_{k}(T^n) = \binom{n}{k}.\boxc
\end{align*}
\end{exenum}
\end{ex}

\begin{cor}
Sei $(M^n,g)$ eine kompakte, orientierte Riemannsche Mannigfaltigkeit. Dann gilt
\begin{align*}
H_{\dR}^k(M) \cong H_{\dR}^{n-k}(M).
\end{align*}
Insbesondere gilt $b_{n-k}(M) = b_{k}(M)$.\fish
\end{cor}

\begin{proof}
Da $*$ mit $\D$ vertauscht, definiert der Hodge-$*$-Operator den gewünschten Isomorphismus von $\Hs^{n-k}(M)$ und $\Hs^k(M)$.\qed
\end{proof}

\begin{rem}
Die Schnittform
\begin{align*}
H_{\dR}^k(M)\times H_{\dR}^k(M)\to \R,\qquad
([\alpha],[\beta]) \mapsto \int_{M} \alpha\wedge \beta
\end{align*}
ist nicht entartet, denn\begin{align*}
\int_{M}\alpha\wedge *\alpha =  \int_{M} g(\alpha,\alpha)\dM = \norm{\alpha}^2.\map
\end{align*}
\end{rem}

\section{Die Weizenböck-Formeln}

\begin{defn}
Der \emph{Bochner-Laplace-Operator (auch rough Laplacian)} ist definiert durch
\begin{align*}
\nabla^*\nabla \colon \Omega^k(M)\to \Omega^k(M),\qquad \nabla^*\nabla\omega = - \tr(\nabla^2\omega).\fish
\end{align*}
\end{defn}

Wie wir noch sehen werden, unterscheidet sich der Bochner-Laplace-Operator vom gewöhnlichen Laplace-Operator nur durch einen Term nullter Ordnung und auf Funktionen stimmen beide Operatoren überein.

\begin{rem}
Seien $E$ und $F$ Vektorbündel, dann ist der Zusammenhang auf $E\otimes F$ gegeben durch
\begin{align*}
\nabla_{X}^{E\otimes F} (e\otimes f) = (\nabla_{X}^E e)\otimes f + e\otimes (\nabla_{X}^F f).
\end{align*}
Somit ist $\nabla^2$ beschrieben durch
\begin{align*}
\Gamma(\Lambda^k T^*M) \overset{\nabla}{\longrightarrow}
\Gamma(T^*M\otimes\Lambda^k T^*M) \overset{\nabla}{\longrightarrow}
\Gamma(T^*M\otimes T^*M\otimes\Lambda^k T^*M).
\end{align*}
Für eine $k$-Form $\omega$ und Vektorfelder $X$ und $Y$ auf $M$ berechnet man explizit
\begin{align*}
(\nabla^2\omega)_{XY} = \nabla_{X}\nabla_{Y}\omega - \nabla_{\nabla_{X}Y}\omega.
\end{align*}
Auf Funktionen (also Nullformen) entspricht $\nabla^2$ also genau der Hessischen. Wählen wir eine lokale Orthonormalbasis $\setd{e_{i}}$, dann schreibt sich
\begin{align*}
\nabla^2 \omega = \sum_{i,j=1}^n e_{i}^* \otimes e_{j}^*\otimes \nabla_{e_{i}e_{j}}^2 \omega.
\end{align*}
Durch Kontraktion des Ausdrucks erhalten wir
\begin{align*}
\nabla^*\nabla \omega = - \tr\left(\nabla^2\omega\right) =  -\sum_{i,j=1}^n \nabla_{e_{i}e_{j}}^2\omega
\end{align*}
und daher ist insbesondere $\nabla^*\nabla f = \D f$ für eine Funktion $f\in\Cs^\infty(M)$.\map
\end{rem}

Wir betrachten für einen Moment die allgemeine Situation eines Vektorbündels $E$ versehen mit einem metrischen Zusammenhang $\nabla^E$, einer Fasermetrik $\lin{\cdot,\cdot}$ und der induzierten $L^2$ Metrik $(\cdot,\cdot)$. Die kovariante Ableitung schreibt sich dann als
\begin{align*}
\nabla^E : \Gamma(E)\to \Gamma(T^*M\otimes E),\qquad e\mapsto \nabla e,
\end{align*}
und ihr Dual schreibt sich als
\begin{align*}
(\nabla^E)^* : \Gamma(T^*M\otimes E)\to \Gamma^E,\qquad X^*\otimes e\mapsto \nabla^{T^*M\otimes E} (X^*\otimes e).
\end{align*}

\begin{lem}
\begin{propenum}
\item Die adjungierte kovariante Ableitung ist gegeben durch
\begin{align*}
(\nabla^E)^* = -\tr(\nabla^{T^*M\otimes E}).
\end{align*}
\item Für einen Schnitt $e$ in $E$ und eine lokale Orthonormalbasis $\setd{e_{i}}$ in $T_{m}M$ gilt
\begin{align*}
((\nabla^E)^*\nabla^E e)_{m} = -\tr\left((\nabla^E)^2e\right)_{m} = 
-\sum_{i=1}^n \left(\nabla_{e_{i}}\nabla_{e_{i}}e - \nabla_{\nabla_{e_{i}}e_{i}}e \right)_{m}.\fish
\end{align*}
\end{propenum}
\end{lem}

\begin{rem}
Die Fasermetrik auf dem Tensorprodukt $E\otimes F$ berechnet sich für Schnitte $e_{1}$, $e_{2}$ in $E$ und $f_{1}$, $f_{2}$ in $F$ zu
\begin{align*}
\lin{e_{1}\otimes f_{1},e_{2}\otimes f_{2}} = \lin{e_{1},f_{1}}\lin{e_{2},f_{2}}.\map
\end{align*}
\end{rem}

\begin{ex}
Seien $\alpha,\beta\in\Gamma(T^*M\otimes E)$ und $\setd{e_{i}}$ eine lokale Orthonormalbasis von $T_{m}M$, dann gilt
\begin{align*}
\alpha = \sum_{i=1}^n e_{i}^* \otimes \alpha(e_{i}),\qquad
\beta = \sum_{i=1}^n e_{i}^* \otimes \beta(e_{i}).
\end{align*}
So gilt dann
\begin{align*}
\lin{\alpha,\beta} = \sum_{i=1}^n \lin{\alpha(e_{i}),\beta(e_{i})}.
\end{align*}
Seien $\omega$, $\eta$ zwei $k$-Formen auf $M$, dann gilt
\begin{align*}
\lin{\nabla \omega, \nabla \eta} = \sum_{i=1}^n \lin{\nabla_{e_{i}}\omega, \nabla_{e_{i}}\eta}.\boxc
\end{align*}
\end{ex}

\begin{proof}[Beweis des Lemmas.]
a): Um zu zeigen, dass $(\nabla^E)^* = -\tr(\nabla^{T^*M\otimes E})$ müssen wir zeigen, dass $-\tr(\nabla^{T^*M\otimes E})$ dual zu $\nabla^E$ ist.

Seien $e$ und $f$ Schnitte in $E$ und $X$ ein Vektofeld auf $M$, dann gilt
\begin{align*}
\lin{\nabla^E e,X^*\otimes f} &= 
\sum_{i=1}^n \lin{e_{i}^*\otimes \nabla_{e_{i}}^E e, X^*\otimes f }\\
&=\sum_{i=1}^n \lin{e_{i}^*,X^*}\lin{\nabla_{e_{i}}^E e, f }\\
&= \lin{\nabla_{X}^E e,f}
= X(\lin{e,f}) - \lin{e,\nabla_{X}^E f},
\end{align*}
denn der Zusammenhang ist metrisch. Andererseits gilt auch
\begin{align*}
\tr(\nabla^{T^*M\otimes E}X^*\otimes f) &=
\tr\left( \sum_{i=1}^n e_{i}^* \otimes \nabla_{e_{i}}^{T^*M\otimes E} (X^*\otimes f) \right) \\
&=
\tr\left( \sum_{i=1}^n e_{i}^* \otimes (\nabla_{e_{i}}^{T^*M}X^*)\otimes f) + 
e_{i}^* \otimes X^*\otimes  (\nabla_{e_{i}}^{E}f) \right) \\
&=
\sum_{i=1}^n\left( \lin{e_{i}^*,\nabla_{e_{i}}^{T*M}X^*}f + 
\lin{e_{i}^*,X^*}\nabla_{e_{i}}^E f\right)\\
&= \div(X) f + \nabla_{X}^Ef.
\end{align*}
Somit gilt
\begin{align*}
&\lin{\nabla^E e,X^*\otimes f}  -
\lin{e,-\tr(\nabla^{T^*M\otimes E}X^*\otimes f)}\\
 &\qquad= 
\div(X)\lin{e,f} + X\lin{e,f} \\
&\qquad=  \div(\lin{e,f}X).
\end{align*}
Bei Integration über $M$ verschwindet der Divergenzterm aufgrund des Satzes von Gauß-Green. Wir haben damit gezeigt, dass
\begin{align*}
(\nabla^E e,X^*\otimes f) =  (e,-\tr(\nabla^{T^*M\otimes E}X^*\otimes f)),
\end{align*}
d.h. es gilt $(\nabla^E)^* = -\tr(\nabla^{T^*M\otimes E})$.

b): Sei $e$ ein Schnitt und $\setd{e_{i}}$ eine lokale Orthonormalbasis, die durch Parallelverschiebung fortgesetzt wird, d.h. $(\nabla e_{i})_{m} = 0$. Dann gilt
\begin{align*}
(\nabla^E)^*(\nabla^E e) &= (\nabla^E)^* \left(\sum_{i=1}^n e_{i}^* \otimes \nabla_{e_{i}} e_{i}\right)
= -\tr\left(\sum_{i=1}^n \nabla^{T^*M\otimes E} (e_{i}^* \otimes \nabla_{e_{i}}e) \right) \\
&= 
 -\tr\left(\sum_{j,i=1}^n e_{j}^*\otimes \nabla_{e_{j}}^{T^*M\otimes E} (e_{i}^* \otimes \nabla_{e_{i}}e)
 \right) \\
&= 
 -\tr\left(\sum_{j,i=1}^n e_{j}^*\otimes e_{i}^* \otimes \nabla_{e_{j}}\nabla_{e_{i}}e \right) \\
 &= - \tr(\nabla^2 e).\qed
\end{align*}
\end{proof}

\begin{rem}
Der Operator $\nabla^*\nabla$ ist nichtnegativ, denn für eine Form $\alpha$ gilt
\begin{align*}
(\nabla^* \nabla\alpha,\alpha)  =
(\nabla\alpha,\nabla\alpha)  = \norm{\nabla\alpha}^2 \ge0.\map
\end{align*}
\end{rem}

Wir wollen nun den Unterschied zwischen $\D$ und $\nabla^*\nabla$ präzisieren. Formeln dieser Art für $\D$ heißen Weitzenböck-Formeln.


\begin{prop}[Weizenböck-Formel I]
Auf $\Omega^k(M)$ gilt
\begin{align*}
\D = \nabla^*\nabla + q(R),
\end{align*}
wobei $q(R)\in\End(\Lambda^kT^*M)$ in einer lokalen Orthonormalbasis bestimmt ist durch
\begin{align*}
q(R) \defl \sum_{i,j=1}^n e_{j}^*\wedge \left(e_{i} \ic R_{e_{i},e_{j}}^{\Lambda^k T^*M}\right).
\end{align*}
Insbesondere gilt für $k=1$, dass
\begin{align*}
\D = \nabla^*\nabla + \Ric.\fish
\end{align*}
\end{prop}


\begin{rem}[Bemerkungen.]
\begin{remenum}
\item Der Endomorphismus $q(R)$ ist unabhängig von der Wahl der Orthonormalbasis.
\item 
Man kann den Raum der 2-Formen über einem Vektorraum $V$ mit den schiefsymmetrischen Matrizen identifizieren
\begin{align*}
\Lambda^2 V^* \cong \so(V).
\end{align*}
Dann operiert $\Lambda^2$ auf $\Lambda^k$ durch
\begin{align*}
(X^*\wedge Y^*)* \omega \defl Y^*\wedge X \ic \omega - X^* \wedge Y\ic \omega \in\Omega^k(M),
\end{align*}
für Vektorfelder $X$ und $Y$ auf $M$ und eine $k$-Form $\omega$. Bezüglich dieser Wirkung schreibt sich auch
\begin{align*}
q(R) = \sum_{i,j=1}^n (e_{i}^*\wedge e_{j}^*)* R(e_{i}\wedge e_{j})*.\map
\end{align*}
\end{remenum}
\end{rem}

\begin{lem}
\begin{propenum}
\item Für die Sphäre $S^n$ gilt auf $\Omega^k(S^n)$
\begin{align*}
q(R) = k(n-k)\Id_{\Lambda^kT^*S^n}.
\end{align*}
\item Falls $k=1$, dann ist $q(R) = \Ric$.\fish
\end{propenum}
\end{lem}
\begin{proof}
b): Sei $\lambda\in\Omega^1(M)=\Gamma(T^*M)$, dann berechnet sich die kovariante Ableitung zu
\begin{align*}
(\nabla_{X}^{T^*M}\lambda)(Y) = X\lambda(Y) - \lambda(\nabla_{X}Y),
\end{align*}
so dass für die Krümmung gilt
\begin{align*}
(R_{XY}^{T^*M}\lambda)(Z) = - \lambda(R_{XY}Z).
\end{align*}
Somit rechnet man für $q(R)$ nach, dass
\begin{align*}
(q(R)\lambda)(X) &= \left(\sum_{i,j=1}^n e_{j}^* \wedge (e_{i} \ic (R_{e_{i}e_{j}}\lambda) )\right)(X)\\
&= -\left(\sum_{i,j=1}^n e_{j}^* \wedge \lambda( R_{e_{i}e_{j}} e_{i} )\right)(X)\\
&= - \sum_{i=1}^n \lambda(R_{e_{i}X}e_{i}) = \lambda(\Ric(X)) = \Ric(\lambda)(X).\qed
\end{align*}
\end{proof}

\begin{proof}[Beweis der Weitzenböck-Formel I.]
Wir wählen in einem Punkt $m\in M$ geodätische Normalkoordinaten, so dass $(\nabla e_{i})_{m} = 0$ gilt. Mit den lokalen Darstellungen von $\ddd$ und $\ddd^*$ erhalten wir,
\begin{align*}
\ddd^*\ddd \omega &= \ddd^* \left(\sum_{i=1}^n e_{i}^* \wedge \nabla_{e_{i}}\omega\right) 
= 
-\sum_{i,j=1}^n e_{j}\ic  \nabla_{e_{j}}\left( e_{i}^* \wedge \nabla_{e_{i}}\omega\right)\\
&=
-\sum_{i,j=1}^n e_{j}\ic  \left( e_{i}^* \wedge \nabla_{e_{j}}\nabla_{e_{i}}\omega\right)\\
&=
-\sum_{i,j=1}^n \nabla_{e_{i}}\nabla_{e_{i}}\omega
+ 
\sum_{i,j=1}^n  e_{i}^* \wedge \left( e_{j} \ic \nabla_{e_{j}}\nabla_{e_{i}}\omega\right)\\
&= \nabla^*\nabla \omega + 
+ 
\sum_{i,j=1}^n  e_{i}^* \wedge \left( e_{j} \ic \nabla_{e_{j}}\nabla_{e_{i}}\omega\right).
\end{align*}
Analog berechnet man
\begin{align*}
\ddd\ddd^*\omega &= 
\ddd\left( -\sum_{j=1}^n e_{j}\ic \nabla_{e_{j}} \omega\right)
= 
 -\sum_{i,j=1}^n e_{i}^* \wedge \nabla_{e_{i}} \left(  e_{j}\ic \nabla_{e_{j}} \omega\right)\\
 &=
 -\sum_{i,j=1}^n e_{i}^* \wedge  \left(  e_{j}\ic \nabla_{e_{i}}\nabla_{e_{j}} \omega\right).
\end{align*}
Addieren dieser Terme ergibt schließlich
\begin{align*}
\D \omega &= \ddd^*\ddd\omega + \ddd\ddd^*\omega \\ 
& = \nabla^*\nabla \omega + \sum_{i,j=1}^n e_{i}^*\wedge \left(
e_{j}\ic \left(\nabla_{e_{j}}\nabla_{e_{i}}\omega - \nabla_{e_{i}}\nabla_{e_{j}}\omega\right)\right)\\
& = \nabla^*\nabla \omega + \sum_{i,j=1}^n e_{i}^*\wedge \left(
e_{j}\ic R_{e_{j}{e_{i}}}\omega\right).\qed
\end{align*}
\end{proof}

Informationen über die Krümmung übersetzen sich mit den Weitzenböck-Formeln in Informationen über den Kern des Laplace Operators, durch den z.B. die Bettizahlen beschrieben werden. Diese Formeln stellen somit eine Verbindung zwischen Topologie und Geometrie her.

\begin{cor}
Sei $(M,g)$ eine kompakte Riemannsche Mannigfaltigkeit mit nichtnegativer Ricci-Krümmung, d.h.
$(\Ric(X),X) \ge 0$ für alle Vektorfelder $X$ auf $M$. Dann sind alle harmonischen 1-Formen parallel, genauer gilt für solch eine Form $\omega$,
\begin{align*}
\nabla\omega = 0,\qquad \text{und}\qquad (\Ric(\omega),\omega) = 0.\fish
\end{align*}
\end{cor}
\begin{proof}
Sei $\omega$ eine harmonische 1-Form, dann gilt nach der Weizenböckformel I,
\begin{align*}
0 = \D \omega = \nabla^*\nabla\omega + \Ric(\omega).
\end{align*}
Da $M$ kompakt ist, ist $\omega$ integrabel und wir erhalten durch Integration
\begin{align*}
0 = (\nabla^*\nabla\omega + \Ric(\omega),\omega) = 
\norm{\nabla\omega}^2 + (\Ric(\omega),\omega).
\end{align*}
Die beiden Summanden auf der rechten Seite sind nichtnegativ, folglich sind beide Null. Insbesondere ist $\nabla\omega = 0$ und  $\omega$ parallel.\qed
\end{proof}

\begin{rem}
Solche Beweise bezeichnet man als ``Anwendung der Bochner Methode''.\map
\end{rem}

\begin{cor}
Sei $(M,g)$ eine kompakte, orientierte Riemannsche Mannigfaltigkeit.
\begin{propenum}
\item Gilt in einem Punkt $\Ric > 0$, so verschwindet die erste de-Rham Kohomologie, d.h. $b_{1}(M) = 0$.
\item Die Dimension der ersten de-Rham Kohomologie ist durch die der Mannigfaltigkeit beschränkt, d.h.
\begin{align*}
b_{1}(M) \le \dim M,
\end{align*}
wobei Gleichheit nur dann gilt, wenn $g$ flach ist.\fish
\end{propenum}
\end{cor}

\begin{proof}
a): Nach vorigem Korollar gilt $(\Ric(\omega),\omega) = 0$ für jede harmonische 1-Form. Die Ricci-Krümmung ist jedoch nach Voraussetzung in mindestens einem Punkt strikt positiv. Daher kann es keine nichttriviale harmonische 1-Form geben und folglich  ist $b_{1}(M) = \dim \Hs^1(M) = 0$.

b): Auf einer $n$-dimensionalen Mannigfaltigkeit kann es höchstens $n$ linear unabhängige parallele 1-Formen geben, denn diese trivialisieren das Bündel $\Lambda^1TM$ lokal.

Gilt Gleichheit, so existieren $n$ linear unabhängige, parallele Vektorfelder $E_{1},\ldots,E_{n}$. Auf diesen verschwindet der Krümmungstensor an jedem Punkt, d.h.
\begin{align*}
R_{E_{i},E_{j}}E_{k} = 0,\qquad 1\le i,j,k\le n.
\end{align*}
Also ist $g$ flach.\qed
\end{proof}

Eine Verschärfung dieses Resultats liefert der folgende

\begin{prop}[Satz von Myers]
Sei $(M,g)$ eine kompakte, orientierte Riemannsche Mannigfaltigkeit und $\Ric \ge \kappa > 0$. Dann ist die Fundamentalgruppe von $M$ endlich und daher insbesondere die erste de-Rham Kohomologie trivial.\fish
\end{prop}

Wir wollen nun den Zusammenhang zwischen der kovarianten Ableitung und dem Differential und dem Kodifferential genauer untersuchen. Dazu benötigen wir noch folgendes Ergebnis aus der linearen Algebra.

\begin{lem}
Sei $(V,\lin{\cdot,\cdot})$ ein euklidischer Vektorraum. Dann gilt
\begin{align*}
V^* \otimes \Lambda^k V^* \cong 
\Lambda^{k+1} V^*\oplus
\Lambda^{k-1} V^*\oplus
\Lambda^{k,1} V^*,
\end{align*}
wobei für die Abbildungen
\begin{align*}
\pr_{k+1} &: V^*\otimes \Lambda^k V^*\to \Lambda^{k+1}V^*, &&
X^*\otimes\omega\mapsto X\wedge\omega,\\
i_{k+1} &: \Lambda^{k+1}V^*\to V^*\otimes \Lambda^k V^*, && \alpha \mapsto
\frac{1}{k+1} \sum_{i=1}^n e_{i}^* \wedge (e_{i}\ic\alpha),
\end{align*}
gilt $\pr_{k+1}\circ i_{k+1}=\Id_{\Lambda^{k+1}}$, und $i_{k+1}\circ \pr_{k+1}$ ist die Projektion auf $i_{k+1}(\Lambda^{k+1}V^*) \subset V^*\otimes \Lambda^k V^*$. Weiterhin erfüllen die Abbildungen
\begin{align*}
\pr_{k-1} &: V^*\otimes \Lambda^k V^* \to \Lambda^{k-1}V^*,&& X^*\otimes \omega \mapsto X\ic \omega,\\
i_{k-1} &: \Lambda^{k-1}V^* \to V^*\Lambda^k V^*, &&
\alpha \mapsto \frac{1}{n-k+1} \sum_{i=1}^n e_{i}^* \otimes (e_{i}^*\wedge \alpha)
\end{align*}
die Relation $\pr_{k-1}\circ i_{k-1} = \Id_{\Lambda^{k-1}}$. Schließlich ist der Raum $\Lambda^{k,1}V^*$ gegeben durch
\begin{align*}
\Lambda^{k,1}V^* = \ker(\pr_{k-1}\cap \pr_{k+1}) \subset V^*\otimes \Lambda^k V^*,
\end{align*}
und seine Projektion $\pr_{k,1} \colon   V^*\otimes \Lambda^{k}V^* \to \Lambda^{k,1}V^*$ ist definiert durch
\begin{align*}
\pr_{k,1} (X^*\otimes \omega)v = 
X^*(v)\omega - \frac{1}{k+1}v\ic (X^*\wedge\omega) - \frac{1}{n-k+1} v^* \wedge(X^* \ic \omega).\fish
\end{align*}
\end{lem}

Wir wollen dies nun auf Mannigfaltigkeiten übertragen. Hier erhalten wir folgende orthogonale Aufspaltung
\begin{align*}
T^*M \otimes \Lambda^k T^*M \cong \Lambda^{k+1}T^*M \oplus \Lambda^{k-1}T^*M \oplus \Lambda^{k,1}T^*M,
\end{align*}
die vom Levi-Civita-Zusammenhang erhalten wird. Man nennt $\Lambda^{k,1}T^*M$ auch den Cartan-Summanden. Für eine $k$-Form $\omega$ schreibt sich ihre kovariante Ableitung als
\begin{align*}
\nabla \omega = \sum_{i=1}^n e_{i}^* \otimes \nabla_{e_{i}}\omega,
\end{align*}
während Differential und Kodifferential gegeben sind durch
\begin{align*}
\ddd \omega = \sum_{i=1}^n e_{i}^* \wedge \nabla_{e_{i}}\omega,\qquad
\ddd^* \omega = -\sum_{i=1}^n e_{i} \ic \nabla_{e_{i}}\omega.
\end{align*}
Wenden wir obige Projektionsabbildungen an, erhalten wir unmittelbar folgendes Ergebnis.

\begin{lem}
Sei $\omega$ eine $k$-Form auf $M$. Dann gelten
\begin{propenum}
\item $\dom = \pr_{k+1}(\nabla \omega)$, und
\item $\ddd^*\omega = -\pr_{k-1}(\nabla \omega)$.\fish
\end{propenum}
\end{lem}

Die Projektion der kovarianten Ableitung auf den Cartan-Summanden $\Lambda^{k,1}T^*M$ hat ebenfalls einen Namen.

\begin{defn}
\index{Twistor-Operator}
Der \emph{Twistor-Operator} ist definiert durch
\begin{align*}
P\defl \pr_{k,1}\circ \nabla \colon \Omega^k(M)\to \Gamma(T^*M\otimes \Lambda^k T^*M),
\end{align*}
d.h. für ein Vektorfeld $X$ auf $M$ und eine $k$-Form $\omega$ gilt
\begin{align*}
(P\omega)(X) = \nabla_{X}\omega - \frac{1}{k+1}X \ic \dom + \frac{1}{n-k+1} X^*\wedge \ddd^*\omega.\fish
\end{align*}
\end{defn}

\begin{lem}
\label{lem:Killing-Twistor}
Sei $X$ ein Vektorfeld auf $M$.
\begin{propenum}
\item Es gilt $PX^* = 0$ genau dann, wenn $X$ ein konformes Vektorfeld ist, d.h. wenn $L_{X}g = f\cdot g$ mit einer glatten Funktion $f$ gilt.
\item Es gilt $PX^* = 0$ und $\ddd^* X^* = 0$ genau dann, wenn $X$ ein Killing-Vektorfeld auf $M$ ist, d.h. wenn $L_{X}g = 0$ gilt.\fish
\end{propenum}
\end{lem}

\begin{proof}
Wir schreiben $\omega = X^*$, dann gilt $P\omega = 0$ genau dann, wenn für jedes Vektorfeld $Y,Z$ auf $M$ gilt,
\begin{align*}
0 &=  (\nabla_{Y}\omega)(Z) - \frac{1}{2}\dom(Y,Z) + \frac{1}{n} Y^*(Z) \cdot \ddd^*\omega\\
&= (\nabla_{Y}\omega)(Z) - \frac{1}{2}\left( (\nabla_{Y}\omega)(Z) - (\nabla_{Z}\omega)(Y) \right) + \frac{1}{n} g(Y,Z) \cdot \ddd^*\omega\\
&= \frac{1}{2}\left( (\nabla_{Y}\omega)(Z) + (\nabla_{Z}\omega)(Y) \right) + \frac{1}{n} g(Y,Z) \cdot \ddd^*\omega\\
&= \frac{1}{2}(L_{X})g(Y,Z) + \frac{1}{n} g(Y,Z) \cdot \ddd^*\omega.
\end{align*}
Also gilt $PX^* = 0$ genau dann, wenn
\begin{align*}
L_{X}g = -\frac{2}{n}\ddd^*X^*\cdot g.
\end{align*}
Somit folgen a) und b).\qed
\end{proof}

\begin{prop}[Weitzenböck Formel II]
Sei $\omega$ eine $k$-Form auf $M$. Dann gelten die Formeln
\begin{propenum}
\item 
\[
\frac{1}{k+1}\ddd^*\ddd\omega + \frac{1}{n-k+1}\ddd\ddd^*\omega + P^*P\omega = \nabla^*\nabla\omega,
\]
\item 
\[
\frac{k}{k+1}\ddd^*\ddd\omega + \frac{n-k}{n-k+1}\ddd\ddd^*\omega - P^*P\omega = q(R)\omega.\fish
\]
\end{propenum}
\end{prop}
Offenbar ergeben a) und b) aufaddiert wiederum $\D\omega = \nabla^*\nabla\omega + q(R)\omega$. Für 1-Formen erhalten wir unmittelbar das folgende Resultat.

\begin{cor}
Sei $\omega$ eine 1-Form auf $M$. So gilt
\begin{align*}
\frac{1}{2}\ddd^*\dom + \frac{n-1}{n}\ddd\ddd^*\omega - P^*P\omega = \Ric(\omega).\fish
\end{align*}
\end{cor}

\begin{cor}[Satz von Lichnerowicz-Obata]
Sei $(M,g)$ eine kompakte Riemannsche Mannigfaltigkeit mit $\Ric \ge \kappa g$ für ein $\kappa > 0$. Dann gilt für den ersten von Null verschiedenen Eigenwert $\lambda_{1}$ des Laplace-Operators auf Funktionen, dass
\begin{align*}
\lambda_{1} \ge \frac{n}{n-1}\kappa,
\end{align*}
wobei Gleichheit genau dann gilt, wenn $(M,g)$ isometrisch zur Standardsphäre ist.~\fish
\end{cor}

\begin{proof}
Sei $\lambda_{1}$ der erste positive Eigenwert, also $\D f = \lambda_{1}f$ mit einer nichtkonstanten Funktion $f$, d.h. $\df \neq 0$. Dann gilt nach der Weitzenböck Formel II, dass
\begin{align*}
\frac{1}{2}\ddd^*\ddd\df + \frac{n-1}{n}\ddd\ddd^*\df = 
\frac{n-1}{n}\D\df = 
\Ric(\df) + P^*P(\df).
\end{align*}
Durch Integration dieser Gleichung und Verwendung von $\Ric \ge \kappa$ und $P^*P \ge 0$, erhalten wir
\begin{align*}
\frac{n}{n-1}\kappa(\df,\df) \le (\D\df,\df) = (\ddd\D f,\df) = \lambda_{1}(\df,\df).
\end{align*}
Folglich ist $\lambda_{1} \ge n/(n+1)\kappa$.\qed
\end{proof}

\section{Charakterisierung von Killing Vektorfeldern}

Sei $(M,g)$ eine $n$-dimensionale Riemannsche Mannigfaltigkeit. Ein Vektorfeld $X$ auf $M$ heißt Killing, wenn $L_{X}g = 0$ gilt. Dies ist äquivalent dazu, dass der lokale Fluss von $X$ aus Isometrien von $g$ besteht.

\begin{defn}
\index{Isometriegruppe}
Die \emph{Isometriegruppe} von $M$ ist definiert durch
\begin{align*}
\Iso(M,g) \defl \setdef{f: M\to M}{f\text{ ist ein Diffeomorphismus mit }f^*g = g}.\fish
\end{align*}
\end{defn}

\begin{prop}
Sei $M$ eine Riemannsche Mannigfaltigkeit mit endlich vielen Zusammenhangskomponenten.
\begin{propenum}
\item 
Die Isometriegruppe ist eine endlichdimensionale Lie-Gruppe, genauer gilt
\begin{align*}
\dim \Iso(M,g) \le \frac{1}{2}n(n+1) - \dim \Iso(S^n,g_{\o}).
\end{align*}
\item Ist die Mannigfaltigkeit $M$ kompakt, so auch $\Iso(M,g)$.
\item Ist $M$ vollständig, so ist die Lie-Algebra von $\Iso(M,g)$ genau die Lie-Algebra der Killing-Vektorfelder auf $M$.\fish
\end{propenum}
\end{prop}

Die Killing-Vektorfelder kodieren also Informationen über die Isometriegruppe. Je mehr man über die Killing-Vektorfelder weiß, desto mehr kann man über die Isometriegruppe aussagen. Die Killing-Vektorfelder sind daher wichtige und häufig untersuchte Objekte der Differentialgeometrie.


\begin{cor}
Sei $X$ ein Killing-Vektorfeld auf $M$. Dann gelten
\begin{propenum}
\item $\D X^* = 2\Ric(X^*)$, und
\item $\ddd^* X^*  = 0$.\fish
\end{propenum}
Ist die Mannigfaltigkeit $M$ kompakt, so charakterisieren die Eigenschaften a) und b) die Killing-Vektorfelder.\fish
\end{cor}
\begin{proof}
Nach Lemma \ref{lem:Killing-Twistor} ist ein Vektorfeld $X$ genau dann Killing, wenn $PX^* = 0$ und $\ddd^* X^* = 0$ gilt. 
Sei $X$ Killing, also $PX^* = 0$, dann folgt mit der Weitzenböck Formel II, dass
\begin{align*}
\frac{1}{2}\D X^* = \frac{1}{2}\ddd^*\ddd X^* = \Ric(X^*).
\end{align*}

Umgekehrt sei nun $\D X^* = 2\Ric(X^*)$ und $\ddd^* X^* = 0$. Dann gilt $\D X^* = \ddd^*\ddd X^* $ und es folgt
\begin{align*}
P^*P(X^*) = \frac{1}{2}\D X^* - \Ric(X^*) = 0.
\end{align*}
Integration liefert nun $(PX^*,PX^*) = 0$ und daher ist $X$ ein Killing Vektorfeld.\qed
\end{proof}

\begin{cor}
Sei $M$ eine kompakte Riemannsche Mannigfaltigkeit.
\begin{propenum}
\item Trägt $M$ in einem Punkt strikt negative Ricci-Krümmung, so ist die Isometriegruppe endlich.
\item Gilt $\Ric \le 0$, so sind alle Killing Vektorfelder parallel und $I(M,g)_{\o}$ ist ein Torus.
\item Gilt $\Ric \equiv 0$, so ist $\dim I(M,g) = b_{1}(M)$.\fish
\end{propenum}
\end{cor}
\begin{proof}
a): Für Killing Vektorfelder gilt $\D X^* = 2\Ric(X^*)$. Da $\D$ ein positiver Operator ist, aber $\Ric < 0$ in einem Punkt gilt, kann es keine nichttrivialen Killing Vektorfelder geben. Also ist $\Lie(I(M,g)) = (0)$ und daher ist $I(M,g)$ diskret und kompakt, also endlich.

b): Sei $X$ ein Killing-Vektorfeld. So gilt
\begin{align*}
2\Ric(X^*) = \D X^* = \nabla^*\nabla X^* + \Ric(X^*),
\end{align*}
d.h. $\nabla^*\nabla X^* = \Ric(X^*)$. Integration liefert nun
\begin{align*}
\norm{\nabla X^*}^2 = (\Ric(X^*),X^*) \le 0,
\end{align*}
also gilt $\nabla X^* = 0$ und $X$ ist parallel.

Da Killing-Vektorfelder parallel sind, kommutieren diese, denn
\begin{align*}
[X,Y] = \nabla_{X}Y - \nabla_{Y}X.
\end{align*}
Somit ist die Lie-Algebra der Killing-Vektorfelder abelsch und folglich $\Lie(I(M,g)_{o})$ eine abelsche, zusammenhängende, kompakte Gruppe. Eine solche Gruppe nennt man Torus.

c): Da die Ricci-Krümmung identisch verschwindet, ist ein Vektorfeld genau dann Killing, wenn
\begin{align*}
\D X^* = 2\Ric(X^*) = 0
\end{align*}
gilt. Somit gilt
\begin{align*}
b_{1}(M) = \dim\Hs^1(M) = \dim\Lie(I(M,g)) = \dim I(M,g).\qed
\end{align*}
\end{proof}

\printindex

\bibliographystyle{plain}
\bibliography{bibliography}

\end{document}